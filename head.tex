%Packages

%unicode formatting
\usepackage[T1]{fontenc}
\pdfinclusioncopyfonts=1
\usepackage[utf8]{inputenc}



%%%%%%%%%%%%%%%%%%%%%%%%%%%%%%%%

%References
\usepackage[english]{babel}
\usepackage{csquotes}
\usepackage[
			sorting=nyc,
            backend=biber, %backend to use
            style=authoryear, %citation style
            uniquelist=false,
            natbib,
            block=space, % allow additional horizontal space between blocks
            maxnames=2, % How many names show up in the in-text citation
            maxbibnames=99 %How many names show up in the references/bibliography section
            ]{biblatex}


%Make the title of the bibliography say 'References'
\DefineBibliographyStrings{english}{%
	bibliography = {References},
}


%Sort by name-year-cite order (biblatex default is to sort by name-year-title).
\DeclareSortingTemplate{nyc}{
	\sort{
		\field{presort}
	}
	\sort[final]{
		\field{sortkey}
	}
	\sort{
		\field{sortname}
		\field{author}
		\field{editor}
		\field{translator}
		\field{sorttitle}
		\field{title}
	}
	\sort{
		\field{sortyear}
		\field{year}
	}
	\sort{\citeorder}
}

% Use last name, initial. unless this is non-unique. If this is non-unique, use full first name.
%From https://tex.stackexchange.com/a/631580
\DeclareNameFormat{always-init}{%
	\ifnumequal{\value{uniquename}}{2}
	{\usebibmacro{name:family-given}
		{\namepartfamily}
		{\namepartgiven}
		{\namepartprefix}
		{\namepartsuffix}}
	{\usebibmacro{name:family-given}
		{\namepartfamily}
		{\namepartgiveni}
		{\namepartprefix}
		{\namepartsuffix}}%
	\usebibmacro{name:andothers}}



\DeclareNameAlias{author}{always-init}
\DeclareNameAlias{editor}{always-init}

\ExecuteBibliographyOptions{useprefix=true}
\DeclareSortingNamekeyTemplate{
	\keypart{\namepart{family}}
	\keypart{\namepart{prefix}}
	\keypart{\namepart{given}}
	\keypart{\namepart{suffix}}}


%% Modify citations to follow the style of Cell
%% from https://tex.stackexchange.com/a/404787
\usepackage{xpatch}

% Some general changes
\DeclareNameAlias{sortname}{always-init}
\renewcommand*{\bibinitdelim}{}
\renewbibmacro*{in:}{%
    \iffieldequalstr{entrytype}{inproceedings}{%
        \printtext{\bibstring{in}\addspace}%
    }{}%
}

% Changes for Book
\csletcs{abx@macro@publisher+location+date@orig}{abx@macro@publisher+location+date}
\renewbibmacro*{publisher+location+date}{%
    \printtext[parens]{\usebibmacro{publisher+location+date@orig}}
}
\DeclareFieldFormat[book]{title}{\mkbibitalic{#1}\printunit{\addspace}}

% Changes for inproceedings
\DeclareFieldFormat[inproceedings]{title}{#1\isdot}
\DeclareFieldFormat{booktitle}{\mkbibitalic{#1}\addcomma}
\xpatchbibmacro{byeditor+others}{%
    \usebibmacro{byeditor+othersstrg}%
    \setunit{\addspace}%
    \printnames[byeditor]{editor}%
    \clearname{editor}%
}{%
    \printnames[byeditor]{editor}%
    \clearname{editor}
    \addcomma\addspace
    \bibstring{editor}
    \setunit{\addspace}%
}{}{}

% Changes in Article
\DeclareFieldFormat[article]{title}{#1}
\DeclareFieldFormat[article]{journaltitle}{\mkbibitalic{#1}\isdot}
\DeclareFieldFormat[article]{volume}{\textit{#1}}
\DeclareFieldFormat[article]{pages}{#1}


% Changes for preprints
\DeclareFieldFormat[misc]{title}{#1}
\DeclareFieldFormat[misc]{publisher}{\mkbibitalic{#1}\isdot}

%Don't put parentheses around the year in the bibliography
%from https://tex.stackexchange.com/a/428193
\xpatchbibmacro{date+extradate}{%
	\printtext[parens]%
}{%
	\setunit*{\addcomma\space}%
	\printtext%
}{}{} 


%Clear some things I dont want going into the bibliography
%from https://tex.stackexchange.com/a/89848
\AtEveryBibitem{\clearfield{month}}
\AtEveryBibitem{\clearfield{day}}
\AtEveryBibitem{\clearfield{url}}
\AtEveryBibitem{\clearfield{urlyear}}
\AtEveryBibitem{\clearfield{note}}



%%%%%%%%%%%%%%%%%%%%
%Adjust some lengths related to bibliography
%from https://tex.stackexchange.com/a/140153

%space between consecutive entries in the bibliography
\setlength{\bibitemsep}{1.5ex}

%space between consecutive entries of authors with DIFFERENT names in the bibliography. This is called in place of \bibitemsep when the names are different.
%\setlength{\bibnamesep}{1.5\bibitemsep}

%space between consecutive entries of authors with same last names but different first names in the bibliography. This is called in place of \bibitemsep when the names are different.
%\setlength{\bibinitsep}{1.5\bibitemsep}

%The indent of follow up lines of the bibliography
\setlength{\bibhang}{1.5\parindent}

%%%%%%%%%%%%%%%%
%Add a new command for citing things with possesive apostrophe. For example, if I want to write:  Smith's (2023) big idea, the apostrophe needs to go after the author but before the year
%From https://tex.stackexchange.com/a/537765
%The relevant command is \posscite
\DeclareNameWrapperFormat{labelname:poss}{#1's}

\newrobustcmd*{\posscitealias}{%
	\AtNextCite{%
		\DeclareNameWrapperAlias{labelname}{labelname:poss}}}

\newrobustcmd*{\posscite}{%
	\posscitealias
	\textcite}

\newrobustcmd*{\Posscite}{\bibsentence\posscite}

\newrobustcmd*{\posscites}{%
	\posscitealias
	\textcites}

%%%%%%%%%%%%%%%%%%%%%%%%%

\usepackage{titlesec} % Format chapter headings
%from https://tex.stackexchange.com/a/52474

%Add a line b/w chapter number and heading
%for numbered chapters
\titleformat{\chapter}[display]
  {\normalfont\bfseries\Huge}
  {\chaptertitlename~\thechapter}{1pc}
  {{\color{gray}\titlerule[2pt]}\vspace{1pc}}

%Don't add a line for unnumbered chapters  
\titleformat{name=\chapter,numberless}[display]
  {\normalfont\bfseries\Huge}{}{1pc}
  {}

%reduce spacing before and after section titles
%this is to be read {left spacing}{before spacing}{after spacing}
%spacing: how to read {12pt plus 4pt minus 2pt}
%           12pt is what we would like the spacing to be
%           plus 4pt means that TeX can stretch it by at most 4pt
%           minus 2pt means that TeX can shrink it by at most 2pt
\titlespacing\section{0pt}{0pt plus 0pt minus 1pt}{0pt plus 0pt minus 1pt}
\titlespacing\subsection{0pt}{0pt plus 0pt minus 1pt}{0pt plus 0pt minus 1pt}
\titlespacing\subsubsection{0pt}{0pt plus 0pt minus 1pt}{0pt plus 0pt minus 1pt}

%Add a custom strut to increase vertical space given to equations when combining with underbrace
%from https://tex.stackexchange.com/a/13864
\newcommand*\mystrut[1]{\vrule width0pt height0pt depth#1\relax}


%%%%%%%%%%%%%
\usepackage{epigraph} %for quotes
\renewcommand{\textflush}{flushright} %quotes are right aligned

\usepackage{amsthm, amsmath, amssymb} % Mathematical typesetting
\usepackage{mathrsfs} %fancy fonts for sigma-algebras
\usepackage{float} % Improved interface for floating objects
\usepackage[final, colorlinks = true, 
            linkcolor = black, 
            citecolor = black,
            urlcolor = blue,
            breaklinks=true]{hyperref} % For hyperlinks in the PDF
\usepackage{multicol} % Enable multiple columns in text
\usepackage{graphicx} % Enhanced support for graphics
\usepackage{xcolor} % Driver-independent color extensions
\usepackage{physics} %some physics symbols
\usepackage{framed}
\usepackage[normalem]{ulem} %underlining
\usepackage{multicol}  %for double column text
\usepackage{color,soul} %for colored text
\usepackage{pdfpages} %for inserting PDFs
\usepackage{ragged2e} %for formatting
\usepackage{amsfonts}
\usepackage{enumitem}
\usepackage{mathtools}

\usepackage[toc, title]{appendix} %Appendix

\usepackage[labelfont=bf]{caption} %bold captions on figures and tables

%for tables
\usepackage{makecell,tabularx}
%\setlength{\extrarowheight}{12pt} %additive padding
\renewcommand{\arraystretch}{2} %multiplicative padding 
\renewcommand\theadfont{\small\bfseries}
\usepackage{rotating}
\usepackage{setspace}

%For pseudocode
\usepackage[boxed]{algorithm2e}
\DontPrintSemicolon

%A bunch of definitions that make my life easier
\newtheorem{theorem}{Theorem}[section]
\newtheorem{corollary}{Corollary}[section]
\theoremstyle{definition}
\newtheorem*{definition}{Definition}
\newtheorem{example}{Example}
\newtheorem*{note}{Note}
\newtheorem*{claim}{Claim}
\newcommand{\bproof}{\bigskip {\bf Proof. }}
\newcommand{\eproof}{\hfill\qedsymbol}
\newcommand{\Disp}{\displaystyle}
\newcommand{\qe}{\hfill\(\bigtriangledown\)}
\setlength{\columnseprule}{1 pt}

%for characters inside circles
%syntax is \circled{character}
\usepackage{tikz}
\newcommand*\circled[1]{\tikz[baseline=(char.base)]{
            \node[shape=circle,draw,inner sep=1pt] (char) {#1};}}


\usepackage{pgfplots}
\pgfplotsset{compat=1.8}


\usepackage[mode=buildnew]{standalone} %for loading precompiled tikz figures

%%%%%%%%%%%%%%%%%%%%%%%%
% Defines the `mycase` environment for cases in proofs
\newcounter{cases}
\newcounter{subcases}[cases]
\newenvironment{mycase}
{
    \setcounter{cases}{0}
    \setcounter{subcases}{0}
    \newcommand{\case}
    {
        \stepcounter{cases}\textbf{Case \thecases.}
    }
    \newcommand{\subcase}
    {
        \par\indent\stepcounter{subcases}\textit{Subcase (\thesubcases):}
    }
}
{
    \par
}
\renewcommand*\thecases{\arabic{cases}}
\renewcommand*\thesubcases{\roman{subcases}}

%For easily making figures
%syntax is:
%\myfig{scaling_factor}{name_of_file}{caption}{label}
\newcommand{\myfig}[4]{\begin{figure}[h] \begin{center} \includegraphics[width=#1\textwidth]{#2} \caption{#3} \label{#4} \end{center} \end{figure}}

% horizontal line across the page
\newcommand{\horz}{
\vspace{-.4in}
\begin{center}
\begin{tabular}{p{\textwidth}}\\
\hline
\end{tabular}
\end{center}
}

%Resize the summation symbol
%syntax is \sum[size] 
\newlength{\depthofsumsign}
\setlength{\depthofsumsign}{\depthof{$\sum$}}
\newlength{\totalheightofsumsign}
\newlength{\heightanddepthofargument}
\newcommand{\bigsum}[1][1.4]{% only for \displaystyle
    \mathop{%
        \raisebox
            {-#1\depthofsumsign+1\depthofsumsign}
            {\scalebox
                {#1}
                {$\displaystyle\sum$}%
            }
    }
}



%Add a footer without a marker/reference in the main text
%From https://tex.stackexchange.com/a/544121
\newcommand\extrafootertext[1]{%
	\bgroup
	\renewcommand\thefootnote{\fnsymbol{footnote}}%
	\renewcommand\thempfootnote{\fnsymbol{mpfootnote}}%
	\footnotetext[0]{#1}%
	\egroup
}