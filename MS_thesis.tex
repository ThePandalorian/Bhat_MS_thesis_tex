\documentclass[twoside, 12pt]{iiser-thesis}

%Bio dept wants all font in Arial (ew)
%Will keep it commented until the v end for my own sanity. I refuse to actively change a tex file to Arial permanently. Change compiler to XeLaTeX or LuaLaTeX for this to work
%\usepackage{fontspec}
%\setmainfont{Arial}

%my own list of packages
%Packages

%unicode formatting
\usepackage[T1]{fontenc}
\pdfinclusioncopyfonts=1
\usepackage[utf8]{inputenc}



%%%%%%%%%%%%%%%%%%%%%%%%%%%%%%%%

%References
\usepackage[english]{babel}
\usepackage{csquotes}
\usepackage[
			sorting=nyc,
            backend=biber, %backend to use
            style=authoryear, %citation style
            uniquelist=false,
            natbib,
            block=space, % allow additional horizontal space between blocks
            maxnames=2, % How many names show up in the in-text citation
            maxbibnames=99 %How many names show up in the references/bibliography section
            ]{biblatex}


%Make the title of the bibliography say 'References'
\DefineBibliographyStrings{english}{%
	bibliography = {References},
}


%Sort by name-year-cite order (biblatex default is to sort by name-year-title).
\DeclareSortingTemplate{nyc}{
	\sort{
		\field{presort}
	}
	\sort[final]{
		\field{sortkey}
	}
	\sort{
		\field{sortname}
		\field{author}
		\field{editor}
		\field{translator}
		\field{sorttitle}
		\field{title}
	}
	\sort{
		\field{sortyear}
		\field{year}
	}
	\sort{\citeorder}
}

% Use last name, initial. unless this is non-unique. If this is non-unique, use full first name.
%From https://tex.stackexchange.com/a/631580
\DeclareNameFormat{always-init}{%
	\ifnumequal{\value{uniquename}}{2}
	{\usebibmacro{name:family-given}
		{\namepartfamily}
		{\namepartgiven}
		{\namepartprefix}
		{\namepartsuffix}}
	{\usebibmacro{name:family-given}
		{\namepartfamily}
		{\namepartgiveni}
		{\namepartprefix}
		{\namepartsuffix}}%
	\usebibmacro{name:andothers}}



\DeclareNameAlias{author}{always-init}
\DeclareNameAlias{editor}{always-init}

\ExecuteBibliographyOptions{useprefix=true}
\DeclareSortingNamekeyTemplate{
	\keypart{\namepart{family}}
	\keypart{\namepart{prefix}}
	\keypart{\namepart{given}}
	\keypart{\namepart{suffix}}}


%% Modify citations to follow the style of Cell
%% from https://tex.stackexchange.com/a/404787
\usepackage{xpatch}

% Some general changes
\DeclareNameAlias{sortname}{always-init}
\renewcommand*{\bibinitdelim}{}
\renewbibmacro*{in:}{%
    \iffieldequalstr{entrytype}{inproceedings}{%
        \printtext{\bibstring{in}\addspace}%
    }{}%
}

% Changes for Book
\csletcs{abx@macro@publisher+location+date@orig}{abx@macro@publisher+location+date}
\renewbibmacro*{publisher+location+date}{%
    \printtext[parens]{\usebibmacro{publisher+location+date@orig}}
}
\DeclareFieldFormat[book]{title}{\mkbibitalic{#1}\printunit{\addspace}}

% Changes for inproceedings
\DeclareFieldFormat[inproceedings]{title}{#1\isdot}
\DeclareFieldFormat{booktitle}{\mkbibitalic{#1}\addcomma}
\xpatchbibmacro{byeditor+others}{%
    \usebibmacro{byeditor+othersstrg}%
    \setunit{\addspace}%
    \printnames[byeditor]{editor}%
    \clearname{editor}%
}{%
    \printnames[byeditor]{editor}%
    \clearname{editor}
    \addcomma\addspace
    \bibstring{editor}
    \setunit{\addspace}%
}{}{}

% Changes in Article
\DeclareFieldFormat[article]{title}{#1}
\DeclareFieldFormat[article]{journaltitle}{\mkbibitalic{#1}\isdot}
\DeclareFieldFormat[article]{volume}{\textit{#1}}
\DeclareFieldFormat[article]{pages}{#1}


% Changes for preprints
\DeclareFieldFormat[misc]{title}{#1}
\DeclareFieldFormat[misc]{publisher}{\mkbibitalic{#1}\isdot}

%Don't put parentheses around the year in the bibliography
%from https://tex.stackexchange.com/a/428193
\xpatchbibmacro{date+extradate}{%
	\printtext[parens]%
}{%
	\setunit*{\addcomma\space}%
	\printtext%
}{}{} 


%Clear some things I dont want going into the bibliography
%from https://tex.stackexchange.com/a/89848
\AtEveryBibitem{\clearfield{month}}
\AtEveryBibitem{\clearfield{day}}
\AtEveryBibitem{\clearfield{url}}
\AtEveryBibitem{\clearfield{urlyear}}
\AtEveryBibitem{\clearfield{note}}



%%%%%%%%%%%%%%%%%%%%
%Adjust some lengths related to bibliography
%from https://tex.stackexchange.com/a/140153

%space between consecutive entries in the bibliography
\setlength{\bibitemsep}{1.5ex}

%space between consecutive entries of authors with DIFFERENT names in the bibliography. This is called in place of \bibitemsep when the names are different.
%\setlength{\bibnamesep}{1.5\bibitemsep}

%space between consecutive entries of authors with same last names but different first names in the bibliography. This is called in place of \bibitemsep when the names are different.
%\setlength{\bibinitsep}{1.5\bibitemsep}

%The indent of follow up lines of the bibliography
\setlength{\bibhang}{1.5\parindent}

%%%%%%%%%%%%%%%%
%Add a new command for citing things with possesive apostrophe. For example, if I want to write:  Smith's (2023) big idea, the apostrophe needs to go after the author but before the year
%From https://tex.stackexchange.com/a/537765
%The relevant command is \posscite
\DeclareNameWrapperFormat{labelname:poss}{#1's}

\newrobustcmd*{\posscitealias}{%
	\AtNextCite{%
		\DeclareNameWrapperAlias{labelname}{labelname:poss}}}

\newrobustcmd*{\posscite}{%
	\posscitealias
	\textcite}

\newrobustcmd*{\Posscite}{\bibsentence\posscite}

\newrobustcmd*{\posscites}{%
	\posscitealias
	\textcites}

%%%%%%%%%%%%%%%%%%%%%%%%%

\usepackage{titlesec} % Format chapter headings
%from https://tex.stackexchange.com/a/52474

%Add a line b/w chapter number and heading
%for numbered chapters
\titleformat{\chapter}[display]
  {\normalfont\bfseries\Huge}
  {\chaptertitlename~\thechapter}{1pc}
  {{\color{gray}\titlerule[2pt]}\vspace{1pc}}

%Don't add a line for unnumbered chapters  
\titleformat{name=\chapter,numberless}[display]
  {\normalfont\bfseries\Huge}{}{1pc}
  {}

%reduce spacing before and after section titles
%this is to be read {left spacing}{before spacing}{after spacing}
%spacing: how to read {12pt plus 4pt minus 2pt}
%           12pt is what we would like the spacing to be
%           plus 4pt means that TeX can stretch it by at most 4pt
%           minus 2pt means that TeX can shrink it by at most 2pt
\titlespacing\section{0pt}{0pt plus 0pt minus 1pt}{0pt plus 0pt minus 1pt}
\titlespacing\subsection{0pt}{0pt plus 0pt minus 1pt}{0pt plus 0pt minus 1pt}
\titlespacing\subsubsection{0pt}{0pt plus 0pt minus 1pt}{0pt plus 0pt minus 1pt}

%Add a custom strut to increase vertical space given to equations when combining with underbrace
%from https://tex.stackexchange.com/a/13864
\newcommand*\mystrut[1]{\vrule width0pt height0pt depth#1\relax}


%%%%%%%%%%%%%
\usepackage{epigraph} %for quotes
\renewcommand{\textflush}{flushright} %quotes are right aligned

\usepackage{amsthm, amsmath, amssymb} % Mathematical typesetting
\usepackage{mathrsfs} %fancy fonts for sigma-algebras
\usepackage{float} % Improved interface for floating objects
\usepackage[final, colorlinks = true, 
            linkcolor = black, 
            citecolor = black,
            urlcolor = blue,
            breaklinks=true]{hyperref} % For hyperlinks in the PDF
\usepackage{multicol} % Enable multiple columns in text
\usepackage{graphicx} % Enhanced support for graphics
\usepackage{xcolor} % Driver-independent color extensions
\usepackage{physics} %some physics symbols
\usepackage{framed}
\usepackage[normalem]{ulem} %underlining
\usepackage{multicol}  %for double column text
\usepackage{color,soul} %for colored text
\usepackage{pdfpages} %for inserting PDFs
\usepackage{ragged2e} %for formatting
\usepackage{amsfonts}
\usepackage{enumitem}
\usepackage{mathtools}

\usepackage[toc, title]{appendix} %Appendix

\usepackage[labelfont=bf]{caption} %bold captions on figures and tables

%for tables
\usepackage{makecell,tabularx}
%\setlength{\extrarowheight}{12pt} %additive padding
\renewcommand{\arraystretch}{2} %multiplicative padding 
\renewcommand\theadfont{\small\bfseries}
\usepackage{rotating}
\usepackage{setspace}

%For pseudocode
\usepackage[boxed]{algorithm2e}
\DontPrintSemicolon

%A bunch of definitions that make my life easier
\newtheorem{theorem}{Theorem}[section]
\newtheorem{corollary}{Corollary}[section]
\theoremstyle{definition}
\newtheorem*{definition}{Definition}
\newtheorem{example}{Example}
\newtheorem*{note}{Note}
\newtheorem*{claim}{Claim}
\newtheorem*{lemma}{Lemma}

\newcommand{\bproof}{\bigskip {\bf Proof. }}
\newcommand{\eproof}{\hfill\qedsymbol}
\newcommand{\Disp}{\displaystyle}
\newcommand{\qe}{\hfill\(\bigtriangledown\)}
\setlength{\columnseprule}{1 pt}

\newcommand{\bignorm}[1]{\left\lVert#1\right\rVert} %for norm symbol

%for characters inside circles
%syntax is \circled{character}
\usepackage{tikz}
\newcommand*\circled[1]{\tikz[baseline=(char.base)]{
            \node[shape=circle,draw,inner sep=1pt] (char) {#1};}}


\usepackage{pgfplots}
\pgfplotsset{compat=1.8}


\usepackage[mode=buildnew]{standalone} %for loading precompiled tikz figures

%%%%%%%%%%%%%%%%%%%%%%%%
% Defines the `mycase` environment for cases in proofs
\newcounter{cases}
\newcounter{subcases}[cases]
\newenvironment{mycase}
{
    \setcounter{cases}{0}
    \setcounter{subcases}{0}
    \newcommand{\case}
    {
        \stepcounter{cases}\textbf{Case \thecases.}
    }
    \newcommand{\subcase}
    {
        \par\indent\stepcounter{subcases}\textit{Subcase (\thesubcases):}
    }
}
{
    \par
}
\renewcommand*\thecases{\arabic{cases}}
\renewcommand*\thesubcases{\roman{subcases}}

%For easily making figures
%syntax is:
%\myfig{scaling_factor}{name_of_file}{caption}{label}
\newcommand{\myfig}[4]{\begin{figure}[h] \begin{center} \includegraphics[width=#1\textwidth]{#2} \caption{#3} \label{#4} \end{center} \end{figure}}

% horizontal line across the page
\newcommand{\horz}{
\vspace{-.4in}
\begin{center}
\begin{tabular}{p{\textwidth}}\\
\hline
\end{tabular}
\end{center}
}

%Resize the summation symbol
%syntax is \sum[size] 
\newlength{\depthofsumsign}
\setlength{\depthofsumsign}{\depthof{$\sum$}}
\newlength{\totalheightofsumsign}
\newlength{\heightanddepthofargument}
\newcommand{\bigsum}[1][1.4]{% only for \displaystyle
    \mathop{%
        \raisebox
            {-#1\depthofsumsign+1\depthofsumsign}
            {\scalebox
                {#1}
                {$\displaystyle\sum$}%
            }
    }
}



%Add a footer without a marker/reference in the main text
%From https://tex.stackexchange.com/a/544121
\newcommand\extrafootertext[1]{%
	\bgroup
	\renewcommand\thefootnote{\fnsymbol{footnote}}%
	\renewcommand\thempfootnote{\fnsymbol{mpfootnote}}%
	\footnotetext[0]{#1}%
	\egroup
}

%Add fancy headers with chapter/section titles
\usepackage{fancyhdr}
\setlength{\headsep}{0.9in} %add spacing between header and body text

%this file contains the bibliography
\addbibresource{refs.bib}

%%%%%%%%%%%%%%%%%%%
% Packages/Macros %
%%%%%%%%%%%%%%%%%%%
\usepackage{fullpage}
\setlength{\parskip}{1em}

\usepackage{setspace}
\onehalfspacing

\setcounter{tocdepth}{3}

\catcode`\@=11
\catcode`\@=12

%%%%%%%%%%%%
% Document %
%%%%%%%%%%%%

\title{Eco-evolutionary dynamics of finite populations from first principles}
\author{Ananda Shikhara Bhat}

%Details of supervisors and experts
\supervisor{Vishwesha Guttal}
\cosupervisor{Rohini Balakrishnan}
\department{Centre for Ecological Sciences, Indian Institute of Science} \reader{Sutirth Dey} %Expert

\dedication{This thesis is dedicated to ?}

%Details of graduation date
\graduationyear{2023}
\academicyear{2022-2023}
\graduationmonth{May}

%The abstract goes here. IISER guidelines say it should be <= 250 words.
\thesisabstract{
Population biology is built on a strong mathematical foundation developed during the Modern Synthesis through fields such as theoretical population genetics, evolutionary game theory, and quantitative genetics. Historically, these formalisms have often worked with infinite populations, ignoring the effects of demographic stochasticity. Finite population models in population genetics usually assume a fixed population size and are of limited applicability in the real world, where population sizes routinely fluctuate. In this thesis, I use ideas from statistical physics to analytically describe evolving populations from biological first principles. Starting from a density-dependent ‘birth-death process’ describing an arbitrary closed population of individuals with discrete traits, I derive a set of stochastic differential equations (SDEs) for how trait frequencies change over time. Along with recovering the effects of the standard evolutionary forces of selection, mutation, and drift, these SDEs also reveal a new directional evolutionary force, ‘noise-induced selection’, that is particular to finite populations and has been largely overlooked in standard formulations of evolution. Noise-induced selection can reverse the direction of evolution predicted by infinite-population frameworks, with implications for simulation studies and real world populations. Well-known results such as the replicator-mutator equation and Fisher’s fundamental theorem are recovered in the infinite population limit. Finally, I extend these ideas to one-dimensional quantitative traits through a ‘stochastic field theory’ that yields frameworks such as Kimura's continuum-of-alleles and gradient dynamics in the infinite population limit. My work thus generalizes the formal structures of population biology to finite fluctuating populations and predicts a new evolutionary force unique to such populations.
}

\acknowledgments{Not more than 250 words}
\allowdisplaybreaks %allow align envts to go across pages   


\begin{document}

\pagestyle{plain}
\thesisfront
\listoftables
\addcontentsline{toc}{chapter}{List of Tables}
\listoffigures
\addcontentsline{toc}{chapter}{List of Figures}

\part{Background}


%Implement the fancy headers for everything from here on out
\pagestyle{fancy}
\fancyhf{}
\fancyhead[LE,RO]{\thepage}
\fancyhead[RE]{\rule[-2ex]{0pt}{4ex}\footnotesize\itshape\nouppercase{\leftmark}}
\fancyhead[LO]{\rule[-2ex]{0pt}{4ex}\footnotesize\itshape\nouppercase{\rightmark}}
\setlength{\headheight}{18pt} % as requested by fancyhdf
\renewcommand{\headrulewidth}{0pt}

\renewcommand{\chaptermark}[1]{\markboth{\chaptername \ \thechapter \ :\ #1}{}}
\renewcommand{\sectionmark}[1]{\markright{\thesection. \ \ #1}{}}

\chapter{Introduction}
\epigraph{\justifying The epistemic aim of science is not truth, but understanding}{Angela Potochnik}

More than 150 years have passed since Ernst Haeckel first coined the term ‘ecology’ in 1866. Today, ecology and evolution are incredibly interdisciplinary, borrowing techniques and ideas from diverse fields such as computer science, statistics, economics, dynamical systems, physics, and information theory. With this development has come a cornucopia of models that use these tools and techniques to try and understand biological phenomena. Though many of these models deal with specific `low-level' factors, there is also value to formulating general, idealized models in abstract terms to underscore these organizing principles and better illustrate the fundamental processes that are required to capture the `essence' of a biological pattern \citep{frank_natural_2012, vellend_theory_2016}. Such abstractions and idealizations are useful in unifying apparently disparate modelling frameworks and are thus vital to theory-building \citep{luque_mirror_2021}.

\section{Idealization and generality in ecology}\label{idealization}
The push and pull between the search for general patterns and the specification of minutiae has a long and torturous history in ecology \citep{kingsland_modeling_1985}. One of the first mathematical idealizations in ecology came in the form of the logistic equation formulated by Pearl and Verhulst, and shortly thereafter, equations for the populations of interacting predator and prey species, put forth independently by Lotka and Volterra. These models were immediately controversial, and for good reason: Many ecologists felt that they were overly idealistic and neglected many important truths about real biological populations \citep{kingsland_modeling_1985}. These models were nevertheless quite good at predicting the patterns of such populations, and have proved valuable to the field of population ecology. Today, these models are viewed as `classical' and are regularly used even by hard-line empiricists, not because we believe them to be true in all their gory biological details, but because we recognize that they can be \emph{useful} despite being blatantly false generalizations. This is a single instance of a more general philosophical idea concerning the goals and ideals of science. As suggested by the epigraph at the beginning of the chapter, philosophers of science \citep{potochnik_idealization_2018} have recently argued that science does not seek truth, but instead seeks understanding. The fact that idealization is part and parcel of science is clear if one looks at the actual practice, be it theorists making unrealistic assumptions on paper to model specific phenomena or experimentalists creating artificially controlled conditions in the laboratory to test specific hypotheses \citep{zuk_models_2018}. Relatively simple models of complex eco-evolutionary processes are therefore desirable as a way to shine light on these phenomena. To reiterate once more, the goal of such simple models is not truth (whatever that is), predictive power (as with models in physics), or detailed description (as with detailed individual-based simulations, very flexible statistical models, or machine learning), but \emph{understanding}. Since the world is complicated and humans are limited, such understanding inevitably comes at the cost of other desirable qualities such as the ability to make precise quantitative predictions. It is important to remember at the outset that the models I speak about in this thesis may seem to be over-idealized and too general, and will make only qualitative predictions. This is \textit{by design}, in pursuit of general insight over precise quantitative prediction.\\
Vellend has recently argued that conceptual synthesis in community ecology requires ``shifting the emphasis away from an organizational structure based on the useful lines of inquiry carved out by researchers, to one based on the fundamental processes that underlie community dynamics and patterns" \citep{vellend_theory_2016}. Vellend's assertion is based on the fact that population genetics has managed to come up with reasonably comprehensive theory due to its focus on the abstract `high-level' processes of selection, mutation, drift, and gene flow instead of the myriad `low-level' processes that may be responsible for generating them. In contrast, he believes that practitioners of community ecology often focus on specific `low-level' processes such as predation rate, limiting resources ($R^*$), storage effects, priority effects, senescence, and niche partitioning, leading to a plethora of models (see Table 5.1 in \citep{vellend_theory_2016} for an in-exhaustive list of 24 such models) and the conclusion that community ecology `is a mess'. Vellend proposes organizing ecological models according to the `high-level' processes of selection, ecological drift (demographic stochasticity), dispersal, and speciation.\\ Of course, no such general organization will be perfect or all-encompassing. As Robert MacArthur once remarked, ``general events are only seen by ecologists with rather blurred vision. The very sharp-sighted always find discrepancies and are able to see that there is no generality, only a spectrum of special cases”\citep{kingsland_modeling_1985}. However, I believe the act of looking past the myriad low-level processes present in biological systems and categorizing theories, models, and concepts in terms of a small number of general fundamental high-level processes provides a powerful unifying tool to organize concepts in biology. This is perhaps best exemplified by Darwin’s theory of natural selection as first proposed in \emph{The Origin of Species}. Darwin famously painstakingly collected a series of ‘low-level’ facts and observations regarding breeds, wild populations, and the geographic record to support his hypothesis. However, ultimately, these observations culminated in a synthesis whereby they were all unified under a single, abstract, ‘high-level’ process, namely evolution by natural selection.
In the spirit of this approach, the questions I seek to answer in this thesis are in terms of rather abstract sounding `high-level' processes. In the next section, I survey some of the existing literature to provide concrete examples which motivate these general models.

\section{A brief history of high-level modelling frameworks in population biology}

In population genetics, these high-level forces are the standard evolutionary forces of natural selection, genetic drift, dispersal, and mutation, and the description of evolution in terms of these forces was first laid out in formal mathematical terms during the Modern Synthesis \citep{ewens_mathematical_2004}. The mathematization of the evolutionary forces proved extremely successful, unifying two major schools of thought - Mendelian genetics and Darwinian evolution - that were, at the time, thought to be incompatible \hl{cite something here}.\\
Classical population genetics regarded forces like selection and mutation as fundamental, and the success of this approach during the Modern Synthesis illustrates the value of formulating high-level, abstract models that only provide a `high-level' description of the fundamental processes required to capture the essence of a biological pattern. However, the drama of evolution famously occurs in the ecological theatre \citep{hutchinson_ecological_1965}, and quantities like fitness are not truly fundamental, but instead emerge as the result of various ecological interactions, tradeoffs, and constraints, a fact that can have important consequences for evolution \citep{coulson_putting_2006, kokko_can_2017}. Trying to model such `eco-evolutionary dynamics' has sprouted a rich body of theoretical literature and led to the development of theoretical frameworks like evolutionary game theory and adaptive dynamics, fields which have greatly enriched our understanding of biological populations \citep{brown_why_2016}. Eco-evolutionary population dynamics can broadly be organized under a single unifying framework, the Price equation, that yields all the relevant formal structures under various limits \citep{page_unifying_2002, lion_theoretical_2018}. The Price equation partitions changes in population composition into multiple terms, each of which lends itself to a straightforward interpretation in terms of the high-level evolutionary forces of selection, mutation, and drift, thus providing a useful conceptual framework for thinking about how populations change over time \citep{frank_natural_2012}. The Price equation also leads to a small number of simple yet insightful `fundamental theorems' of population biology \citep{queller_fundamental_2017, lion_theoretical_2018, lehtonen_price_2018} and unifies several various seemingly disjoint formal structures under a single theoretical banner \citep{ lehtonen_price_2020, luque_mirror_2021}\\
The general guiding philosophy of much of this mathematization has been that incorporating the reality of finite population sizes into models leads to no major qualitative differences in behavior, only `adding noise' or `blurring out' the predictions of simpler infinite population models \citep{page_unifying_2002}. Consequently, several major theoretical frameworks in the field, such as adaptive dynamics, are explicitly formulated in deterministic terms at the infinite population size limit. However, this assumption is largely unjustified, and since populations in the real world are finite and stochastic, checking whether stochastic models differ from their deterministic analogs is vital to furthering our understanding of the fundamentals of population biology \citep{hastings_transients_2004, coulson_skeletons_2004, shoemaker_integrating_2020}. Today, we increasingly recognize that incorporating the finite and stochastic nature of the real world routinely has much stronger consequences than simply `adding noise' to deterministic expectations \citep{boettiger_noise_2018} in both ecological \citep{schreiber_does_2022} and evolutionary \citep{delong_stochasticity_2023} models. Stochastic models often exhibit phenomena that do not occur in infinite-population models \citep{rogers_demographic_2012, rogers_spontaneous_2012, veller_drift-induced_2017}, prevent phenomena that occur in infinite populations from occurring \citep{proulx_what_2005, johansson_will_2006, claessen_delayed_2007,  wakano_evolutionary_2013, debarre_evolutionary_2016, johnson_two-dimensional_2021}, or even completely reverse the predictions of deterministic models \citep{houchmandzadeh_selection_2012,houchmandzadeh_fluctuation_2015,constable_demographic_2016,mcleod_social_2019}. Studies of neutral or near-neutral dynamics in population and quantitative genetics usually do take stochasticity seriously, explicitly modeling finite populations that follow stochastic dynamics. However, the classic stochastic models in population and quantitative genetics typically assume a fixed total population size \citep{fisher_genetical_1930, crow_introduction_1970, lande_natural_1976} and their validity is therefore rather restrictive since population sizes routinely fluctuate in the real world. Importantly, the Price equation itself is usually formulated in a deterministic, dynamically-insufficient manner (but see \cite{rice_stochastic_2008} for a stochastic formulation of the Price equation). Since real-life populations are stochastic, finite, and of non-constant population size, it is thus imperative that we develop a theoretical framework that can handle such systems directly, instead of only working with deterministic, infinite-population approximations.\\
Incorporating stochasticity into deterministic systems is a tricky business, and, if done in a phenomenological manner by adding noise to a `deterministic skeleton' \citep{coulson_skeletons_2004} in an ad-hoc fashion, can lead to nonsensical predictions and inconsistencies \citep{strang_how_2019}. Stochastic individual-based models, in which (probabilistic) rules are specified at the level of the individual and population level dynamics are systematically derived from first principles, are much more natural \citep{black_stochastic_2012}, lead to sensible predictions \citep{strang_how_2019}, and can fundamentally differ from the predictions made by simply adding noise terms to a deterministic model \citep{strang_how_2019} (This is also true when incorporating spatial structure into non-spatial models, see for example \cite{durrett_importance_1994}). Formulating the fundamental formal structures of evolutionary biology in terms of the mechanistic demographic processes of birth and death at the individual level is also greatly desirable for biological reasons \citep{metcalf_why_2007,geritz_mathematical_2012}: Since demographic processes such as birth and death rates explicitly account for the ecology of the system, they can more accurately reflect the complex interplay between ecological and evolutionary processes \citep{doebeli_towards_2017} and provide a more fundamental mechanistic description of the relevant evolutionary forces. In other words, `all paths to fitness lead through demography' \citep{metcalf_all_2007}.

\section{Outline of the rest of this thesis}

In this thesis, I present a formulation of population dynamics constructed from mechanistic first principles grounded in individual-level birth and death. Part \ref{part_theory} presents the mathematical formalism in a detailed, self-contained, pedagogical manner. To facilitate readership by a broad audience, I only assume passing familiarity with calculus (derivatives, integrals, Taylor expansions) and probability. Familiarity with stochastic calculus is helpful for some sections but is not required. I present a brief introduction to the relevant mathematics in section \ref{sec_math_background} and present a toy example of tracking population size of a population of identical individuals in section \ref{sec_1D_processes}. I introduce a description of the system via a `master equation', and then conduct a `system-size expansion' to obtain a Fokker-Planck equation for the system. Finally, I conduct a weak noise approximation to arrive at a linear Fokker-Planck equation which can be solved exactly using some stochastic calculus to arrive at a closed-form solution given by a time-dependent Ornstein-Uhlenbeck process, thus illustrating all the major tools required. In section \ref{sec_nD_processes}, I present a multivariate process to describe the evolution of discretely varying traits, and, as before, illustrate the system size expansion and the weak noise approximation. Under mild assumptions, I show that the deterministic limit of this process is the well-known replicator-mutator equation (or equivalently, Eigen's quasispecies equation), thus establishing the microscopic basis of well-known equations in evolutionary game theory. I also show that the mean value of the trait in the population changes according to the Price equation in the deterministic limit. Chapter \ref{chap_infD_processes} introduces a function-valued process to model the evolution of quantitative traits such as body size, which can take on uncountably many values. This function-valued process can then also be analyzed via a system-size approximation to arrive at a `functional' Fokker-Planck equation, in terms of functional derivatives. Under mild assumptions, I show that classic equations such as Kimura's infinite alleles model and the canonical equation of adaptive dynamics can be derived as the deterministic limits of this stochastic process. I also conduct a weak noise approximation to arrive at a linear functional Fokker-Planck equation.  Sections \ref{sec_1D_processes} and \ref{sec_nD_processes} apply techniques that are well-known in physics, as applied to population biology. Chapter \ref{chap_infD_processes} generalizes the work of Tim Rogers and colleagues \citep{rogers_demographic_2012,rogers_spontaneous_2012,rogers_modes_2015}, and to the best of my knowledge, is entirely original. Chapter \ref{chap_infD_processes} also presents a simple heuristic derivation of quantitative genetics models and adaptive dynamics from stochastic first principles that is much simpler than the rigorous mathematical derivations grounded in measure theory and martingale/markov theory that are currently standard reference in theoretical ecology \citep{champagnat_individual_2008}. Examples for one-dimensional, multi-dimensional, and quantitative trait models that can be analyzed in this formalism are presented in Appendix \ref{App_examples}.\\
Part \ref{part_summary} summarizes the major results of this formalism and presents some simple equations that can be argued to be `fundamental theorems' of population biology in the sense of \cite{queller_fundamental_2017}. These theorems reduce to well-known results such as the Price equation, the replicator-mutator equation from evolutionary game theory, and Fisher's fundamental theorem from population genetics in the infinite population limit. For finite populations, these same theorems predict a new evolutionary force, `noise-induced selection', that has still not found its way into the standard formal canon of evolutionary biology such as the Price equation, and whose significance is only recently being recognized \citep{constable_demographic_2016,mcleod_social_2019,mazzolini_universality_2022, kuosmanen_turnover_2022}.

%\section{Coexistence of eco-variants in diverse contexts}\label{synthesis}
%The phenomenon of coexistence of multiple phenotypically distinct, ecologically relevant variants of an organism in sympatry occurs in several different contexts in the natural world. This is best illustrated through examples, which I provide below.
%\subsection{Alternative Reproductive Tactics}
%Alternative reproductive tactics (ARTs) are discrete polymorphisms that occur in sympatry in one or both sexes of a species in response to intrasexual competition for mates. More precisely, two or more traits which are expressed in individuals of a given sex of a given species can be said to be ARTs if \citep{oliveira_alternative_2008}:
%\begin{enumerate}[label=(\alph*)]
%    \item They are \emph{alternative}, in the sense that a given individual can only express one of the behaviors at any given point in time.
%    \item They are \emph{reproductive}, in the sense that the traits are directly relevant to the process of obtaining a mate and are intended to mitigate conspecific intrasexual competition.
%    \item They are \emph{tactics}, in the sense of serving a well-defined adaptive function and thus having fitness consequences. Usually, different ARTs have different associated fitness effects \emph{ceteris paribus}.
%\end{enumerate}
%ARTs are widespread in the animal kingdom, and may be seen either in morphological traits, or as more complex behavioral polymorphisms (see \citep{oliveira_alternative_2008} for an overview). Most (if not all) species which exhibit ARTs only show a small number of polymorphisms (usually 2-3, 5 in a handful of cases, and very rarely 7-8) - \hl{I can maybe make a table for this: Somehow wasn't able to find one in the literature}. ARTs have obvious consequences for reproductive life history, and can strongly influence the ecology and evolutionary trajectory of organisms.
%
%\subsection{Trophic resource polymorphisms}
%
%Trophic resource polymorphism is defined as ``the occurrence of discrete intraspecific morphs showing differential niche use, usually through differences in feeding biology and habitat use" \citep{skulason_resource_1995}. Species which exhibit trophic resource polymorphisms often show marked, discontinuous morphological variations adapted for specific niche usage. For example, arctic charr and african cichlid fishes both often exhibit discrete trophic polymorphisms in which different morphs (of the same species) are specialized for feeding at different strata of the lakes they live in \citep{recknagel_ecosystem_2017}. Such polymorphisms also generally occur in relatively small numbers in sympatry (see Table 1 in \citep{smith_evolutionary_1996}). They are relatively widespread in vertebrates, and are thought to have important evolutionary consequences by acting as the `substrate' for adaptive radiations \citep{smith_evolutionary_1996}. 
%
%\subsection{Mating types in isogamous species}
%Isogamous organisms are sexual, but do not have gametes which can be classified into `male' and `female' based on size. Gametes of such species can, however, be divided into distinct `mating types' according to chemical self-incompatibility. There is substantial variation in the number of mating types present in a given species: While most organisms have only two mating types, this is by no means the rule \citep{phadke_rapid_2009,constable_rate_2018}, with a handful of organisms having 7-10 mating types, and one species of fungus (\textit{Schizophyllum commune}) exhibiting over 23,000 mating types. The study of mating types in isogamous species is ecologically and evolutionarily relevant, not only due to the intrinsic importance of mating and self-incompatibility, but due to the fact that isogamy is thought to be the evolutionary precursor to anisogamy: Thus, studying the factors governing the number of mating types in isogamous species may shed light on why anisogamous species have only two distinct types of gametes.
%
%\subsection{Phenotypic clustering and sympatric speciation}
%\hl{Write stuff about sympatric speciation, adaptive radiations, and phenotypic clustering.}
%\\
%\\
%In all these cases, we are interested in monitoring the changes in the number of distinct phenotypic entities, in a scenario in which the phenotype is ecologically relevant for fitness. Existing literature has tended to use the word `morph' for this concept, and I will bow down to convention and do the same. However, it is important to remember that in this context, `morphs' are only defined in terms of traits that \textit{affect fitness}, in contrast to the more general use of the word in biology to mean any difference whatsoever (as in single nucleotide polymorphisms, for example). Note that in many of these cases, even though the emergent alternative morphs are discretely distributed, the underlying traits need not be, and indeed are often continuously varying, and only categorized into discrete bins by human experimentalists. Such discrete morphs are often persistent in phylogenetically similar lineages \citep{jamie_persistence_2020}, suggesting they are evolutionarily predictable to some degree, and are also thought to be important for very general evolutionary processes \citep{west-eberhard_alternative_1986}. Intraspecific variation also provides an important metric for the diversity harboured by populations and can have important ecological consequences. Two important questions immediately arise:
%
%\begin{enumerate}
%    \item How does a population that is initially monomorphic for a (selectively non-neutral) trait evolve to become polymorphic for this trait in a seemingly discrete manner?
%    \item What governs the number of discrete morphs that co-occur in a population?
%\end{enumerate}
%
%This very general setting motivates a view in which an `individual' is simply characterized by its phenotype as a point in some abstract `trait space'. These questions then concern the distribution of points in this space (analogous to the frequency of each `type' of individual) changes over evolutionary time due to ecological fitness determinants (through the traits that the individuals possess). More formally, we can visualize an abstract trait space $\mathcal{T}$ which captures all possible phenotypes that an individual may exhibit. Any phenotype can then be viewed as some point $x \in \mathcal{T}$. We are interested in the distribution of trait values in the population subject to some ecological and evolutionary rules, and a `morph' or `eco-variant' is just a cluster within this trait space (a mode in the distribution).
% \subsection{Sympatric speciation}
% The origin of species is relatively well understood when populations diverge in allopatry (separated by geographic barriers, for example). In contrast, sympatric speciation, first formally proposed by Maynard Smith \hl{cite}, is much less well-understood, despite empirical evidence that it has occurred in nature \hl{cite}. Since speciation is one of the fundamental processes in 

% \section{Frequency-dependent selection}\label{FDS}
% One well-known mechanism of coexistence is `niche partitioning', whereby coexisting types occupy different areas in niche space and thereby reduce competition between types. This of course immediately immediately begs the question of how niche partitioning occurs in the first place. In general, coexistence is also possible without niche partitioning. Such coexistence in the presence of selective pressure (\textit{i.e} when eco-variants are not selectively neutral) occurs through frequency-dependent selection.


%Raymond Pearl, one of the pioneers of mathematical ecology as a discipline, wrote in 1927 that ``What we want to know is how the biological forces of natality and mortality are so integrated and correlated in their action as to lead to a final result in size of population which follows this particular curve rather than some other one'' \citep{pearl_growth_1976}. Pearl realized 95 years ago that population dynamics must be ultimately explained by the mechanistic processes of birth and death. Today, there are mounting calls for more mechanistic models of evolution that are grounded in these fundamental processes of birth and death \citep{geritz_mathematical_2012,doebeli_towards_2017}. Ecologists also increasingly recognize that incorporating stochasticity is vital to developing more realistic ecological models \citep{hastings_transients_2004, coulson_skeletons_2004, boettiger_noise_2018, shoemaker_integrating_2020,schreiber_does_2022} and does more than `add noise' to deterministic expectations. Individual-based models, where ecological rules are specified at the level of the individual, are a powerful mathematical tool for mechanistic descriptions of stochastic population dynamics \citep{black_stochastic_2012}. Birth-death processes are a very general class of stochastic processes that can be used to capture a wide range of eco-evolutionary processes. `System-size expansions' and their subsequent analysis using the `weak noise approximation' are common tools for analyzing stochastic birth-death processes that are well-known in the statistical physics and applied mathematics communities \citep{gardiner_stochastic_2009}. However, these tools remain relatively underappreciated in ecology, despite being relatively easy to understand and extremely well-motivated in scenarios germane to ecology and evolution. Here, I present a formulation of population dynamics constructed from first principles grounded in birth-death processes. To facilitate readership by a broad audience, only passing familiarity with calculus (derivatives, integrals, Taylor expansions) and probability are assumed. Familiarity with stochastic calculus is helpful for some sections but is not required.  In section \ref{sec_1D_processes}, I show how fluctuating population size of populations of identical individuals can be tracked through a one-dimensional birth-death process. I introduce a description of the system via a `master equation', and then conduct a `system-size expansion' to obtain a Fokker-Planck equation for the system. Finally, I conduct a weak noise approximation to arrive at a linear Fokker-Planck equation which can be solved exactly using some stochastic calculus to arrive at a closed-form solution given by a time-dependent Ornstein-Uhlenbeck process. As an example, I illustrate the complete process for a stochastic version of the logistic equation. In section \ref{sec_nD_processes}, I present a multivariate process to describe the evolution of discretely varying traits, and, as before, illustrate the system size expansion and the weak noise approximation. Under mild assumptions, I show that the deterministic limit of this process is the well-known replicator-mutator equation (or equivalently, Eigen's quasispecies equation), thus establishing the microscopic basis of well-known equations in evolutionary game theory. I also show that the mean value of the trait in the population changes according to the Price equation in the deterministic limit. Chapter \ref{chap_infD_processes} introduces a function-valued process to model the evolution of quantitative traits such as body size, which can take on uncountably many values. This function-valued process can then also be analyzed via a system-size approximation to arrive at a `functional' Fokker-Planck equation, in terms of functional derivatives. Under mild assumptions, I show that classic equations such as Kimura's infinite alleles model and the canonical equation of adaptive dynamics can be derived as the deterministic limits of this stochastic process. I also conduct a weak noise approximation to arrive at a linear functional Fokker-Planck equation.  Sections \ref{sec_1D_processes} and \ref{sec_nD_processes} apply techniques that are well-known in physics, as applied to population biology. Chapter \ref{chap_infD_processes} generalizes the work of Tim Rogers and colleagues \citep{rogers_demographic_2012,rogers_spontaneous_2012,rogers_modes_2015}, and to the best of my knowledge, is entirely original. Chapter \ref{chap_infD_processes} also presents a simple heuristic derivation of quantitative genetics models and adaptive dynamics from stochastic first principles that is much simpler than the rigorous mathematical derivations grounded in measure theory and martingale/markov theory that are currently standard reference in theoretical ecology \citep{champagnat_individual_2008}. I illustrate the utility of all this abstract formalism through examples in Chapter \ref{chap_examples}, and discuss the major implications in \ref{chap_unification}.



\part{Theory}\label{part_theory}
\chapter{Population dynamics from stochastic first principles}\label{chap_BD}
\input{ind_files/02_BD_processes}
\chapter{Stochastic field equations for the evolution of quantitative traits}\label{chap_infD_processes}
\input{ind_files/03_field_eqns}

\part{Summary \& Discussion}\label{part_summary}
\chapter{A unified view of population dynamics}\label{chap_unification}
\epigraph{\justifying The grand aim of all science [is] to cover the greatest number of empirical facts by logical deduction from the smallest number of hypotheses or axioms}{Albert Einstein}
%\epigraph{\justifying Not only is algebraic reasoning exact; it imposes an exactness on the verbal postulates made before algebra can start which is usually lacking in the first verbal formulations of scientific principles.}{J.B.S. Haldane}
\justifying
In this thesis, we have seen how stochastic birth-death processes can be used to construct and analyze mechanistic individual-based models for the dynamics of finite populations. In doing so, we have also seen that various well-known equations of evolutionary dynamics can be recovered in the infinite population size limit. In the finite-dimensional case, the infinite population limit corresponds to the equations of population genetics and evolutionary game theory. In the infinite-dimensional case, we instead obtain the equations of quantitative genetics, and, in some further limits, gradient dynamics. In both cases, the mean value of the trait in the population changes according to an equation resembling the Price equation. My derivation highlights the natural connections between the various equations of population dynamics - For example, the same procedures that lead to the replicator-mutator equation in the case of discretely varying traits yield Kimura's model in the quantitative case, underscoring the broad similarities between evolutionary game theory and quantitative genetics and extending known similarities to the finite population case. The major formulations are summarized in Table \ref{table_summary}.

{\centering\begin{sideways}
    \begin{minipage}{\textheight}
        \resizebox{\textheight}{!}{%
            \setstretch{1.5}
            \begin{tabular}{ %I manually specified the width of each column by trial and error
  |p{\dimexpr.25\linewidth-2\tabcolsep-1.3333\arrayrulewidth}% column 1
  |p{\dimexpr.27\linewidth-2\tabcolsep-1.3333\arrayrulewidth}% column 2
  |p{\dimexpr.33\linewidth-2\tabcolsep-1.3333\arrayrulewidth}% column 3
  |p{\dimexpr.25\linewidth-2\tabcolsep-1.3333\arrayrulewidth}% column 4
  |p{\dimexpr.4\linewidth-2\tabcolsep-1.3333\arrayrulewidth}|% column 5
  }
            \hline
         \centering \textbf{Number of possible distinct trait variants ($m$)} & \centering \textbf{State Space} &
\centering \textbf{Model parameters} & \centering \textbf{Mesoscopic description} & \centering\arraybackslash \textbf{Infinite population limit} \\
        \hline
        $m = 1$ \newline (Identical individuals)  & $[0,1,2,3,\ldots]$ \newline (Population size) & Two real-valued functions, $b(N)$ and $d(N)$, describing the birth and death rate of individuals when the population size is $N$ & Univariate Fokker-Planck equation \newline (one-dimensional SDEs) & Single species population dynamics\\ 
        \hline
        $1 < m < \infty$ \newline (Discrete traits) & $[0,1,2,3,\ldots]^{m}$ \newline (Number of individuals of each trait variant) & $2m$ real-valued functions, $b_i(\mathbf{v})$ and $d_i(\mathbf{v})$ (for $1 \leq i \leq m$) describing the birth and death rate of trait variant $i$ when the population is $\mathbf{v}$ & Multivariate Fokker-Planck equation \newline ($m$-dimensional SDEs) &  Evolutionary game theory \newline
        Lotka-Volterra competition \newline Quasispecies equation \newline Price equation (discrete traits) \\
        \hline
        $m = \infty$ \newline (Quantitative traits) & $\left\{\sum\limits_{i=1}^{n}\delta_{x_i} \ | \ n \in \mathbb{N}\right\}$ \newline \newline (Each Dirac mass $\delta_{x_i}$ is an individual with trait value $x_i$) & Two real-valued functionals $b(x|\nu)$ and $d(x|\nu)$ describing the birth and death rate of trait variant $x$ when the population is $\nu$ & Functional Fokker-Planck equation/Field theory \newline (SPDEs) & Kimura's continuum-of-alleles model \newline \cite{sasaki_oligomorphic_2011}'s Oligomorphic Dynamics \newline \cite{wickman_theoretical_2022}'s Trait Space Equations for intraspecific trait variation \newline Gradient Dynamics \newline Price equation (quantitative traits)\\
        \hline
            \end{tabular}
        }
        %\renewcommand\thetable{1}
        \captionof{table}{Summary of the various birth-death processes studied in this thesis}
        \label{table_summary}
    \end{minipage}
\end{sideways}\par}
\clearpage
\section{Fundamental theorems of evolution in finite population}
\subsection{The fundamental theorem for changes in type frequencies in the population}\label{sec_fun_theorem_freq}
Equation \eqref{nD_eqn_for_frequencies}, which we derived in chapter \ref{chap_BD}, is a very general equation for how frequencies change over time in stochastic populations. To recap, we started with a population which can contain up to $m$ different types of individuals, and used ecological arguments to posit the existence of a `system-size' parameter $K$ that leads to density-dependent growth and prevents the population from growing infinitely large. The population as a whole is characterized by a vector $\mathbf{x} = [x_1, \ldots, x_m]$ indexing the density (\emph{i.e.} number divided by $K$) of each type of individual. Changes of the population are through either birth or death of individuals. Each type has a per-capita birth rate $b^{\textrm{(ind)}}(\mathbf{x})$, a per-capita death rate $d^{\textrm{(ind)}}(\mathbf{x})$, and an additional term $\mu Q_{i}(\mathbf{x})$ representing mutational effects. All three of these functions depend on the density (and \emph{not} just the total number) of indidivuals of each type in the population, and may in general also be frequency-dependent. In the regime where $K$ is not too small (corresponding to `medium sized' populations), we identified two quantities, $w_i(\mathbf{x}) = b^{\textrm{(ind)}}_{i}(\mathbf{x}) - d^{\textrm{(ind)}}_{i}(\mathbf{x})$ and $\tau_i(\mathbf{x}) = b^{\textrm{(ind)}}_{i}(\mathbf{x}) + d^{\textrm{(ind)}}_{i}(\mathbf{x})$, the Malthusian fitness and per-capita turnover rate of the $i\textsuperscript{th}$ type respectively, that emerge as being important for trait frequency dynamics. In particular, we saw that $p_i$, the frequency of the $i\textsuperscript{th}$ type in the population, changes according to the equation:
\begin{equation}
\label{nD_stochastic_RM}
\begin{aligned}
dp_i(t) &= \underbrace{\left[(w_i(\mathbf{x}) - \overline{w})p_i + \mu\left\{Q_i(\mathbf{p}) - p_i\left(\sum\limits_{j=1}^{m}Q_j(\mathbf{p})\right)\right\}\right]dt}_{\substack{\text{Infinite population predictions: selection-mutation balance} \\ \text{for higher fitness}}}\\
&- \frac{1}{K}\underbrace{\frac{1}{N_{K}(t)}\left[(\tau_i(\mathbf{x}) - \overline{\tau})p_i + \mu\left\{Q_i(\mathbf{p}) - p_i\left(\sum\limits_{j=1}^{m}Q_j(\mathbf{p})\right)\right\}\right]dt}_{\substack{\text{Directional noise-induced effects: selection-mutation balance}\\\text{for lower turnover rates}}}\\
&+ \frac{1}{\sqrt{K}N_{K}(t)}\underbrace{\left[\left(A^{+}_{i}\right)^{1/2}dB^{(i)}_t - p_i\sum\limits_{j=1}^{m}\left(A^{+}_{j}\right)^{1/2}dB^{(j)}_t\right]}_{\substack{\text{Non-directional noise-induced effects}\\\text{due to stochastic fluctuations}}}
\end{aligned}
\end{equation}
where $N_K = \sum x_i$ is the total population size scaled by $K$ (and thus $KN_K$ is the total population size), $A_i^{+} = x_i\tau_i(\mathbf{x}) + \mu Q_i(\mathbf{x})$, and each $B^{(i)}_t$ is an independent one-dimensional standard Brownian motion. Equation \eqref{nD_stochastic_RM} is in `replicator-mutator' form, and letting $K \to \infty$ recovers the standard replicator-mutator equation. The first term represents the direct effects of forces captured in classic deterministic models, and reflects a selection-mutation balance. However, finite populations experience a new directional force dependent on $\tau_i(\mathbf{x})$, the per-capita turnover rate of type $i$, that cannot be captured in infinite population models \citep{kuosmanen_turnover_2022}. Remarkably, this term acts in a way that is mathematically identical to the classical action of selection and mutation in infinite population models as captured by the first term in \eqref{nD_stochastic_RM}, but in the opposite direction - A higher relative $\tau_i$ leads to a decrease in frequency (Notice the minus sign before the second term in \eqref{nD_stochastic_RM}).
\subsection{The fundamental theorem for the mean value of a type-level quantity in the population}\label{sec_fun_theorems_mean}

We can now calculate how the statistical mean value of a type-level quantity changes over time. Let $f$ be any type level quantity, with value $f_i(t)$ for the $i\textsuperscript{th}$ type. We allow for the possibility of $f_i$ to vary over time. By multiplying both sides of
equation \eqref{nD_stochastic_RM} by $f_i$ and summing over all $i$ (The same steps as going from \eqref{nD_replicator_mutator} to \eqref{nD_Price_time_dependent}), we see that the statistical mean $\overline{f}$ of the quantity in the population varies as:
\begin{equation}
\label{nD_stochastic_Price}
\begin{aligned}
%The \vphantom is to vertically align the texts under the \underbraces
d\overline{f} &= {\underbrace{\vphantom{ \frac{1}{KN_K(t)} }\textrm{Cov}(w,f)dt}_{\substack{\text{Classical} \\ \text{selection}}}} \ - \ {\underbrace{\frac{1}{KN_K(t)}\textrm{Cov}(\tau,f)dt}_{\substack{\text{Noise-induced} \\ \text{selection}}}} \ +  \  {\underbrace{\vphantom{ \frac{1}{KN_K(t)} }\overline{\left(\frac{\partial f}{\partial t}\right)}dt}_{\substack{\text{Ecological timescale} \\ \text{feedbacks due to} \\ \text{time-dependence of $f_i$}}}}\\[15pt]
&+ \underbrace{\mu\left(1-\frac{1}{KN_K(t)}\right)\left(\sum\limits_{i=1}^{m}f_iQ_i(\mathbf{p}) - \overline{f}\sum\limits_{i=1}^{m}Q_i(\mathbf{p})\right)dt}_{\text{Transmission bias/mutational effects}}\\[15pt]
&+ \underbrace{\frac{1}{\sqrt{K}N_{K}(t)}\left(\sum\limits_{i=1}^{m}\left(f_i-\overline{f}\right)\sqrt{A_i^+}dB_{t}^{(i)}\right)}_{\text{Stochastic fluctuations}}
\end{aligned} 
\end{equation}
where all covariances are understood in the statistical sense (Note that since $w_i$, $\tau_i$, $\overline{w}$, and $\overline{\tau}$ are stochastic processes depending on $\mathbf{p}$, the terms $\textrm{Cov}(w,f)$ and $\textrm{Cov}(\tau,f)$ are themselves stochastic processes). Taking $K \to \infty$ in equation \eqref{nD_stochastic_Price} recovers the standard Price equation as the infinite population limit (either \eqref{nD_Price_time_dependent} or \eqref{nD_Price} based on whether $f_i$ varies with time). We saw in chapter \ref{chap_infD_processes} using field equations that very similar methods of attack to those used outlined in chapter \eqref{chap_BD} also hold for quantitative traits. For example, equation \eqref{cts_replicator_mutator} and \eqref{cts_price} are respectively exactly the infinite dimensional analogs of the deterministic replicator-mutator equation \eqref{nD_replicator_mutator} and the deterministic Price equation \eqref{nD_Price} when $f$ is the trait value. We may therefore expect to find equations similar to \eqref{nD_stochastic_RM} and \eqref{nD_stochastic_Price} for quantitative traits. Indeed, measure-theoretic tools have recently been used to rigorously show that an infinite-dimensional version of \eqref{nD_stochastic_Price} holds for one-dimensional quantitative traits when $f$ is the trait value and $b(x|\phi) \pm d(x|\phi)$ are Gaussian (see equation (21b) in \cite{week_white_2021}). Equations \eqref{nD_stochastic_RM} and \eqref{nD_stochastic_Price} are thus fundamental theorems for the evolution of finite populations, with the replicator-mutator and Price equations as their respective infinite population limits (also see \citep{rice_universal_2020} for a stochastic Price equation in a discrete-time setting).\\
Each term in equation \eqref{nD_stochastic_Price} lends itself to a simple biological interpretation. The first term, $\textrm{Cov}(w,f)$, is well-understood in the classical Price equation and represents the effect of natural selection in the infinite population setting. In the stochastic Price equation \eqref{nD_stochastic_Price}, the effects of the second term of 
\eqref{nD_stochastic_RM} decompose into a selection term $\textrm{Cov}(\tau,f)$ for reduced turnover rates and a transmission bias term that vanishes in the weak mutation ($\mu \to 0$) limit. Following \cite{week_white_2021}, we refer to the effect of the covariance (the second term of equation \eqref{nD_stochastic_Price}) as \emph{noise-induced selection} since it occurs exactly analogously to classical natural selection (but for lower $\tau$) and is induced purely by the finiteness of the population. Since this evolutionary force is unique to finite populations and has therefore been overlooked in classical population genetics, it warrants some more detailed discussion. Biologically, the $\textrm{Cov}(\tau,f)$ term (with a negative sign) describes a biasing effect due to differential turnover rates and can intuitively be understood as being similar to gambler's ruin in probability theory through the following reasoning: If a type $i$ has a higher $\tau_i$, it experiences greater turnover due to a generally higher birth and death rate and thus experience more births and deaths in a given time interval than an otherwise equivalent species with a lower $\tau_i$. More events mean greater demographic stochasticity, and types with a higher $\tau_i$ thus tend to be eliminated by a stochastic analog of selection because they experience more chance events (births and deaths) in a given time period. This effect is less visible if the total population size is higher because larger populations generally experience less stochasticity, which is reflected in the $1/N_K$ factor in this term. This stochastic analog of selection for reduced turnover rates, captured by the second term of equation \eqref{nD_stochastic_RM}, is the force responsible for the `reversal of the direction of deterministic selection' induced by demographic noise in previous studies \citep{houchmandzadeh_selection_2012, houchmandzadeh_fluctuation_2015, constable_demographic_2016, mcleod_social_2019}. Note that types that tend to increase the \emph{total} population size $KN_K(t)$ (such as altruists in evolutionary theory and mutualists in ecological communities) will reduce the magnitude of this effect compared to types that do not facilitate such an increase, such as cheaters and highly competitive species, which could explain why this effect preferentially favors the former types in reversing the direction of deterministic selection in finite population models with fluctuating population sizes \citep{houchmandzadeh_fluctuation_2015, chotibut_evolutionary_2015, mcleod_social_2019}. The fact that total population size controls the strength of noise-induced selection also explains why cooperation is favoured in the early transient period of population growth \citep{melbinger_evolutionary_2010} when simulations are initiated from a small population size - In the early transient period, $N_K(t)$ is small, and the biasing effect of differential turnover rates is stronger, thus favouring cooperation. More generally, selection for reduced turnover rate could explain why cooperation often persists for a long time in finite population IbMs (and the real world) despite infinite population models predicting their extinction. The fact that the entire term is multiplied by $(KN_K(t))^{-1}$ suggests that the effect of this force is weak for medium to large populations, which explains why the persistence of cooperators is often only observed in restrictive sounding conditions such as quasi-neutrality, rapid attraction to a slow manifold, or a weak selection + weak mutation limit \citep{mcleod_social_2019}. In all three of these cases, the first term on the RHS of \eqref{nD_eqn_for_frequencies} becomes identically 0. It therefore no longer contributes to the trait frequency dynamics, thus allowing us to see the contributions of the second term.\\
The third term of \eqref{nD_stochastic_Price} is relevant in both finite and infinite populations whenever $f_i$ can vary over time and represents feedback effects on the quantity $f_i$ of the $i\textsuperscript{th}$ species over short (`ecological') time-scales. Such feedback is often through a changing environment or phenotypic/behavioral plasticity, but other biological phenomena may also be at play. The fourth term of \eqref{nD_stochastic_Price} is a transmission bias term, with a correction factor due to noise-induced selection. Finally, the last term of \eqref{nD_stochastic_Price} describes the role of stochastic fluctuations. The contributions of this last term are `directionless' due to the $dB_t$ factors, and this term vanishes when we take a conditional expectation value over the underlying probability space. We denote this probabilistic expectation value operation by $\mathbb{E}[\cdot]$ to distinguish it from the statistical mean \eqref{nD_mean}. Note that this expectation is conditioned on the initial state of the population, and thus $\mathbb{E}[\cdot]$ is really shorthand for $\mathbb{E}[\ \cdot \ | \ \mathbf{X}_0 = \mathbf{x}_0]$.
\\
Two particularly interesting implications of \eqref{nD_stochastic_Price} are realized upon ignoring mutations by setting $\mu = 0$ and then substituting $f=w$ and $f = \tau$. We first note that:
\begin{align}
\textrm{Cov}(w,\tau) &=\textrm{Cov}\left( b^{\textrm{(ind)}}(\mathbf{x}) - d^{\textrm{(ind)}}(\mathbf{x}) \ , \   b^{\textrm{(ind)}}(\mathbf{x}) + d^{\textrm{(ind)}}(\mathbf{x})\right)\\
&= \sigma^2_{b^{\textrm{(ind)}}(\mathbf{x})} - \sigma^2_{d^{\textrm{(ind)}}(\mathbf{x})}\label{nD_cross_covariance}
\end{align}
Note that just like the statistical mean, the statistical variance is a random variable and is not to be confused with the probabilistic/ensemble variance. Upon substituting $f = w$ in \eqref{nD_stochastic_Price} and taking expectations over the underlying probability space, we obtain:
\begin{align}
\label{nD_stochastic_Fisher}
%The \vphantom is to vertically align the texts under the underbraces of different terms
\mathbb{E}\left[\frac{d\overline{w}}{dt}\right] &= 
\negmedspace {\underbrace{\mystrut{4ex} \ \vphantom{ \frac{\sigma^2_{b^{\textrm{(ind)}}} - \sigma^2_{d^{\textrm{(ind)}}}}{KN_K(t)} } \mathbb{E}\left[\sigma^2_{w}\right] \ }_{\substack{\text{Fisher's} \\ \text{Fundamental theorem}}}} \ - \ {\underbrace{\mystrut{4ex}\mathbb{E}\left[\frac{\sigma^2_{b^{\textrm{(ind)}}} - \sigma^2_{d^{\textrm{(ind)}}}}{KN_K(t)}\right]}_{\substack{\text{Noise-induced} \\ \text{selection}}}} \ + {\underbrace{\mystrut{4ex} \vphantom{ \frac{\sigma^2_{b^{\textrm{(ind)}}} - \sigma^2_{d^{\textrm{(ind)}}}}{KN_K(t)} } \mathbb{E}\left[\hphantom{a}\overline{\hphantom{a}\frac{\partial w}{\partial t}\hphantom{a}}\hphantom{a}\right]}_{\substack{\text{Short-term (ecological)} \\ \text{feedbacks to fitness}}}}
\end{align}
Taking $K \to \infty$ in \eqref{nD_stochastic_Fisher} recovers a well-known equation in population genetics upon noting that the process tends to a deterministic process as $K \to \infty$, as noted in section \ref{sec_nD_det_limit}, and thus the expectation value in the infinite population case is superfluous.The first term, $\sigma^2_w$, is the subject of Fisher's fundamental theorem \citep{fisher_genetical_1930,  price_fishers_1972, frank_fishers_1992, kokko_stagnation_2021}. The second term of equation \eqref{nD_stochastic_Fisher} is a manifestation of noise-induced selection and vanishes in the infinite population limit, and is thus particular to finite populations. Finally, the last term arises in both finite and infinite populations whenever $w_i$ can vary over time \citep{baez_fundamental_2021}, be it through frequency-dependent selection, phenotypic plasticity, varying environments, or other ecological mechanisms, and represents feedback effects on the fitness $w_i$ of the $i\textsuperscript{th}$ species over short (`ecological') time-scales. The fact that Fisher appears to have been rather vague and dismissive of this feedback \citep{fisher_genetical_1930} has led to much discussion, debate, and confusion about the interpretation, importance, and implications of his `fundamental theorem' (see \cite{kokko_stagnation_2021} and sources cited therein).\\
Carrying out the same steps with $f = \tau$ in \eqref{nD_stochastic_Price} yields a new equation/theorem due to \cite{kuosmanen_turnover_2022} that has only recently been recognized as important. This theorem is an analog of Fisher's fundamental theorem for the turnover rates, and reads:
\begin{equation}
\label{nD_stochastic_Fisher_turnover}
\mathbb{E}\left[\frac{d\overline{\tau}}{dt}\right] = \underbrace{\vphantom{ \mathbb{E}\left[\frac{\sigma^2_{\tau}}{KN_K(t)}\right] } \mathbb{E}\left[\sigma^2_{b^{\textrm{(ind)}}} - \sigma^2_{d^{\textrm{(ind)}}}\right]}_{\substack{\text{Classical selection} \\ \text{effects}}} \ - \ \underbrace{ \mathbb{E}\left[\frac{\sigma^2_{\tau}}{KN_K(t)}\right]}_{\substack{\text{Noise-induced selection} \\ \text{effects}}} \ + \ \underbrace{ \vphantom{ \mathbb{E}\left[\frac{\sigma^2_{\tau}}{KN_K(t)}\right] } \mathbb{E}\left[\hphantom{a}\overline{\hphantom{a}\frac{\partial \tau}{\partial t}\hphantom{a}}\hphantom{a}\right]}_{\substack{\text{Short-term (ecological)} \\ \text{feedbacks to $\tau_i$}}}
\end{equation}
The implications of this theorem have been extensively discussed in \citep{kuosmanen_turnover_2022}, which is where we refer the interested reader.

\subsection{The fundamental theorem for the variance of a type-level quantity in the population}\label{sec_fun_theorems_var}
Equation \eqref{nD_stochastic_Price} is a general equation for the mean value of an arbitrary type level quantity $f$ in the population. In many real-life situations, especially those pertaining to finite populations, we are interested in not just the mean, but also the variance of a type-level quantity. In appendix \ref{App_stoch_var_eqns}, I show that the statistical variance of any type level quantity $f$ obeys
\begin{equation}
\label{nD_stochastic_Price_variance}
\begin{aligned}
d\sigma^2_{f} &= \textrm{Cov}\left(w,(f - \overline{f})^2\right)dt - \frac{1}{KN_K}\left[ \ \overline{\tau}\sigma^2_{f} +  2\textrm{Cov}\left(\tau,(f - \overline{f})^2\right) \ \right]dt\\[12pt]
& + 2\textrm{Cov}\left(\frac{\partial f}{\partial t},f\right)dt + M_{\sigma^2_f}(\mathbf{p},N_K)dt + dB_{\sigma^2_{f}}
\end{aligned}
\end{equation}
where
\begin{equation}
\label{variance_price_mutation_term}
M_{\sigma^2_f}(\mathbf{p},N_K) = \mu\left[\left(1 - \frac{2}{KN_K}\right)\left(\sum\limits_{i=1}^{m}(f_i - \overline{f})^2Q_i(\mathbf{p})\right) + \sigma^2_f\left(1 - \frac{1}{KN_K}\right)\sum\limits_{i=1}^{m}Q_i(\mathbf{p})\right]
\end{equation}
is a mutational term that vanishes in the $\mu \to 0$ limit and
\begin{equation}
\label{variance_price_diffusion_term}
dB_{\sigma^2_f} = \frac{1}{\sqrt{K}N_{K}(t)}\left(\sum\limits_{i=1}^{m}\left(f_i - \overline{f}\right)^2\sqrt{A_i^+}dB_{t}^{(i)}\right)
\end{equation}
is a stochastic integral term measuring the (non-directional) effect of stochastic fluctuations that vanishes upon taking an expectation over the probability space. In the case of one-dimensional quantitative traits, an infinite-dimensional version of \eqref{nD_stochastic_Price_variance} has recently been rigorously derived \citep{week_white_2021} using measure-theoretic tools under certain additional assumptions (See equation (21c) in \cite{week_white_2021}). Taking expectations over the probability space in \eqref{nD_stochastic_Price_variance} and substituting mutation as acting via a Gaussian kernel also recovers equations previously derived \citep{debarre_evolutionary_2016} in the context of evolutionary branching in finite populations as a special case (Equation A.23 in \cite{debarre_evolutionary_2016} is equivalent to equation \eqref{nD_stochastic_Price_variance} for their choice of functional forms upon converting their change in variance to an infinitesimal rate of change \emph{i.e.} a derivative). An infinite population ($K \to \infty$) version of equation \eqref{nD_stochastic_Price_variance} also appears in \cite{lion_theoretical_2018}.\\
Once again, terms of equation \eqref{nD_stochastic_Price_variance} lend themselves to straightforward biological interpretation. The quantity $(f_i-\overline{f})^2$ is a measure of the distance of $f_i$ from the population mean value $\overline{f}$, and thus covariance with $(f-\overline{f})^2$ quantifies the type of selection operating in the system: A negative correlation is indicative of stabilizing selection, and a positive correlation is indicative of disruptive (\emph{i.e.} diversifying) selection. An extreme case of diversifying selection for fitness occurs if the mean fitness is at a local minimum for fitness but $f_i \not\equiv \overline{f}$ (\emph{i.e.} the population still exhibits some variation in $f$). In this case, if the variation in $f$ is associated with a variation in fitness, then $\textrm{Cov}(w,(f - \overline{f})^2)$ is strongly positive and the population experiences a sudden explosion in variance. If $\textrm{Cov}(w,(f - \overline{f})^2)$ is still positive after the initial emergence of multiple morphs, evolution eventually leads to the emergence of stable coexisting polymorphisms in the population. The $\textrm{Cov}\left(\partial f/\partial t,f\right)$ term once again represents the effect of eco-evolutionary feedback loops due to rapid change in $f$ that is not solely due to changes in $\mathbf{p}$. The $M_{\sigma^2_f}(\mathbf{p},N_K)$ term quantifies the effect of mutations on the variance of $f$.Note that each $Q_i(\mathbf{p}) \geq 0$ by its definition in \eqref{nD_functional_forms_for_replicator} and thus $\sum_i Q_i(\mathbf{p}) > 0$ if there are any mutational effects (and $=0$ otherwise). Furthermore, the total population size $KN_K > 2$ for most interesting evolutionary questions. Thus, from \eqref{variance_price_mutation_term}, it is clear that when $\mu > 0$ (\emph{i.e.} there is mutation in the system), we have $M_{\sigma^2_f}(\mathbf{p},N_K) > 0$, meaning that mutations always increase the variance of $f$ in the population.\\
The $\overline{\tau}\sigma^2_{f}$ term represents a loss of diversity due to stochastic extinctions (i.e. demographic stochasticity). To see this, it is instructive to consider the case in which this is the only force at play. Let us imagine a population of asexual organisms in which each $f_i$ is simply a label or mark arbitrarily assigned to individuals in the population at the start of an experiment/observational study and subsequently passed on to offspring - For example, a neutral genetic tag in a part of the genome that experiences a negligible mutation rate. Let us set $\mu = 0$ so that the labels cannot change between parents to offspring. This means that we have $M_{\sigma^2_f}(\mathbf{p},N_K) \equiv 0$. Further, since the labels are arbitrary and have no effect whatsoever on the biology of the organisms, we have $\textrm{Cov}\left(w,(f - \overline{f})^2\right) \equiv \textrm{Cov}\left(\tau,(f - \overline{f})^2\right) \equiv 0$. Since the labels do not change over time, we also have $\textrm{Cov}\left(\partial f/\partial t,f\right) = 0$. From \eqref{nD_stochastic_Price_variance}, we see that in this case, the variance changes as
\begin{equation}
d\sigma^2_f = - \frac{\overline{\tau}\sigma^2_{f}}{KN_K}dt + dB_{\sigma^2_{f}}
\end{equation}
On taking expectations, we see that the expected variance in the population obeys
\begin{equation}
\label{neutral_example_for_variances}
\frac{d \mathbb{E}[\sigma^2_f]}{dt} = - \left(\mathbb{E}\left[\frac{\overline{\tau}}{KN_K}\right]\right)\mathbb{E}[\sigma^2_{f}]
\end{equation}
where we have decomposed the expectation of the product on the RHS into a product of expectations, which is admissible since the label $f$ is completely arbitrary and thus independent of both $\overline{\tau}$ and $N_K(t)$. Equation \eqref{neutral_example_for_variances} is a simple linear ODE and has the solution
\begin{equation}
\mathbb{E}[\sigma^2_f](t) = \sigma^2_f(0)e^{-\mathbb{E}\left[\frac{\overline{\tau}}{KN_K}\right]t}
\end{equation}
which tells us that the expected diversity (variance) of labels in the population decreases exponentially over time. The rate of loss is $\mathbb{E}\left[\overline{\tau}(KN_K)^{-1}\right]$, and thus, populations with higher mean turnover $\overline{\tau}$  and/or lower population size $KN_K$ lose diversity faster. This is because populations with higher $\overline{\tau}$ experience more stochastic events per unit time (a gambler's ruin type scenario), while extinction is `easier' in smaller populations because a smaller number of deaths is required to eliminate a label from the population completely. Note that \emph{which} labels/individuals are lost is entirely random (since all labels are arbitrary), but nevertheless, labels tend to be stochastically lost until only a small number of labels remain in the population.


\section{Discussion \& Outlook}\label{sec_disc}

Stochastic finite population models often exhibit behaviors that are markedly different from their deterministic limits. Since real-life populations are stochastic and finite, it is thus imperative that modellers work with stochastic first-principles models instead of their deterministic limits, lest they risk missing important phenomena that are unique to stochastic systems \citep{black_stochastic_2012,schreiber_does_2022,hastings_transients_2004,shoemaker_integrating_2020}. In the context of eco-evolutionary theory, \cite{doebeli_towards_2017} have called for a reformulation of evolutionary dynamics starting from stochastic birth-death processes on the grounds that such a formulation is more fundamental and mechanistic, a view that has also been echoed by theorists before \citep{metcalf_why_2007,geritz_mathematical_2012}. In this thesis, I present and derive from first principles some fundamental relations (equations \eqref{nD_stochastic_RM}, \eqref{nD_stochastic_Price}, and \eqref{nD_stochastic_Price_variance}) that any such reformulation must satisfy. Further, these relations deal with biologically important quantities, lend themselves to simple biological interpretation, and are very general. They thus fulfill the criteria to be called `fundamental theorems' (in the sense of \cite{queller_fundamental_2017}), or `unifying principles' (in the sense of \cite{lion_theoretical_2018}) for the dynamics of finite populations. \cite{lion_theoretical_2018} has pointed out that in the dynamic setting (for infinite populations), the replicator-mutator equation \eqref{nD_replicator_mutator} is in some sense the `most fundamental' of the lot, and equations like the Price equation are best viewed as a hierarchy of moment equations for the population mean, population variance, etc. of a type-level quantity. This is also true in our framework - Equation \eqref{nD_stochastic_RM} is the most fundamental equation for population dynamics, and equations like \eqref{nD_stochastic_Price} and \eqref{nD_stochastic_Price_variance} can then be derived from \eqref{nD_stochastic_RM} through repeated application of It\^o's formula (in principle for any moment of the distribution of $f$ in the population, though this quickly becomes too tedious to actually carry out in practice). Just like in infinite population models, the various terms of equations \eqref{nD_stochastic_RM}, \eqref{nD_stochastic_Price}, and \eqref{nD_stochastic_Price_variance} lend themselves to simple biological interpretation. Note that actually solving these equations analytically for equilibrium/stationary state distributions of $\mathbf{p}$, $\overline{f}$, and $\sigma^2_f$ will quickly become impossibly difficult if the birth and death rate functions are complicated. Indeed, previous studies indicate that high dimensional evolutionary birth-death models can exhibit a dizzying array of complicated phenomena, including limit cycles and evolutionary chaos \citep{doebeli_diversity_2017}. We have also neglected any potential complications introduced by genotype-phenotype maps and other such genetic factors, since they can be absorbed into the per-capita birth and death rates. Nevertheless, our equations will still hold for arbitrarily complicated birth and death rates. Like the Price equation, the utility of these equations lies not in their solutions, but instead in their generality and the fact that their terms help us clearly understand the various forces operating in biological populations \citep{frank_natural_2012,luque_one_2017, luque_mirror_2021}. The general approach of working with `high-level' processes without specifying system-specific details is part of a broader pursuit of `model-independent' eco-evolutionary theory that has recently been gaining popularity in the literature  \citep{grafen_formal_2014, queller_fundamental_2017, doebeli_towards_2017, lion_theoretical_2018, allen_mathematical_2019, rice_universal_2020, week_white_2021, wickman_theoretical_2022, kuosmanen_turnover_2022, mazzolini_universality_2022}. Such model-independent descriptions that `abstract away' system-specific details almost inevitably come at the cost of precision \citep{levins_strategy_1966, potochnik_idealization_2018}, and are thus intended to complement rather than supersede specific models of specific systems.\\
Noise-induced selection due to variation in per-capita turnover rate $\tau$ is an evolutionary force only seen in finite populations. Mathematically, this term arises essentially due to It\^o's formula adding an extra term when conducting a change of variables in going from type densities to type frequencies. Biologically, it arises due to different types experiencing a different number of stochastic events (birth and death) in a given time interval. The effects of noise-induced selection are increasingly being recognized in various specific models \citep{houchmandzadeh_selection_2012, houchmandzadeh_fluctuation_2015, constable_demographic_2016, parsons_pathogen_2018}, and, recently, have also been shown in some more general settings \citep{week_white_2021,mazzolini_universality_2022,kuosmanen_turnover_2022}. This force has some important practical implications. For one, simulation studies that work with evolutionary individual-based or agent-based models should be careful about whether interactions effects are incorporated into birth rates or death rates, since this seemingly arbitrary choice can have unintended consequences due to noise-induced selection \citep{mcleod_social_2019,kuosmanen_turnover_2022}. Being mindful of noise-induced selection is also important for applied fields like conservation and population management that regularly deal with small populations (\emph{i.e.} populations which likely experience stronger noise-induced selection). For example, when trying to increase the population of a hypothetical desired species in a multispecies community, increasing the birth rate is \emph{not} equivalent to reducing the death rate even though both result in an increase in the Malthusian fitness (growth rate) $w_i$ - Decreasing the death rate leads to a decrease in $\tau_i$, which leads to positive noise-induced selection (from \eqref{nD_stochastic_RM}), whereas increasing the birth rate leads to an increase in $\tau_i$, which leads to noise-induced selection acting to reduce the abundance of the focal species from the community. If the total community size is small, increasing the birth rate of a species can thus lead to noise-induced selection completely eliminating the focal species from the community despite the fact that our actions \emph{increased} the growth rate of this species. Similarly, when trying to infer or predict the future trajectory of the relative abundance of a species (or phenotype, allele, etc.) from empirical data, our equations indicate that measuring the growth rate of populations is not, in general, sufficient for accurate prediction even in completely controlled environments. The growth rate $w_i = b^{\textrm{(ind)}}_{i} - d^{\textrm{(ind)}}_{i}$ of a species $i$ only specifies the difference between its per-capita birth and death rates, whereas the complete stochastic dynamics also depend on the total turnover $\tau_i = b^{\textrm{(ind)}}_{i} + d^{\textrm{(ind)}}_{i}$ (\emph{i.e.} the sum of the per-capita birth and death rates). Two species can have very different $\tau_i$ while having the same $w_i$, leading to discrepancies from predictions that only account for growth rates ($w$) in smaller communities due to noise-induced selection causing a systematic deviation from true neutrality. Thus, equations \eqref{nD_stochastic_RM} and \eqref{nD_stochastic_Price} imply that for evolution to be \emph{truly} neutral in finite populations, it is not sufficient for the trait in question to be neutral with respect to fitness $w$. Instead, we also require the trait to be neutral with respect to turnover rate $\tau$. A similar phenomenon has also been recognized in some finite population ecological models, where seemingly neutral models (equal growth rate) nevertheless select for individuals with greater longevity \citep{lin_features_2012, oliveira_advantage_2017} \\
At first glance, the idea that individuals with lower birth and death rates are preferred over individuals with higher birth and death rates may be reminiscent of ideas in life-history evolution like $r$ vs $K$ selection or selection for the pace of life in stochastic models  \citep{stearns_evolution_1977}. However, it is unclear whether this similarity reflects some deep principle or whether it is just superficial. Models in life-history theory are often primarily concerned with spatiotemporally fluctuating (external) environments, and thus the stochasticity in those models is extrinsic to the population. I have entirely neglected such extrinsic factors in my formalisms. In principle, it is possible to make the birth and death rates \ref{nD_functional_forms_for_replicator} in our model also depend on a temporally varying external environment $E(t)$ (whose variation may possibly depend on the population $\mathbf{v}(t)$). Incorporating such a term would ensure that the `ecological feedback' terms in equations \eqref{nD_stochastic_Price} and \eqref{nD_stochastic_Price_variance} are non-zero, but may also lead to much more complex dynamics. Furthermore, if the variation of the environment $E(t)$ has some associated stochasticity, the final dynamics would be the result of interactions between two qualitatively different forms of noise --- \emph{extrinsic} noise from the environment, and \emph{intrinsic} noise from the finiteness of the population --- and thus will be rather complicated and likely analytically intractable. Ecological frameworks such as modern coexistence theory \citep{chesson_multispecies_1994}, which deal with questions about ecological population dynamics and would benefit from a first principles stochastic birth-death formulation, also generally work with fluctuating external environments. Thus, while the idea of integrating my birth-death framework with ecological ideas such as the pace-of-life syndrome \citep{mathot_models_2018, wright_life-history_2019} or modern coexistence theory \citep{chesson_multispecies_1994, johnson_resolving_2022} is biologically appealing, it is far from trivial and may present a promising avenue for future work.

%\chapter{Towards a stochastic evolutionary theory}\label{chap_stochasticity}
%\epigraph{\justifying Not only is algebraic reasoning exact; it imposes an exactness on the verbal postulates made before algebra can start which is usually lacking in the first verbal formulations of scientific principles.}{J.B.S. Haldane}

One striking feature that repeatedly shows up in our derivations is that finite populations exhibit phenomena that are not visible in infinite population models. Stochastic systems exhibit many interesting and biologically relevant phenomena which cannot be captured in the deterministic limit. For example, in both the stochastic logistic equation \ref{ex_1D_stoch_logistic_BD_eqns} and in two-strategy games with finite population sizes \citep{tao_stochastic_2007}, demographic noise ensures that all populations are guaranteed to go extinct given enough time, even if the deterministic limit predicts a stable state far from extinction. In the case of quantitative traits, demographic noise can hinder adaptive diversification by increasing the time before evolutionary branching occurs \citep{claessen_delayed_2007, wakano_evolutionary_2013, debarre_evolutionary_2016}, causing stochastic extinction of existing evolutionary branches \citep{rogers_demographic_2012, johansson_will_2006}, or preventing branching altogether if the population is too small \citep{rogers_modes_2015, johnson_two-dimensional_2021}. Stochastic systems also routinely exhibit evolution towards attractors that cannot be attained in the deterministic limit \citep{delong_stochasticity_2023}, sometimes even completely reversing the direction of evolution predicted by deterministic dynamics \citep{constable_demographic_2016,mcleod_social_2019}. Since real-life populations are stochastic and finite, it is thus imperative that modellers work with stochastic first-principles models instead of their deterministic limits, lest they risk missing important phenomena that are unique to stochastic systems \citep{black_stochastic_2012,schreiber_does_2022,hastings_transients_2004,shoemaker_integrating_2020}. In the context of our models, we have seen that if we observe the change in trait frequencies instead of the change in densities, finite populations are subject to an additional evolutionary force that vanishes in infinite population models.
% \chapter{}
% \input{ind_files/chapter_08}
% \chapter{} <-- uncomment and add as required
% \input{ind_files/chapter_09}

%You need the below commands to elevate the appendix to 'Part' level in the TOC
%from https://tex.stackexchange.com/a/286452
\cleardoublepage
\phantomsection
\appendix
\addcontentsline{toc}{part}{Appendices}

%make sure the title is 'Appendix' and not 'Chapter'
\renewcommand{\chaptername}{Appendix} 
\chapter{Deriving the Fokker-Planck equations for It\^{o} SDEs}\label{App_SDE_FPE}
\section*{Appendix A: From It\^{o} to Fokker-Planck}\label{sec_Ito_to_FPE}
Here, I present a simple (informal) derivation of the Fokker-Planck equation (FPE) for a one-dimensional It\^{o} process. The result for the multi-dimensional case follows from the same logic but is more notationally cumbersome.
\\
Consider a one-dimensional real It\^{o} process given by $dX_t = \mu(X_t,t)dt + \sigma(X_t,t)dB_t$ on a filtered probability space $\Omega \subseteq \mathbb{R}$ with probability measure $\mathbb{P}$ such that $\mathbb{P}(\cdot) \equiv 0$ on $\partial \Omega$ and $\mathbb{P} \ll m$, where $m$ is the Lebesgue measure. The latter requirement allows us to use the Radon-Nikodym theorem to write $\int \ \cdot \ d\mathbb{P} = \int \ \cdot \ P(x,t)dx$, where $P(x,t)$ is a `probability density function' defined at every point in $\Omega \times [0,\infty)$. Now, Let $f:\mathbb{R}\to\mathbb{R}$ be an arbitrary $C^2(\mathbb{R})$ function. By It\^{o}'s lemma, we have:
\begin{equation*}
    df(X_t) = f'dX_t + \frac{1}{2}f''d\langle X\rangle_t
\end{equation*}
where $\langle \cdot \rangle$ denotes the quadratic variation. For $dX_t = \mu dt + \sigma dB_t$, it is clear that $d\langle X\rangle_t = \sigma^2d\langle B\rangle_t = \sigma^2dt$, and thus, we obtain:
\begin{align*}
    df(X_t) = \left(\mu f' + \frac{\sigma^2}{2}f''\right)dt + \sigma f' dB_t
\end{align*}
Writing this in integral form and taking expectations on both sides yields:
\begin{align}
\label{eq_expectation_Ito}
    \mathbb{E}[f(X_t)] = \mathbb{E}\left[\int\limits_{0}^{t}\left(\mu f' + \frac{\sigma^2}{2}f''\right)ds\right] + \mathbb{E}\left[\int\limits_{0}^{t}\sigma f' dB_s\right]
\end{align}
Since the Brownian motion is a martingale, as long as $X_t$ and $\sigma(X_t,t)$ are reasonably `nice'\footnote{We require $\sigma(X_t,t)f'(X_t) \in \mathcal{L}^*(B)$, which is a highly technical condition. Since existence/uniqueness of solutions for the SDE already requires Lipschitz continuity of $\sigma(X_t,t)$, this seems like a reasonable assumption to make.}, the stochastic integral in the second term of the RHS of \eqref{eq_expectation_Ito} will be a continuous $L^2(\mathbb{P})$ martingale starting at the origin, and its expectation will therefore be 0. Using the definition of the expectation value, we are thus left with:
\begin{equation*}
    \int\limits_{\Omega}f(X_t)P(x,t)dx = \int\limits_{\Omega}\left(\int\limits_{0}^{t}\mu f' + \frac{\sigma^2}{2}f''ds\right)P(x,t)dx
\end{equation*}
Assuming derivatives and expectations commute, we can now differentiate with respect to time on both sides and use the fundamental theorem of calculus to write
\begin{equation}
\label{eq_Ito_to_FPE_for_parts}
\int\limits_{\Omega}f(X_t)\frac{\partial P}{\partial t}(x,t)dx = \underbrace{\int\limits_{\Omega}\mu f'P(x,t)dx}_{I(x,t)} + \underbrace{\int\limits_{\Omega}\frac{\sigma^2}{2}f''P(x,t)dx}_{J(x,t)}
\end{equation}
We will now use integration by parts to further evaluate $I(x,t)$ and $J(x,t)$. Recall that the general formula for integration by parts is given by:
\begin{equation*}
    \int\limits_{\Omega}u_{x_i}vdx = -\int\limits_{\Omega}uv_{x_i}dx + \int\limits_{\partial\Omega}uv\gamma_{i}dS(x)
\end{equation*}
where subscript indicates differentiation and $\gamma$ is the unit outward normal. In our case, assuming that $P(x,t) \equiv 0$ on $\partial \Omega$, the boundary term (second term of the RHS) vanishes and we can use integration by parts once on $I(x,t)$ to obtain
\begin{equation}
\label{eq_Ito_to_FPE_I_term}
    I(x,t) = - \int\limits_{\Omega}f(X_t)\left(\frac{\partial}{\partial x}\mu P(x,t)\right)dx
\end{equation}
and twice on $J(x,t)$ to obtain
\begin{align}
    J(x,t) &= - \frac{1}{2}\int\limits_{\Omega}f'(X_t)\left(\frac{\partial}{\partial x}\sigma^2 P(x,t)\right)dx\nonumber\\
    &= \frac{1}{2}\int\limits_{\Omega}f(X_t)\left(\frac{\partial^2}{\partial x^2}\sigma^2 P(x,t)\right)dx\label{eq_Ito_to_FPE_J_term}
\end{align}
Substituting \eqref{eq_Ito_to_FPE_I_term} and \eqref{eq_Ito_to_FPE_J_term} into \eqref{eq_Ito_to_FPE_for_parts} and collecting terms yields
\begin{equation*}
    \int\limits_{\Omega}f(X_t)\frac{\partial P}{\partial t}(x,t)dx = \int\limits_{\Omega}f(X_t)\left[-\frac{\partial}{\partial x}(\mu P(x,t)) + \frac{1}{2}\frac{\partial^2}{\partial x^2}(\sigma^2P(x,t))\right]dx
\end{equation*}
Since this is true for an arbitrary choice of $f(x)$ (as long as $f$ is $C^2$), we are thus led to conclude that the density function $P(x,t)$ must satisfy:
\begin{equation}
\label{FPE}
\frac{\partial P}{\partial t}(x,t) =-\frac{\partial}{\partial x}\left(\mu(x,t) P(x,t)\right) + \frac{1}{2}\frac{\partial^2}{\partial x^2}\left((\sigma(x,t))^2P(x,t)\right)
\end{equation}
Equation \eqref{FPE} is the Fokker-Planck equation in one dimension. Using the exact same strategy, the multidimensional Fokker-Planck equation for the $n$ dimensional It\^{o} Process $d\mathbf{X}_t = \mu(\mathbf{X}_t,t)dt + \sigma(\mathbf{X}_t,t)dB_t$ is found to be:
\begin{equation}
\label{FPE_ndim}
\frac{\partial P}{\partial t}(\mathbf{x},t) =-\sum\limits_{i=1}^{n}\frac{\partial}{\partial x_i}\left(\mu_i(\mathbf{x},t) P(\mathbf{x},t)\right) + \frac{1}{2}\sum\limits_{i=1}^{n}\sum\limits_{j=1}^{n}\frac{\partial^2}{\partial x_ix_j}\left(D_{ij}P(\mathbf{x},t)\right)
\end{equation}
where $\mathbf{D} = \mathbf{\sigma}\mathbf{\sigma}^T$.
\chapter{Deriving stochastic trait frequency dynamics using It\^{o}'s formula}\label{App_density_to_freq}
We first recall the version of the multi-dimensional It\^{o}'s formula that will be relevant to us. Consider an $m$-dimensional real It\^{o} process $\mathbf{X}_t$ given by the solution to
\begin{equation*}
d\mathbf{X}_t = \boldsymbol{\mu}(\mathbf{X}_t)dt + \boldsymbol{\sigma}(\mathbf{X}_t)d\mathbf{B}_t
\end{equation*}
where $\boldsymbol{\mu}: \mathbb{R}^m \to \mathbb{R}^m$ is the `drift vector' and $\boldsymbol{\sigma}: \mathbb{R}^{m} \to \mathbb{R}^{m \times m}$ is the `diffusion matrix'. Let $f: \mathbb{R}^m \to \mathbb{R}$ be an arbitrary $C^2(\mathbb{R}^m)$ function. Then, It\^{o}'s formula (\cite{oksendal_stochastic_1998}, Section 4.2) states that  the stochastic process $f(\mathbf{X}_t)$ must satisfy:
\begin{equation}
\label{nD_Ito_formula}
df(\mathbf{X}_t) = \left[\left(\nabla_{\mathbf{X}}f\right)^{\mathrm{T}}\boldsymbol{\mu} + \frac{1}{2}\mathrm{Tr}[\boldsymbol{\sigma}^{\mathrm{T}}(H_{\mathbf{X}}f)\boldsymbol{\sigma}]\right]dt + \left(\nabla_{\mathbf{X}}f\right)^{\mathrm{T}}\boldsymbol{\sigma}d\mathbf{B}_t
\end{equation}
where $\mathrm{Tr}[\cdot]$ denotes the trace of a matrix, $\left(\cdot\right)^{\mathrm{T}}$ denotes the transpose, and we have suppressed the $\mathbf{X}_t$ dependence of $\boldsymbol{\mu}$ and $\boldsymbol{\sigma}$ to reduce clutter. Here, $\nabla_{\mathbf{X}}f$ is the $m$ dimensional \emph{gradient vector} of $f$ and $H_{\mathbf{X}}f$ is the $m \times m$ \emph{Hessian matrix} of $f$, respectively defined for $f(x_1,\ldots,x_m)$ as:
\begin{align*}
\left(\nabla_{\mathbf{x}} f\right)_j &= \frac{\partial f}{\partial x_j}\\
\left(H_{\mathbf{x}} f\right)_{jk} &= \frac{\partial^2 f}{\partial x_j \partial x_k } 
\end{align*}
In our case, we have the It\^{o} process given by \eqref{nD_Ito_SDE}, which defines how the density of each type of individual changes over time. We thus have $\boldsymbol{\mu}(\mathbf{X}_t) = \mathbf{A}^{-}(\mathbf{X}_t)$ and $\boldsymbol{\sigma}(\mathbf{X}_t) = \mathbf{D}(\mathbf{X}_t)/\sqrt{K}$.  For each fixed $i \in \{1,2,\ldots,m\}$, let us define a scalar function $f_i: \mathbb{R}^m \to \mathbb{R}$ as
\begin{equation*}
f_i(\mathbf{x}) = \frac{x_i}{\sum\limits_{j=1}^{m}x_j}
\end{equation*}
Thus, $f_i(\mathbf{X}_t)$ gives us the frequency of type $i$ individuals when the population is described by the vector $\mathbf{X}_t$. This function is obviously $C^2(\mathbb{R}^m)$, and we can thus use It\^{o}'s formula \eqref{nD_Ito_formula} to describe how it changes over time. The $j\textsuperscript{th}$ element of the gradient of $f_i$ is given by:
\begin{align}
\left(\nabla_{\mathbf{x}} f_i\right)_j &= \frac{\partial }{\partial x_j}\left(\frac{x_i}{\sum\limits_{k=1}^{m}x_k}\right)\nonumber\\
&= \left(\frac{1}{N}\frac{\partial x_i}{\partial x_j} 
- \frac{x_i}{N^2}\sum\limits_{k=1}^{m}\frac{\partial x_k}{\partial x_j}\right)\nonumber\\
&= \frac{1}{N}\left(\delta_{ij}-p_i\right)\label{nD_jacobian_for_ito}
\end{align}
where we have defined the total (scaled) population size\footnote{This is $N_{K}(t)$ in the main text, but we omit the subscript here to reduce notational clutter}  $N = \sum_i x_i$ and the frequency of the $i\textsuperscript{th}$ type $p_i = f_i(x)$ and used the fact that $\frac{\partial x_j}{\partial x_k} = \delta_{jk}$. The $jk\textsuperscript{th}$ element of the Hessian is given by:
\begin{align}
\left(H_{\mathbf{x}} f_i\right)_{jk} &= \frac{\partial^2 }{\partial x_j \partial x_k}\left( \frac{x_i}{\sum\limits_{l=1}^{m}x_l}\right)\nonumber\\
&= \frac{\partial}{\partial x_j}\left(\frac{\delta_{ik}}{N}-\frac{x_i}{N^2}\right)\nonumber\\
&= \frac{1}{N^2}\left(2p_i - \delta_{ij}-\delta_{ik}\right)\label{nD_hessian_for_ito}
\end{align}
Thus, for the first term of \eqref{nD_Ito_formula}, we have:
\begin{align}
\left(\nabla_{\mathbf{X}}f_i\right)^{\mathrm{T}}\boldsymbol{\mathbf{A}^{-}} &= \sum\limits_{j=1}^{m}\left(\left(\nabla_{\mathbf{x}} f_i\right)_j\right)A^{-}_{j} \nonumber\\
&= \frac{1}{N}\sum\limits_{j=1}^{m}\left(\delta_{ij}-p_i\right)A^{-}_{j}\nonumber\\
&= \frac{1}{N}\left(A^{-}_{i} - p_i\sum\limits_{j=1}^{m}A^{-}_{j}\right)\label{nD_for_Ito_first_term}
\end{align}
If the birth and death rates take the form of \eqref{nD_functional_forms_for_replicator}, then it is easy to see using \eqref{nD_det_limit_fitess_defn} that equation \eqref{nD_for_Ito_first_term} is exactly the RHS of \eqref{nD_replicator_intermediate_1}. Thus, when the birth and death rates take the form of \eqref{nD_functional_forms_for_replicator}, then \eqref{nD_for_Ito_first_term} describes the deterministic component of the dynamics as described by the replicator-mutator equation, Price equation, etc. in the infinite population limit. These are the effects of selection and mutation at the infinite population limit. However, the finiteness of the population adds a second term to these dynamics, described by the second term that multiplies $dt$ in \eqref{nD_Ito_formula}. To calculate it, we first calculate:
\begin{align}
\frac{1}{\sqrt{K}}\left(H_{\mathbf{x}} f_i \mathbf{D}\right)_{jk} &= \frac{1}{\sqrt{K}}\sum\limits_{l=1}^{m} \left(H_{\mathbf{x}} f_i \right)_{jl}\left(\mathbf{D}\right)_{lk}\nonumber\\
&= \frac{1}{\sqrt{K}N^2}\sum\limits_{l=1}^{m}\left(2p_i - \delta_{ij} - \delta_{il}\right)\delta_{lk}\left(A^{+}_{l}A^{+}_{k}\right)^{\frac{1}{4}}\\
&=  \frac{1}{\sqrt{K}N^2}\left(\left(2p_i -\delta_{ij}\right)(A^{+}_{k})^{\frac{1}{2}} -\delta_{ik}\left(A^{+}_{i}A^{+}_{k}\right)^{\frac{1}{4}}\right)\\
&= \frac{1}{\sqrt{K}N^2}\left(2p_i -\delta_{ij} -\delta_{ik}\right)(A^{+}_{k})^{\frac{1}{2}}
\end{align}
and thus:
\begin{align}
\frac{1}{K}\left(\mathbf{D}^{\mathrm{T}} H_{\mathbf{x}} f_i \mathbf{D}\right)_{lk} &=\frac{1}{K}\sum\limits_{j=1}^{m}\left(\mathbf{D}^{\mathrm{T}}\right)_{lj}\left(H_{\mathbf{x}} f_i \mathbf{D}\right)_{jk}\nonumber\\
&=  \frac{1}{KN^2}\sum\limits_{j=1}^{m}\delta_{lj}\left(A^{+}_{l}A^{+}_{j}\right)^{\frac{1}{4}}(A^{+}_{k})^{\frac{1}{2}}\left(2p_i -\delta_{ij} -\delta_{ik}\right)\\
&=  \frac{1}{KN^2}(A^{+}_{k})^{\frac{1}{2}}\left(2p_i(A^{+}_{l})^{\frac{1}{2}} - (A^{+}_{i})^{\frac{1}{2}}\delta_{il} - (A^{+}_{l})^{\frac{1}{2}}\delta_{ik}\right)
\end{align}
Using this, we see that the trace of this matrix is given by:
\begin{align}
\frac{1}{K}\mathrm{Tr}[\mathbf{D}^{\mathrm{T}} H_{\mathbf{x}} f_i \mathbf{D}] &= \frac{1}{K}\sum\limits_{k=1}^{m}\left(\mathbf{D}^{\mathrm{T}} H_{\mathbf{x}} f_i \mathbf{D}\right)_{kk}\nonumber\\
&= \frac{1}{KN^2}\sum\limits_{k=1}^{m}\left(2p_i(A^{+}_{k}A^{+}_{k})^{\frac{1}{2}} - (A^{+}_{i}A^{+}_{k})^{\frac{1}{2}}\delta_{ik} - (A^{+}_{k}A^{+}_{k})^{\frac{1}{2}}\delta_{ik}\right)\\
&= \frac{1}{KN^2}\left(2p_i\left(\sum\limits_{k=1}^{m} A^{+}_k\right) - 2A^{+}_{i}\right)
\end{align}
and thus, the second term multiplying $dt$ in \eqref{nD_Ito_formula} is given by:
\begin{equation}
\frac{1}{2K}\mathrm{Tr}[\mathbf{D}^{\mathrm{T}} H_{\mathbf{x}} f_i \mathbf{D}] =  \frac{-1}{KN^2}\left(A^{+}_{i}-p_i\left(\sum\limits_{k=1}^{m} A^{+}_k\right)\right)\label{nD_for_Ito_second_term}
\end{equation}
Finally, denoting $d\mathbf{B}_t = [dB^{(1)}_t,dB^{(2)}_t, \ldots, dB^{(m)}_t]^{\mathrm{T}}$ where each $dB^{(j)}_t$ is an independent one dimensional Wiener process, we have:
\begin{align}
\left(\mathbf{D}d\mathbf{B}_t\right)_j &= \sum\limits_{k=1}^{m}\mathbf{D}_{jk}dB^{(k)}_t\nonumber\\
&= \sum\limits_{k=1}^{m}\delta_{jk}\left(A^{+}_{j}A^{+}_{k}\right)^{\frac{1}{4}}dB^{(k)}_t\\
&= \left(A^{+}_{j}\right)^{1/2}dB^{(j)}_t
\end{align}
Thus, using \eqref{nD_jacobian_for_ito}, we see that the last term on the RHS of \eqref{nD_Ito_formula} is given by
\begin{align}
\frac{1}{\sqrt{K}}\left(\nabla_{\mathbf{X}}f\right)^{\mathrm{T}}\mathbf{D}d\mathbf{B}_t &= \frac{1}{\sqrt{K}}\sum\limits_{j=1}^{m}\left(\nabla_{\mathbf{x}} f_i\right)_j\left(\mathbf{D}d\mathbf{B}_t\right)_j\nonumber\\
&=  \frac{1}{N\sqrt{K}}\sum\limits_{j=1}^{m}\left(\delta_{ij}-p_i\right)\left(A^{+}_{j}\right)^{1/2}dB^{(j)}_t\\
&= \frac{1}{N\sqrt{K}}\left(A^{+}_{i}\right)^{1/2}dB^{(i)}_t - p_i\sum\limits_{j=1}^{m}\left(A^{+}_{j}\right)^{1/2}dB^{(j)}_t\label{nD_for_Ito_third_term}
\end{align}
Putting equations \eqref{nD_for_Ito_first_term}, \eqref{nD_for_Ito_second_term} and \eqref{nD_for_Ito_third_term} into \eqref{nD_Ito_formula}, we see that $p_i = f_i(\mathbf{X})_t$, the frequency of the $i\textsuperscript{th}$ type in the population $\mathbf{X}_t$, changes according to the equation:
\begin{equation}
\begin{aligned}
N(t) dp_i &= \underbrace{\left(A^{-}_{i} - p_i\sum\limits_{j=1}^{m}A^{-}_{j}\right)dt}_{\text{$K \to \infty$ prediction}} - \frac{1}{K}\underbrace{\frac{1}{N(t)}\left(A^{+}_{i}-p_i\left(\sum\limits_{k=1}^{m} A^{+}_k\right)\right)dt}_{\substack{\text{Directional finite size effects}\\\text{due to differential turnover rates}}}\\
&+ \frac{1}{\sqrt{K}}\underbrace{\left[\left(A^{+}_{i}\right)^{1/2}dB^{(i)}_t - p_i\sum\limits_{j=1}^{m}\left(A^{+}_{j}\right)^{1/2}dB^{(j)}_t\right]}_{\substack{\text{Non-directional finite size effects}\\\text{due to stochastic fluctuations}}}
\end{aligned}
\end{equation}
Substituting the functional forms given by \eqref{nD_functional_forms_for_replicator} and repeating calculations for the $A^{+}_i$ terms exactly analogous to those done in going from \eqref{nD_replicator_intermediate_1} to \eqref{nD_replicator_mutator} now yields equation \eqref{nD_eqn_for_frequencies} in the main text.
\chapter{A Price-like equation for the variance of a type-level quantity}\label{App_stoch_var_eqns}
Let $\sigma^2_{f}$ denote the statistical variance of a type-level quantity, defined as:
\begin{equation}
    \sigma^2_{f} \coloneqq \overline{(f^2)} - (\overline{f})^2
\end{equation}
where $\overline{X}$ is the statistical mean value defined by \eqref{nD_mean}. By the product rule, we have
\begin{equation}
\label{prod_rule_for_variances}
\frac{d\sigma^2_{f}}{dt} = 2\overline{f\frac{\partial f}{\partial t}} + \sum\limits_{i=1}^{m}f_i^2\frac{dp_i}{dt} - \frac{d}{dt}(\overline{f}^2)
\end{equation}
We will evaluate the RHS term by term. The first term is as simplified as can be. For the second term, we can substitute $dp_i$ from \eqref{nD_stochastic_RM} and use the same steps used in going from \eqref{nD_replicator_mutator} to \eqref{nD_Price} to write
\begin{equation}
\label{2nd_term_for_variances}
\begin{aligned}
\sum\limits_{i=1}^{m}f_i^2dp_i &= \textrm{Cov}(w,f^2)dt - \frac{1}{KN_K}\textrm{Cov}(\tau,f^2)dt\\
&+ \mu\left(1-\frac{1}{KN_K(t)}\right)\left(\sum\limits_{i=1}^{m}f^2_iQ_i(\mathbf{p}) - \overline{f^2}\sum\limits_{i=1}^{m}Q_i(\mathbf{p})\right)dt \\
&+\frac{1}{\sqrt{K}N_{K}(t)}\left(\sum\limits_{i=1}^{m}f^2_i\sqrt{A_i^+}dB_{t}^{(i)} - \overline{f^2}\sum\limits_{i=1}^{m}\sqrt{A_i^+}dB_{t}^{(i)}\right)
\end{aligned}
\end{equation}
For the third term, we need to use It\^{o}'s formula. Here, the relevant version of It\^{o}'s formula is the one-dimensional version of \eqref{nD_Ito_formula}. Given a one-dimensional process $dX_t = S(X_t)dt + \sum D_j(X_t)dB^{(j)}_t$ with $S, D_j$ being suitable real functions and each $B^{(j)}_t$ being an independent Wiener process, It\^{o}'s formula says that given any $C^2(\mathbb{R})$ function $g(x)$, we have the relation:
\begin{equation}
\label{1D_Ito_formula}
dg(X_t) = \left(S(X_t)g'(X_t) + \frac{g''(X_t)}{2}\sum\limits_{j}D_j^2(X_t)\right)dt + \sum\limits_{j}D_j(X_t)g'(X_t)dB^{(j)}_t 
\end{equation}
In our case, we have a one-dimensional process for the mean value $d\overline{f}$ of the type level quantity, and the $C^2(\mathbb{R})$ function $g(x) = x^2$. It\^{o}'s formula thus says that the third term of \eqref{prod_rule_for_variances} is given by:
\begin{equation}
\label{3rd_term_for_variances}
d(\overline{f}^2) = \left(2\overline{f}S(X_t) + \sum\limits_{j}D_j^2(X_t)\right)dt + \sum\limits_{j}2\overline{f}D_j(X_t)dB^{(j)}_t
\end{equation}
where the relevant functions $S$ and $D_j$ can be read off from \eqref{nD_stochastic_Price}. Since the $dB$ terms are unwieldy and do not contribute to the expected dynamics, we will denote the contribution of all the $dB_t$ terms collectively by $dB_{\sigma^2_{f}}$ and not explicitly calculate these terms. We also note that the covariance operator is a bilinear form, \emph{i.e.} given any three quantities $X$, $Y$ and $Z$ and any constant $a \neq 0$, we have the relations:
\begin{align*}
\textrm{Cov}(aX,Y) &= a\textrm{Cov}(X,Y) = \textrm{Cov}(X,aY)\\
\textrm{Cov}(X,Y+Z) &= \textrm{Cov}(X,Y)+\textrm{Cov}(X,Z)
\end{align*}
Substituting equations \eqref{2nd_term_for_variances} and \eqref{3rd_term_for_variances} into equation \eqref{prod_rule_for_variances} and using this property of covariances, we obtain:
\begin{equation}
\label{intermediate_1_for_variances}
\begin{aligned}
d\sigma^2_{f} &= \textrm{Cov}(w,f^2 - 2\overline{f}f)dt - \frac{1}{KN_K}\left(\textrm{Cov}(\tau,f^2 - 2\overline{f}f)\right)dt + 2\left(\overline{f\frac{\partial f}{\partial t}} - \overline{f}\overline{\left(\frac{\partial f}{\partial t}\right)}\right)dt\\
&+ \mu\left(1-\frac{1}{KN_K(t)}\right)\left(\sum\limits_{i=1}^{m}(f^2_i - 2\overline{f}f_i)Q_i(\mathbf{p}) - (\overline{f^2}-2\overline{f}^2)\sum\limits_{i=1}^{m}Q_i(\mathbf{p})\right)dt\\
&+ \frac{1}{KN^2_{K}(t)}\left(\sum\limits_{i=1}^{m}f^2_iA_i^+ - \overline{f}^2\sum\limits_{i=1}^{m}A_i^+\right)dt\\
&+ dB_{\sigma^2_{f}}
\end{aligned}
\end{equation}
Now, we note that
\begin{align}
\frac{1}{N_K}A_i^+ &= \frac{1}{N_K}\left(\tau_ix_i + \mu Q_i(\mathbf{x})\right)\\
&= \tau_ip_i + \mu Q_i(\mathbf{p})
\end{align}
and thus the third line of \eqref{intermediate_1_for_variances} is
\begin{align}
\frac{1}{KN^2_{K}(t)}\left(\sum\limits_{i=1}^{m}f^2_iA_i^+ - \overline{f}^2\sum\limits_{i=1}^{m}A_i^+\right)dt &= \frac{1}{KN_{K}}\sum\limits_{i=1}^{m}f_i^2\left(\tau_ip_i + \mu Q_i(\mathbf{p})\right) - \frac{\overline{f}^2}{KN_{K}}\sum\limits_{i=1}^{m}\left(\tau_ip_i + \mu Q_i(\mathbf{p})\right)\\
&= \frac{1}{KN_K}\sum\limits_{i=1}^{m}\left(f_i^2 - \overline{f}^2\right)\left(\tau_ip_i + \mu Q_i(\mathbf{p})\right)\\
&= \frac{1}{KN_K}\left(\textrm{Cov}(\tau,f^2)+\mu \sum\limits_{i=1}^{m}\left(f_i^2 - \overline{f}^2\right)Q_i(\mathbf{p})\right)
\end{align}
Substituting this into \eqref{intermediate_1_for_variances} and using $M_{\sigma^2_f}(\mathbf{p},N_K)$ to denote the contributions of all the mutational terms (\emph{i.e.} all terms with a $\mu$ factor) for notational brevity, we obtain
\begin{equation}
\begin{aligned}
d\sigma^2_{f} &= \textrm{Cov}(w,f^2 - 2\overline{f}f)dt - \frac{1}{KN_K}\left(\textrm{Cov}(\tau,2f^2 - 2\overline{f}f)\right)dt\\
&+ 2\textrm{Cov}\left(\frac{\partial f}{\partial t},f\right)dt + M_{\sigma^2_f}(\mathbf{p},N_K)dt + dB_{\sigma^2_{f}}
\end{aligned}
\end{equation}
We can now complete the square inside the covariance terms of the first line of the RHS by writing $f^2 - 2\overline{f}f = (f - \overline{f})^2 - \overline{f}^2$ to obtain
\begin{equation}
\label{intermediate_2_for_variances}
\begin{aligned}
d\sigma^2_{f} &= \left[ \ \textrm{Cov}\left(w,(f - \overline{f})^2\right)-\textrm{Cov}\left(w, {\left(\overline{f}\right)}^2\right) \ \right]dt\\[12pt]
&- \frac{1}{KN_K}\left[ \ \textrm{Cov}\left(\tau,(f - \overline{f})^2\right) + \textrm{Cov}\left(\tau,f^2 - {\left(\overline{f}\right)}^2\right) \ \right]dt\\[12pt]
& + 2\textrm{Cov}\left(\frac{\partial f}{\partial t},f\right)dt + M_{\sigma^2_f}(\mathbf{p},N_K)dt + dB_{\sigma^2_{f}}
\end{aligned}
\end{equation}
Finally, we can now recognize that
\begin{align}
\textrm{Cov}\left(w, {\left(\overline{f}\right)}^2\right) &= \overline{\left(w{\left(\overline{f}\right)}^2\right)} - \overline{w}\overline{\left({\left(\overline{f}\right)}^2\right)}\\
&= {\left(\overline{f}\right)}^2\sum\limits_{i=1}^{m}w_ip_i - \overline{w}{\left(\overline{f}\right)}^2\sum\limits_{i=1}^{m}p_i\\
&= {\left(\overline{f}\right)}^2\overline{w} - \overline{w}{\left(\overline{f}\right)}^2 = 0\label{cov_term_1_for_variances}
\end{align}
and
\begin{align}
\textrm{Cov}\left(\tau,f^2 - {\left(\overline{f}\right)}^2\right) &= \overline{\tau\left(f^2 - {\left(\overline{f}\right)}^2\right)} - \overline{\tau}\left(\overline{f^2 - {\left(\overline{f}\right)}^2}\right)\\
&= \overline{\tau f^2} - \overline{\tau}{\left(\overline{f}\right)}^2  - \overline{\tau}\overline{f^2} + \overline{\tau}{\left(\overline{f}\right)}^2\\
&=\overline{\tau f^2} - \overline{\tau}\overline{f^2} = \textrm{Cov}(\tau,f^2)\label{cov_term_2_for_variances}
\end{align}
Substituting \eqref{cov_term_1_for_variances} and \eqref{cov_term_2_for_variances} into \eqref{intermediate_2_for_variances}, we thus arrive at:
\begin{equation}
\label{intermediate_3_for_variances}
\begin{aligned}
d\sigma^2_{f} &= \textrm{Cov}\left(w,(f - \overline{f})^2\right)dt\\
&- \frac{1}{KN_K}\left[ \ \textrm{Cov}\left(\tau,(f - \overline{f})^2\right) + \textrm{Cov}(\tau,f^2) \ \right]dt\\
& + 2\textrm{Cov}\left(\frac{\partial f}{\partial t},f\right)dt + M_{\sigma^2_f}(\mathbf{p},N_K)dt + dB_{\sigma^2_{f}}
\end{aligned}
\end{equation}
Covariance with $(f-\overline{f})^2$ is a measure of covariance with the existing `spread' of $f_i$ around the population mean value $\overline{f}$ and is thus readily understood to represent existing variation. However, the $\textrm{Cov}(\tau,f^2)$ term is difficult to interpret since it depends on the square of $f$, which may not have any direct biological interpretation. We will convert this into a form whose interpretation is more clear by repeatedly adding and subtracting a quantity from our expression (these quantities are shown in red below for clarity), noting that this leaves the net value unchanged. We have:
\begin{align}
\textrm{Cov}(\tau, f^2) &= \overline{\tau f^2} - \overline{\tau}\overline{f^2}\nonumber\\
&= \overline{\tau f^2} {\color{red} + \overline{\tau{\left(\overline{f}\right)}^2} - \overline{\tau f \overline{f}}}  - \overline{\tau}\overline{f^2} \nonumber\\
&= \overline{\tau f^2} + \overline{\tau{\left(\overline{f}\right)}^2} - 2\overline{\tau f \overline{f}}\nonumber\\
&= \overline{\tau\left(f-\overline{f}\right)^2}\nonumber\\
&= \overline{\tau\left(f-\overline{f}\right)^2} {\color{red}- \overline{\tau}{\left(\overline{f}\right)}^2 + 2 \overline{\tau}\overline{f\overline{f}} - \overline{\tau}{\left(\overline{f}\right)}^2}\nonumber\\
&= \overline{\tau\left(f-\overline{f}\right)^2} {\color{red} - \overline{\tau}\overline{f^2}} -\overline{\tau}\overline{{\left(\overline{f}\right)}^2} + 2 \overline{\tau}\overline{f\overline{f}} {\color{red} + \overline{\tau}\overline{f^2}} - \overline{\tau}{\left(\overline{f}\right)}^2\nonumber\\
&=  \overline{\tau\left(f-\overline{f}\right)^2} - \overline{\tau}\overline{\left(f-\overline{f}\right)^2} + \overline{\tau}\left(\overline{f^2} - {\left(\overline{f}\right)}^2\right)\nonumber\\
&= \textrm{Cov}(\tau, \left(f-\overline{f}\right)^2) + \overline{\tau}\sigma^2_{f}\label{cov_to_var_sub_for_variances}
\end{align}
Finally, substituting \eqref{cov_to_var_sub_for_variances} into \eqref{intermediate_3_for_variances}, we see that the rate of change of the variance of any type-level quantity $f$ in the population satisfies:
\begin{equation}
\begin{aligned}
d\sigma^2_{f} &= \textrm{Cov}\left(w,(f - \overline{f})^2\right)dt\\[12pt]
&- \frac{1}{KN_K}\left[ \ 2\textrm{Cov}\left(\tau,(f - \overline{f})^2\right) + \overline{\tau}\sigma^2_{f}(t) \ \right]dt\\[12pt]
& + 2\textrm{Cov}\left(\frac{\partial f}{\partial t},f\right)dt + M_{\sigma^2_f}(\mathbf{p},N_K)dt + dB_{\sigma^2_{f}}
\end{aligned}
\end{equation}
\chapter{Some Examples}\label{App_examples}
\section{An example in one dimension: The stochastic logistic equation}
Here, we analyze example \ref{ex_1D_stoch_logistic}. To recap, we had a population of individuals that exhibit a constant per-capita birth rate $\lambda > 0$, and a per-capita death rate that had the linear density-dependence $\mu + (\lambda-\mu)\frac{n}{K}$, where $\mu$ and $K$ are positive constants. Thus, we have the birth and death rates
\begin{equation}
\label{ex_1D_stoch_logistic_BD_eqns}
\begin{aligned}
    b(n) &= \lambda n\\
    d(n) &= \left(\mu + (\lambda-\mu)\frac{n}{K}\right)n
\end{aligned}
\end{equation}
Here, $K$ is the system-size parameter. Introducing the population density $x\coloneqq n/K$, we obtain
\begin{align*}
    b_K(x) &= \frac{1}{K}b(n) = \frac{1}{K}\lambda Kx\\
    d_K(x) &= \frac{1}{K}d(n) = \frac{1}{K}\left(\mu + (\lambda-\mu)\frac{Kx}{K}\right)Kx
\end{align*}
Thus, we have
\begin{equation*}
    A^{\pm}(x) = b_K(x)\pm d_K(x) = x\left(\lambda \pm \left(\left(\mu + (\lambda-\mu)x\right)\right) \right)
\end{equation*}
Defining $r=\lambda-\mu$ and $v=\lambda+\mu$ and using equation \eqref{1D_SDE}, we 
see that the `mesoscopic view' of the system is given by the solution of the SDE
\begin{equation}\label{ex_1D_stoch_logistic_full_SDE}
dX_t =  rX_t(1-X_t)dt + \sqrt{\frac{X_t(v+rX_t)}{K}}dW_t
\end{equation}
From equation \eqref{1D_det_limit}, we see that the deterministic dynamics are
\begin{equation}\label{ex_1D_stoch_logistic_det_limit}
\frac{dx}{dt} = A^-(x) = rx(1-x)
\end{equation}
showing that in the infinite population limit, we obtain the logistic equation. This derivation also makes it clear that two systems with very different stochastic dynamics can nevertheless converge to the same infinite population limit \eqref{ex_1D_stoch_logistic_det_limit}, since equation \eqref{ex_1D_stoch_logistic_det_limit} only depends on the difference $\lambda-\mu$.
\myfig{0.6}{figures/App_stoch_logistic_same_w_different_t.png}{\textbf{Comparison of time series produced by Gillespie simulations of the stochastic logistic equation for two different parameter values.} Simulations with the parameter values $\lambda = 4, \mu = 2$ are plotted in red, and simulations with the parameter values $\lambda = 31, \mu = 29$ are plotted in blue. Both simulations have $K=200$. The infinite population limit of both simulations is plotted in black dotted lines. The graph shows 10 realizations each for the two sets of parameter values.}{fig_1D_same_w_different_t_comparison}
To illustrate the effects of this seemingly innocent fact, figure \ref{fig_1D_same_w_different_t_comparison} compares two simulations which have the same value of $\lambda - \mu$ but a ten-fold difference in $\lambda+\mu$. As is clear from the figure, though both populations have the same behavior at the infinite population limit, populations with a higher value of $\lambda+\mu$ exhibit much wilder fluctuations and are therefore more prone to stochastic extinction. This is the root cause of the noise-induced selection that occurs in higher dimensions, discussed in detail in part \ref{part_summary}.

Letting $\alpha(t)$ be the solution of the logistic equation \eqref{ex_1D_stoch_logistic_det_limit}, We can Taylor expand $A^{\pm}(x)$ for the weak noise approximation, and we find:
\begin{align*}
A^-_1(x) &= \frac{d}{dx}(rx(1-x))\biggl{|}_{x=\alpha} = r(1 - 2\alpha(t))\\
A^+_0(x) &= \alpha(t)(v+r\alpha(t))
\end{align*}
Thus, the weak noise approximation of \ref{ex_1D_stoch_logistic_BD_eqns} is given by
\begin{equation}
    X_t = \alpha(t) + \frac{1}{\sqrt{K}}Y_t
\end{equation}
where the stochastic process $Y_t$ is an Ornstein-Uhlenbeck process given by the solution to the linear SDE
\begin{align}
    dY_t &= A^-_1(t)Y_tdt + \sqrt{A^+_0(t)}dW_t\nonumber\\
    \Rightarrow dY_t &= r(1 - 2\alpha(t))Y_tdt + \sqrt{\alpha(t)(v+r\alpha(t))}dW_t\label{ex_1D_stoch_logistic_WNA}
\end{align}
The time series predicted by these three processes look qualitatively similar and all seem to fluctuate about the deterministic steady state (Figure \ref{fig_1D_stoch_logistic_timeseries}).
\myfig{1}{figures/App_stoch_logistic_timeseries.png}{\textbf{Comparison of a single realization} of the exact birth-death process \eqref{ex_1D_stoch_logistic_BD_eqns}, the deterministic trajectory \eqref{ex_1D_stoch_logistic_det_limit}, the non-linear Fokker-Planck equation \eqref{ex_1D_stoch_logistic_full_SDE}, and the weak noise approximation \eqref{ex_1D_stoch_logistic_WNA} for \textbf{(A)} $K = 500$, \textbf{(B)} $K = 1000$, and \textbf{(C)} $K = 10000$. $\lambda = 2, \mu = 1$ for all thee cases.}{fig_1D_stoch_logistic_timeseries}
The deterministic trajectory \eqref{ex_1D_stoch_logistic_det_limit} has two fixed points, one at $x=0$ (extinction) and one at $x=1$ (corresponding to a population size of $n=K$). For $r > 0$, $x=0$ is unstable and $x=1$ is a global attractor, meaning in the deterministic limit, when $r > 0$, all populations end up at $x=1$ given enough time.\myfig{0.85}{figures/App_stoch_logistic_distributions.png}{\textbf{Comparison of the steady-state densities} given by \eqref{ex_1D_stoch_logistic_BD_eqns}, \eqref{ex_1D_stoch_logistic_full_SDE}, and \eqref{ex_1D_stoch_logistic_WNA} for \textbf{(A)} $K = 500$, \textbf{(B)} $K = 1000$, and \textbf{(C)} $K = 10000$. $\lambda = 2, \mu = 1$ for all thee cases. Each curve was obtained using $1000$ independent realizations.}{fig_1D_stoch_logistic_densities}
The stochastic dynamics \eqref{ex_1D_stoch_logistic_full_SDE} and \eqref{ex_1D_stoch_logistic_WNA}, however, depend not only on $r$, but also on $v$, the sum of the birth and death rates. It has been proven that $X_t = 0$ is the only recurrent state for the full stochastic dynamics \eqref{ex_1D_stoch_logistic_full_SDE}, meaning that every population is guaranteed to go extinct\footnote{This can be proven using tools from Markov chain theory. For those interested, the proof uses ergodicity to arrive at a contradiction if any state other than $0$ exhibits a non-zero density at steady state.} given enough time~\citep{nasell_extinction_2001}, thus illustrating an important difference between finite and infinite populations. $X_t = 0$ is also an `absorbing' state since once a population goes extinct, it has no way of being revived in this model. However, if $K$ is large enough, the eventual extinction of the population may take a very long time. In fact, we can make the expected time to extinction arbitrarily long by making $K$ sufficiently large. Thus, for moderately large values of $K$, it is biologically meaningful only to look at a weaker version of the steady state distribution by imposing the condition that the population does not go extinct and looking at the `transient' dynamics~\citep{hastings_transients_2004}.
Conditioned on non-extinction, the solution to \eqref{ex_1D_stoch_logistic_full_SDE} has a `quasistationary' distribution about the deterministic attractor $X_t = 1$, with some variance reflecting the effect of noise-induced fluctuations in population size~\citep{nasell_extinction_2001} due to the finite size of the population. The weak-noise approximation \eqref{ex_1D_stoch_logistic_WNA} implicitly assumes non-extinction by only measuring small fluctuations from the deterministic solution to \eqref{ex_1D_stoch_logistic_det_limit} and thus, at steady state, naturally describes a quasistationary distribution centered about $X_t = 1$. The steady-state density (probability density function as $t \to \infty$) of the exact birth-death process \eqref{ex_1D_stoch_logistic_BD_eqns} is compared with that predicted by \eqref{ex_1D_stoch_logistic_full_SDE} and \eqref{ex_1D_stoch_logistic_WNA} for various values of $K$ in figure \ref{fig_1D_stoch_logistic_densities}.

\section{An example for discrete traits: Lotka-Volterra and matrix games in finite populations}

The methods outlined in the above section have recently been used to study the population dynamics of a finite population playing a so-called `matrix game' (An evolutionary game for which you can write down a payoff matrix) with 2 pure strategies~\citep{tao_stochastic_2007}. Based on the interpretation of what each type represents, this is mathematically equivalent to studying frequency-dependent selection on a one-locus two-allele gene (with a bijective genotype-phenotype map and no mutations) or studying two-species competitive Lotka-Volterra dynamics, as we will show below. The stochastic Lotka-Volterra competition model shown below has also been proved to be equivalent to an $m$-allele Moran model under certain limits~\citep{constable_mapping_2017}.

Let us imagine a population with $m$ types of individuals that are interacting according to some ecological rules. Let the state of the population be characterized by the vector $\mathbf{n}(t) = [n_1(t),n_2(t),\ldots,n_m(t)]^{\mathrm{T}}$, where $n_i(t)$ is the number of type $i$ individuals at time $t$. Let the birth and death rates of the $i$th type be given by:
\begin{equation}
\label{nD_example_numbers_b_d_rates}
\begin{aligned}
b_i(\mathbf{n}) &= \left(\lambda + \frac{1}{K}\left(\sum\limits_{j=1}^{m}\beta_{ij}n_j\right)\right)n_i\\
d_i(\mathbf{n}) &= \left(\mu + \frac{1}{K}\left(\sum\limits_{j=1}^{m}\delta_{ij}n_j\right)\right)n_i
\end{aligned}
\end{equation}
where $K > 0$ is our system size parameter (and represents a global carrying capacity across all types), $\lambda > 0$ and $\mu > 0$ are suitable positive constants representing the baseline natality and mortality common to all types, and $\beta_{ij}$ and $\delta_{ij}$ are constants describing the effect of type $j$ individuals on the birth and death rate of type $i$ individuals respectively. The sign of $M_{ij} \coloneqq \beta_{ij} - \delta_{ij}$ determines whether type $j$ has a net positive or negative effect on the growth of type $i$. In ecological communities, this is a per-capita ecological interaction effect. In game-theoretic terms, we can interpret $M_{ij}$ as the payoff obtained by a type $j$ individual playing against a type $i$ individual. I assume that $| M_{ij} | \ll K$. The values $M_{ij}$ are often collected in an $m \times m$ matrix $\mathbf{M}$ called the `payoff matrix' (in evolutionary game theory) or `interaction matrix' (in Lotka-Volterra models). Lotka-Volterra models also frequently assume that the diagonal elements $M_{ii}$ are all equal, though I will not make that assumption here.

Going from population numbers $\mathbf{n}$ to densities $\mathbf{x} = \mathbf{n}/K$, we obtain the birth and death rates:
\begin{equation}
\label{nD_example_density_b_d_rates}
\begin{aligned}
b^{(K)}_i(\mathbf{x}) &= \left(\lambda + \sum\limits_{j=1}^{m}\beta_{ij}x_j\right)x_i\\
d^{(K)}_i(\mathbf{x}) &= \left(\mu + \sum\limits_{j=1}^{m}\delta_{ij}x_j\right)x_i
\end{aligned}
\end{equation}
Thus, we have
\begin{equation*}
A^{\pm}_{i} = x_i\left(\left(\lambda \pm \mu\right) + \sum\limits_{j=1}^{m}\left(\beta_{ij} \pm \delta_{ij}\right)x_j\right)
\end{equation*}
Defining $r \coloneqq \lambda - \mu$, $\nu \coloneqq \lambda + \mu$, and $T_{ij} \coloneqq \beta_{ij} + \delta_{ij}$, we see from equation \eqref{nD_Ito_SDE} that the mesoscopic view is the $m$ dimensional SDE given by
\begin{equation}
\label{nD_example_SDE}
d\mathbf{X}_{t} = \mathbf{A^-}(\mathbf{X}_t)dt + \frac{1}{\sqrt{K}}\mathbf{D}(\mathbf{X}_t)d\mathbf{W}_t
\end{equation}
where 
\begin{equation*}
\mathbf{A^-}_i = {(\mathbf{X}_{t})}_i(r + \sum\limits_{j=1}^{m}M_{ij}{(\mathbf{X}_{t})}_j) 
\end{equation*}
and
\begin{equation*}
(\mathbf{D}\mathbf{D}^{\mathrm{T}})_i = {(\mathbf{X}_{t})}_i(\nu + \sum\limits_{j=1}^{m}T_{ij}{(\mathbf{X}_{t})}_j) 
\end{equation*}
From \eqref{nD_det_limit}, we see that the deterministic limit is a set of $m$ coupled ODEs given by
\begin{equation}
\label{nD_example_det_limit}
\frac{d x_i}{dt} = x_i\left(r + \sum\limits_{j=1}^{m}M_{ij}x_j\right)
\end{equation}
These are precisely the Lotka-Volterra equations for a system of $m$ species. By matching the terms of \eqref{nD_example_density_b_d_rates} with those of \eqref{nD_functional_forms_for_replicator}, we can identify that we have $\mu = 0$ and
\begin{equation}
\label{nD_example_fitness}
w_i(\mathbf{x}) = r + \sum\limits_{j=1}^{m}M_{ij}x_j
\end{equation}
If $\mathbf{p}(t) = [p_1(t),\ldots,p_m(t)]^{\mathrm{T}}$ is the frequency vector at time $t$ and $N_K(t) = \sum_i x_i(t)$, then the mean fitness is given by
\begin{align}
\overline{w}(t) &= \sum\limits_{i=1}^{m}w_ip_i\\
&= \sum\limits_{i=1}^{m}\left(r + \sum\limits_{j=1}^{m}M_{ij}x_j\right)p_i\\
&= r + \sum\limits_{i=1}^{m}p_i\left(\sum\limits_{j=1}^{m}M_{ij}x_j\right)
\end{align}
where we have used the fact that $\sum_i p_i = 1$ in the last line. Using \eqref{nD_replicator_mutator} to write down the equations for the frequencies $p_i$, we obtain
\begin{equation}
\frac{1}{N_K(t)}\frac{dp_i}{dt} = \left[(\mathbf{Mp})_i - \mathbf{p}\cdot\mathbf{Mp}\right]p_i   
\end{equation}
which is the familiar version of the replicator equation seen in most textbooks, with an extra $N_K(t)$ factor to account for the fact that $\sum_i x_i$ is allowed to fluctuate in our model. If instead $N_K$ was a constant for all time, it could simply be absorbed into the definition of the payoff matrix $M$ to obtain exactly the replicator equation as presented in most ecology/evolution textbooks. Both the stochastic dynamics \eqref{nD_example_SDE} and the deterministic limit \eqref{nD_example_det_limit} can be simplified from an $m$ dimensional system to an $m-1$ dimensional system by a coordinate transformation that projects the dynamics onto an appropriate curve: If we go from the variables $x_1,\ldots,x_m$ to the variables $p_1,\ldots,p_{m-1},N_K$, we can exploit the fact that $N_K$ varies much less than the $p_i$ terms to project the system onto a `slow manifold' in which $N_K$ is approximately constant, thus obtaining an $m-1$ dimensional system of equations and recovering the relation between the Lotka-Volterra equations for $m$ species and the replicator equation for $m-1$ tactics~\citep{constable_mapping_2017,parsons_dimension_2017}. However, I will not explore such dimensional reduction techniques further in this manuscript, and refer the reader to~\cite{constable_stochastic_2013} and~\cite{parsons_dimension_2017} for a review of the ideas of (stochastic) dynamics on slow manifolds.

Let the solution to the equations \eqref{nD_example_det_limit} be given by $\mathbf{a}(t) = [a_1(t),\ldots,a_m(t)]^{\mathrm{T}}$. For the weak noise approximation, we can Taylor expand $A^{\pm}_i$ and use \eqref{nD_WNA_directional_derivative_for_replicator_eqns} to compute the directional derivative as:
\begin{align}
D_i &= y_iw_i(\mathbf{a}) + a_i\sum\limits_{k=1}^{m}y_k\left(\frac{\partial w_i}{\partial x_k}\bigg{|}_{\mathbf{x}=\mathbf{a}(t)}\right)\\
&= y_iw_i(\mathbf{a}) + a_i\sum\limits_{k=1}^{m}y_k\left(\frac{\partial}{\partial x_k}(r+\sum\limits_{j=1}^{m}M_{ij}x_j)\bigg{|}_{\mathbf{x}=\mathbf{a}(t)}\right)\\
&= y_iw_i(\mathbf{a}) + a_i\sum\limits_{k=1}^{m}y_kM_{ik}\\
\Rightarrow D_i &= y_iw_i(\mathbf{a}) + a_iw_i(\mathbf{y}) - ra_i\label{nD_example_directional_derivative}
\end{align}
where we have used the fact that $w_i(\mathbf{y}) = r + \sum\limits_{k=1}^{m}y_kM_{ik}$ (from \eqref{nD_example_fitness}) in the last step. Thus, in the weak noise approximation of our process, the dynamics are given by
\begin{equation}
\mathbf{x}(t) = \mathbf{a}(t) + \frac{1}{\sqrt{K}}\mathbf{y}(t)
\end{equation}
where the stochastic fluctuations $\mathbf{y}(t)$ satisfy the linear Fokker-Planck equation
\begin{equation}
\resizebox{1.1\textwidth}{!}{$\displaystyle\frac{\partial P}{\partial t}(\mathbf{y},t) = \sum\limits_{i=1}^{m}\left(-\frac{\partial}{\partial y_i}\left\{\left(y_iw_i(\mathbf{a}) + a_iw_i(\mathbf{y}) - ra_i\right)P(\mathbf{y},t)\right\}+\frac{1}{2}\left(a_i\left(\nu + \sum\limits_{j=1}^{m}T_{ij}a_j\right)\right)\frac{\partial^2}{\partial{y_i}^2}P(\mathbf{y},t)\right)$}
\end{equation}
Using \eqref{nD_example_directional_derivative} in \eqref{nD_moment_eqn_mean}, we see that the fluctuations are expected to evolve as:
\begin{equation}
\label{nD_example_moment_eqn_mean}
\frac{d}{dt}\mathbb{E}[y_i] = w_i(\mathbf{a})\mathbb{E}[y_i] + a_i\sum\limits_{k=1}^{m}M_{ik}\mathbb{E}[y_k]
\end{equation}
or, in matrix form:
\begin{equation}
\resizebox{0.93\textwidth}{!}{$\displaystyle
	\frac{d}{dt}\begin{bmatrix}
	\mathbb{E}[y_1]\\
	\mathbb{E}[y_2]\\
	\vdots\\
	\mathbb{E}[y_i]\\
	\vdots\\
	\mathbb{E}[y_m]
	\end{bmatrix}
	=
	\begin{bmatrix}
	(r + \sum\limits_{j=1}^{m}M_{1j}a_j + a_1M_{11}) & a_1M_{12} & a_1M_{13} & \dots & \dots & \dots & a_1M_{1m}\\
	a_2M_{21} & (r + \sum\limits_{j=1}^{m}M_{2j}a_j + a_2M_{22}) & a_2M_{23} & \dots & \dots & \dots & a_2M_{2m}\\
	\vdots &  & \ddots & &  & & \vdots\\
	a_{i}M_{i1} & a_iM_{i2} & a_iM_{i3} & \dots & (r + \sum\limits_{j=1}^{m}M_{ij}a_j + a_iM_{ii}) & \dots & a_iM_{im}\\
	\vdots &  &  & & & \ddots & \vdots\\
	a_mM_{m1} & a_mM_{m2} & a_mM_{m3} & \dots & \dots & \dots & (r + \sum\limits_{j=1}^{m}M_{mj}a_j + a_mM_{mm})
	\end{bmatrix}
	\begin{bmatrix}
	\mathbb{E}[y_1]\\
	\mathbb{E}[y_2]\\
	\vdots\\
	\mathbb{E}[y_i]\\
	\vdots\\
	\mathbb{E}[y_m]
	\end{bmatrix}
	$}
\end{equation}
The eigenvalues of the first matrix on the RHS will tell us whether the fixed point $\mathbb{E}[y_i] = 0 \ \forall \ i$ (the only fixed point of this system) is stable, or whether fluctuations are expected to grow (up to the point where the fluctuations are so large that the WNA is no longer valid). In the $m=2$ case,~\cite{tao_stochastic_2007} have shown that $\mathbb{E}[y_i] = 0 \ \forall \ i$ is a stable fixed point for this system iff the point $\mathbf{y}$ is an ESS (in the usual game-theoretic sense) for the matrix game defined by the payoff matrix $\mathbf{M}$.

\section{An example of noise-induced selection causing deviations from neutrality despite equal fitness}\label{App_deviation_from_neutrality}

To illustrate the biasing effects of noise-induced selection in otherwise neutral dynamics, I will use a simple 2 species Lotka-Volterra competition-like model where the effects of competition are on birth rates of one species but on the death rates of the other.

To motivate this, consider a community that contains two types of birds, say type 1 and type  2. These birds compete for limited resources, but in a peculiar manner: Though the two birds feed on different food sources, the trees that type 1 birds use for nesting are the same as those that the type 2 birds rely on for food. Both types are fiercely territorial and do not tolerate other individuals of either type on either their nesting or feeding sites. Thus, competition between the two types affects the \emph{birth rate} of type 1 birds (because they can't find good nesting sites) but the \emph{death rate} of type 2 birds (because of starvation), whereas intratype competition affects the death rate in both cases (due to competition for food sources). Occasionally, each type can give birth to babies of the other type due to mutations. Let us construct the simplest possible model for such a system.

Let each type of bird have a constant per-capita intrinsic birth rate (rate of birth of individuals, not rate at which individuals give birth) of 1 due to reproduction. Additionally, type 1 birds face a reduction in birth rates due to competition with type 2 birds. Let us assume that the magnitude of this competition (per-capita) is equal to the per-capita competition experienced from other individuals of the same type. Both types have some additional birth rate due to rare mutations of the other type, parameterized by a mutation rate $\mu > 0$. Let $n_i$ be the number of type $i$ individuals (which may vary over time). Assuming trees and birds are both randomly distributed through the landscape, we arrive at the birth rates
\begin{equation}
\label{App_example_stoch_LV_birth_rates}
\begin{aligned}
	b_{1}({n_1},{n_2}) &= {n_1} + \mu n_2 - \frac{{n_1}{n_2}}{K}\\
	b_{2}({n_1},{n_2}) &= {n_2} + \mu n_1
\end{aligned}
\end{equation}
Here, the ${n_1}{n_2}/K$ term represents the effect of competition between types; The product ${n_1}{n_2}$ quantifies how often a type 1 bird and a type 2 bird are expected to interact, and $K$ is a carrying capacity for the habitat (in analogy to logistic growth or Lotka-Volterra competition), and can be thought of as a proxy for the amount of tree cover in the landscape.

For the death rates, I assume that the effect of intra-type competition on the death rate is linearly density-dependent, and thus arrive at the equations:
\begin{equation}
\label{App_example_stoch_LV_death_rates}
\begin{aligned}
	d_{1}({n_1},{n_2}) &= \frac{{n_1^2}}{K}\\
	d_{2}({n_1},{n_2}) &= \frac{{n_2^2}}{K} + \frac{{n_1}{n_2}}{K}
\end{aligned}
\end{equation}
Note that the effect of competition between types manifests here in an increased death rate of type 2 birds due to starvation.

Moving to density space via the change of variables $x_i = n_i/K$, letting $\mathbf{x} = [x_1, x_2]^{\mathrm{T}}$, and comparing terms with equations \eqref{nD_functional_forms_for_replicator}, we can see that the per-capita fitness $w_i$ of each type is:
\begin{equation*}
	w_{1}(\mathbf{x}) = w_{2}(\mathbf{x}) = 1 - x_1 - x_2
\end{equation*}
The two types of birds have the same fitness! This implies that $w_1 = w_2 = \overline{w}$ and the selection term in \eqref{nD_replicator_mutator} vanishes.
However, if we now compute the per-capita turnover rates $\tau_i$  of each type, we see that we have
\begin{align*}
	\tau_{1}(\mathbf{x}) &= 1 + x_1 - x_2\\
	\tau_{2}(\mathbf{x}) &= 1 + x_1 + x_2
\end{align*}
Thus, $\tau_{1} < \tau_{2}$, and from equation \eqref{nD_stochastic_RM}, we know that this means noise-induced selection favors type 1 over type 2.

Direct simulations of the individual-based model indeed reveal that for low values of $K$, the fraction of individuals in the population that are of type 1 is significantly biased to be greater than 0.5, showing the effect of noise-induced selection (Figure \ref{App_stoch_LV_dens_plots}). This bias disappears for high values of $K$, as expected.
\myfig{1}{figures/App_stoch_LV_density_plots.png}{\textbf{(A)} Time series and \textbf{(B)} Density estimates for $p$, the fraction of type 1 individuals in the population for various values of $K$, obtained from a direct individual-based simulation of the model defined by equations \eqref{App_example_stoch_LV_birth_rates} and \eqref{App_example_stoch_LV_death_rates}, simulated via the (exact) Gillespie algorithm. Dotted lines are at $p=0.5$. At high $K$, the population conforms to deterministic (infinite population) predictions, but at low K, the distribution is biased towards $p > 0.5$. The time series are from single realizations. The density plots in panel $(B)$ are estimated from 100 independent realizations, each of which were run for $10^4$ timesteps. All simulations were initialized with $n_1 = n_2 = K/2$. In all simulations, $\mu = 0.05$.}{App_stoch_LV_dens_plots}

For this model, we can in fact quantitatively derive the effects of noise-induced selection by explicitly calculating each term of equation \eqref{nD_stochastic_RM}. Let $p = x_1/(x_1 + x_2)$ be the frequency of type 1 individuals and let $q = 1-p$. Then, it is easy to check by direct substitution of our functional forms that we have
\begin{align*}
	\overline{w} &= w_1(\mathbf{x}) = w_2(\mathbf{x})\\
	\overline{\tau} &= 1 + x_1 + x_2(q-p)\\
	\mu Q_i(\mathbf{p}) &= \mu p_{j}\textrm{ , where $i \neq j$}\\
	\mu(Q_1(\mathbf{p}) - p\left(\sum\limits_{j=1}^{2} Q_j(\mathbf{p})\right)) &= \mu(q-p)
\end{align*}
And thus, equation \eqref{nD_stochastic_RM} becomes
\begin{equation}
	\label{App_stoch_LV_RM_eqn}
	dp =  \left[\frac{2}{K}p^2q + \mu\left(1-\frac{1}{KN_K}\right)\left(q-p\right)\right]dt + q\sqrt{A^{+}_1}dW_t^{(1)} - p\sqrt{A^{+}_2}dW_t^{(2)}
\end{equation}
where $A_i^{+}$ is as defined throughout this thesis and each $W^{(i)}_t$ is an independent Wiener process. This equation clearly shows the biasing effect of noise-induced selection in the first component of the $dt$ term of the RHS. Since $p^2(1-p) > 0$ for $p \in (0,1)$, this term always tends to increase the fraction of type 1 individuals in the population. Note that the difference in fitness between the two types remains zero if every competition term (\emph{i.e.} every $n_in_j/K$ term) is multiplied by some constant $\alpha > 0$ parameterizing the strength of competition, meaning that the two types still have equal fitness. However, this constant affects the strength of noise-induced selection, and the corresponding term in equation \ref{App_stoch_LV_RM_eqn} becomes $2\alpha p^2q/K$ instead of $2p^2q/K$. Thus, for $\alpha > 1$, the strength of noise-induced selection (and thus the extent to which the distribution of types in the population is biased in favor of type 1 in plots like \ref{App_stoch_LV_dens_plots}) can be made arbitrarily high simply by modulating the strength of competition. The second component in the $dt$ term captures the effects of mutations, and simply reflects the fact that we assumed that $\textrm{(type }1) \to \textrm{(type }2)$ and $\textrm{(type }2) \to \textrm{(type }1)$ mutations occur at the same rate $\mu$, and thus, the net effect of mutational effects depends on the difference between the frequencies of the two types and flows towards the type with lower frequency. Finally, the two $dW^{(i)}_t$ terms are non-directional and vanish upon taking an expectation over the probability space, and therefore have no net contribution other than `blurring out the results' if we look at the dynamics averaged over many realizations.

\section{Interlude: Detecting modes in quantitative trait distributions through Fourier analysis}

In Chapter \ref{chap_infD_processes}, we used various approximations to arrive at the linear functional Fokker-Planck equation
\begin{equation}
\label{functional_WNE_for_App}
    \frac{\partial P}{\partial t}(\zeta,t) = \int\limits_{\mathcal{T}}\left(-\frac{\delta}{\delta \zeta(x)}\left\{\mathcal{D}_{\zeta}[\mathcal{A}^{-}](x)P(\zeta,t)\right\}+\frac{1}{2}\mathcal{A}^{+}(x|\psi)\frac{\delta^2}{\delta\zeta(x)^2}\{P(\zeta,t)\}\right)dx
\end{equation}
for describing stochastic fluctuations $\zeta$ from the deterministic solution obtained by solving \eqref{deterministic_traj}. Our goal is now to find a method to effectively detect and describe evolutionary branches (modes in trait space, corresponding to individual morphs) for this process. Following the methods used by Tim Rogers and colleagues for various special cases~\citep{rogers_demographic_2012, rogers_spontaneous_2012, rogers_modes_2015}, we will do this in a general manner by measuring the autocorrelation of the distribution of the population over trait space, a task made easier by moving to Fourier space.
\myfig{0.8}{figures/App_fourier_fig.png}{\textbf{Schematic description of Fourier analysis}. A function $\phi(x)$ (shown in red) over the trait space can be decomposed as the sum of infinitely many Fourier modes (shown in blue) $\phi_k$. In the Fourier dual space, we can look at the peaks of each of these Fourier modes: The magnitude of $\phi_k$ tells us how much it contributes to the actual function of interest $\phi$.}{fig_Fourier}
Specifically, a convenient theorem due to Weiner and Khinchin relates the autocorrelation of a probability distribution to its power spectral density via Fourier transformation. This has been extensively used in spatial ecology, and we too will make use of it here. We will thus restrict ourselves to cases in which we can express our focal function $\phi$ in terms of the Fourier basis $\{e^{ikx}\}_{k\in\mathbb{Z}}$ (Figure \ref{fig_Fourier}). For example, this can be done by restricting ourselves to cases where $\mathcal{T}$ is an interval with `periodic boundary conditions' (\emph{i.e.} we will extend all our functions from $\mathcal{T}$ to $\mathbb{R}$ in a way that they appear periodic with period given by the length of the interval $\mathcal{T}$). We may also need to restrict ourselves to a `nice' subspace of $\mathcal{M}(\mathcal{T})$, for example by intersecting with $L^2(\mathcal{T})$. In any case, we will assume all the prerequisites required for a Fourier basis expansion are satisfied. If $\mathcal{D}_{\zeta}[\mathcal{A}^{-}]$ is a translation-invariant\footnote{This is horrible nomenclature by the mathematicians. Though `invariant' is the conventional name for this concept, the intended meaning is not really invariant but `equivariant'. Formally, let $\mathcal{F}$ be a suitable function space of real valued functions. For any $c \in \mathbb{R}$, let $T_c: \mathcal{F} \to \mathcal{F}$ be the translation operator on this space, defined by $T_c[f(x)] = f(x+c)$. An operator $L: \mathcal{F} \to \mathcal{F}$ is said to be translation-invariant if it commutes with $T_c$ for every $c \in \mathbb{R}$, \emph{i.e.} $T_c[L[f]] = L[T_c[f]] \ \forall \ f \in \mathcal{F} \ \forall \ c \in \mathbb{R}$.} linear operator, then $\exp(ikx)$ acts as an eigenfunction, significantly simplifying the calculations. We therefore assume that $\mathcal{D}_{\zeta}[\mathcal{A}^{-}]$ takes the form:
\begin{equation*}
 \mathcal{D}_{\zeta}[\mathcal{A}^-](x,t) = L[\zeta(x,t)]   
\end{equation*}
for a translation-invariant linear operator $L$ that only depends on $x$ and $t$. This is not as restrictive as it initially sounds. For example, both the Laplacian operator and the convolution operator are linear and translation invariant. The presence of phenotypic clustering and polymorphisms can be analyzed by examining the power spectrum of $\Tilde{P}_{0}(\zeta,s)$ over the trait space, which is precisely what we will do.

As mentioned before, we assume that $\zeta$, and $\mathcal{A}^{+}(x|\psi)$ admit the Fourier basis representations:
\begin{equation}
\label{fourier_representations_functions}
\begin{aligned}
\zeta(x,t) &= \sum\limits_{k=-\infty}^{\infty}e^{ikx}\zeta_k(t) \ \ ; \ \ \zeta_k(t) = \int\limits_{\mathcal{T}}\zeta(x,t)e^{-ikx}dx\\
\mathcal{A}^{+}(x|\psi) &= \sum\limits_{k=-\infty}^{\infty}e^{ikx}A_k(t) \ \ ; \ \ A_k(t) = \int\limits_{\mathcal{T}}\mathcal{A}^{+}(x|\psi)e^{-ikx}dx
\end{aligned}
\end{equation}
In this case, the functional derivative operator obeys:
\begin{equation}
\label{fourier_representations_derivative}
    \frac{\delta}{\delta \zeta(x)} = \sum\limits_{k=-\infty}^{\infty}e^{-ikx}\frac{\partial}{\partial \zeta_k}
\end{equation}
and since $L$ is linear and translation-invariant, we also have the relation\footnote{This is because $\exp(ikx)$ acts as an eigenfunction for translation invariant linear operators, and therefore, for any function $\varphi = \sum\varphi_k\exp(ikx)$, we have the relation $L[\varphi] = L[\sum\varphi_k\exp(ikx)]=\sum\varphi_kL[\exp(ikx)]=\sum\varphi_kL_k\exp(ikx)$, where $L_k$ is the eigenvalue of $L$ associated with the eigenfunction $\exp(ikx)$. It is helpful to draw the analogy with eigenvectors of matrices and view $L_k\varphi_k$ as the projection of $L[\varphi]$ along the $k$th eigenvector $e_k = \exp(ikx)$.}:
\begin{equation}
\label{fourier_representation_linear_operator}
    L[\zeta] = \sum\limits_{k=-\infty}^{\infty}L_{k}\zeta_ke^{ikx}
\end{equation}
where 
\begin{equation*}
    L_k = e^{-ikx}L[e^{ikx}]
\end{equation*}
Lastly, by definition of Fourier modes, we have, for any differentiable real function $F$ and any fixed time $t > 0$:
\begin{equation}
\label{fourier_mode_relation}
\frac{\partial}{\partial \zeta_j(t)}F(\zeta_i(t)) = \delta_{ij}F'(\zeta_j(t))
\end{equation}
where $\delta_{ij}$ is the Kronecker delta symbol.
Using \eqref{fourier_representations_functions}, \eqref{fourier_representations_derivative}, and \eqref{fourier_representation_linear_operator} in \eqref{functional_WNE_for_App}, we get, for the first term of the RHS:
\begin{gather}
-\int\limits_{\mathcal{T}}\frac{\delta}{\delta \zeta(x)}\left\{L[\zeta(x,t)]P(\zeta,t)\right\}dx\nonumber\\
= -\int\limits_{\mathcal{T}}\sum\limits_{k}e^{-ikx}\frac{\partial}{\partial \zeta_k}\{\sum\limits_{n}e^{inx}L_n\zeta_nP\}dx\nonumber\\
= -\int\limits_{\mathcal{T}}\sum\limits_{k}\sum\limits_{n}e^{-i(k-n)x}\frac{\partial}{\partial \zeta_k}\{L_n\zeta_nP\}dx\nonumber\\
= -2\pi\sum\limits_{k}L_{k}\frac{\partial}{\partial \zeta_k}\{\zeta_kP\}\label{fourier_FPE_first_term}
\end{gather}
and for the second:
\begin{gather}
\int\limits_{\mathcal{T}}\sum\limits_{k}e^{ikx}A_k\left(\sum\limits_{m}\sum\limits_{n}e^{-i(m+n)x}\frac{\partial}{\partial \zeta_m}\frac{\partial}{\partial \zeta_n}P\right)dx\nonumber\\
= \int\limits_{\mathcal{T}}\sum\limits_{k}\sum\limits_{m}\sum\limits_{n}e^{i(k-m-n)x}A_k\frac{\partial}{\partial \zeta_m}\frac{\partial}{\partial \zeta_n}\{P\}dx\nonumber\\
= 2\pi\sum\limits_{m}\sum\limits_{n}A_{m+n}\frac{\partial}{\partial \zeta_m}\frac{\partial}{\partial \zeta_{n}}\{P\}\label{fourier_FPE_second_term}
\end{gather}
Substituting \eqref{fourier_FPE_first_term} and \eqref{fourier_FPE_second_term} into \eqref{functional_WNE_for_App}, we see that the Fokker-Planck equation \eqref{functional_WNE_for_App} in Fourier space reads:
\begin{equation}
\label{fourier_FPE}
\frac{\partial P}{\partial t} = -2\pi\sum\limits_{k}L_{k}\frac{\partial}{\partial \zeta_k}\{\zeta_kP\} + \pi\sum\limits_{m}\sum\limits_{n}A_{m+n}\frac{\partial}{\partial \zeta_m}\frac{\partial}{\partial \zeta_{n}}\{P\}
\end{equation}
It is important to remember that since $\zeta(x,t)$ is a stochastic process, $\zeta_i$ is really a stochastic process and thus $\zeta_i(t)$ is actually shorthand for the random variable $(\zeta_i)_{t}(\omega)$, where $\omega$ is a sample path in the Fourier dual of our original probability space. Multiplying both sides of \eqref{fourier_FPE} by $\zeta_r$ and integrating over the probability space to obtain expectation values, we see that
\begin{align}
\frac{d}{dt}\mathbb{E}[\zeta_r] &= - 2\pi \sum\limits_{k}\int\zeta_rL_k\frac{\partial}{\partial \zeta_k}\{\zeta_k P\}d\omega + \pi\sum\limits_{m}\sum\limits_{n}A_{m+n}\int\zeta_r\frac{\partial}{\partial \zeta_m}\frac{\partial}{\partial \zeta_{n}}(P)d\omega\nonumber\\
&=  2\pi \sum\limits_{k}L_k\int\zeta_k\frac{\partial \zeta_r}{\partial \zeta_k}Pd\omega + \pi\sum\limits_{m}\sum\limits_{n}A_{m+n}\int\frac{\partial^2 \zeta_r}{\partial \zeta_m\partial \zeta_{n}}Pd\omega\nonumber\\
&=  2\pi L_{r}\mathbb{E}[\zeta_r]\label{fourier_mode_mean}
\end{align}
where we have used integration by parts and neglected the boundary term in the second step (assuming once again that $P$ decays rapidly enough near the boundaries that this is doable), and then used \eqref{fourier_mode_relation} to arrive at the final expression. Similarly, multiplying \eqref{fourier_FPE} by $\zeta_r\zeta_s$, integrating over the probability space and using integration by parts, we get:
\begin{align}
\frac{d}{dt}\mathbb{E}[\zeta_r\zeta_s] &= 2\pi \sum\limits_{k}L_{k}\int\zeta_kP\frac{\partial}{\partial \zeta_k}\{\zeta_r\zeta_s\}d\omega + \pi\sum\limits_{m}\sum\limits_{n}A_{m+n}\int\limits_{-\infty}^{\infty}P\frac{\partial}{\partial \zeta_m}\frac{\partial}{\partial \zeta_{n}}\{\zeta_r\zeta_s\}d\omega\nonumber\\
&= 2\pi (L_{r} + L_{s})\mathbb{E}[\zeta_r\zeta_s] + \pi (A_{2r}+A_{2s})\label{fourier_mode_covariance}
\end{align}
At the stationary state, the LHS must be zero by definition, and we must therefore have, for every $r,s \in \mathbb{Z}$,:
\begin{equation}
\label{fourier_mode_covariance_stationary}
\mathbb{E}[\zeta_r\zeta_s] = -   \frac{A_{2r}+A_{2s}}{2(L_{r}+L_{s})}
\end{equation}
Recall that the Fourier modes of any real function $\varphi$ must satisfy $\varphi_{-r} = \overline{\varphi}_r$. Since $\zeta$, $A$ and $L$ are all real, we can substitute $s=-r$ in equation \eqref{fourier_mode_covariance_stationary} to obtain the autocovariance relation:
\begin{equation}
\label{fourier_mode_autocovariance}
\mathbb{E}[|\zeta_r|^2] =- \frac{\mathrm{Re}(A_{2r})}{2\mathrm{Re}(L_{r})}
\end{equation}

The presence of phenotypic clustering can be detected using the `spatial covariance' of our original process $\phi$, defined as~\citep{rogers_demographic_2012}:
\begin{equation}
\label{spatial_covariance_defn}
\Xi[x] = m(\mathcal{T})\int\limits_{\mathcal{T}}\mathbb{E}[\phi_{\infty}(x)\phi_{\infty}(y-x)]dy
\end{equation}
where $\phi_{\infty}$ is the stationary state distribution of $\{\phi_t\}_{t}$ and $m$ is the Lebesgue measure. We can use a spatial analogue of the Wiener-Khinchin theorem to calculate:
\begin{equation}
\label{spatial_covariance_zeta}
\Xi[x] = m(\mathcal{T})\left[\int\limits_{\mathcal{T}}\psi_{\infty}(x)\psi_{\infty}(y-x)dy + \frac{1}{K}\sum\limits_{r=-\infty}^{\infty}\mathbb{E}[|\zeta_r|^2]e^{irx}\right]
\end{equation}
where the expectations in the second term are for the stationary state. A flat $\Xi[x]$ indicates that there are no clusters, and peaks indicate the presence of clusters.

\section{An example for quantitative traits: The quantitative logistic equation}
Recall the birth and death functionals given by \eqref{Rogers_logistic_BD}. That is, the functionals
\begin{equation}
\begin{aligned}
b(x|\nu) &= r\int\limits_{\mathcal{T}}m(x,y)\nu(y)dy; \ m(x,y) = \exp\left(\frac{-(x-y)^2}{\sigma_{m}^{2}}\right)\\
d(x|\nu) &= \frac{\nu(x)}{Kn(x)}\int\limits_{\mathcal{T}}\alpha(x,y)\nu(y)dy; \ \alpha(x,y) = \exp\left(\frac{-(x-y)^2}{\sigma_{\alpha}^{2}}\right)
\end{aligned}
\end{equation}
corresponding to an asexual population having a constant (per-capita) birth rate $r$ and mutations controlled by a Gaussian kernel $m(x,y)$. The death rate is density-dependent, mediated by a Gaussian competition kernel $\alpha(x,y)$, and also contains a phenotype-dependent carrying capacity controlled by $n(x)$, scaled by a constant $K$. The biological interpretation of the death rate is through ecological specialization for limiting resources - Individuals have different intrinsic advantages (controlled by $n(x)$), and experience greater competition from conspecifics that are closer to them in phenotype space (controlled by $\alpha(x,y)$). In terms of the scaled variable $\phi = K\nu$, these functions read:
\begin{equation}
\label{Rogers_logistic_BD_scaled}
\begin{aligned}
b_K(x|\phi) &= \frac{1}{K}b(x|\nu) = \frac{1}{K}\left( r\int\limits_{\mathcal{T}}m(x,y)K\phi(y)dy\right)\\
    d_K(x|\phi) &= \frac{1}{K}d(x|\nu) =  \frac{1}{K}\left(\frac{K\phi(x)}{Kn(x)}\int\limits_{\mathcal{T}}\alpha(x,y)K\phi(y)dy\right)
\end{aligned}
\end{equation}
Thus, using equation \eqref{deterministic_traj}, the deterministic trajectory becomes:
\begin{equation}
\label{Rogers_logistic_BD_deterministic}
\frac{\partial \psi}{\partial t}(x,t) = r\int\limits_{\mathcal{T}}m(x,y)\psi(y,t)dy-\frac{1}{n(x)}\psi(x,t)\int\limits_{\mathcal{T}}\alpha(x,y)\psi(y,t)dy
\end{equation}
Note that if we employ the change of variables $\Psi = K\psi$ to go back from $\mathcal{M}_{K}$ (\textit{i.e} $\phi^{(t)}$) to $\mathcal{M}$ (\textit{i.e} $\nu^{(t)}$), we recover the familiar quantitative logistic equation as the deterministic limit:
\begin{align*}
\frac{\partial \Psi}{\partial t}(x,t) &= r\int\limits_{\mathcal{T}}m(x,y)\Psi(y,t)dy-\frac{\Psi(x,t)}{Kn(x)}\int\limits_{\mathcal{T}}\alpha(x,y)\Psi(y,t)dy \\
&\approx r\Psi(x,t) -\frac{\Psi(x,t)}{K(x)}\int\limits_{\mathcal{T}}\alpha(x,y)\Psi(y,t)dy + D_m\nabla^2_{x}\Psi(x,t)
\end{align*}
where $K(x) = Kn(x)$ is the carrying capacity experienced by an individual of phenotype $x$, and $D_m = r \sigma_m^2/2$ measures the `diffusion rate' of the population in trait space.

We can also calculate $\mathcal{D}_{\zeta}[\mathcal{A}^-]$ as
\begin{align*}
\mathcal{D}_{\zeta}[\mathcal{A}^-] &= \frac{d}{d\epsilon}\left( r\int\limits_{\mathcal{T}} m(x,y)(\psi(y)+\epsilon\zeta(y))dy - \frac{\psi(x)+\epsilon\zeta(x)}{n(x)}\int\limits_{\mathcal{T}}\alpha(x,y)(\psi(y)+\epsilon\zeta(y))dy\right) \biggl{|}_{\epsilon = 0}\\
&= r\int\limits_{\mathcal{T}}m(x,y)\zeta(y)dy - \frac{1}{n(x)}\left(\psi(x)\int\limits_{\mathcal{T}}\alpha(x,y)\zeta(y)dy + \zeta(x)\int\limits_{\mathcal{T}}\alpha(x,y)\psi(y)dy\right)
\end{align*}
Using this in equation \eqref{spatial_covariance_zeta},
\cite{rogers_demographic_2012} (and later~\cite{rogers_modes_2015}) have shown that the contribution of demographic stochasticity can lead to inhibition of branching, and thus, while the population undergoes infinitely many branching events in the infinite population prediction, this does not happen for finite populations.
\myfig{0.8}{figures/App_dem_stoch_effects.png}{\textbf{Effect of population size on evolutionary branching}. Two different realizations of the system \eqref{Rogers_logistic_BD} with $n(x) = \exp(-x^2/\sigma_{K}^2)$. Simulation parameters are $\sigma_{K} = 1.9, \sigma_{\alpha} = 0.7, \sigma^2_{m} = 0.05$ for \textbf{top:} $K = 1000$ and \textbf{bottom:} $K = 10000$. Each point represents an individual. Note that the model on top remains monomorphic whereas the model on the bottom exhibits evolutionary branching, where an initially monomorphic population evolves to become dimorphic.}{fig_dem_stoch}
An alternative `moment-based' method that avoids moving to Fourier space has also been used to study this phenomenon of evolutionary branching and clustering in finite populations~\citep{wakano_evolutionary_2013,debarre_evolutionary_2016}. These studies use the equation we derived in section \ref{sec_fun_theorems_var} for the variance of the trait in the population and compute the conditions required for the variance to `explode' (Equation A.23 in~\cite{debarre_evolutionary_2016} is exactly equivalent to equation \eqref{nD_stochastic_Price_variance} for their choice of functional forms upon converting their change in variance to an infinitesimal rate of change \emph{i.e.} a derivative). The method itself is relatively straightforward in principle (complications arise if the particular models are complicated) and I therefore do not explore it further in this thesis, but the broad results of such moment-based approaches is in agreement with the predictions made from the spectral methods employed in~\cite{rogers_demographic_2012} and~\cite{rogers_modes_2015}.

It is left as an exercise for the reader to verify by the same steps that if we instead have the birth rate functional $b(x|\phi) = \lambda\int m(x,y)\phi(y)dy$ (with $m(x,y)$ as defined in \eqref{Rogers_logistic_BD}) and the death rate functional $d(x|\phi) = \phi(x)\left(\mu+(\lambda-\mu)\phi(x)/K\right)$, the infinite-population limit yields the famous Fisher-KPP equation with growth rate $r=\lambda-\mu$ and diffusion constant $D = \lambda \sigma_m^2/2$.

\cleardoublepage
\phantomsection
\printbibliography
\addcontentsline{toc}{part}{References}
\end{document}
