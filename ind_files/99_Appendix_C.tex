Let $\sigma^2_{f}$ denote the statistical variance of a type-level quantity, defined as:
\begin{equation}
    \sigma^2_{f} \coloneqq \overline{(f^2)} - (\overline{f})^2
\end{equation}
where $\overline{X}$ is the statistical mean value defined by \eqref{nD_mean}. By the product rule, we have
\begin{equation}
\label{prod_rule_for_variances}
\frac{d\sigma^2_{f}}{dt} = 2\overline{f\frac{\partial f}{\partial t}} + \sum\limits_{i=1}^{m}f_i^2\frac{dp_i}{dt} - \frac{d}{dt}(\overline{f}^2)
\end{equation}
We will evaluate the RHS term by term. The first term is as simplified as can be. For the second term, we can substitute $dp_i$ from \eqref{nD_stochastic_RM} and use the same steps used in going from \eqref{nD_replicator_mutator} to \eqref{nD_Price} to write
\begin{equation}
\label{2nd_term_for_variances}
\begin{aligned}
\sum\limits_{i=1}^{m}f_i^2dp_i &= \textrm{Cov}(w,f^2)dt - \frac{1}{KN_K}\textrm{Cov}(\tau,f^2)dt\\
&+ \mu\left(1-\frac{1}{KN_K(t)}\right)\left(\sum\limits_{i=1}^{m}f^2_iQ_i(\mathbf{p}) - \overline{f^2}\sum\limits_{i=1}^{m}Q_i(\mathbf{p})\right)dt \\
&+\frac{1}{\sqrt{K}N_{K}(t)}\left(\sum\limits_{i=1}^{m}f^2_i\sqrt{A_i^+}dB_{t}^{(i)} - \overline{f^2}\sum\limits_{i=1}^{m}\sqrt{A_i^+}dB_{t}^{(i)}\right)
\end{aligned}
\end{equation}
For the third term, we need to use It\^{o}'s formula. Here, the relevant version of It\^{o}'s formula is the one-dimensional version of \eqref{nD_Ito_formula}. Given a one-dimensional process $dX_t = S(X_t)dt + \sum D_j(X_t)dB^{(j)}_t$ with $S, D_j$ being suitable real functions and each $B^{(j)}_t$ being an independent Wiener process, It\^{o}'s formula says that given any $C^2(\mathbb{R})$ function $g(x)$, we have the relation:
\begin{equation}
\label{1D_Ito_formula}
dg(X_t) = \left(S(X_t)g'(X_t) + \frac{g''(X_t)}{2}\sum\limits_{j}D_j^2(X_t)\right)dt + \sum\limits_{j}D_j(X_t)g'(X_t)dB^{(j)}_t 
\end{equation}
In our case, we have a one-dimensional process for the mean value $d\overline{f}$ of the type level quantity, and the $C^2(\mathbb{R})$ function $g(x) = x^2$. It\^{o}'s formula thus says that the third term of \eqref{prod_rule_for_variances} is given by:
\begin{equation}
\label{3rd_term_for_variances}
d(\overline{f}^2) = \left(2\overline{f}S(X_t) + \sum\limits_{j}D_j^2(X_t)\right)dt + \sum\limits_{j}2\overline{f}D_j(X_t)dB^{(j)}_t
\end{equation}
where the relevant functions $S$ and $D_j$ can be read off from \eqref{nD_stochastic_Price}. Since the $dB$ terms are unwieldy and do not contribute to the expected dynamics, we will denote the contribution of all the $dB_t$ terms collectively by $dB_{\sigma^2_{f}}$ and not explicitly calculate these terms. We also note that the covariance operator is a bilinear form, \emph{i.e.} given any three quantities $X$, $Y$ and $Z$ and any constant $a \neq 0$, we have the relations:
\begin{align*}
\textrm{Cov}(aX,Y) &= a\textrm{Cov}(X,Y) = \textrm{Cov}(X,aY)\\
\textrm{Cov}(X,Y+Z) &= \textrm{Cov}(X,Y)+\textrm{Cov}(X,Z)
\end{align*}
Substituting equations \eqref{2nd_term_for_variances} and \eqref{3rd_term_for_variances} into equation \eqref{prod_rule_for_variances} and using this property of covariances, we obtain:
\begin{equation}
\label{intermediate_1_for_variances}
\begin{aligned}
d\sigma^2_{f} &= \textrm{Cov}(w,f^2 - 2\overline{f}f)dt - \frac{1}{KN_K}\left(\textrm{Cov}(\tau,f^2 - 2\overline{f}f)\right)dt + 2\left(\overline{f\frac{\partial f}{\partial t}} - \overline{f}\overline{\left(\frac{\partial f}{\partial t}\right)}\right)dt\\
&+ \mu\left(1-\frac{1}{KN_K(t)}\right)\left(\sum\limits_{i=1}^{m}(f^2_i - 2\overline{f}f_i)Q_i(\mathbf{p}) - (\overline{f^2}-2\overline{f}^2)\sum\limits_{i=1}^{m}Q_i(\mathbf{p})\right)dt\\
&+ \frac{1}{KN^2_{K}(t)}\left(\sum\limits_{i=1}^{m}f^2_iA_i^+ - \overline{f}^2\sum\limits_{i=1}^{m}A_i^+\right)dt\\
&+ dB_{\sigma^2_{f}}
\end{aligned}
\end{equation}
Now, we note that
\begin{align}
\frac{1}{N_K}A_i^+ &= \frac{1}{N_K}\left(\tau_ix_i + \mu Q_i(\mathbf{x})\right)\\
&= \tau_ip_i + \mu Q_i(\mathbf{p})
\end{align}
and thus the third line of \eqref{intermediate_1_for_variances} is
\begin{align}
\frac{1}{KN^2_{K}(t)}\left(\sum\limits_{i=1}^{m}f^2_iA_i^+ - \overline{f}^2\sum\limits_{i=1}^{m}A_i^+\right)dt &= \frac{1}{KN_{K}}\sum\limits_{i=1}^{m}f_i^2\left(\tau_ip_i + \mu Q_i(\mathbf{p})\right) - \frac{\overline{f}^2}{KN_{K}}\sum\limits_{i=1}^{m}\left(\tau_ip_i + \mu Q_i(\mathbf{p})\right)\\
&= \frac{1}{KN_K}\sum\limits_{i=1}^{m}\left(f_i^2 - \overline{f}^2\right)\left(\tau_ip_i + \mu Q_i(\mathbf{p})\right)\\
&= \frac{1}{KN_K}\left(\textrm{Cov}(\tau,f^2)+\mu \sum\limits_{i=1}^{m}\left(f_i^2 - \overline{f}^2\right)Q_i(\mathbf{p})\right)
\end{align}
Substituting this into \eqref{intermediate_1_for_variances} and using $M_{\sigma^2_f}(\mathbf{p},N_K)$ to denote the contributions of all the mutational terms (\emph{i.e.} all terms with a $\mu$ factor) for notational brevity, we obtain
\begin{equation}
\begin{aligned}
d\sigma^2_{f} &= \textrm{Cov}(w,f^2 - 2\overline{f}f)dt - \frac{1}{KN_K}\left(\textrm{Cov}(\tau,2f^2 - 2\overline{f}f)\right)dt\\
&+ 2\textrm{Cov}\left(f, \frac{\partial f}{\partial t}\right)dt + M_{\sigma^2_f}(\mathbf{p},N_K)dt + dB_{\sigma^2_{f}}
\end{aligned}
\end{equation}
We can now complete the square inside the covariance terms of the first line of the RHS by writing $f^2 - 2\overline{f}f = (f - \overline{f})^2 - \overline{f}^2$ to obtain
\begin{equation}
\label{intermediate_2_for_variances}
\begin{aligned}
d\sigma^2_{f} &= \left[ \ \textrm{Cov}\left(w,(f - \overline{f})^2\right)-\textrm{Cov}\left(w, {\left(\overline{f}\right)}^2\right) \ \right]dt\\[12pt]
&- \frac{1}{KN_K}\left[ \ \textrm{Cov}\left(\tau,(f - \overline{f})^2\right) + \textrm{Cov}\left(\tau,f^2 - {\left(\overline{f}\right)}^2\right) \ \right]dt\\[12pt]
& + 2\textrm{Cov}\left(f, \frac{\partial f}{\partial t}\right)dt + M_{\sigma^2_f}(\mathbf{p},N_K)dt + dB_{\sigma^2_{f}}
\end{aligned}
\end{equation}
To simplify the covariance terms of the first line of the RHS, we observe that
\begin{align}
\textrm{Cov}\left(w, {\left(\overline{f}\right)}^2\right) &= \overline{\left(w{\left(\overline{f}\right)}^2\right)} - \overline{w}\overline{\left({\left(\overline{f}\right)}^2\right)}\nonumber\\
&= {\left(\overline{f}\right)}^2\sum\limits_{i=1}^{m}w_ip_i - \overline{w}{\left(\overline{f}\right)}^2\sum\limits_{i=1}^{m}p_i\nonumber\\
&= {\left(\overline{f}\right)}^2\overline{w} - \overline{w}{\left(\overline{f}\right)}^2 = 0\label{cov_term_1_for_variances}
\end{align}
and
\begin{align}
\textrm{Cov}\left(\tau,f^2 - {\left(\overline{f}\right)}^2\right) &= \overline{\tau\left(f^2 - {\left(\overline{f}\right)}^2\right)} - \overline{\tau}\left(\overline{f^2 - {\left(\overline{f}\right)}^2}\right)\nonumber\\
&= \overline{\tau f^2} - \overline{\tau}{\left(\overline{f}\right)}^2  - \overline{\tau}\overline{f^2} + \overline{\tau}{\left(\overline{f}\right)}^2\nonumber\\
&=\overline{\tau f^2} - \overline{\tau}\overline{f^2} = \textrm{Cov}(\tau,f^2)\label{cov_term_2_for_variances}
\end{align}
Substituting \eqref{cov_term_1_for_variances} and \eqref{cov_term_2_for_variances} into \eqref{intermediate_2_for_variances}, we thus arrive at:
\begin{equation}
\label{intermediate_3_for_variances}
\begin{aligned}
d\sigma^2_{f} &= \textrm{Cov}\left(w,(f - \overline{f})^2\right)dt\\
&- \frac{1}{KN_K}\left[ \ \textrm{Cov}\left(\tau,(f - \overline{f})^2\right) + \textrm{Cov}(\tau,f^2) \ \right]dt\\
& + 2\textrm{Cov}\left(f, \frac{\partial f}{\partial t}\right)dt + M_{\sigma^2_f}(\mathbf{p},N_K)dt + dB_{\sigma^2_{f}}
\end{aligned}
\end{equation}
Covariance with $(f-\overline{f})^2$ is a measure of covariance with the existing `spread' of $f_i$ around the population mean value $\overline{f}$ and is thus readily understood to represent existing variation. However, the $\textrm{Cov}(\tau,f^2)$ term is difficult to interpret since it depends on the square of $f$, which may not have any direct biological interpretation. We will convert this into a form whose interpretation is more clear
% by repeatedly adding and subtracting a quantity from our expression (these quantities are shown in red below for clarity), noting that this leaves the net value unchanged. We have:
% \begin{align}
% \textrm{Cov}(\tau, f^2) &= \overline{\tau f^2} - \overline{\tau}\overline{f^2}\nonumber\\
% &= \overline{\tau f^2} {\color{red} + \overline{\tau{\left(\overline{f}\right)}^2} - \overline{\tau f \overline{f}}}  - \overline{\tau}\overline{f^2} \nonumber\\
% &= \overline{\tau f^2} + \overline{\tau{\left(\overline{f}\right)}^2} - 2\overline{\tau f \overline{f}}\nonumber\\
% &= \overline{\tau\left(f-\overline{f}\right)^2}\nonumber\\
% &= \overline{\tau\left(f-\overline{f}\right)^2} {\color{red}- \overline{\tau}{\left(\overline{f}\right)}^2 + 2 \overline{\tau}\overline{f\overline{f}} - \overline{\tau}{\left(\overline{f}\right)}^2}\nonumber\\
% &= \overline{\tau\left(f-\overline{f}\right)^2} {\color{red} - \overline{\tau}\overline{f^2}} -\overline{\tau}\overline{{\left(\overline{f}\right)}^2} + 2 \overline{\tau}\overline{f\overline{f}} {\color{red} + \overline{\tau}\overline{f^2}} - \overline{\tau}{\left(\overline{f}\right)}^2\nonumber\\
% &=  \overline{\tau\left(f-\overline{f}\right)^2} - \overline{\tau}\overline{\left(f-\overline{f}\right)^2} + \overline{\tau}\left(\overline{f^2} - {\left(\overline{f}\right)}^2\right)\nonumber\\
% &= \textrm{Cov}(\tau, \left(f-\overline{f}\right)^2) + \overline{\tau}\sigma^2_{f}\label{cov_to_var_sub_for_variances}
% \end{align}
% Finally, substituting \eqref{cov_to_var_sub_for_variances} into \eqref{intermediate_3_for_variances}, we see that the rate of change of the variance of any type-level quantity $f$ in the population satisfies:
% \begin{equation}
% \label{stochastic_Price_variance}
% \begin{aligned}
% d\sigma^2_{f} &= \textrm{Cov}\left(w,(f - \overline{f})^2\right)dt - \frac{1}{KN_K}\left[ \ 2\textrm{Cov}\left(\tau,(f - \overline{f})^2\right) + \overline{\tau}\sigma^2_{f}(t) \ \right]dt\\[12pt]
% & + 2\textrm{Cov}\left(\frac{\partial f}{\partial t},f\right)dt + M_{\sigma^2_f}(\mathbf{p},N_K)dt + dB_{\sigma^2_{f}}
% \end{aligned}
% \end{equation}
% In the case of one-dimensional quantitative traits, an infinite-dimensional version of \eqref{stochastic_Price_variance} has been rigorously derived \citep{week_white_2021} in a recent paper for the case where $f$ is the phenotypic value of the quantitative trait - Equation (21b) of \cite{week_white_2021} is precisely the $m \to \infty$ version of my equation \eqref{stochastic_Price_variance} when $f$ is the phenotypic value of the quantitative trait under consideration.
by using the fact that $f^2 = (f-\overline{f})^2 - \overline{f}^2 + 2f\overline{f}$. we have
\begin{align}
\textrm{Cov}(\tau, f^2) &= \textrm{Cov}(\tau, (f-\overline{f})^2) + \textrm{Cov}(\tau, - {\left(\overline{f}\right)}^2) + 2f\overline{f}\nonumber\\
&=  \textrm{Cov}(\tau, (f-\overline{f})^2) \ \  - \underbrace{\textrm{Cov}(\tau, {\left(\overline{f}\right)}^2)}_{\substack{ =0 \\ {\text{by same argument as in \eqref{cov_term_1_for_variances}}}}} + \ \  2\overline{f}\textrm{Cov}(\tau, f)\label{cov_term_3_for_variances}
\end{align}
and thus, substituting \eqref{cov_term_3_for_variances} into \eqref{intermediate_3_for_variances},  we see that the rate of change of the variance of any type-level quantity $f$ in the population satisfies:
\begin{equation}
\label{stochastic_Price_variance}
\begin{aligned}
d\sigma^2_{f} &= \textrm{Cov}\left(w,(f - \overline{f})^2\right)dt - \frac{2}{KN_K}\left[ \ \overline{f}\textrm{Cov}(\tau, f) +  \textrm{Cov}\left(\tau,(f - \overline{f})^2\right) \ \right]dt\\[12pt]
& + 2\textrm{Cov}\left(f,\frac{\partial f}{\partial t}\right)dt + M_{\sigma^2_f}(\mathbf{p},N_K)dt + dB_{\sigma^2_{f}}
\end{aligned}
\end{equation}.