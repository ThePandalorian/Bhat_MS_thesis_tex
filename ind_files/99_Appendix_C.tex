Let $\sigma^2_{f}$ denote the statistical variance of a type-level quantity, defined as:
\begin{equation}
    \sigma^2_{f} \coloneqq \overline{(f^2)} - (\overline{f})^2
\end{equation}
where $\overline{X}$ is the statistical mean value defined by \eqref{nD_mean}. By the product rule, we have
\begin{equation}
\label{prod_rule_for_variances}
\frac{d\sigma^2_{f}}{dt} = 2\overline{f\frac{\partial f}{\partial t}} + \sum\limits_{i=1}^{m}f_i^2\frac{dp_i}{dt} - \frac{d}{dt}(\overline{f}^2)
\end{equation}
We will evaluate the RHS term by term. The first term is as simplified as can be. For the second term, we can substitute $dp_i$ from \eqref{nD_stochastic_RM} and use the same steps used in going from \eqref{nD_replicator_mutator} to \eqref{nD_Price} to write
\begin{equation}
\label{2nd_term_for_variances}
\begin{aligned}
\sum\limits_{i=1}^{m}f_i^2dp_i &= \textrm{Cov}(w,f^2)dt - \frac{1}{KN_K}\textrm{Cov}(\tau,f^2)dt\\
&+ \mu\left(1-\frac{1}{KN_K(t)}\right)\left(\sum\limits_{i=1}^{m}f^2_iQ_i(\mathbf{p}) - \overline{f^2}\sum\limits_{i=1}^{m}Q_i(\mathbf{p})\right)dt \\
&+\frac{1}{\sqrt{K}N_{K}(t)}\left(\sum\limits_{i=1}^{m}f^2_i\sqrt{A_i^+}dB_{t}^{(i)} - \overline{f^2}\sum\limits_{i=1}^{m}\sqrt{A_i^+}dB_{t}^{(i)}\right)
\end{aligned}
\end{equation}
For the third term, we need to use It\^{o}'s formula. Here, the relevant version of It\^{o}'s formula is the one-dimensional version of \eqref{nD_Ito_formula}. Given a one-dimensional process $dX_t = S(X_t)dt + \sum D_j(X_t)dB^{(j)}_t$ with $S, D_j$ being suitable real functions and each $B^{(j)}_t$ being an independent Wiener process, It\^{o}'s formula says that given any $C^2(\mathbb{R})$ function $g(x)$, we have the relation:
\begin{equation}
\label{1D_Ito_formula}
dg(X_t) = \left(S(X_t)g'(X_t) + \frac{g''(X_t)}{2}\sum\limits_{j}D_j^2(X_t)\right)dt + \sum\limits_{j}D_j(X_t)g'(X_t)dB^{(j)}_t 
\end{equation}
In our case, we have a one-dimensional process for the mean value $d\overline{f}$ of the type level quantity, and the $C^2(\mathbb{R})$ function $g(x) = x^2$. It\^{o}'s formula thus says that the third term of \eqref{prod_rule_for_variances} is given by:
\begin{equation}
\label{3rd_term_for_variances}
d(\overline{f}^2) = \left(2\overline{f}S(X_t) + \sum\limits_{j}D_j^2(X_t)\right)dt + \sum\limits_{j}2\overline{f}D_j(X_t)dB^{(j)}_t
\end{equation}
where the relevant functions $S$ and $D_j$ can be read off from \eqref{nD_stochastic_Price}. Since the $dB$ terms are unwieldy and do not contribute to the expected dynamics, we will denote the contribution of all the $dB_t$ terms collectively by $dB_{\sigma^2_{f}}$ and not explicitly calculate these terms. We also note that the covariance operator is a bilinear form, \emph{i.e.} given any three quantities $X$, $Y$ and $Z$ and any constant $a \neq 0$, we have the relations:
\begin{align*}
\textrm{Cov}(aX,Y) &= a\textrm{Cov}(X,Y) = \textrm{Cov}(X,aY)\\
\textrm{Cov}(X,Y+Z) &= \textrm{Cov}(X,Y)+\textrm{Cov}(X,Z)
\end{align*}
Substituting equations \eqref{2nd_term_for_variances} and \eqref{3rd_term_for_variances} into equation \eqref{prod_rule_for_variances} and using this property of covariances, we obtain:
\begin{equation}
\label{intermediate_1_for_variances}
\begin{aligned}
d\sigma^2_{f} &= \textrm{Cov}(w,f^2 - 2\overline{f}f)dt - \frac{1}{KN_K}\left(\textrm{Cov}(\tau,f^2 - 2\overline{f}f)\right)dt + 2\left(\overline{f\frac{\partial f}{\partial t}} - \overline{f}\overline{\left(\frac{\partial f}{\partial t}\right)}\right)dt\\
&+ \mu\left(1-\frac{1}{KN_K(t)}\right)\left(\sum\limits_{i=1}^{m}(f^2_i - 2\overline{f}f_i)Q_i(\mathbf{p}) - (\overline{f^2}-2\overline{f}^2)\sum\limits_{i=1}^{m}Q_i(\mathbf{p})\right)dt\\
&- \frac{1}{KN^2_{K}(t)}\left(\sum\limits_{i=1}^{m}(f_i - \overline{f})^2A_i^+\right)dt\\
&+ dB_{\sigma^2_{f}}
\end{aligned}
\end{equation}
Now, we note that
\begin{align}
\frac{1}{N_K}A_i^+ &= \frac{1}{N_K}\left(\tau_ix_i + \mu Q_i(\mathbf{x})\right)\\
&= \tau_ip_i + \mu Q_i(\mathbf{p})
\end{align}
and thus the third line of \eqref{intermediate_1_for_variances} is
\begin{align}
\frac{1}{KN^2_{K}(t)}\left(\sum\limits_{i=1}^{m}(f_i-\overline{f})^2A_i^+\right)dt &= \frac{1}{KN_{K}}\sum\limits_{i=1}^{m}(f_i-\overline{f})^2\left(\tau_ip_i + \mu Q_i(\mathbf{p})\right)\\
&= \frac{1}{KN_K}\sum\limits_{i=1}^{m}\left(f_i - \overline{f}\right)^2\left(\tau_ip_i + \mu Q_i(\mathbf{p})\right)\\
&= \frac{1}{KN_K}\left(\overline{\left(f - \overline{f}\right)^2\tau}+\mu \sum\limits_{i=1}^{m}\left(f_i - \overline{f}\right)^2Q_i(\mathbf{p})\right)\\
&= \frac{1}{KN_K}\left(\textrm{Cov}(\tau,\left(f - \overline{f}\right)^2) + \overline{\tau}\overline{\left(f - \overline{f}\right)^2} +\mu \sum\limits_{i=1}^{m}\left(f_i - \overline{f}\right)^2Q_i(\mathbf{p})\right)\\
&= \frac{1}{KN_K}\left(\textrm{Cov}(\tau,\left(f - \overline{f}\right)^2) + \overline{\tau}\sigma^2_{f} +\mu \sum\limits_{i=1}^{m}\left(f_i - \overline{f}\right)^2Q_i(\mathbf{p})\right)
\end{align}
where we have used the definition of statistical covariance in the second to last line and used the definition of statistical variance in the last line. Substituting this into \eqref{intermediate_1_for_variances} and using $M_{\sigma^2_f}(\mathbf{p},N_K)$ to denote the contributions of all the mutational terms (\emph{i.e.} all terms with a $\mu$ factor) for notational brevity, we obtain
\begin{equation}
\begin{aligned}
d\sigma^2_{f} &= \textrm{Cov}(w,f^2 - 2\overline{f}f)dt - \frac{1}{KN_K}\left(\textrm{Cov}(\tau,f^2 - 2\overline{f}f) + \textrm{Cov}(\tau,\left(f - \overline{f}\right)^2) + \overline{\tau}\sigma^2_{f}\right)dt\\
&+ 2\textrm{Cov}\left(f, \frac{\partial f}{\partial t}\right)dt + M_{\sigma^2_f}(\mathbf{p},N_K)dt + dB_{\sigma^2_{f}}
\end{aligned}
\end{equation}
We can now complete the square inside the covariance terms of the first line of the RHS by writing $f^2 - 2\overline{f}f = (f - \overline{f})^2 - \overline{f}^2$ to obtain
\begin{equation}
\label{intermediate_2_for_variances}
\begin{aligned}
d\sigma^2_{f} &= \left[ \ \textrm{Cov}\left(w,(f - \overline{f})^2\right)-\textrm{Cov}\left(w, {\left(\overline{f}\right)}^2\right) \ \right]dt\\[12pt]
&- \frac{1}{KN_K}\left[ \ \textrm{Cov}\left(\tau,(f - \overline{f})^2\right) - \textrm{Cov}\left(\tau, {\left(\overline{f}\right)}^2\right) + \textrm{Cov}(\tau,\left(f - \overline{f}\right)^2) + \overline{\tau}\sigma^2_{f} \ \right]dt\\[12pt]
& + 2\textrm{Cov}\left(\frac{\partial f}{\partial t},f\right)dt + M_{\sigma^2_f}(\mathbf{p},N_K)dt + dB_{\sigma^2_{f}}
\end{aligned}
\end{equation}
To simplify the covariance terms of the first line of the RHS, we observe that
\begin{align*}
\textrm{Cov}\left(w, {\left(\overline{f}\right)}^2\right) &= \overline{\left(w{\left(\overline{f}\right)}^2\right)} - \overline{w}\overline{\left({\left(\overline{f}\right)}^2\right)}\\
&= {\left(\overline{f}\right)}^2\sum\limits_{i=1}^{m}w_ip_i - \overline{w}{\left(\overline{f}\right)}^2\sum\limits_{i=1}^{m}p_i\\
&= {\left(\overline{f}\right)}^2\overline{w} - \overline{w}{\left(\overline{f}\right)}^2 = 0
\end{align*}
and similarly,
\begin{equation*}
\textrm{Cov}\left(\tau, {\left(\overline{f}\right)}^2\right) = 0
\end{equation*}
and thus, using this in \eqref{intermediate_2_for_variances},  we see that the rate of change of the variance of any type-level quantity $f$ in the population satisfies:
\begin{equation}
\label{stochastic_Price_variance}
\begin{aligned}
d\sigma^2_{f} &= \textrm{Cov}\left(w,(f - \overline{f})^2\right)dt - \frac{1}{KN_K}\left[ \ \overline{\tau}\sigma^2_{f} +  2\textrm{Cov}\left(\tau,(f - \overline{f})^2\right) \ \right]dt\\[12pt]
& + 2\textrm{Cov}\left(\frac{\partial f}{\partial t},f\right)dt + M_{\sigma^2_f}(\mathbf{p},N_K)dt + dB_{\sigma^2_{f}}
\end{aligned}
\end{equation}
In the case of one-dimensional quantitative traits, an infinite-dimensional version of \eqref{stochastic_Price_variance} has been rigorously derived \citep{week_white_2021} in a recent paper for the case where $f$ is the phenotypic value of the quantitative trait - Equation (21b) of \cite{week_white_2021} is precisely the $m \to \infty$ version of my equation \eqref{stochastic_Price_variance} when $f$ is the phenotypic value of the quantitative trait under consideration.\\
Once again, terms of equation \eqref{stochastic_Price_variance} lend themselves to straightforward biological interpretation. The quantity $(f-\overline{f})^2$ is a measure of the `spread' of the quantity $f$ around the population mean value $\overline{f}$, and thus covariance with $(f-\overline{f})^2$ is a measure of disruptive (\emph{i.e.} diversifying) selection: If values of $f$ that are further away from the mean value are associated with higher fitness, the variance of that quantity in the population increases. The $\overline{\tau}\sigma^2_{f}$ term represents a loss of diversity due to stochastic extinctions (i.e. demographic stochasticity). To see this, let us consider the case in which each $f_i$ is simply a label or mark arbitrarily assigned to individuals in the population at the start of an experiment/observation and subsequently passed on to offspring - For example, a neutral tag in the genome of the organism in an area that does not mutate. Let us set $\mu = 0$ so that the labels cannot change between parents to offspring. This means that we have $M_{\sigma^2_f}(\mathbf{p},N_K) \equiv 0$. Further, since the labels are arbitrary and have no effect whatsoever on the biology of the organisms, we have $\textrm{Cov}\left(w,(f - \overline{f})^2\right) \equiv \textrm{Cov}\left(\tau,(f - \overline{f})^2\right) \equiv 0$. Since the labels do not change over time, we also have $\textrm{Cov}\left(\frac{\partial f}{\partial t},f\right) = 0$. From \eqref{stochastic_Price_variance}, we see that in this case, the variance changes as
\begin{equation}
d\sigma^2_f = - \frac{\overline{\tau}\sigma^2_{f}}{KN_K}dt + dB_{\sigma^2_{f}}
\end{equation}
In particular, on taking expectations, we see that the expected variance in the population obeys
\begin{equation}
\frac{d \mathbb{E}[\sigma^2_f]}{dt} = - \left(\mathbb{E}\left[\frac{\overline{\tau}}{KN_K}\right]\right)\mathbb{E}[\sigma^2_{f}]
\end{equation}
We were able to decompose the expectation on the RHS because the label $f$ is completely arbitrary and thus independent of both $\overline{\tau}$ and $N_K(t)$. This is a simple linear ODE and has the solution
\begin{equation}
\mathbb{E}[\sigma^2_f(t)] = \sigma^2_f(0)\exp\left(-\mathbb{E}\left[\frac{\overline{\tau}}{KN_K}\right]\right)
\end{equation}
In this case, demographic stochasticity is expected to lead to an exponential loss of diversity in the population. Populations with higher mean turnover $\overline{\tau}$  and/or lower population size $KN_K$ lose diversity faster, which again should be intuitive since these populations experience more extreme stochastic events and are therefore more likely to experience loss of labels. Note that \emph{which} label is lost is entirely random (since labelling is arbitrary), but nevertheless, labels tend to be stochastically lost until only a small number of labels remain in the population.