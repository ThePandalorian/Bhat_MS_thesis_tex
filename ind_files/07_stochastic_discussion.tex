\epigraph{\justifying Not only is algebraic reasoning exact; it imposes an exactness on the verbal postulates made before algebra can start which is usually lacking in the first verbal formulations of scientific principles.}{J.B.S. Haldane}

One striking feature that repeatedly shows up in our derivations is that finite populations exhibit phenomena that are not visible in infinite population models. Stochastic systems exhibit many interesting and biologically relevant phenomena which cannot be captured in the deterministic limit. For example, in both the stochastic logistic equation \ref{ex_1D_stoch_logistic_BD_eqns} and in two-strategy games with finite population sizes \citep{tao_stochastic_2007}, demographic noise ensures that all populations are guaranteed to go extinct given enough time, even if the deterministic limit predicts a stable state far from extinction. In the case of quantitative traits, demographic noise can hinder adaptive diversification by increasing the time before evolutionary branching occurs \citep{claessen_delayed_2007, wakano_evolutionary_2013, debarre_evolutionary_2016}, causing stochastic extinction of existing evolutionary branches \citep{rogers_demographic_2012, johansson_will_2006}, or preventing branching altogether if the population is too small \citep{rogers_modes_2015, johnson_two-dimensional_2021}. Stochastic systems also routinely exhibit evolution towards attractors that cannot be attained in the deterministic limit \citep{delong_stochasticity_2023}, sometimes even completely reversing the direction of evolution predicted by deterministic dynamics \citep{constable_demographic_2016,mcleod_social_2019}. Since real-life populations are stochastic and finite, it is thus imperative that modellers work with stochastic first-principles models instead of their deterministic limits, lest they risk missing important phenomena that are unique to stochastic systems \citep{black_stochastic_2012,schreiber_does_2022,hastings_transients_2004,shoemaker_integrating_2020}. In the context of our models, we have seen that if we observe the change in trait frequencies instead of the change in densities, finite populations are subject to an additional evolutionary force that vanishes in infinite population models.