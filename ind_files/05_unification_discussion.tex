\epigraph{\justifying Not only is algebraic reasoning exact; it imposes an exactness on the verbal postulates made before algebra can start which is usually lacking in the first verbal formulations of scientific principles.}{J.B.S. Haldane~\citep{haldane_defense_1964}}

So far, we have seen a lot of relatively abstract formalism for describing evolution in finite populations. Some concrete and hopefully illustrative examples are also presented in Appendix \ref{App_examples}. Have we gained anything from this (re)formulation? I show below that the formalism developed in part \ref{part_theory} naturally yields some `fundamental equations' for evolutionary population dynamics that are very general, help us clearly understand how evolution operates in finite fluctuating populations, and recover well-known results in the infinite population limit, thus showing that I have generalized these infinite population results. 

\section{Fundamental equations for evolution in finite populations}\label{sec_fun_theorems}
\subsection{The fundamental equation for changes in type frequencies}\label{sec_fun_theorem_freq}
Equation \eqref{nD_eqn_for_frequencies}, which we derived in chapter \ref{chap_BD}, is a very general equation for how frequencies change over time in stochastic populations. To recap, we started with a population which can contain up to $m$ different types of individuals. The population as a whole is characterized by a vector $\mathbf{n}(t) = [n_1(t), \ldots, n_m(t)]$ indexing the number of individuals of each type in the population. Changes of the population are through either birth or death of single individuals. On ecological grounds, we postulated the existence of a `system-size' parameter $K>0$ that leads to density-dependent growth and prevents the population from growing infinitely large. We then moved from numbers $\mathbf{n}(t)$ to population densities $\mathbf{x}(t)$ by assuming that birth and death rates depend on population density $x$ and not only on population numbers $n$. In the regime where $K$ is not too small (corresponding to `medium sized' populations), we found a continuous description of how $x$ changes stochastically. We further assumed that each type has a per-capita birth rate $b^{\textrm{(ind)}}(\mathbf{x})$, a per-capita death rate $d^{\textrm{(ind)}}(\mathbf{x})$, and an additional term $\mu Q_{i}(\mathbf{x})$ representing mutational effects. All three of these functions may in general vary in an arbitrarily complicated frequency-dependent manner (or be frequency independent constants, of course). We then saw that $p_i$, the frequency of the $i\textsuperscript{th}$ type in the population, changes according to the stochastic differential equation:
\begin{equation}
\label{nD_stochastic_RM}
\begin{aligned}
dp_i(t) &= \underbrace{\left[(w_i(\mathbf{x}) - \overline{w})p_i + \mu\left\{Q_i(\mathbf{p}) - p_i\left(\sum\limits_{j=1}^{m}Q_j(\mathbf{p})\right)\right\}\right]dt}_{\substack{\text{Infinite population predictions: selection-mutation balance} \\ \text{for higher fitness}}}\\
&- \frac{1}{K}\underbrace{\frac{1}{N_{K}(t)}\left[(\tau_i(\mathbf{x}) - \overline{\tau})p_i + \mu\left\{Q_i(\mathbf{p}) - p_i\left(\sum\limits_{j=1}^{m}Q_j(\mathbf{p})\right)\right\}\right]dt}_{\substack{\text{Directional noise-induced effects: selection-mutation balance}\\\text{for lower turnover rates}}}\\
&+ \frac{1}{\sqrt{K}N_{K}(t)}\underbrace{\left[\sqrt{A^{+}_{i}}dW^{(i)}_t - p_i\sum\limits_{j=1}^{m}\sqrt{A^{+}_{j}}dW^{(j)}_t\right]}_{\substack{\text{Non-directional noise-induced effects}\\\text{due to stochastic fluctuations}}}
\end{aligned}
\end{equation}
where $N_K = \sum x_i$ is the total population size scaled by $K$ (and thus $KN_K$ is the total population size), $w_i(\mathbf{x}) = b^{\textrm{(ind)}}_{i}(\mathbf{x}) - d^{\textrm{(ind)}}_{i}(\mathbf{x})$ and $\tau_i(\mathbf{x}) = b^{\textrm{(ind)}}_{i}(\mathbf{x}) + d^{\textrm{(ind)}}_{i}(\mathbf{x})$ are respectively the Malthusian fitness and per-capita turnover rate of the $i\textsuperscript{th}$ type, and $A_i^{+} \coloneqq x_i\tau_i(\mathbf{x}) + \mu Q_i(\mathbf{x})$. Each $W^{(i)}_t$ is an independent one-dimensional Wiener process (standard Brownian motion). Letting $K \to \infty$  in \eqref{nD_stochastic_RM} recovers the replicator-mutator equation in the infinite population limit. The first term of \eqref{nD_stochastic_RM} represents the direct effects of forces captured in classic deterministic models, and reflects a selection-mutation balance. However, finite populations experience a new directional force dependent on $\tau_i(\mathbf{x})$, the per-capita turnover rate of type $i$, that cannot be captured in infinite population models~\citep{kuosmanen_turnover_2022}. Remarkably, this term acts in a way that is mathematically identical to the classical action of selection and mutation in infinite population models as captured by the first term in \eqref{nD_stochastic_RM}, but in the opposite direction - A higher relative $\tau_i$ leads to a decrease in frequency (Notice the minus sign before the second term in \eqref{nD_stochastic_RM}).
\subsection{The fundamental equation for the mean value of a type-level quantity}\label{sec_fun_theorems_mean}

We can also calculate how the statistical mean value of any type-level quantity (see section \ref{sec_stat_measures}) changes over time. Let $f$ be any type level quantity, with value $f_i(t)$ for the $i\textsuperscript{th}$ type. We allow for the possibility of $f_i$ to vary over time. By multiplying both sides of
equation \eqref{nD_stochastic_RM} by $f_i$ and summing over all $i$ (The same steps as going from \eqref{nD_replicator_mutator} to \eqref{nD_Price_time_dependent}), we see that the statistical mean $\overline{f}$ of the quantity in the population varies as:
\begin{equation}
\label{nD_stochastic_Price}
\begin{aligned}
%The \vphantom is to vertically align the texts under the \underbraces
d\overline{f} &= {\underbrace{\vphantom{ \frac{1}{KN_K(t)} }\textrm{Cov}(w,f)dt}_{\substack{\text{Classical} \\ \text{selection}}}} \ - \ {\underbrace{\frac{1}{KN_K(t)}\textrm{Cov}(\tau,f)dt}_{\substack{\text{Noise-induced} \\ \text{selection}}}} \ +  \  {\underbrace{\vphantom{ \frac{1}{KN_K(t)} }\overline{\left(\frac{\partial f}{\partial t}\right)}dt}_{\substack{\text{Ecological timescale} \\ \text{feedbacks due to} \\ \text{time-dependence of $f_i$}}}}\\[15pt]
&+ \underbrace{\mu\left(1-\frac{1}{KN_K(t)}\right)\left(\sum\limits_{i=1}^{m}f_iQ_i(\mathbf{p}) - \overline{f}\sum\limits_{i=1}^{m}Q_i(\mathbf{p})\right)dt}_{\text{Transmission bias/mutational effects}}\\[15pt]
&+ \underbrace{\frac{1}{\sqrt{K}N_{K}(t)}\left(\sum\limits_{i=1}^{m}\left(f_i-\overline{f}\right)\sqrt{A_i^+}dW_{t}^{(i)}\right)}_{\text{Stochastic fluctuations}}
\end{aligned} 
\end{equation}
where all covariances are understood in the statistical sense (Note that since $w_i$, $\tau_i$, $\overline{w}$, and $\overline{\tau}$ are stochastic processes depending on $\mathbf{p}$, the terms $\textrm{Cov}(w,f)$ and $\textrm{Cov}(\tau,f)$ are themselves stochastic processes). Taking $K \to \infty$ in equation \eqref{nD_stochastic_Price} recovers the standard Price equation as the infinite population limit (either \eqref{nD_Price_time_dependent} or \eqref{nD_Price} based on whether $f_i$ varies with time; also see~\cite{rice_universal_2020} for a stochastic Price equation in a discrete-time setting).

Each term in equation \eqref{nD_stochastic_Price} lends itself to a simple biological interpretation. The first term, $\textrm{Cov}(w,f)$, is well-understood in the classical Price equation and represents the effect of natural selection in the infinite population setting. In the stochastic Price equation \eqref{nD_stochastic_Price}, the effects of the second term of 
\eqref{nD_stochastic_RM} decompose into a selection term $\textrm{Cov}(\tau,f)$ for reduced turnover rates and a transmission bias term that vanishes in the weak mutation ($\mu \to 0$) limit. Following~\cite{constable_demographic_2016} and~\cite{week_white_2021}, we refer to the effect of the covariance term (the second term of equation \eqref{nD_stochastic_Price}) as \emph{noise-induced selection} since it occurs exactly analogously to classical natural selection (but for lower $\tau$) and is induced purely by the finiteness of the population. Since this evolutionary force is unique to finite populations and has therefore been overlooked in classical population genetics, it warrants some more detailed discussion. Biologically, the $\textrm{Cov}(\tau,f)$ term (with a negative sign) describes a biasing effect due to differential turnover rates and can intuitively be understood as being similar to gambler's ruin in probability theory through the following reasoning: If a type $i$ has a higher $\tau_i$, it experiences greater turnover due to a generally higher birth and death rate and thus experience more births and deaths in a given time interval than an otherwise equivalent species with a lower $\tau_i$. More events mean greater demographic stochasticity, and types with a higher $\tau_i$ thus tend to be eliminated by a stochastic analog of selection because they experience more chance events (births and deaths) in a given time period. This effect is less visible if the total population size is higher because larger populations generally experience less stochasticity, which is reflected in the $1/N_K$ factor in this term.

This stochastic analog of selection for reduced turnover rates, captured by the second term of equation \eqref{nD_stochastic_RM}, is the force responsible for the `reversal of the direction of deterministic selection' induced by demographic noise in previous studies~\citep{houchmandzadeh_selection_2012, houchmandzadeh_fluctuation_2015, constable_demographic_2016, mcleod_social_2019}. Note that populations composed largely of types that tend to increase the \emph{total} population size $KN_K(t)$ (such as altruists in evolutionary theory and mutualists in ecological communities) will experience a lower magnitude of noise-induced selection compared to populations largely composed of types that do not facilitate such an increase, such as cheaters and highly competitive species. Further, if altruists/mutualists act by reducing the death rate (rather than increasing the birth rate) of other individuals, their presence causes higher $w$ and lower $\tau$ in the beneficiary individuals, both of which are favored by selection (but note that if they act by increasing the birth rate, they increase the magnitude of negative noise-induced selection disfavoring the beneficiary individual), which explains why this effect preferentially favors mutualists in reversing the direction of deterministic selection in finite population models with fluctuating population sizes if interaction effects are on the death rate~\citep{mcleod_social_2019}. Thus, selection for reduced turnover rate could help explain why cooperation often persists in fluctuating populations in laboratory experiments~\citep{sanchez_feedback_2013} finite population IbMs of social evolution~\citep{houchmandzadeh_selection_2012,houchmandzadeh_fluctuation_2015,chotibut_evolutionary_2015,behar_fluctuations-induced_2016,mcavoy_public_2018,mcleod_social_2019} despite infinite population models predicting their extinction.

The fact that total population size controls the strength of noise-induced selection also explains why cooperation is favored in the early transient period of population growth~\citep{melbinger_evolutionary_2010} when simulations are initiated from a small population size - In the early transient period, $N_K(t)$ is small, and the biasing effect of differential turnover rates is stronger, thus favoring cooperation. The fact that the entire term scales inversely with the total population size $KN_K(t)$ suggests that the effect of this force is weak for large populations, which explains why the persistence of cooperators is often only observed in restrictive sounding conditions such as quasi-neutrality, timescale separation, or a weak selection + weak mutation limit~\citep{mcleod_social_2019}. In all three of these cases, the first term on the RHS of \eqref{nD_eqn_for_frequencies} becomes identically 0. It therefore no longer contributes to the trait frequency dynamics, thus allowing us to see the (otherwise weak) contributions of the second term.

The third term of \eqref{nD_stochastic_Price} is relevant in both finite and infinite populations whenever $f_i$ can vary over time and represents feedback effects on the quantity $f_i$ of the $i\textsuperscript{th}$ species over short (`ecological') time-scales. Such feedback is often through a changing environment or phenotypic/behavioral plasticity, but other biological phenomena may also be at play. The fourth term of \eqref{nD_stochastic_Price} is a transmission bias term, with a correction factor due to noise-induced selection. Finally, the last term of \eqref{nD_stochastic_Price} describes the role of stochastic fluctuations. The contributions of this last term are `directionless' due to the $dW_t$ factors, and this term vanishes when we take a conditional expectation value over the underlying probability space. I will denote this probabilistic expectation value operation by $\mathbb{E}[\cdot]$ to distinguish it from the statistical mean $\overline{f}$ of a type-level quantity in the population (see section \ref{sec_stat_measures} for a brief discussion on the difference between statistical averages and probabilistic expectations in this context). Note that this expectation is conditioned on the initial state of the population, and thus $\mathbb{E}[\cdot]$ is really shorthand for $\mathbb{E}[\ \cdot \ | \ \mathbf{X}_0 = \mathbf{x}_0]$.

Two particularly interesting implications of \eqref{nD_stochastic_Price} are realized upon ignoring mutations by setting $\mu = 0$ and then substituting $f=w$ and $f = \tau$. We first note that:
\begin{align}
\textrm{Cov}(w,\tau) &=\textrm{Cov}\left( b^{\textrm{(ind)}}(\mathbf{x}) - d^{\textrm{(ind)}}(\mathbf{x}) \ , \   b^{\textrm{(ind)}}(\mathbf{x}) + d^{\textrm{(ind)}}(\mathbf{x})\right)\\
&= \sigma^2_{b^{\textrm{(ind)}}(\mathbf{x})} - \sigma^2_{d^{\textrm{(ind)}}(\mathbf{x})}\label{nD_cross_covariance}
\end{align}
It is important to remember once again that just like the statistical mean, the statistical variance $\sigma^2_{f}(t)$ of a type-level quantity $f$ is a random variable obtained by calculating the variance of the quantity \emph{in the population} at time $t$, and is not to be confused with the probabilistic/ensemble variance obtained by calculating the variance of a quantity \emph{over different realizations} of the stochastic process (see section \ref{sec_stat_measures}). Upon substituting $f = w$ in \eqref{nD_stochastic_Price} and taking expectations over the underlying probability space, we obtain:
\begin{align}
\label{nD_stochastic_Fisher}
%The \vphantom is to vertically align the texts under the underbraces of different terms
\mathbb{E}\left[\frac{d\overline{w}}{dt}\right] &= 
\negmedspace {\underbrace{\mystrut{4ex} \ \vphantom{ \frac{\sigma^2_{b^{\textrm{(ind)}}} - \sigma^2_{d^{\textrm{(ind)}}}}{KN_K(t)} } \mathbb{E}\left[\sigma^2_{w}\right] \ }_{\substack{\text{Fisher's} \\ \text{Fundamental theorem}}}} \ - \ {\underbrace{\mystrut{4ex}\mathbb{E}\left[\frac{\sigma^2_{b^{\textrm{(ind)}}} - \sigma^2_{d^{\textrm{(ind)}}}}{KN_K(t)}\right]}_{\substack{\text{Noise-induced} \\ \text{selection}}}} \ + {\underbrace{\mystrut{4ex} \vphantom{ \frac{\sigma^2_{b^{\textrm{(ind)}}} - \sigma^2_{d^{\textrm{(ind)}}}}{KN_K(t)} } \mathbb{E}\left[\hphantom{a}\overline{\hphantom{a}\frac{\partial w}{\partial t}\hphantom{a}}\hphantom{a}\right]}_{\substack{\text{Short-term (ecological)} \\ \text{feedbacks to fitness}}}}
\end{align}
Taking $K \to \infty$ in \eqref{nD_stochastic_Fisher} recovers a well-known equation in population genetics upon noting that the process tends to a deterministic process as $K \to \infty$, as noted in section \ref{sec_nD_det_limit}, and thus the expectation value in the infinite population case is superfluous. The first term, $\sigma^2_w$, is the subject of Fisher's fundamental theorem~\citep{fisher_genetical_1930,  price_fishers_1972, frank_fishers_1992, kokko_stagnation_2021}. The second term of equation \eqref{nD_stochastic_Fisher} is a manifestation of noise-induced selection and vanishes in the infinite population limit, and is thus particular to finite populations. Finally, the last term arises in both finite and infinite populations whenever $w_i$ can vary over time~\citep{frank_fishers_1992,kokko_stagnation_2021,baez_fundamental_2021}, be it through frequency-dependent selection, phenotypic plasticity, varying environments, or other ecological mechanisms, and represents feedback effects on the fitness $w_i$ of the $i\textsuperscript{th}$ species over short (`ecological') time-scales. The fact that Fisher appears to have been rather vague and dismissive of this feedback~\citep{fisher_genetical_1930} has led to much discussion, debate, and confusion about the interpretation, importance, and implications of his `fundamental theorem' (see~\cite{kokko_stagnation_2021} and sources cited therein).

Carrying out the same steps with $f = \tau$ in \eqref{nD_stochastic_Price} yields a new equation due to~\cite{kuosmanen_turnover_2022}. The result is an analog of Fisher's fundamental theorem for the turnover rates, and reads:
\begin{equation}
\label{nD_stochastic_Fisher_turnover}
\mathbb{E}\left[\frac{d\overline{\tau}}{dt}\right] = \underbrace{\vphantom{ \mathbb{E}\left[\frac{\sigma^2_{\tau}}{KN_K(t)}\right] } \mathbb{E}\left[\sigma^2_{b^{\textrm{(ind)}}} - \sigma^2_{d^{\textrm{(ind)}}}\right]}_{\substack{\text{Classical selection} \\ \text{effects}}} \ - \ \underbrace{ \mathbb{E}\left[\frac{\sigma^2_{\tau}}{KN_K(t)}\right]}_{\substack{\text{Noise-induced selection} \\ \text{effects}}} \ + \ \underbrace{ \vphantom{ \mathbb{E}\left[\frac{\sigma^2_{\tau}}{KN_K(t)}\right] } \mathbb{E}\left[\hphantom{a}\overline{\hphantom{a}\frac{\partial \tau}{\partial t}\hphantom{a}}\hphantom{a}\right]}_{\substack{\text{Short-term (ecological)} \\ \text{feedbacks to $\tau_i$}}}
\end{equation}
The implications of this equation have been extensively discussed in~\cite{kuosmanen_turnover_2022}, which is where I refer the interested reader.

\subsection{The fundamental equation for the variance of a type-level quantity}\label{sec_fun_theorems_var}
Equation \eqref{nD_stochastic_Price} is a general equation for the mean value of an arbitrary type level quantity $f$ in the population. In many real-life situations, especially those pertaining to finite populations, we are interested in not just the mean, but also the variance of a type-level quantity. In Appendix \ref{App_stoch_var_eqns}, I show that the statistical variance of any type level quantity $f$ obeys
\begin{equation}
\label{nD_stochastic_Price_variance}
\begin{aligned}
d\sigma^2_{f} &= \textrm{Cov}\left(w,(f - \overline{f})^2\right)dt - \frac{1}{KN_K}\left[ \ \overline{\tau}\sigma^2_{f} +  2\textrm{Cov}\left(\tau,(f - \overline{f})^2\right) \ \right]dt\\[12pt]
& + 2\textrm{Cov}\left(\frac{\partial f}{\partial t},f\right)dt + M_{\sigma^2_f}(\mathbf{p},N_K)dt + \frac{1}{\sqrt{K}N_{K}(t)}dW_{\sigma^2_{f}}
\end{aligned}
\end{equation}
where
\begin{equation}
\label{variance_price_mutation_term}
M_{\sigma^2_f}(\mathbf{p},N_K) = \mu\left[\left(1 - \frac{2}{KN_K}\right)\left(\sum\limits_{i=1}^{m}(f_i - \overline{f})^2Q_i(\mathbf{p})\right) + \sigma^2_f\left(1 - \frac{1}{KN_K}\right)\sum\limits_{i=1}^{m}Q_i(\mathbf{p})\right]
\end{equation}
is a mutational term that vanishes in the $\mu \to 0$ limit and
\begin{equation}
\label{variance_price_diffusion_term}
dW_{\sigma^2_f} = \sum\limits_{i=1}^{m}\left(f_i - \overline{f}\right)^2\sqrt{A_i^+}dW_{t}^{(i)}
\end{equation}
is a stochastic integral term measuring the (non-directional) effect of stochastic fluctuations that vanishes upon taking an expectation over the probability space. In the case of one-dimensional quantitative traits, an infinite-dimensional version of \eqref{nD_stochastic_Price_variance} has recently been rigorously derived~\citep{week_white_2021} using measure-theoretic tools under certain additional assumptions (See equation (21c) in~\cite{week_white_2021}). Taking expectations over the probability space in \eqref{nD_stochastic_Price_variance} and substituting mutation as acting via a Gaussian kernel also recovers equations previously derived~\citep{debarre_evolutionary_2016} in the context of evolutionary branching in finite populations as a special case. An infinite population ($K \to \infty$) version of equation \eqref{nD_stochastic_Price_variance} also appears in~\cite{lion_theoretical_2018}.

Once again, terms of equation \eqref{nD_stochastic_Price_variance} lend themselves to straightforward biological interpretation. The quantity $(f_i-\overline{f})^2$ is a measure of the distance of $f_i$ from the population mean value $\overline{f}$, and thus covariance with $(f-\overline{f})^2$ quantifies the type of selection operating in the system: A negative correlation is indicative of stabilizing selection, and a positive correlation is indicative of disruptive (\emph{i.e.} diversifying) selection. An extreme case of diversifying selection for fitness occurs if the mean fitness is at a local minimum for fitness but $f_i \not\equiv \overline{f}$ (\emph{i.e.} the population still exhibits some variation in $f$). In this case, if the variation in $f$ is associated with a variation in fitness, then $\textrm{Cov}(w,(f - \overline{f})^2)$ is strongly positive and the population experiences a sudden explosion in variance, causing the emergence of polymorphism in the population. If $\textrm{Cov}(w,(f - \overline{f})^2)$ is still positive after the initial emergence of multiple morphs, evolution eventually leads to the emergence of stable coexisting polymorphisms in the population - This phenomenon is a slight generalization of the idea of evolutionary branching that occurs in frameworks such as adaptive dynamics \citep{geritz_evolutionarily_1998}, and a special case of the exact equation \eqref{nD_stochastic_Price_variance} has been used before to study evolutionary branching in finite populations~\citep{debarre_evolutionary_2016}.

The $\textrm{Cov}\left(\partial f/\partial t,f\right)$ term once again represents the effect of eco-evolutionary feedback loops due to rapid change in $f$ that is not solely due to changes in $\mathbf{p}$. The $M_{\sigma^2_f}(\mathbf{p},N_K)$ term quantifies the effect of mutations on the variance of $f$. Note that each $Q_i(\mathbf{p}) \geq 0$ by its definition in \eqref{nD_functional_forms_for_replicator} and thus $\sum_i Q_i(\mathbf{p}) > 0$ if there are any mutational effects (and $=0$ otherwise). Furthermore, the total population size $KN_K > 2$ for most interesting evolutionary questions. Thus, from \eqref{variance_price_mutation_term}, it is clear that when $\mu > 0$ (\emph{i.e.} there is mutation in the system), we have $M_{\sigma^2_f}(\mathbf{p},N_K) > 0$, meaning that mutations always increase the variance of $f$ in the population.

The $\overline{\tau}\sigma^2_{f}$ term represents a loss of diversity due to stochastic extinctions (i.e. demographic stochasticity). To see this, it is instructive to consider the case in which this is the only force at play. Let us imagine a population of asexual organisms in which each $f_i$ is simply a label or mark arbitrarily assigned to individuals in the population at the start of an experiment/observational study and subsequently passed on to offspring - For example, a neutral genetic tag in a part of the genome that experiences a negligible mutation rate. Let us set $\mu = 0$ so that the labels cannot change between parents to offspring. This means that we have $M_{\sigma^2_f}(\mathbf{p},N_K) \equiv 0$. Further, since the labels are arbitrary and have no effect whatsoever on the biology of the organisms, we have $\textrm{Cov}\left(w,(f - \overline{f})^2\right) \equiv \textrm{Cov}\left(\tau,(f - \overline{f})^2\right) \equiv 0$. Since the labels do not change over time, we also have $\textrm{Cov}\left(\partial f/\partial t,f\right) = 0$. From \eqref{nD_stochastic_Price_variance}, we see that in this case, the variance changes as
\begin{equation}
d\sigma^2_f = - \frac{\overline{\tau}\sigma^2_{f}}{KN_K}dt + \frac{1}{\sqrt{K}N_{K}(t)}dW_{\sigma^2_{f}}
\end{equation}
On taking expectations, the second term on the RHS vanishes and we see that the expected variance in the population obeys
\begin{equation}
\label{neutral_example_for_variances}
\frac{d \mathbb{E}[\sigma^2_f]}{dt} = - \left(\mathbb{E}\left[\frac{\overline{\tau}}{KN_K}\right]\right)\mathbb{E}[\sigma^2_{f}]
\end{equation}
where I have decomposed the expectation of the product on the RHS into a product of expectations, which is admissible since the label $f$ is completely arbitrary and thus independent of both $\overline{\tau}$ and $N_K(t)$. Equation \eqref{neutral_example_for_variances} is a simple linear ODE and has the solution
\begin{equation}
\mathbb{E}[\sigma^2_f](t) = \sigma^2_f(0)e^{-\mathbb{E}\left[\frac{\overline{\tau}}{KN_K}\right]t}
\end{equation}
which tells us that the expected diversity (variance) of labels in the population decreases exponentially over time. The rate of loss is $\mathbb{E}\left[\overline{\tau}(KN_K)^{-1}\right]$, and thus, populations with higher mean turnover $\overline{\tau}$  and/or lower population size $KN_K$ lose diversity faster. This is because populations with higher $\overline{\tau}$ experience more stochastic events per unit time (a gambler's ruin type scenario), while extinction is `easier' in smaller populations because a smaller number of deaths is required to eliminate a label from the population completely. Note that \emph{which} labels/individuals are lost is entirely random (since all labels are arbitrary), but nevertheless, labels tend to be stochastically lost until only a small number of labels remain in the population.

\section{A stochastic field theory for quantitative traits}\label{sec_disc_field_eqns}

In chapter \ref{chap_infD_processes}, I formulated a `field equation' for the evolution of one-dimensional quantitative traits in populations. To recap, given a one-dimensional quantitative trait that takes values in a trait space $\mathcal{T} \subseteq \mathbb{R}$, I defined the set $\mathcal{M}(\mathcal{T}) =  \left\{\sum_{i=1}^{n}\delta_{x_i} \ | \ n \in \mathbb{N}, x_i \in \mathcal{T}\right\}$, where each $\delta_{x_i}$ is a Dirac mass centered at $x_i$. I then formulated a model in which the population at time $t$ is characterized as a whole by a randomly varying density distribution (`stochastic field') $\nu^{(t)}: \mathcal{M}(\mathcal{T}) \to [0,\infty)$ such that for any subset $A \subseteq \mathcal{T}$, the number of individuals that have trait value in $A$ is given by integrating $\nu^{(t)}$ over $A$. The change of this field is determined entirely by two functionals, $b(x|\nu)$ and $d(x|\nu)$ from $\mathcal{T}$ to $[0,\infty)$, that respectively describe the birth rate and death rate of individuals of type $x$ in a population $\nu$. Under the assumption that there exists a suitable system size parameter $K > 0$, I moved from the space of `number' distribution functions $\mathcal{M}(
\mathcal{T})$ to the space of `density' distribution functions $\mathcal{M}_{K}(\mathcal{T}) =  \left\{\sum_{i=1}^{n}\delta_{x_i}/K \ | \ n \in \mathbb{N}, x_i \in \mathcal{T}\right\}$ via a functional analog of the system size expansion. By appropriately rescaling the birth and death rate functionals, I then determined that $P(\phi,t)$, the probability that the population is described by the distribution $\phi \in \mathcal{M}_{K}(\mathcal{T})$ at time $t$, (approximately) satisfies the `stochastic field equation':
\begin{equation}
\label{disc_functional_field_eqns}
\resizebox{\textwidth}{!}{$\displaystyle
\frac{\partial P}{\partial t}(\phi,t) = \int\limits_{\mathcal{T}}\left[-
\frac{\delta}{\delta\phi(x)} \big{\{} \big{[}\phi(x) w(x|\phi) + \mu Q(x|\phi)\big{]} P(\phi,t) \big{\}} + \frac{1}{2K}\frac{\delta^2}{\delta\phi(x)^2}\big{\{}\big{[}\phi(x) \tau(x|\phi) + \mu Q(x|\phi)\big{]}P(\phi,t)\big{\}}\right]dx$}
\end{equation}
where $w(x|\phi)$, $\tau(x|\phi)$, and $Q(x|\phi)$ are functionals that respectively describe the Malthusian fitness, per-capita turnover rate, and birth rate due to mutations (with mutation rate $\mu$) of type $x \in \mathcal{T}$ individuals (now a continuous variable) in a population $\phi$. We then saw that this equation yields some well-known frameworks of quantitative genetics in the infinite population ($K \to \infty$) limit, thus illustrating consistency with known theories.

If the `intuitive' version of It\^{o}'s formula for $\mathcal{M}_K(\mathcal{T})$ valued stochastic processes obtained by `taking the limit' $m \to \infty$ in the It\^{o}'s formula for $m$-dimensional stochastic processes holds, then the exact same steps carried out in Appendices \ref{App_density_to_freq} and \ref{App_stoch_var_eqns} will `go through' essentially unchanged for quantitative traits and yield the equations obtained by simply taking $m \to \infty$ in \eqref{nD_stochastic_RM}, \eqref{nD_stochastic_Price} and \eqref{nD_stochastic_Price_variance} (with sums replaced by integrals and derivatives replaced by functional derivatives as appropriate) as the `fundamental theorems' for the evolution of quantitative traits. Indeed,~\cite{week_white_2021} have recently proposed exactly the It\^o formula we would need via a heuristic It\^o multiplication table for $L^2(\mathbb{R}, m)$ valued processes with a.s. finite Hilbert-Schmidt norm and have shown that their heuristics are equivalent to the rigorous infinite-dimensional stochastic calculus proposed by~\cite{da_prato_stochastic_2014} for more general Hilbert space valued processes\footnote{If this sentence sounds like abstract gibberish to you, don't worry too much about the details - The essence is that~\cite{week_white_2021} have proven the `correct' formula we need holds whenever $b(x|\nu), d(x|\nu)$ and $\mathcal{M}_K(\mathcal{T})$ together fulfill certain technical requirements. My heuristics here are only a `first step' and have focused on accessibility over mathematical propriety, and thus make no attempt to verify whether and when these technical requirements are satisfied.}. Furthermore, using these `spacetime white noise heuristics' together with a particular functional form for (weak) solutions to certain SPDEs,~\cite{week_white_2021} have arrived at precisely the infinite-dimensional version of equations \eqref{nD_stochastic_Price} and \eqref{nD_stochastic_Price_variance} obtained by `taking the limit' $m \to \infty$ in \eqref{nD_stochastic_Price} and \eqref{nD_stochastic_Price_variance} from the `SPDE side' under certain additional assumptions required to ensure existence/uniqueness of solutions to the relevant SPDEs. Equation \eqref{disc_functional_field_eqns} can thus be argued to be the (functional) Fokker-Planck view of the stochastic processes for the evolutionary ecology of quantitative traits recently studied in~\cite{week_white_2021}. Like~\cite{week_white_2021}, I believe that rigorously establishing the precise criteria needed for general existence and uniqueness of solutions to equations of the form \eqref{disc_functional_field_eqns} (or the SPDE versions presented in~\cite{week_white_2021}, which resemble my equation \eqref{functional_langevin}) presents an important mathematical direction for future work that may yield new insights into both the biology of quantitative traits and the mathematics of measure-valued stochastic processes.

More generally, the well-known intimate relation between It\^o SDEs and Fokker-Planck equations (see section \ref{sec_math_background}) is extremely useful, and indeed has been extensively exploited~\citep{van_kampen_stochastic_1981,oksendal_stochastic_1998, gardiner_stochastic_2009}, because many problems are much more easily attacked using probabilistic tools that operate on It\^o process, whereas others are more amenable to study using tools from PDE theory that operate on the Fokker-Planck equations. In the infinite-dimensional case, working with the `physics' side through functional Fokker-Planck equations such as \eqref{disc_functional_field_eqns} allows us to use tools with an essentially `differential equation' flavor such as spectral methods (\cite{rogers_demographic_2012, rogers_modes_2015}; See also Appendix \ref{App_examples} for a general model-independent pedagogical example for studying phenotypic clustering and evolutionary branching), slow manifold (`adiabatic') approximations~\citep{parsons_dimension_2017}, and WKB style expansions~(\cite{rogers_demographic_2012,assaf_wkb_2017}; though it bears noting that WKB theory is closely related to large deviation principles that occur in the study of Markov processes and come in a distinctly `probabilistic' flavor, see for example~\cite{dembo_large_1998}). On the other hand, many biologically interesting questions are much easier to approach from the `mathematics'\footnote{The `mathematics' vs `physics' distinction is just a caricature and is not intended to be read into or thought about too deeply :)} side by working with branching processes/SPDEs and using tools that have an essentially `probabilistic' flavor such as duality~\citep{dawson_stochastic_1975,greenman_duality_2020}, infinitesimal generators~\citep{ethier_markov_1986,etheridge_mathematical_2011}, martingale theory~\citep{champagnat_unifying_2006,etheridge_mathematical_2011}, and weak/mild solutions to SPDEs~\citep{week_white_2021}. Rigorously establishing and exploiting relations between functional Fokker-Planck equations of the form \eqref{disc_functional_field_eqns} and SPDEs of the form \eqref{functional_langevin} (or those studied in~\cite{week_white_2021}) for measure-valued branching processes could thus prove very fruitful for developing more integrative eco-evolutionary theory since it allows us to seamlessly transition between complementary views of the same object to suit the problem at hand.

To the best of my knowledge, a general formulation of stochastic field equations for the population dynamics of quantitative traits from the functional Fokker-Planck side in the manner I have carried out here has not been done before. Similar equations have been formulated for some specific stochastic models of quantitative trait evolution~\citep{rogers_demographic_2012,rogers_modes_2015} and population ecology of size-structured communities~\citep{odwyer_integrative_2009}. Stochastic field equations are also known in mathematical neurobiology~\citep{buice_field-theoretic_2007,bressloff_stochastic_2010,coombes_neural_2014}, and have recently been proposed in a model of collective motion~\citep{o_laighleis_minimal_2018}. Broadly similar deterministic field theoretic approaches have also been proposed to study bio-geography and spatial biodiversity patterns such as the species-area relationship~(\cite{odwyer_field_2010}; But see~\cite{grilli_absence_2012} for an explanation of why the calculations presented in~\cite{odwyer_field_2010} are not exactly correct\footnote{The essential problem is that an assumption about time reversibility (`detailed balance' in more accurate physics lingo), usually made for equilibrium systems, is not satisfied. Nevertheless, the general \emph{approach} is still interesting and seems reasonable}). The ubiquity of such approaches in diverse areas of theoretical biology suggests that the formal systematic study and analysis of such equations could have wide-spread applications.

Currently, (stochastic) field equations are primarily used by physicists working in areas such as statistical field theory and quantum field theory, and are largely attacked using ingenious and often somewhat system-specific tools that may not necessarily be mathematically well understood~(\cite{carmona_stochastic_1999}; but see \cite{bogachev_fokker-planck-kolmogorov_2015} for a rigorous treatment of some infinite-dimensional Fokker-Planck equations). Equation \eqref{disc_functional_field_eqns} opens up the study of quantitative trait dynamics in finite fluctuating populations to analysis using some of these tools - For example, a careful reading of the literature provides many hints and recent attempts illustrating how the major ideas behind the Fock space representation~\citep{del_razo_probabilistic_2022}, perturbative `loop diagrams/expansions'~\citep{dodd_many-body_2009}, and the path integral formalism~\citep{doi_second_1976, peliti_path_1985, chow_path_2015, weber_master_2017} can all be co-opted to coax biological insights from the sort of stochastic processes modeled by \eqref{disc_functional_field_eqns}.

Equation \eqref{disc_functional_field_eqns} thus provides a (relatively) more accessible alternative formulation that does not (explicitly) rely on tools from relatively `advanced' mathematical fields such as martingale theory and stochastic analysis that are typically used to study the evolution of quantitative traits in finite, fluctuating populations~\citep{dawson_stochastic_1975,fleming_measure-valued_1979,ethier_markov_1986,champagnat_unifying_2006,etheridge_mathematical_2011,week_white_2021}. This is intended to make the models and ideas more accessible to theorists without formal training in mathematical areas such as measure theory and functional analysis and encourage the use of techniques from physics and related domains, as outlined in the paragraphs above. 

Importantly, the formalism I develop here in terms of functional Fokker-Planck equations likely does \emph{not} carry over to the study of higher dimensional quantitative traits (or populations which vary in two or more one-dimensional quantitative traits) because these processes are routinely badly behaved in higher dimensions: In particular, a probability density $P(\phi,t)$ with respect to the reals frequently does not even exist in higher dimensions if one has any biologically non-trivial features such as interactions between types (\cite{fleming_measure-valued_1979, walsh_introduction_1986}; Also see for example~\cite{evans_measure-valued_1994}), rendering equation \eqref{disc_functional_field_eqns} entirely meaningless. My (admittedly limited) understanding is that such processes are also rather difficult to study in two or more dimensions from the SPDE side for similar reasons since one is forced to work with distribution valued (rather than function valued) solutions and thus needs to use a considerable amount of advanced functional analysis to establish even basic existence-uniqueness results~\citep{walsh_introduction_1986,carmona_stochastic_1999,prevot_concise_2007,liu_stochastic_2015,balan_gentle_2018}. It may well be the case that concrete biologically useful progress in this direction requires new mathematics altogether, a situation increasingly also encountered in other areas of mathematical biology~\citep{vittadello_open_2022}. Just as theoretical physics has done in the past, the pursuit of general mathematical descriptions of biological theory may therefore also inspire new ideas that are of independent interest to `pure' mathematicians~\citep{cohen_mathematics_2004}.
