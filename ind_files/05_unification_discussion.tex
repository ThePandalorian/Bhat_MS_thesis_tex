\epigraph{\justifying The grand aim of all science [is] to cover the greatest number of empirical facts by logical deduction from the smallest number of hypotheses or axioms}{Albert Einstein}
%\epigraph{\justifying Not only is algebraic reasoning exact; it imposes an exactness on the verbal postulates made before algebra can start which is usually lacking in the first verbal formulations of scientific principles.}{J.B.S. Haldane}
\justifying
In this thesis, we have seen how stochastic birth-death processes can be used to construct and analyze mechanistic individual-based models for the dynamics of finite populations. In doing so, we have also seen that various well-known equations of evolutionary dynamics can be recovered in the infinite population size limit. In the finite-dimensional case (corresponding to discrete trait variants), the infinite population limit corresponds to the equations of population genetics and evolutionary game theory. In the infinite-dimensional case (corresponding to quantitative traits), we instead obtain the equations of quantitative genetics, and, in some further limits, adaptive dynamics. In both cases, the mean value of the trait in the population changes according to an equation resembling the Price equation. My derivatioon highlights the natural connections between the various equations of population dynamics - For example, the same procedures that lead to the replicator-mutator equation in the case of discretely varying traits yield Kimura's model in the quantitative case, underscoring the broad similarities between evolutionary game theory and quantitative genetics. The major formulations are summarized in Table \ref{table_summary}.

{\centering\begin{sideways}
    \begin{minipage}{\textheight}
        \resizebox{\textheight}{!}{%
            \setstretch{1.5}
            \begin{tabular}{ %I manually specified the width of each column by trial and error
  |p{\dimexpr.25\linewidth-2\tabcolsep-1.3333\arrayrulewidth}% column 1
  |p{\dimexpr.27\linewidth-2\tabcolsep-1.3333\arrayrulewidth}% column 2
  |p{\dimexpr.33\linewidth-2\tabcolsep-1.3333\arrayrulewidth}% column 3
  |p{\dimexpr.25\linewidth-2\tabcolsep-1.3333\arrayrulewidth}% column 4
  |p{\dimexpr.4\linewidth-2\tabcolsep-1.3333\arrayrulewidth}|% column 5
  }
            \hline
         \centering \textbf{Number of possible distinct trait variants ($m$)} & \centering \textbf{State Space} &
\centering \textbf{Model parameters} & \centering \textbf{Mesoscopic description} & \centering\arraybackslash \textbf{Infinite population limit} \\
        \hline
        $m = 1$ \newline (Identical individuals)  & $[0,1,2,3,\ldots]$ \newline (Population size) & Two real-valued functions, $b(N)$ and $d(N)$, describing the birth and death rate of individuals when the population size is $N$ & Univariate Fokker-Planck equation \newline (one-dimensional SDEs) & Single species population dynamics\\ 
        \hline
        $1 < m < \infty$ \newline (Discrete traits) & $[0,1,2,3,\ldots]^{m}$ \newline (Number of individuals of each trait variant) & $2m$ real-valued functions, $b_i(\mathbf{v})$ and $d_i(\mathbf{v})$ (for $1 \leq i \leq m$) describing the birth and death rate of trait variant $i$ when the population is $\mathbf{v}$ & Multivariate Fokker-Planck equation \newline ($m$-dimensional SDEs) &  Evolutionary game theory \newline
        Lotka-Volterra competition \newline Quasispecies equation \newline Price equation (discrete traits) \\
        \hline
        $m = \infty$ \newline (Quantitative traits) & $\left\{\sum\limits_{i=1}^{n}\delta_{x_i} \ | \ n \in \mathbb{N}\right\}$ \newline \newline (Each Dirac mass $\delta_{x_i}$ is an individual with trait value $x_i$) & Two real-valued functionals $b(x|\nu)$ and $d(x|\nu)$ describing the birth and death rate of trait variant $x$ when the population is $\nu$ & Functional Fokker-Planck equation/Field theory \newline (SPDEs) & Kimura's continuum-of-alleles model \newline \cite{sasaki_oligomorphic_2011}'s Oligomorphic Dynamics \newline \cite{wickman_theoretical_2022}'s Trait Space Equations for intraspecific trait variation \newline Gradient Dynamics \newline Price equation (quantitative traits)\\
        \hline
            \end{tabular}
        }
        %\renewcommand\thetable{1}
        \captionof{table}{Summary of the various birth-death processes studied in this thesis}
        \label{table_summary}
    \end{minipage}
\end{sideways}\par}
\clearpage
\section{The fundamental theorem for changes in type frequencies in the population}\label{sec_fun_theorem_freq}
Equation \eqref{nD_eqn_for_frequencies}, which we derived in chapter \ref{chap_BD}, is a very general equation for how frequencies change over time in stochastic populations. To recap, we started with a population which can contain up to $m$ different types of individuals, and used ecological arguments to posit the existence of a `system-size' parameter $K$ that leads to density-dependent growth and prevents the population from growing infinitely large. The population as a whole is characterized by a vector $\mathbf{x} = [x_1, \ldots, x_m]$ indexing the density (\emph{i.e.} number divided by $K$) of each type of individual. Changes of the population are through either birth or death of individuals. Each type has a per-capita birth rate $b^{\textrm{(ind)}}(\mathbf{x})$, a per-capita death rate $d^{\textrm{(ind)}}(\mathbf{x})$, and an additional term $\mu Q_{i}(\mathbf{x})$ representing mutational effects. All three of these functions depend on the density (and \emph{not} just the total number) of indidivuals of each type in the population, and may in general also be frequency-dependent. In the regime where $K$ is not too small (corresponding to `medium sized' populations), we identified two quantities, $w_i(\mathbf{x}) = b^{\textrm{(ind)}}_{i}(\mathbf{x}) - d^{\textrm{(ind)}}_{i}(\mathbf{x})$ and $\tau_i(\mathbf{x}) = b^{\textrm{(ind)}}_{i}(\mathbf{x}) + d^{\textrm{(ind)}}_{i}(\mathbf{x})$, the Malthusian fitness and per-capita turnover rate of the $i\textsuperscript{th}$ type respectively, that emerge as being important for trait frequency dynamics. In particular, we saw that $p_i$, the frequency of the $i\textsuperscript{th}$ type in the population, changes according to the equation:
\begin{equation}
\label{nD_stochastic_RM}
\begin{aligned}
dp_i(t) &= \underbrace{\left[(w_i(\mathbf{x}) - \overline{w})p_i + \mu\left\{Q_i(\mathbf{p}) - p_i\left(\sum\limits_{j=1}^{m}Q_j(\mathbf{p})\right)\right\}\right]dt}_{\substack{\text{Infinite population predictions: selection-mutation balance} \\ \text{for higher fitness}}}\\
&- \frac{1}{K}\underbrace{\frac{1}{N_{K}(t)}\left[(\tau_i(\mathbf{x}) - \overline{\tau})p_i + \mu\left\{Q_i(\mathbf{p}) - p_i\left(\sum\limits_{j=1}^{m}Q_j(\mathbf{p})\right)\right\}\right]dt}_{\substack{\text{Directional noise-induced effects: selection-mutation balance}\\\text{for lower turnover rates}}}\\
&+ \frac{1}{\sqrt{K}N_{K}(t)}\underbrace{\left[\left(A^{+}_{i}\right)^{1/2}dB^{(i)}_t - p_i\sum\limits_{j=1}^{m}\left(A^{+}_{j}\right)^{1/2}dB^{(j)}_t\right]}_{\substack{\text{Non-directional noise-induced effects}\\\text{due to stochastic fluctuations}}}
\end{aligned}
\end{equation}
where $N_K = \sum x_i$ is the total population size scaled by $K$ (and thus $KN_K$ is the total population size), $A_i^{+} = x_i\tau_i(\mathbf{x}) + \mu Q_i(\mathbf{x})$, and each $B^{(i)}_t$ is an independent one-dimensional standard Brownian motion. Equation \eqref{nD_stochastic_RM} is in `replicator-mutator' form, and letting $K \to \infty$ recovers the standard replicator-mutator equation. The first term represents the direct effects of forces captured in classic deterministic models, and reflects a selection-mutation balance. However, finite populations experience a new directional force dependent on $\tau_i(\mathbf{x})$, the per-capita turnover rate of type $i$, that cannot be captured in infinite population models \citep{kuosmanen_turnover_2022}. Remarkably, this term acts in a way that is mathematically identical to the classical action of selection and mutation in infinite population models as captured by the first term in \eqref{nD_stochastic_RM}, but in the opposite direction - A higher relative $\tau_i$ leads to a decrease in frequency (Notice the minus sign before the second term in \eqref{nD_stochastic_RM}).
\section{The fundamental theorem for the mean value of a type-level quantity in the population}\label{sec_fun_theorems_mean}

We can now calculate how the statistical mean value of a type-level quantity changes over time. Let $f$ be any type level quantity, with value $f_i(t)$ for the $i\textsuperscript{th}$ type. We allow for the possibility of $f_i$ to vary over time. By multiplying both sides of
equation \eqref{nD_stochastic_RM} by $f_i$ and summing over all $i$ (The same steps as going from \eqref{nD_replicator_mutator} to \eqref{nD_Price_time_dependent}), we see that the statistical mean $\overline{f}$ of the quantity in the population varies as:
\begin{equation}
\label{nD_stochastic_Price}
\begin{aligned}
%The \vphantom is to vertically align the texts under the \underbraces
d\overline{f} &= {\underbrace{\vphantom{ \frac{1}{KN_K(t)} }\textrm{Cov}(w,f)dt}_{\substack{\text{Classical} \\ \text{selection}}}} \ - \ {\underbrace{\frac{1}{KN_K(t)}\textrm{Cov}(\tau,f)dt}_{\substack{\text{Noise-induced} \\ \text{selection}}}} \ +  \  {\underbrace{\vphantom{ \frac{1}{KN_K(t)} }\overline{\left(\frac{\partial f}{\partial t}\right)}dt}_{\substack{\text{Ecological timescale} \\ \text{feedbacks due to} \\ \text{time-dependence of $f_i$}}}}\\[15pt]
&+ \underbrace{\mu\left(1-\frac{1}{KN_K(t)}\right)\left(\sum\limits_{i=1}^{m}f_iQ_i(\mathbf{p}) - \overline{f}\sum\limits_{i=1}^{m}Q_i(\mathbf{p})\right)dt}_{\text{Transmission bias/mutational effects}}\\[15pt]
&+ \underbrace{\frac{1}{\sqrt{K}N_{K}(t)}\left(\sum\limits_{i=1}^{m}\left(f_i-\overline{f}\right)\sqrt{A_i^+}dB_{t}^{(i)}\right)}_{\text{Stochastic fluctuations}}
\end{aligned} 
\end{equation}
where all covariances are understood in the statistical sense (Note that since $w_i$, $\tau_i$, $\overline{w}$, and $\overline{\tau}$ are stochastic processes depending on $\mathbf{p}$, the terms $\textrm{Cov}(w,f)$ and $\textrm{Cov}(\tau,f)$ are themselves stochastic processes). Taking $K \to \infty$ in equation \eqref{nD_stochastic_Price} recovers the standard Price equation as the infinite population limit (either \eqref{nD_Price_time_dependent} or \eqref{nD_Price} based on whether $f_i$ varies with time). We saw in chapter \ref{chap_infD_processes} using field equations that very similar methods of attack to those used outlined in chapter \eqref{chap_BD} also hold for quantitative traits. For example, equation \eqref{cts_replicator_mutator} and \eqref{cts_price} are respectively exactly the infinite dimensional analogs of the deterministic replicator-mutator equation \eqref{nD_replicator_mutator} and the deterministic Price equation \eqref{nD_Price} when $f$ is the trait value. We may therefore expect to find equations similar to \eqref{nD_stochastic_RM} and \eqref{nD_stochastic_Price} for quantitative traits. Indeed, measure-theoretic tools have recently been used to rigorously show that an infinite-dimensional version of \eqref{nD_stochastic_Price} holds for one-dimensional quantitative traits when $f$ is the trait value and $b(x|\phi) \pm d(x|\phi)$ are Gaussian \citep{week_white_2021}. Equations \eqref{nD_stochastic_RM} and \eqref{nD_stochastic_Price} are thus fundamental theorems for the evolution of finite populations, with the replicator-mutator and Price equations as their respective infinite population limits (also see \citep{rice_universal_2020} for a stochastic Price equation in a discrete-time setting).\\
Each term in equation \eqref{nD_stochastic_Price} lends itself to a simple biological interpretation. The first term, $\textrm{Cov}(w,f)$, is well-understood in the classical Price equation and represents the effect of natural selection in the infinite population setting. In the stochastic Price equation \eqref{nD_stochastic_Price}, the effects of the second term of 
\eqref{nD_stochastic_RM} decompose into a selection term $\textrm{Cov}(\tau,f)$ for reduced turnover rates and a transmission bias term that vanishes in the weak mutation ($\mu \to 0$) limit. Following \cite{week_white_2021}, we refer to the effect of the covariance (the second term of equation \eqref{nD_stochastic_Price}) as \emph{noise-induced selection} since it occurs exactly analogously to classical natural selection (but for lower $\tau$) and is induced purely by the finiteness of the population. Since this evolutionary force is unique to finite populations and has therefore been overlooked in classical population genetics, it warrants some more detailed discussion. Biologically, the $\textrm{Cov}(\tau,f)$ term (with a negative sign) describes a biasing effect due to differential turnover rates and can intuitively be understood as being similar to gambler's ruin in probability theory through the following reasoning: If a type $i$ has a higher $\tau_i$, it experiences greater turnover due to a generally higher birth and death rate and thus experience more births and deaths in a given time interval than an otherwise equivalent species with a lower $\tau_i$. More events mean greater demographic stochasticity, and types with a higher $\tau_i$ thus tend to be eliminated by a stochastic analog of selection because they experience more chance events (births and deaths) in a given time period. This effect is less visible if the total population size is higher because larger populations generally experience less stochasticity, which is reflected in the $1/N_K$ factor in this term. This stochastic analog of selection for reduced turnover rates, captured by the second term of equation \eqref{nD_stochastic_RM}, is the force responsible for the `reversal of the direction of deterministic selection' induced by demographic noise in previous studies \citep{houchmandzadeh_selection_2012, houchmandzadeh_fluctuation_2015, constable_demographic_2016, mcleod_social_2019}. Note that types that tend to increase the \emph{total} population size $KN_K(t)$ (such as altruists in evolutionary theory and mutualists in ecological communities) will reduce the magnitude of this effect compared to types that do not facilitate such an increase, such as cheaters and highly competitive species, which could explain why this effect preferentially favors the former types in reversing the direction of deterministic selection in finite population models \citep{houchmandzadeh_fluctuation_2015, mcleod_social_2019}. The fact that total population size controls the strength of the second term of \eqref{nD_stochastic_RM} also explains why cooperation is favoured in the early transient period of population growth \citep{melbinger_evolutionary_2010} when simulations are initiated from a small population size - In the early transient period, $N_K(t)$ is small, and the biasing effect of differential turnover rates is stronger, thus favouring cooperation. More generally, selection for reduced turnover rate could explain why cooperation often persists for a long time in finite population IbMs (and the real world) despite infinite population models predicting their extinction. The fact that the entire term is multiplied by $(KN_K(t))^{-1}$ suggests that the effect of this force is weak for medium to large populations, which explains why the persistence of cooperators is often only observed in restrictive sounding conditions such as quasi-neutrality, rapid attraction to a slow manifold, or a weak selection + weak mutation limit. In all three of these cases, the first term on the RHS of \eqref{nD_eqn_for_frequencies} becomes identically 0. It therefore no longer contributes to the trait frequency dynamics, thus allowing us to see the contributions of the second term.\\
The third term of \eqref{nD_stochastic_Price} is relevant in both finite and infinite populations whenever $f_i$ can vary over time and represents feedback effects on the quantity $f_i$ of the $i\textsuperscript{th}$ species over short (`ecological') time-scales. Such feedback is often through a changing environment or phenotypic/behavioral plasticity, but other biological phenomena may also be at play. The fourth term of \eqref{nD_stochastic_Price} is a transmission bias term, with a correction factor due to noise-induced selection. Finally, the last term of \eqref{nD_stochastic_Price} describes the role of stochastic fluctuations. The contributions of this last term are `directionless' due to the $dB_t$ factors, and this term vanishes when we take a conditional expectation value over the underlying probability space. We denote this probabilistic expectation value operation by $\mathbb{E}[\cdot]$ to distinguish it from the statistical mean \eqref{nD_mean}. Note that this expectation is conditioned on the initial state of the population, and thus $\mathbb{E}[\cdot]$ is really shorthand for $\mathbb{E}[\ \cdot \ | \ \mathbf{X}_0 = \mathbf{x}_0]$.
\\
Two particularly interesting implications of \eqref{nD_stochastic_Price} are realized upon ignoring mutations by setting $\mu = 0$ and then substituting $f=w$ and $f = \tau$. We first note that:
\begin{align}
\textrm{Cov}(w,\tau) &=\textrm{Cov}\left( b^{\textrm{(ind)}}(\mathbf{x}) - d^{\textrm{(ind)}}(\mathbf{x}) \ , \   b^{\textrm{(ind)}}(\mathbf{x}) + d^{\textrm{(ind)}}(\mathbf{x})\right)\\
&= \sigma^2_{b^{\textrm{(ind)}}(\mathbf{x})} - \sigma^2_{d^{\textrm{(ind)}}(\mathbf{x})}\label{nD_cross_covariance}
\end{align}
where we have defined the statistical variance of a quantity $f$ as
\begin{equation}
\sigma^2_{f} \coloneqq \overline{(f^2)} - (\overline{f})^2
\end{equation}
Note that just like the statistical mean, the statistical variance is a random variable and is not to be confused with the probabilistic/ensemble variance. Upon substituting $f = w$ in \eqref{nD_stochastic_Price} and taking expectations over the underlying probability space, we obtain:
\begin{align}
\label{nD_stochastic_Fisher}
%The \vphantom is to vertically align the texts under the underbraces of different terms
\mathbb{E}\left[\frac{d\overline{w}}{dt}\right] &= 
\negmedspace {\underbrace{\mystrut{4ex} \ \vphantom{ \frac{\sigma^2_{b^{\textrm{(ind)}}} - \sigma^2_{d^{\textrm{(ind)}}}}{KN_K(t)} } \mathbb{E}\left[\sigma^2_{w}\right] \ }_{\substack{\text{Fisher's} \\ \text{Fundamental theorem}}}} \ - \ {\underbrace{\mystrut{4ex}\mathbb{E}\left[\frac{\sigma^2_{b^{\textrm{(ind)}}} - \sigma^2_{d^{\textrm{(ind)}}}}{KN_K(t)}\right]}_{\substack{\text{Noise-induced} \\ \text{selection}}}} \ + {\underbrace{\mystrut{4ex} \vphantom{ \frac{\sigma^2_{b^{\textrm{(ind)}}} - \sigma^2_{d^{\textrm{(ind)}}}}{KN_K(t)} } \mathbb{E}\left[\hphantom{a}\overline{\hphantom{a}\frac{\partial w}{\partial t}\hphantom{a}}\hphantom{a}\right]}_{\substack{\text{Short-term (ecological)} \\ \text{feedbacks to fitness}}}}
\end{align}
Taking $K \to \infty$ in \eqref{nD_stochastic_Fisher} recovers a well-known equation in population genetics upon noting that the process tends to a deterministic process as $K \to \infty$, as noted in section \ref{sec_nD_det_limit}, and thus the expectation value in the infinite population case is superfluous.The first term, $\sigma^2_w$, is the subject of Fisher's fundamental theorem \citep{fisher_genetical_1930,  price_fishers_1972, frank_fishers_1992, kokko_stagnation_2021}. The last term arises in both finite and infinite populations whenever $w_i$ can vary over time \citep{baez_fundamental_2021}, be it through frequency-dependent selection, phenotypic plasticity, varying environments, or other ecological mechanisms, and represents feedback effects on the fitness $w_i$ of the $i\textsuperscript{th}$ species over short (`ecological') time-scales. Fisher appears to have been rather vague and dismissive of this feedback \citep{fisher_genetical_1930}, and this has led to much discussion, debate, and confusion about the interpretation, importance, and implications of his `fundamental theorem' (see \cite{kokko_stagnation_2021} and sources cited therein). Finally, the second term of equation \eqref{nD_stochastic_Fisher} is a manifestation of noise-induced selection and vanishes in the infinite population limit, and is thus particular to finite populations.\\
Carrying out the same steps with $f = \tau$ in \eqref{nD_stochastic_Price} yields a new equation/theorem due to \cite{kuosmanen_turnover_2022} that has only recently been recognized as important. This theorem is an analog of Fisher's fundamental theorem for the turnover rates, and reads:
\begin{equation}
\label{nD_stochastic_Fisher_turnover}
\mathbb{E}\left[\frac{d\overline{\tau}}{dt}\right] = \underbrace{\vphantom{ \mathbb{E}\left[\frac{\sigma^2_{\tau}}{KN_K(t)}\right] } \mathbb{E}\left[\sigma^2_{b^{\textrm{(ind)}}} - \sigma^2_{d^{\textrm{(ind)}}}\right]}_{\substack{\text{Classical selection} \\ \text{effects}}} \ - \ \underbrace{ \mathbb{E}\left[\frac{\sigma^2_{\tau}}{KN_K(t)}\right]}_{\substack{\text{Noise-induced selection} \\ \text{effects}}} \ + \ \underbrace{ \vphantom{ \mathbb{E}\left[\frac{\sigma^2_{\tau}}{KN_K(t)}\right] } \mathbb{E}\left[\hphantom{a}\overline{\hphantom{a}\frac{\partial \tau}{\partial t}\hphantom{a}}\hphantom{a}\right]}_{\substack{\text{Short-term (ecological)} \\ \text{feedbacks to $\tau_i$}}}
\end{equation}
The implications of this theorem have been extensively discussed in \citep{kuosmanen_turnover_2022}, which is where we refer the interested reader.
\hl{Not sure if there's much more to write?}

\section{The fundamental theorem for the variance of a type-level quantity in the population}\label{sec_fun_theorems_var}
Equation \eqref{nD_stochastic_Price} is a general equation for the mean value of an arbitrary type level quantity $f$ in the population. In many real-life situations, especially those pertaining to finite populations, we are interested in not just the mean, but also the variance of a type-level quantity. In appendix \ref{App_stoch_var_eqns}, I show that the statistical variance of any type level quantity $f$ obeys
\begin{equation}
\label{nD_stochastic_Price_variance}
\begin{aligned}
d\sigma^2_{f} &= \textrm{Cov}\left(w,(f - \overline{f})^2\right)dt - \frac{1}{KN_K}\left[ \ \overline{\tau}\sigma^2_{f} +  2\textrm{Cov}\left(\tau,(f - \overline{f})^2\right) \ \right]dt\\[12pt]
& + 2\textrm{Cov}\left(\frac{\partial f}{\partial t},f\right)dt + M_{\sigma^2_f}(\mathbf{p},N_K)dt + dB_{\sigma^2_{f}}
\end{aligned}
\end{equation}
where
\begin{equation}
\label{variance_price_mutation_term}
M_{\sigma^2_f}(\mathbf{p},N_K) = \mu\left[\left(1 - \frac{2}{KN_K}\right)\left(\sum\limits_{i=1}^{m}(f_i - \overline{f})^2Q_i(\mathbf{p})\right) + \sigma^2_f\left(1 - \frac{1}{KN_K}\right)\sum\limits_{i=1}^{m}Q_i(\mathbf{p})\right]
\end{equation}
is a mutational term that vanishes in the $\mu \to 0$ limit and
\begin{equation}
\label{variance_price_diffusion_term}
dB_{\sigma^2_f} = \frac{1}{\sqrt{K}N_{K}(t)}\left(\sum\limits_{i=1}^{m}\left(f_i - \overline{f}\right)^2\sqrt{A_i^+}dB_{t}^{(i)}\right)
\end{equation}
is a stochastic integral term measuring the (non-directional) effect of stochastic fluctuations that vanishes upon taking an expectation over the probability space. In the case of one-dimensional quantitative traits, an infinite-dimensional version of \eqref{nD_stochastic_Price_variance} has recently been rigorously derived \citep{week_white_2021} using measure-theoretic tools under certain additional assumptions (See equation (21c) in \cite{week_white_2021}). Taking expectations over the probability space in \eqref{nD_stochastic_Price_variance} and substituting mutation as acting via a Gaussian kernel also recovers equations previously derived \citep{debarre_evolutionary_2016} in the context of evolutionary branching in finite populations as a special case (Equation A.23 in \cite{debarre_evolutionary_2016} is equivalent to equation \eqref{nD_stochastic_Price_variance} for their choice of functional forms upon converting their change in variance to an infinitesimal rate of change \emph{i.e.} a derivative).\\
Once again, terms of equation \eqref{nD_stochastic_Price_variance} lend themselves to straightforward biological interpretation. The quantity $(f_i-\overline{f})^2$ is a measure of the distance of $f_i$ from the population mean value $\overline{f}$, and thus covariance with $(f-\overline{f})^2$ quantifies the type of selection operating in the system: A negative correlation is indicative of stabilizing selection, and a positive correlation is indicative of disruptive (\emph{i.e.} diversifying) selection. An extreme case of diversifying selection for fitness occurs if the mean fitness is at a local minimum for fitness but $f_i \not\equiv \overline{f}$ (\emph{i.e.} the population still exhibits some variation in $f$). In this case, if the variation in $f$ is associated with a variation in fitness, then $\textrm{Cov}(w,(f - \overline{f})^2)$ is strongly positive and the population experiences a sudden explosion in variance. If $\textrm{Cov}(w,(f - \overline{f})^2)$ is still positive after the initial emergence of multiple morphs, evolution eventually leads to the emergence of stable coexisting polymorphisms in the population. The $\textrm{Cov}\left(\partial f/\partial t,f\right)$ term once again represents the effect of eco-evolutionary feedback loops due to rapid change in $f$ that is not solely due to changes in $\mathbf{p}$. The $M_{\sigma^2_f}(\mathbf{p},N_K)$ term quantifies the effect of mutations on the variance of $f$.Note that each $Q_i(\mathbf{p}) \geq 0$ by its definition in \eqref{nD_functional_forms_for_replicator} and thus $\sum_i Q_i(\mathbf{p}) > 0$ if there are any mutational effects (and $=0$ otherwise). Furthermore, the total population size $KN_K > 2$ for most interesting evolutionary questions. Thus, from \eqref{variance_price_mutation_term}, it is clear that when $\mu > 0$ (\emph{i.e.} there is mutation in the system), we have $M_{\sigma^2_f}(\mathbf{p},N_K) > 0$, meaning that mutations always increase the variance of $f$ in the population.\\
The $\overline{\tau}\sigma^2_{f}$ term represents a loss of diversity due to stochastic extinctions (i.e. demographic stochasticity). To see this, it is instructive to consider the case in which this is the only force at play. Let us imagine a population of asexual organisms in which each $f_i$ is simply a label or mark arbitrarily assigned to individuals in the population at the start of an experiment/observational study and subsequently passed on to offspring - For example, a neutral genetic tag in a part of the genome that experiences a negligible mutation rate. Let us set $\mu = 0$ so that the labels cannot change between parents to offspring. This means that we have $M_{\sigma^2_f}(\mathbf{p},N_K) \equiv 0$. Further, since the labels are arbitrary and have no effect whatsoever on the biology of the organisms, we have $\textrm{Cov}\left(w,(f - \overline{f})^2\right) \equiv \textrm{Cov}\left(\tau,(f - \overline{f})^2\right) \equiv 0$. Since the labels do not change over time, we also have $\textrm{Cov}\left(\partial f/\partial t,f\right) = 0$. From \eqref{nD_stochastic_Price_variance}, we see that in this case, the variance changes as
\begin{equation}
d\sigma^2_f = - \frac{\overline{\tau}\sigma^2_{f}}{KN_K}dt + dB_{\sigma^2_{f}}
\end{equation}
On taking expectations, we see that the expected variance in the population obeys
\begin{equation}
\label{neutral_example_for_variances}
\frac{d \mathbb{E}[\sigma^2_f]}{dt} = - \left(\mathbb{E}\left[\frac{\overline{\tau}}{KN_K}\right]\right)\mathbb{E}[\sigma^2_{f}]
\end{equation}
where we have decomposed the expectation of the product on the RHS into a product of expectations, which is admissible since the label $f$ is completely arbitrary and thus independent of both $\overline{\tau}$ and $N_K(t)$. Equation \eqref{neutral_example_for_variances} is a simple linear ODE and has the solution
\begin{equation}
\mathbb{E}[\sigma^2_f](t) = \sigma^2_f(0)e^{-\mathbb{E}\left[\frac{\overline{\tau}}{KN_K}\right]t}
\end{equation}
which tells us that the expected diversity (variance) of labels in the population decreases exponentially over time. The rate of loss is $\mathbb{E}\left[\overline{\tau}(KN_K)^{-1}\right]$, and thus, populations with higher mean turnover $\overline{\tau}$  and/or lower population size $KN_K$ lose diversity faster. This is because populations with higher $\overline{\tau}$ experience more stochastic events per unit time (a gambler's ruin type scenario), while extinction is `easier' in smaller populations because a smaller number of deaths is required to eliminate a label from the population completely. Note that \emph{which} labels/individuals are lost is entirely random (since all labels are arbitrary), but nevertheless, labels tend to be stochastically lost until only a small number of labels remain in the population.


\section{Weak-Noise approximation}
If the population is large (but finite) and the stochasticity is sufficiently weak, stochastic dynamics can be studied analytically using the weak noise approximation. Usually, this approximation is valid if we are studying a stochastic trajectory that is fluctuating about a point that is stable in the deterministic limit \citep{van_kampen_stochastic_1981}. Such situations occur often, since many systems of the forms studied here quickly relax to stable equilibria.


\section{Towards a stochastic evolutionary theory}

One striking feature that repeatedly shows up in our derivations is that finite populations exhibit phenomena that are not visible in infinite population models. For example, in both the stochastic logistic equation \ref{ex_1D_stoch_logistic_BD_eqns} and in two-strategy games with finite population sizes \citep{tao_stochastic_2007}, demographic noise ensures that all populations are guaranteed to go extinct given enough time, even if the deterministic limit predicts a stable state far from extinction. In the case of quantitative traits, demographic noise can hinder adaptive diversification by increasing the time before evolutionary branching occurs \citep{claessen_delayed_2007, wakano_evolutionary_2013, debarre_evolutionary_2016}, causing stochastic extinction of existing evolutionary branches \citep{rogers_demographic_2012, johansson_will_2006}, or preventing branching altogether if the population is too small \citep{rogers_modes_2015, johnson_two-dimensional_2021}. Stochastic systems also routinely exhibit evolution towards attractors that cannot be attained in the deterministic limit \citep{delong_stochasticity_2023}, sometimes even completely reversing the direction of evolution predicted by deterministic dynamics \citep{constable_demographic_2016,mcleod_social_2019}. Since real-life populations are stochastic and finite, it is thus imperative that modellers work with stochastic first-principles models instead of their deterministic limits, lest they risk missing important phenomena that are unique to stochastic systems \citep{black_stochastic_2012,schreiber_does_2022,hastings_transients_2004,shoemaker_integrating_2020}. In the context of our models, we have seen that if we observe the change in trait frequencies instead of the change in densities, finite populations are subject to an additional evolutionary force that vanishes in infinite population models.

