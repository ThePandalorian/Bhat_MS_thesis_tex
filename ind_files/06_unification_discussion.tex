\epigraph{\justifying The grand aim of all science [is] to cover the greatest number of empirical facts by logical deduction from the smallest number of hypotheses or axioms}{Albert Einstein}

\justifying
In this thesis, we have seen how stochastic birth-death processes can be used to construct and analyze mechanistic individual-based models for the dynamics of finite populations. In doing so, we have also seen that various well-known equations of evolutionary dynamics can be recovered in the infinite population size limit, sometimes also called the `macroscopic' description. In the finite-dimensional case (corresponding to discrete trait variants), the macroscopic descriptions are the equations of population genetics and evolutionary game theory. In the infinite-dimensional case, they are the equations of quantitative genetics, and, in some further limits, adaptive dynamics. In both cases, the mean value of the trait in the population changes according to an equation resembling the Price equation. This highlights the natural connections between these various equations - For example, the same procedures that lead to the replicator-mutator equation in the case of discretely varying traits yield Kimura's model in the quantitative case, underscoring the broad similarities between evolutionary game theory and quantitative genetics. The major formulations are summarized in Table \ref{table_summary}.

{\centering\begin{sideways}
    \begin{minipage}{\textheight}
        \resizebox{\textheight}{!}{%
            \setstretch{1.5}
            \begin{tabular}{ %I manually specified the width of each column by trial and error
  |p{\dimexpr.25\linewidth-2\tabcolsep-1.3333\arrayrulewidth}% column 1
  |p{\dimexpr.27\linewidth-2\tabcolsep-1.3333\arrayrulewidth}% column 2
  |p{\dimexpr.33\linewidth-2\tabcolsep-1.3333\arrayrulewidth}% column 3
  |p{\dimexpr.25\linewidth-2\tabcolsep-1.3333\arrayrulewidth}% column 4
  |p{\dimexpr.4\linewidth-2\tabcolsep-1.3333\arrayrulewidth}|% column 5
  }
            \hline
         \centering \textbf{Number of possible distinct trait variants ($m$)} & \centering \textbf{State Space} &
\centering \textbf{Model parameters} & \centering \textbf{Mesoscopic description} & \centering\arraybackslash \textbf{Infinite population limit} \\
        \hline
        $m = 1$ \newline (Identical individuals)  & $[0,1,2,3,\ldots]$ \newline (Population size) & Two real-valued functions, $b(N)$ and $d(N)$, describing the birth and death rate of individuals when the population size is $N$ & Univariate Fokker-Planck equation \newline (one-dimensional SDEs) & Single species population dynamics\\ 
        \hline
        $1 < m < \infty$ \newline (Discrete traits) & $[0,1,2,3,\ldots]^{m}$ \newline (Number of individuals of each trait variant) & $2m$ real-valued functions, $b_i(\mathbf{v})$ and $d_i(\mathbf{v})$ (for $1 \leq i \leq m$) describing the birth and death rate of trait variant $i$ when the population is $\mathbf{v}$ & Multivariate Fokker-Planck equation \newline ($m$-dimensional SDEs) &  Evolutionary game theory \newline
        Lotka-Volterra competition \newline Quasispecies equation \newline Price equation (discrete traits) \\
        \hline
        $m = \infty$ \newline (Quantitative traits) & $\left\{\sum\limits_{i=1}^{n}\delta_{x_i} \ | \ n \in \mathbb{N}\right\}$ \newline \newline (Each Dirac mass $\delta_{x_i}$ is an individual with trait value $x_i$) & Two real-valued functionals $b(x|\nu)$ and $d(x|\nu)$ describing the birth and death rate of trait variant $x$ when the population is $\nu$ & Functional Fokker-Planck equation \newline (SPDEs/Field equations) & Kimura's continuum-of-alleles model \newline \cite{sasaki_oligomorphic_2011}'s Oligomorphic Dynamics \newline \cite{wickman_theoretical_2022}'s Trait Space Equations for intraspecific trait variation \newline Adaptive Dynamics  \newline Price equation (quantitative traits)\\
        \hline
            \end{tabular}
        }
        %\renewcommand\thetable{1}
        \captionof{table}{Summary of the various birth-death processes studied in this chapter}
        \label{table_summary}
    \end{minipage}
\end{sideways}\par}

Equation \eqref{nD_eqn_for_frequencies}, which we derived in chapter \ref{chap_BD}, is a very general equation for how frequencies change over time in stochastic populations. To recap, we derived the equation:
\begin{equation}
\label{nD_stochastic_RM}
\begin{aligned}
dp_i(t) &= \underbrace{\left[(w_i(\mathbf{x}) - \overline{w})p_i + \mu\left\{Q_i(\mathbf{p}) - p_i\left(\sum\limits_{j=1}^{m}Q_j(\mathbf{p})\right)\right\}\right]dt}_{\substack{\text{Infinite population predictions} \\ \text{(Replicator-mutator, Price, etc.)}}}\\
&- \frac{1}{K}\underbrace{\frac{1}{N_{K}(t)}\left[(\tau_i(\mathbf{x}) - \overline{\tau})p_i + \mu\left\{Q_i(\mathbf{p}) - p_i\left(\sum\limits_{j=1}^{m}Q_j(\mathbf{p})\right)\right\}\right]dt}_{\substack{\text{Directional finite population effects}\\\text{due to differential turnover rates}}}\\
&+ \frac{1}{\sqrt{K}}\underbrace{\left[\left(A^{+}_{i}\right)^{1/2}dB^{(i)}_t - p_i\sum\limits_{j=1}^{m}\left(A^{+}_{j}\right)^{1/2}dB^{(j)}_t\right]}_{\substack{\text{Non-directional finite population effects}\\\text{due to stochastic fluctuations}}}
\end{aligned}
\end{equation}
This equation is in `replicator-mutator' form, and letting $K \to \infty$ recovers the replicator-mutator equation. If we wish, we could also express the same equation in `Price' form by calculating how the statistical mean value of a type level quantity changes over time. Let $f$ be any type level quantity, with constant value $f_i$ for the $i\textsuperscript{th}$ type. By multiplying both sides of
equation \eqref{nD_stochastic_RM} by $f_i$ and summing over all $i$ (essentially the same steps as going from \eqref{nD_replicator_mutator} to \eqref{nD_Price}), we see that the statistical mean $\overline{f}$ of the quantity varies as:
\begin{equation}
\label{nD_stochastic_Price}
\begin{aligned}
d\overline{f} &= \textrm{Cov}(w,f)dt - \frac{1}{KN_K(t)}\textrm{Cov}(\tau,f)dt\\
&+ \mu\left(1-\frac{1}{KN_K(t)}\right)\left(\sum\limits_{i=1}^{m}f_iQ_i(\mathbf{p}) - \overline{f}\sum\limits_{i=1}^{m}Q_i(\mathbf{p})\right)dt\\
&+ \frac{1}{\sqrt{K}}\left(\sum\limits_{i=1}^{m}f_i\sqrt{A_i^+}dB_{t}^{(i)} - \overline{f}\sum\limits_{i=1}^{m}\sqrt{A_i^+}dB_{t}^{(i)}\right)
\end{aligned} 
\end{equation}
where all covariances are understood in the statistical sense (Note that since $w_i$, $\tau_i$, $\overline{w}$, and $\overline{\tau}$ are stochastic processes depending on $\mathbf{p}$, the terms $ \textrm{Cov}(w,f)$ and $\textrm{Cov}(\tau,f)$ are themselves stochastic processes). Taking $K \to \infty$ in equation \eqref{nD_stochastic_Price} recovers the standard Price equation as the infinite population limit. In some sense, \eqref{nD_stochastic_RM} and \eqref{nD_stochastic_Price} are therefore `fundamental' equations for the evolution of finite populations, with the replicator-mutator and Price equations as the infinite population limits respectively. In both equations, the first term represents the direct effects of forces captured in classic deterministic models - the selection and mutation are partitioned into two terms in \eqref{nD_stochastic_Price}, but since mutation acts symmetrically in both the first and second term of \eqref{nD_stochastic_RM}, this has no major bearing on our interpretations. As is evident in both \eqref{nD_stochastic_RM} and \eqref{nD_stochastic_Price}, finite populations experience a new directional force dependent on $\tau_i(\mathbf{x}) = b^{\textrm{(ind)}}_{i}(\mathbf{x}) + b^{\textrm{(ind)}}_{i}(\mathbf{x})$, the per-capita turnover rate of type $i$, that cannot be captured in infinite population models. Remarkably, this term acts in a way that is mathematically exactly analogously to the classical evolutionary force of selection and mutation as captured by the first term in \eqref{nD_stochastic_RM}, but in the opposite direction - A higher relative $\tau_i$ leads to a decrease in frequency (Notice the minus sign before the second term in \eqref{nD_stochastic_RM}). Biologically, this term describes a biasing effect due to differential turnover rates and can intuitively be understood as being similar to the gambler's ruin through the following reasoning: If a type $i$ has a higher $\tau_i$, it experiences greater turnover due to a generally higher birth and death rate and thus experience more births and deaths in a given time interval than an otherwise equivalent species with a lower $\tau_i$. More events mean greater demographic stochasticity, and types with a higher $\tau_i$ thus tend to be eliminated by a stochastic analog of selection because they experience more chance events (births and deaths) in a given time period. This effect is less visible if the total population size is higher because larger populations generally experience less stochasticity, which is reflected in the $1/N_K$ factor in this term. This stochastic analog of selection for reduced turnover rates, captured by the second term of equation \eqref{nD_stochastic_RM}, is the force responsible for the `reversal of the direction of deterministic selection' induced by demographic noise in previous studies \citep{houchmandzadeh_selection_2012, houchmandzadeh_fluctuation_2015, constable_demographic_2016, mcleod_social_2019}. Note that types that tend to increase the \emph{total} population size $N_K(t)$ (such as altruists in evolutionary theory and mutualists in ecological communities) will reduce the magnitude of this effect compared to types that do not facilitate such an increase, such as cheaters and highly competitive species, which could explain why this effect preferentially favors the former types in reversing the direction of deterministic selection in finite population models \citep{houchmandzadeh_fluctuation_2015, mcleod_social_2019}. The fact that total population size controls the strength of the second term of \eqref{nD_stochastic_RM} also explains why cooperation is favoured in the early transient period of population growth \citep{melbinger_evolutionary_2010} when simulations are initiated from a small population size - In the early transient period, $N_K(t)$ is small, and the biasing effect of differential turnover rates is stronger, thus favouring cooperation. More generally, selection for reduced turnover rate could explain why cooperation often persists for a long time in finite population IbMs (and the real world) despite infinite population models predicting their extinction. The fact that the entire term is multiplied by $(KN_K(t))^{-1}$ suggests that the effect of this force is weak for medium to large populations, which explains why the persistence of cooperators is often only observed in restrictive sounding conditions such as quasi-neutrality, rapid attraction to a slow manifold, or a weak selection + weak mutation limit. In all three of these cases, the first term on the RHS of \eqref{nD_eqn_for_frequencies} becomes identically 0. It therefore no longer contributes to the total dynamics and allows us to see the contributions of the second term.\\
If the population is large (but finite) and the stochasticity is sufficiently weak, stochastic dynamics can be studied analytically using the weak noise approximation. Usually, this approximation is valid if we are studying a stochastic trajectory that is fluctuating about a point that is stable in the deterministic limit \citep{van_kampen_stochastic_1981}. Such situations occur often, since many systems of the forms studied here quickly relax to stable equilibria.
\clearpage

