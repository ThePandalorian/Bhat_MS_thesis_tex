In chapter \ref{chap_unification}, we arrived at three stochastic differential equations (equations \eqref{nD_stochastic_RM}, \eqref{nD_stochastic_Price}, and  \eqref{nD_stochastic_Price_variance}) for the frequency of a type, the population mean value of a type-level quantity, and the population variance of a type-level quantity. All three of these equations contained stochastic fluctuation terms which were of the form of a sum of stochastic integrals of several independent functions against independent Wiener processes. In this appendix, I will derive a more elegant representation of these terms via a useful lemma. I am sure this lemma is well-known, but I could not easily find a reference for it and soon realized it is quicker to prove it myself than find the appropriate reference text, and so I present a short proof below.

\begin{lemma}
Let $m \in \mathbb{N}$. Let $W^{(1)}_t, W^{(2)}_t, \ldots, W^{(m)}_t$ be $m$ independent one-dimensional Wiener processes. Let $g_1(x), g_2(x), \ldots, g_m(x)$ be $m$ `nice' ($L^2(\mathbb{R})$, Lipschitz, \emph{etc}.) real functions. Let
\begin{equation*}
	dX_t = \sum\limits_{i=1}^{m} g_i(X_t)dW^{(i)}_t
\end{equation*}
Then, we can always find a \emph{single} one-dimensional Wiener process $W_t$ (on the same probability space) such that
\begin{equation*}
	dX_t = \left(\sum\limits_{i=1}^{m} g^2_i(X_t)\right)^{1/2}dW_t
\end{equation*}
\end{lemma}
\begin{proof}
It suffices to prove the $m=2$ case.\\
Let $dX_t$ = $g_1(X_t) dW^{(1)}_t + g_2(X_t)dW^{(2)}_t$. Let us consider the \emph{two}-dimensional process $\mathbf{W}_t = [W^{(1)}_t, W^{(2)}_t]^{\mathrm{T}}$ on $\mathbb{R}^2$. Let us define a function $G:\mathbb{R} \to \mathbb{R}^2$ as
\begin{equation}
\label{lemma_G_defn}
G(x) = \frac{1}{\sqrt{g_1^2(x) + g_2^2(x)}}\begin{bmatrix}g_1(x) \\ g_2(x)\end{bmatrix}
\end{equation}
Now, by definition, we have
\begin{equation}
	\label{lemma_G_integral}
\int\limits_{0}^{t}G(X_s) \cdot d\mathbf{W}_s = \int\limits_{0}^{t}\frac{g_1(X_s)}{\sqrt{g_1^2(X_s) + g_2^2(X_s)}} dW^{(1)}_t + \int\limits_{0}^{t}\frac{g_2(X_s)}{\sqrt{g_1^2(X_s) + g_2^2(X_s)}}dW^{(2)}_s
\end{equation}
By a simple corollary of the It\^o isometry, we can calculate the quadratic variation of $\int G \cdot d\mathbf{W}$ as
\begin{equation}
\bigg\langle \int\limits G(X_s) \cdot d\mathbf{W}_s \bigg\rangle_t = \int\limits_{0}^{t} \bignorm{ G(X_s) }^2 d \langle \mathbf{W} \rangle_s 
= \int\limits_{0}^{t}  \frac{1}{g_1^2 + g_2^2} \cdot (g_1^2 + g_2^2) ds = \int\limits_{0}^{t} ds  = t \label{lemma_ito_isometry_step}
\end{equation}
Further, since $\int G \cdot d\mathbf{W}$ is a stochastic integral, the process $M_t = \int_0^t G(X_s) \cdot d\mathbf{W}_s$ is guaranteed to be a continuous square-integrable martingale.  But, by L\'evy's characterization of Brownian motion, the only continuous martingale $M_t$ that satisfies $\langle M \rangle_t = t$ is the Brownian motion. Thus, from equation \eqref{lemma_ito_isometry_step}, we are led to conclude that there is a one-dimensional Wiener process $W_t$ on the same probability space such that we can write
\begin{equation}
	\label{lemma_G_is_W}
	G(X_t) \cdot d\mathbf{W}_t = dW_t
\end{equation}
We can now use equation \eqref{lemma_G_integral} on the LHS of equation \eqref{lemma_G_is_W} to write
\begin{equation}
	\frac{g_1(X_t)}{\sqrt{g_1^2(X_t) + g_2^2(X_t)}} dW^{(1)}_t + \frac{g_2(X_t)}{\sqrt{g_1^2(X_t) + g_2^2(X_t)}}dW^{(2)}_t = dW_t
\end{equation}
By definition of our original process $X_t$, we can now conclude that
\begin{equation}
	dX_t = \sqrt{g_1^2(X_t) + g_2^2(X_t)}dW_t
\end{equation}
thus completing the proof.
\end{proof}

Using this lemma, we can now calculate the stochastic integral terms of our equations. For equation \eqref{nD_stochastic_RM}, the stochastic analog of the replicator-mutator equation, we can use this lemma and the functional form $A_i^+ = x_i\tau_i(\mathbf{x}) + \mu Q_i(\mathbf{x})$ to find that the noise term can be written as a stochastic integral against a single Wiener process $W_t$ is given by
\begin{equation}
\label{nD_stoch_RM_noise_term_alt_representation}
\frac{1}{\sqrt{KN_{K}(t)}}\left[p_i(1-p_i)\tau_i + p_i^2\left(\sum\limits_{j\neq i}\tau_j p_j\right) + \mu\left\{(1-p_i)^2Q_i(\mathbf{p}) + p_i^2 \sum\limits_{j\neq i}Q_j(\mathbf{p})\right\}\right]^{1/2}dW_t
\end{equation}
For equation \eqref{nD_stochastic_Price}, the stochastic analog of the Price equation, we have:
\begin{equation}
dW_{\overline{f}} = \sum\limits_{i=1}^{m}\left(f_i-\overline{f}\right)\sqrt{A_i^+}dW_{t}^{(i)} = \left(\sum\limits_{i=1}^{m}\left(f_i-\overline{f}\right)^2A_i^+\right)^{1/2}dW_t
\end{equation}
where $dW_t$ is now a single one-dimensional Wiener process. This is precisely the term calculated in equation \eqref{intermediate_2_for_variances} (barring the $1/KN_K^2$ pre-factor), and thus the stochastic term for the mean value is given by:
\begin{equation}
	\label{nD_stoch_Price_noise_term_alt_representation}
	dW_{\overline{f}} = \sqrt{N_K(t) \left(\textrm{Cov}(\tau,\left(f - \overline{f}\right)^2) + \overline{\tau}\sigma^2_{f} +\mu \sum\limits_{i=1}^{m}\left(f_i - \overline{f}\right)^2Q_i(\mathbf{p})\right)}dW_t
\end{equation}

Similarly, for the variance equation \eqref{nD_stochastic_Price_variance}, we can use our lemma to write
\begin{equation}
dW_{\sigma^2_f} = \sum\limits_{i=1}^{m}\left(f_i - \overline{f}\right)^2\sqrt{A_i^+}dW_{t}^{(i)} = \left(\sum\limits_{i=1}^{m}\left(f_i - \overline{f}\right)^4 A_i^+\right)^{1/2}dW_t
\end{equation}
where $dW_t$ is now a single one-dimensional Wiener process. A calculation exactly analogous to that done in obtaining \eqref{intermediate_2_for_variances} reveals that this term can be written as
\begin{equation}
	\label{nD_stoch_variance_Price_noise_term_alt_representation}
	dW_{\sigma^2_f} = \sqrt{N_K(t) \left(\textrm{Cov}(\tau,\left(f - \overline{f}\right)^4) + \overline{\tau}(\sigma^2_{f})^2 +\mu \sum\limits_{i=1}^{m}\left(f_i - \overline{f}\right)^4Q_i(\mathbf{p})\right)}dW_t
\end{equation}