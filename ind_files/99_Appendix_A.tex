Here, I present a simple (informal) derivation of the Fokker-Planck equation (FPE) for a one-dimensional It\^{o} process. The result for the multi-dimensional case follows from the same logic but is more notationally cumbersome.

Consider a one-dimensional real It\^{o} process given by $dX_t = \mu(X_t,t)dt + \sigma(X_t,t)dB_t$ on a filtered probability space $\Omega \subseteq \mathbb{R}$ that admits a probability density function $P(x,t)$ on $\Omega \times [0,\infty)$. Let $f:\mathbb{R}\to\mathbb{R}$ be an arbitrary $C^2(\mathbb{R})$ function. By It\^{o}'s formula, we have:
\begin{align*}
    df(X_t) = \left(\mu f' + \frac{\sigma^2}{2}f''\right)dt + \sigma f' dB_t
\end{align*}
Writing this in integral form and taking expectations on both sides yields:
\begin{align}
\label{eq_expectation_Ito}
    \mathbb{E}[f(X_t)] = \mathbb{E}\left[\int\limits_{0}^{t}\left(\mu f' + \frac{\sigma^2}{2}f''\right)ds\right] + \mathbb{E}\left[\int\limits_{0}^{t}\sigma f' dB_s\right]
\end{align}
As long as $X_t$ and $\sigma(X_t,t)$ are reasonably `nice', the stochastic integral in the second term of the RHS of \eqref{eq_expectation_Ito} will vanish upon taking an expectation (see \ref{sec_math_background}). Using the definition of the expectation value, we are thus left with:
\begin{equation*}
    \int\limits_{\Omega}f(X_t)P(x,t)dx = \int\limits_{\Omega}\left(\int\limits_{0}^{t}\mu f' + \frac{\sigma^2}{2}f''ds\right)P(x,t)dx
\end{equation*}
Assuming derivatives and expectations commute, we can now differentiate with respect to time on both sides and use the fundamental theorem of calculus to write
\begin{equation}
\label{eq_Ito_to_FPE_for_parts}
\int\limits_{\Omega}f(X_t)\frac{\partial P}{\partial t}(x,t)dx = \underbrace{\int\limits_{\Omega}\mu f'P(x,t)dx}_{M(x,t)} + \underbrace{\int\limits_{\Omega}\frac{\sigma^2}{2}f''P(x,t)dx}_{N(x,t)}
\end{equation}
We will now use integration by parts to further evaluate the two parts $M(x,t)$ and $N(x,t)$. Recall that the general formula for integration by parts is given by:
\begin{equation*}
    \int\limits_{\Omega}u_{x_i}vdx = -\int\limits_{\Omega}uv_{x_i}dx + \int\limits_{\partial\Omega}uv\gamma_{i}dS(x)
\end{equation*}
where subscript indicates differentiation and $\gamma$ is the unit outward normal. In our case, assuming that $P(x,t) \equiv 0$ on $\partial \Omega$, the boundary term (second term of the RHS) vanishes and we can use integration by parts once on $M(x,t)$ to obtain
\begin{equation}
\label{eq_Ito_to_FPE_I_term}
    M(x,t) = - \int\limits_{\Omega}f(X_t)\left(\frac{\partial}{\partial x}\mu P(x,t)\right)dx
\end{equation}
and twice on $N(x,t)$ to obtain
\begin{align}
    N(x,t) &= - \frac{1}{2}\int\limits_{\Omega}f'(X_t)\left(\frac{\partial}{\partial x}\sigma^2 P(x,t)\right)dx\nonumber\\
    &= \frac{1}{2}\int\limits_{\Omega}f(X_t)\left(\frac{\partial^2}{\partial x^2}\sigma^2 P(x,t)\right)dx\label{eq_Ito_to_FPE_J_term}
\end{align}
Substituting \eqref{eq_Ito_to_FPE_I_term} and \eqref{eq_Ito_to_FPE_J_term} into \eqref{eq_Ito_to_FPE_for_parts} and collecting terms yields
\begin{equation*}
    \int\limits_{\Omega}f(X_t)\frac{\partial P}{\partial t}(x,t)dx = \int\limits_{\Omega}f(X_t)\left[-\frac{\partial}{\partial x}(\mu P(x,t)) + \frac{1}{2}\frac{\partial^2}{\partial x^2}(\sigma^2P(x,t))\right]dx
\end{equation*}
Since this is true for an arbitrary choice of $f(x)$, we are thus led to conclude that the density function $P(x,t)$ must satisfy:
\begin{equation}
\label{FPE}
\frac{\partial P}{\partial t}(x,t) =-\frac{\partial}{\partial x}\left(\mu(x,t) P(x,t)\right) + \frac{1}{2}\frac{\partial^2}{\partial x^2}\left((\sigma(x,t))^2P(x,t)\right)
\end{equation}
Equation \eqref{FPE} is the Fokker-Planck equation in one dimension. Using the exact same strategy, the multidimensional Fokker-Planck equation for the $m$-dimensional It\^{o} Process $d\mathbf{X}_t = \boldsymbol{\mu}(\mathbf{X}_t,t)dt + \boldsymbol{\sigma}(\mathbf{X}_t,t)d\mathbf{B}_t$ can be easily found to be:
\begin{equation}
\label{app_FPE_ndim}
\frac{\partial P}{\partial t}(\mathbf{x},t) =-\sum\limits_{i=1}^{m}\frac{\partial}{\partial x_i}\left(\mu_i(\mathbf{x},t) P(\mathbf{x},t)\right) + \frac{1}{2}\sum\limits_{i=1}^{m}\sum\limits_{j=1}^{m}\frac{\partial^2}{\partial x_i \partial x_j}\left(D_{ij}P(\mathbf{x},t)\right)
\end{equation}
where $\mathbf{D} = \mathbf{\sigma}\mathbf{\sigma}^T$. Some authors like to define the `probability current' $\mathbf{J}(\mathbf{x},t)$, an $m$-dimensional function with $i\textsuperscript{th}$ element
\begin{equation*}
J_i(\mathbf{x},t) \coloneqq \mu_i(\mathbf{x},t)P(\mathbf{x},t) - \frac{1}{2}\sum\limits_{j=1}^{m}\frac{\partial}{\partial x_j}\left(D_{ij}P(\mathbf{x},t)\right)
\end{equation*}
In this notation, equation \eqref{app_FPE_ndim} can be written in the more compact form:
\begin{equation}
\label{app_FPE_ndim_continuity_eqn}
\frac{\partial P}{\partial t}  + \div{\mathbf{J}} = 0
\end{equation}
where $\div$ is the divergence operator (also denoted $\mathrm{div}$ in some texts). Those familiar with physics should immediately recognize that equation \eqref{app_FPE_ndim_continuity_eqn} is in the form of a so-called `continuity equation' for a conserved quantity. Continuity equations turn up in various areas of applied mathematics, most famously in electromagnetism (conservation of charge from Maxwell's equations), fluid dynamics (continuity equations for mass of a flowing fluid from the Euler equations), and molecular diffusion (Fick's law). This explains the name `probability current' as an analogy to currents in physics such as electrical current or fluid current. The continuity equation representation also  makes it clear that the Fokker-Planck equation describes the `flow of probability' in the system. In particular, equation \eqref{app_FPE_ndim_continuity_eqn} says that the total probability in the system is `conserved', and is thus simply a mathematical formalization of the common-sense idea that whenever the probability of the system of being in a given state decreases, the probability of it being in some other state must increase (and vice versa).