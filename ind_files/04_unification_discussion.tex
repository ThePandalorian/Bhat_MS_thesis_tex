\epigraph{\justifying The grand aim of all science [is] to cover the greatest number of empirical facts by logical deduction from the smallest number of hypotheses or axioms}{Albert Einstein}
%\epigraph{\justifying Not only is algebraic reasoning exact; it imposes an exactness on the verbal postulates made before algebra can start which is usually lacking in the first verbal formulations of scientific principles.}{J.B.S. Haldane}
\justifying
In this thesis, we have seen how stochastic birth-death processes can be used to construct and analyze mechanistic individual-based models for the dynamics of finite populations. In doing so, we have also seen that various well-known equations of evolutionary dynamics can be recovered in the infinite population size limit. In the finite-dimensional case, the infinite population limit corresponds to the equations of population genetics and evolutionary game theory. In the infinite-dimensional case, we instead obtain the equations of quantitative genetics, and, in some further limits, gradient dynamics. In both cases, the mean value of the trait in the population changes according to an equation resembling the Price equation. My derivation highlights the natural connections between the various equations of population dynamics - For example, the same procedures that lead to the replicator-mutator equation in the case of discretely varying traits yield Kimura's model in the quantitative case, underscoring the broad similarities between evolutionary game theory and quantitative genetics and extending known similarities to the finite population case. The major formulations are summarized in Table \ref{table_summary}.

{\centering\begin{sideways}
    \begin{minipage}{\textheight}
        \resizebox{\textheight}{!}{%
            \setstretch{1.5}
            \begin{tabular}{ %I manually specified the width of each column by trial and error
  |p{\dimexpr.25\linewidth-2\tabcolsep-1.3333\arrayrulewidth}% column 1
  |p{\dimexpr.27\linewidth-2\tabcolsep-1.3333\arrayrulewidth}% column 2
  |p{\dimexpr.33\linewidth-2\tabcolsep-1.3333\arrayrulewidth}% column 3
  |p{\dimexpr.25\linewidth-2\tabcolsep-1.3333\arrayrulewidth}% column 4
  |p{\dimexpr.4\linewidth-2\tabcolsep-1.3333\arrayrulewidth}|% column 5
  }
            \hline
         \centering \textbf{Number of possible distinct types/trait variants ($m$)} & \centering \textbf{State Space} &
\centering \textbf{Model parameters} & \centering \textbf{Mesoscopic description} & \centering\arraybackslash \textbf{Infinite population limit} \\
        \hline
        $m = 1$ \newline (Identical individuals)  & $[0,1,2,3,\ldots]$ \newline (Population size) & Two non-negative functions, $b(N)$ and $d(N)$, describing the birth and death rate of individuals when the population size is $N$ & Univariate Fokker-Planck equation \newline (one-dimensional SDEs) & Dynamics of populations of identical individuals\\ 
        \hline
        $1 < m < \infty$ \newline (Discrete traits) & $[0,1,2,3,\ldots]^{m}$ \newline (Number of individuals of each trait variant) & $2m$ non-negative functions, $b_i(\mathbf{v})$ and $d_i(\mathbf{v})$ (for $1 \leq i \leq m$) describing the birth and death rate of trait variant $i$ when the population is $\mathbf{v}$ & Multivariate Fokker-Planck equation \newline ($m$-dimensional SDEs) &  Evolutionary game theory \newline
        Lotka-Volterra competition \newline Quasispecies equation \newline Price equation (discrete traits) \\
        \hline
        $m = \infty$ \newline (Quantitative traits) & $ \left\{\sum\limits_{i=1}^{n}\delta_{x_i} \ \bigg{|} \ n \in \mathbb{N}, \ x_i \in \mathcal{T} \subseteq \mathbb{R}\right\}$ \newline \newline (Each Dirac mass $\delta_{x_i}$ is an individual with trait value $x_i$ in the trait space $\mathcal{T}$) & Two non-negative functionals $b(x|\nu)$ and $d(x|\nu)$ describing the birth and death rate of trait variant $x$ when the population is $\nu$ & Functional Fokker-Planck equation/Field theory \newline (SPDEs) & Kimura's continuum-of-alleles model \newline~\cite{sasaki_oligomorphic_2011}'s Oligomorphic Dynamics \newline~\cite{wickman_theoretical_2022}'s Trait Space Equations for intraspecific trait variation \newline Gradient Dynamics \newline Price equation (quantitative traits)\\
        \hline
            \end{tabular}
        }
        %\renewcommand\thetable{1}
        \captionof{table}{Summary of the various birth-death processes studied in this thesis}
        \label{table_summary}
    \end{minipage}
\end{sideways}\par}
\clearpage
\section{Fundamental theorems of evolution in finite population}\label{sec_fun_theorems}
\subsection{The fundamental theorem for changes in type frequencies in the population}\label{sec_fun_theorem_freq}
Equation \eqref{nD_eqn_for_frequencies}, which we derived in chapter \ref{chap_BD}, is a very general equation for how frequencies change over time in stochastic populations. To recap, we started with a population which can contain up to $m$ different types of individuals, and used ecological arguments to posit the existence of a `system-size' parameter $K$ that leads to density-dependent growth and prevents the population from growing infinitely large. The population as a whole is characterized by a vector $\mathbf{x} = [x_1, \ldots, x_m]$ indexing the density (\emph{i.e.} number divided by $K$) of each type of individual. Changes of the population are through either birth or death of individuals. Each type has a per-capita birth rate $b^{\textrm{(ind)}}(\mathbf{x})$, a per-capita death rate $d^{\textrm{(ind)}}(\mathbf{x})$, and an additional term $\mu Q_{i}(\mathbf{x})$ representing mutational effects. All three of these functions depend on the density (and \emph{not} just the total number) of indidivuals of each type in the population, and may in general also be frequency-dependent. In the regime where $K$ is not too small (corresponding to `medium sized' populations), we identified two quantities, $w_i(\mathbf{x}) = b^{\textrm{(ind)}}_{i}(\mathbf{x}) - d^{\textrm{(ind)}}_{i}(\mathbf{x})$ and $\tau_i(\mathbf{x}) = b^{\textrm{(ind)}}_{i}(\mathbf{x}) + d^{\textrm{(ind)}}_{i}(\mathbf{x})$, the Malthusian fitness and per-capita turnover rate of the $i\textsuperscript{th}$ type respectively, that emerge as being important for trait frequency dynamics. In particular, we saw that $p_i$, the frequency of the $i\textsuperscript{th}$ type in the population, changes according to the equation:
\begin{equation}
\label{nD_stochastic_RM}
\begin{aligned}
dp_i(t) &= \underbrace{\left[(w_i(\mathbf{x}) - \overline{w})p_i + \mu\left\{Q_i(\mathbf{p}) - p_i\left(\sum\limits_{j=1}^{m}Q_j(\mathbf{p})\right)\right\}\right]dt}_{\substack{\text{Infinite population predictions: selection-mutation balance} \\ \text{for higher fitness}}}\\
&- \frac{1}{K}\underbrace{\frac{1}{N_{K}(t)}\left[(\tau_i(\mathbf{x}) - \overline{\tau})p_i + \mu\left\{Q_i(\mathbf{p}) - p_i\left(\sum\limits_{j=1}^{m}Q_j(\mathbf{p})\right)\right\}\right]dt}_{\substack{\text{Directional noise-induced effects: selection-mutation balance}\\\text{for lower turnover rates}}}\\
&+ \frac{1}{\sqrt{K}N_{K}(t)}\underbrace{\left[\left(A^{+}_{i}\right)^{1/2}dB^{(i)}_t - p_i\sum\limits_{j=1}^{m}\left(A^{+}_{j}\right)^{1/2}dB^{(j)}_t\right]}_{\substack{\text{Non-directional noise-induced effects}\\\text{due to stochastic fluctuations}}}
\end{aligned}
\end{equation}
where $N_K = \sum x_i$ is the total population size scaled by $K$ (and thus $KN_K$ is the total population size), $A_i^{+} = x_i\tau_i(\mathbf{x}) + \mu Q_i(\mathbf{x})$, and each $B^{(i)}_t$ is an independent one-dimensional standard Brownian motion. Equation \eqref{nD_stochastic_RM} is in `replicator-mutator' form, and letting $K \to \infty$ recovers the standard replicator-mutator equation in the infinite population limit. The first term represents the direct effects of forces captured in classic deterministic models, and reflects a selection-mutation balance. However, finite populations experience a new directional force dependent on $\tau_i(\mathbf{x})$, the per-capita turnover rate of type $i$, that cannot be captured in infinite population models~\citep{kuosmanen_turnover_2022}. Remarkably, this term acts in a way that is mathematically identical to the classical action of selection and mutation in infinite population models as captured by the first term in \eqref{nD_stochastic_RM}, but in the opposite direction - A higher relative $\tau_i$ leads to a decrease in frequency (Notice the minus sign before the second term in \eqref{nD_stochastic_RM}).
\subsection{The fundamental theorem for the mean value of a type-level quantity in the population}\label{sec_fun_theorems_mean}

We can now calculate how the statistical mean value of a type-level quantity changes over time. Let $f$ be any type level quantity, with value $f_i(t)$ for the $i\textsuperscript{th}$ type. We allow for the possibility of $f_i$ to vary over time. By multiplying both sides of
equation \eqref{nD_stochastic_RM} by $f_i$ and summing over all $i$ (The same steps as going from \eqref{nD_replicator_mutator} to \eqref{nD_Price_time_dependent}), we see that the statistical mean $\overline{f}$ of the quantity in the population varies as:
\begin{equation}
\label{nD_stochastic_Price}
\begin{aligned}
%The \vphantom is to vertically align the texts under the \underbraces
d\overline{f} &= {\underbrace{\vphantom{ \frac{1}{KN_K(t)} }\textrm{Cov}(w,f)dt}_{\substack{\text{Classical} \\ \text{selection}}}} \ - \ {\underbrace{\frac{1}{KN_K(t)}\textrm{Cov}(\tau,f)dt}_{\substack{\text{Noise-induced} \\ \text{selection}}}} \ +  \  {\underbrace{\vphantom{ \frac{1}{KN_K(t)} }\overline{\left(\frac{\partial f}{\partial t}\right)}dt}_{\substack{\text{Ecological timescale} \\ \text{feedbacks due to} \\ \text{time-dependence of $f_i$}}}}\\[15pt]
&+ \underbrace{\mu\left(1-\frac{1}{KN_K(t)}\right)\left(\sum\limits_{i=1}^{m}f_iQ_i(\mathbf{p}) - \overline{f}\sum\limits_{i=1}^{m}Q_i(\mathbf{p})\right)dt}_{\text{Transmission bias/mutational effects}}\\[15pt]
&+ \underbrace{\frac{1}{\sqrt{K}N_{K}(t)}\left(\sum\limits_{i=1}^{m}\left(f_i-\overline{f}\right)\sqrt{A_i^+}dB_{t}^{(i)}\right)}_{\text{Stochastic fluctuations}}
\end{aligned} 
\end{equation}
where all covariances are understood in the statistical sense (Note that since $w_i$, $\tau_i$, $\overline{w}$, and $\overline{\tau}$ are stochastic processes depending on $\mathbf{p}$, the terms $\textrm{Cov}(w,f)$ and $\textrm{Cov}(\tau,f)$ are themselves stochastic processes). Taking $K \to \infty$ in equation \eqref{nD_stochastic_Price} recovers the standard Price equation as the infinite population limit (either \eqref{nD_Price_time_dependent} or \eqref{nD_Price} based on whether $f_i$ varies with time). We saw in chapter \ref{chap_infD_processes} using field equations that very similar methods of attack to those used outlined in chapter \eqref{chap_BD} also hold for quantitative traits. For example, equation \eqref{cts_replicator_mutator} and \eqref{cts_price} are respectively exactly the infinite dimensional analogs of the deterministic replicator-mutator equation \eqref{nD_replicator_mutator} and the deterministic Price equation \eqref{nD_Price} when $f$ is the trait value. We may therefore expect to find equations similar to \eqref{nD_stochastic_RM} and \eqref{nD_stochastic_Price} for quantitative traits. Indeed, measure-theoretic tools have recently been used to rigorously show that an infinite-dimensional version of \eqref{nD_stochastic_Price} holds for one-dimensional quantitative traits when $f$ is the trait value and $b(x|\phi) \pm d(x|\phi)$ are Gaussian (see equation (21b) in~\cite{week_white_2021}). Equations \eqref{nD_stochastic_RM} and \eqref{nD_stochastic_Price} are thus fundamental theorems for the evolution of finite populations, with the replicator-mutator and Price equations as their respective infinite population limits (also see~\citep{rice_universal_2020} for a stochastic Price equation in a discrete-time setting).

Each term in equation \eqref{nD_stochastic_Price} lends itself to a simple biological interpretation. The first term, $\textrm{Cov}(w,f)$, is well-understood in the classical Price equation and represents the effect of natural selection in the infinite population setting. In the stochastic Price equation \eqref{nD_stochastic_Price}, the effects of the second term of 
\eqref{nD_stochastic_RM} decompose into a selection term $\textrm{Cov}(\tau,f)$ for reduced turnover rates and a transmission bias term that vanishes in the weak mutation ($\mu \to 0$) limit. Following~\cite{constable_demographic_2016} and ~\cite{week_white_2021}, we refer to the effect of the covariance term (the second term of equation \eqref{nD_stochastic_Price}) as \emph{noise-induced selection} since it occurs exactly analogously to classical natural selection (but for lower $\tau$) and is induced purely by the finiteness of the population. Since this evolutionary force is unique to finite populations and has therefore been overlooked in classical population genetics, it warrants some more detailed discussion. Biologically, the $\textrm{Cov}(\tau,f)$ term (with a negative sign) describes a biasing effect due to differential turnover rates and can intuitively be understood as being similar to gambler's ruin in probability theory through the following reasoning: If a type $i$ has a higher $\tau_i$, it experiences greater turnover due to a generally higher birth and death rate and thus experience more births and deaths in a given time interval than an otherwise equivalent species with a lower $\tau_i$. More events mean greater demographic stochasticity, and types with a higher $\tau_i$ thus tend to be eliminated by a stochastic analog of selection because they experience more chance events (births and deaths) in a given time period. This effect is less visible if the total population size is higher because larger populations generally experience less stochasticity, which is reflected in the $1/N_K$ factor in this term. This stochastic analog of selection for reduced turnover rates, captured by the second term of equation \eqref{nD_stochastic_RM}, is the force responsible for the `reversal of the direction of deterministic selection' induced by demographic noise in previous studies~\citep{houchmandzadeh_selection_2012, houchmandzadeh_fluctuation_2015, constable_demographic_2016, mcleod_social_2019}.Note that types that tend to increase the \emph{total} population size $KN_K(t)$ (such as altruists in evolutionary theory and mutualists in ecological communities) will reduce the magnitude of this effect compared to types that do not facilitate such an increase, such as cheaters and highly competitive species. Further, if altruists/mutualists act by reducing the death rate (rather than increasing the birth rate) of other individuals, their presence causes higher $w$ and lower $\tau$ in the beneficiary individuals, both of which are favored by selection (but note that if they act by increasing the birth rate, they increase the magnitude of negative noise-induced selection disfavoring the beneficiary individual), which explains why this effect preferentially favors mutualists in reversing the direction of deterministic selection in finite population models with fluctuating population sizes if interaction effects are on the death rate~\citep{mcleod_social_2019}. Thus, selection for reduced turnover rate could help explain why cooperation often persists in finite population IbMs of social evolution~\citep{houchmandzadeh_selection_2012,houchmandzadeh_fluctuation_2015,chotibut_evolutionary_2015,behar_fluctuations-induced_2016,mcleod_social_2019} despite infinite population models predicting their extinction. The fact that total population size controls the strength of noise-induced selection also explains why cooperation is favored in the early transient period of population growth~\citep{melbinger_evolutionary_2010} when simulations are initiated from a small population size --- In the early transient period, $N_K(t)$ is small, and the biasing effect of differential turnover rates is stronger, thus favoring cooperation. The fact that the entire term scales inversely with the total population size $KN_K(t)$ suggests that the effect of this force is weak for large populations, which explains why the persistence of cooperators is often only observed in restrictive sounding conditions such as quasi-neutrality, timescale separation, or a weak selection + weak mutation limit~\citep{mcleod_social_2019}. In all three of these cases, the first term on the RHS of \eqref{nD_eqn_for_frequencies} becomes identically 0. It therefore no longer contributes to the trait frequency dynamics, thus allowing us to see the (otherwise weak) contributions of the second term.

The third term of \eqref{nD_stochastic_Price} is relevant in both finite and infinite populations whenever $f_i$ can vary over time and represents feedback effects on the quantity $f_i$ of the $i\textsuperscript{th}$ species over short (`ecological') time-scales. Such feedback is often through a changing environment or phenotypic/behavioral plasticity, but other biological phenomena may also be at play. The fourth term of \eqref{nD_stochastic_Price} is a transmission bias term, with a correction factor due to noise-induced selection. Finally, the last term of \eqref{nD_stochastic_Price} describes the role of stochastic fluctuations. The contributions of this last term are `directionless' due to the $dB_t$ factors, and this term vanishes when we take a conditional expectation value over the underlying probability space. We denote this probabilistic expectation value operation by $\mathbb{E}[\cdot]$ to distinguish it from the statistical mean \eqref{nD_mean}. Note that this expectation is conditioned on the initial state of the population, and thus $\mathbb{E}[\cdot]$ is really shorthand for $\mathbb{E}[\ \cdot \ | \ \mathbf{X}_0 = \mathbf{x}_0]$.

Two particularly interesting implications of \eqref{nD_stochastic_Price} are realized upon ignoring mutations by setting $\mu = 0$ and then substituting $f=w$ and $f = \tau$. We first note that:
\begin{align}
\textrm{Cov}(w,\tau) &=\textrm{Cov}\left( b^{\textrm{(ind)}}(\mathbf{x}) - d^{\textrm{(ind)}}(\mathbf{x}) \ , \   b^{\textrm{(ind)}}(\mathbf{x}) + d^{\textrm{(ind)}}(\mathbf{x})\right)\\
&= \sigma^2_{b^{\textrm{(ind)}}(\mathbf{x})} - \sigma^2_{d^{\textrm{(ind)}}(\mathbf{x})}\label{nD_cross_covariance}
\end{align}
It is important to remember once again that just like the statistical mean, the statistical variance $\sigma^2_{f}(t)$ of a type-level quantity $f$ is a random variable obtained by calculating the variance of the quantity \emph{in the population} at time $t$, and is not to be confused with the probabilistic/ensemble variance obtained by calculating the variance of a quantity \emph{over different realizations} of the stochastic process (see section \ref{sec_stat_measures}). Upon substituting $f = w$ in \eqref{nD_stochastic_Price} and taking expectations over the underlying probability space, we obtain:
\begin{align}
\label{nD_stochastic_Fisher}
%The \vphantom is to vertically align the texts under the underbraces of different terms
\mathbb{E}\left[\frac{d\overline{w}}{dt}\right] &= 
\negmedspace {\underbrace{\mystrut{4ex} \ \vphantom{ \frac{\sigma^2_{b^{\textrm{(ind)}}} - \sigma^2_{d^{\textrm{(ind)}}}}{KN_K(t)} } \mathbb{E}\left[\sigma^2_{w}\right] \ }_{\substack{\text{Fisher's} \\ \text{Fundamental theorem}}}} \ - \ {\underbrace{\mystrut{4ex}\mathbb{E}\left[\frac{\sigma^2_{b^{\textrm{(ind)}}} - \sigma^2_{d^{\textrm{(ind)}}}}{KN_K(t)}\right]}_{\substack{\text{Noise-induced} \\ \text{selection}}}} \ + {\underbrace{\mystrut{4ex} \vphantom{ \frac{\sigma^2_{b^{\textrm{(ind)}}} - \sigma^2_{d^{\textrm{(ind)}}}}{KN_K(t)} } \mathbb{E}\left[\hphantom{a}\overline{\hphantom{a}\frac{\partial w}{\partial t}\hphantom{a}}\hphantom{a}\right]}_{\substack{\text{Short-term (ecological)} \\ \text{feedbacks to fitness}}}}
\end{align}
Taking $K \to \infty$ in \eqref{nD_stochastic_Fisher} recovers a well-known equation in population genetics upon noting that the process tends to a deterministic process as $K \to \infty$, as noted in section \ref{sec_nD_det_limit}, and thus the expectation value in the infinite population case is superfluous. The first term, $\sigma^2_w$, is the subject of Fisher's fundamental theorem~\citep{fisher_genetical_1930,  price_fishers_1972, frank_fishers_1992, kokko_stagnation_2021}. The second term of equation \eqref{nD_stochastic_Fisher} is a manifestation of noise-induced selection and vanishes in the infinite population limit, and is thus particular to finite populations. Finally, the last term arises in both finite and infinite populations whenever $w_i$ can vary over time~\citep{frank_fishers_1992,kokko_stagnation_2021,baez_fundamental_2021}, be it through frequency-dependent selection, phenotypic plasticity, varying environments, or other ecological mechanisms, and represents feedback effects on the fitness $w_i$ of the $i\textsuperscript{th}$ species over short (`ecological') time-scales. The fact that Fisher appears to have been rather vague and dismissive of this feedback~\citep{fisher_genetical_1930} has led to much discussion, debate, and confusion about the interpretation, importance, and implications of his `fundamental theorem' (see~\cite{kokko_stagnation_2021} and sources cited therein).

Carrying out the same steps with $f = \tau$ in \eqref{nD_stochastic_Price} yields a new equation/theorem due to~\cite{kuosmanen_turnover_2022} that has only recently been recognized as important. This theorem is an analog of Fisher's fundamental theorem for the turnover rates, and reads:
\begin{equation}
\label{nD_stochastic_Fisher_turnover}
\mathbb{E}\left[\frac{d\overline{\tau}}{dt}\right] = \underbrace{\vphantom{ \mathbb{E}\left[\frac{\sigma^2_{\tau}}{KN_K(t)}\right] } \mathbb{E}\left[\sigma^2_{b^{\textrm{(ind)}}} - \sigma^2_{d^{\textrm{(ind)}}}\right]}_{\substack{\text{Classical selection} \\ \text{effects}}} \ - \ \underbrace{ \mathbb{E}\left[\frac{\sigma^2_{\tau}}{KN_K(t)}\right]}_{\substack{\text{Noise-induced selection} \\ \text{effects}}} \ + \ \underbrace{ \vphantom{ \mathbb{E}\left[\frac{\sigma^2_{\tau}}{KN_K(t)}\right] } \mathbb{E}\left[\hphantom{a}\overline{\hphantom{a}\frac{\partial \tau}{\partial t}\hphantom{a}}\hphantom{a}\right]}_{\substack{\text{Short-term (ecological)} \\ \text{feedbacks to $\tau_i$}}}
\end{equation}
The implications of this theorem have been extensively discussed in~\citep{kuosmanen_turnover_2022}, which is where we refer the interested reader.

\subsection{The fundamental theorem for the variance of a type-level quantity in the population}\label{sec_fun_theorems_var}
Equation \eqref{nD_stochastic_Price} is a general equation for the mean value of an arbitrary type level quantity $f$ in the population. In many real-life situations, especially those pertaining to finite populations, we are interested in not just the mean, but also the variance of a type-level quantity. In Appendix \ref{App_stoch_var_eqns}, I show that the statistical variance of any type level quantity $f$ obeys
\begin{equation}
\label{nD_stochastic_Price_variance}
\begin{aligned}
d\sigma^2_{f} &= \textrm{Cov}\left(w,(f - \overline{f})^2\right)dt - \frac{1}{KN_K}\left[ \ \overline{\tau}\sigma^2_{f} +  2\textrm{Cov}\left(\tau,(f - \overline{f})^2\right) \ \right]dt\\[12pt]
& + 2\textrm{Cov}\left(\frac{\partial f}{\partial t},f\right)dt + M_{\sigma^2_f}(\mathbf{p},N_K)dt + \frac{1}{\sqrt{K}N_{K}(t)}dB_{\sigma^2_{f}}
\end{aligned}
\end{equation}
where
\begin{equation}
\label{variance_price_mutation_term}
M_{\sigma^2_f}(\mathbf{p},N_K) = \mu\left[\left(1 - \frac{2}{KN_K}\right)\left(\sum\limits_{i=1}^{m}(f_i - \overline{f})^2Q_i(\mathbf{p})\right) + \sigma^2_f\left(1 - \frac{1}{KN_K}\right)\sum\limits_{i=1}^{m}Q_i(\mathbf{p})\right]
\end{equation}
is a mutational term that vanishes in the $\mu \to 0$ limit and
\begin{equation}
\label{variance_price_diffusion_term}
dB_{\sigma^2_f} = \sum\limits_{i=1}^{m}\left(f_i - \overline{f}\right)^2\sqrt{A_i^+}dB_{t}^{(i)}
\end{equation}
is a stochastic integral term measuring the (non-directional) effect of stochastic fluctuations that vanishes upon taking an expectation over the probability space. In the case of one-dimensional quantitative traits, an infinite-dimensional version of \eqref{nD_stochastic_Price_variance} has recently been rigorously derived~\citep{week_white_2021} using measure-theoretic tools under certain additional assumptions (See equation (21c) in~\cite{week_white_2021}). Taking expectations over the probability space in \eqref{nD_stochastic_Price_variance} and substituting mutation as acting via a Gaussian kernel also recovers equations previously derived~\citep{debarre_evolutionary_2016} in the context of evolutionary branching in finite populations as a special case (Equation A.23 in~\cite{debarre_evolutionary_2016} is equivalent to equation \eqref{nD_stochastic_Price_variance} for their choice of functional forms upon converting their change in variance to an infinitesimal rate of change \emph{i.e.} a derivative). An infinite population ($K \to \infty$) version of equation \eqref{nD_stochastic_Price_variance} also appears in~\cite{lion_theoretical_2018}.

Once again, terms of equation \eqref{nD_stochastic_Price_variance} lend themselves to straightforward biological interpretation. The quantity $(f_i-\overline{f})^2$ is a measure of the distance of $f_i$ from the population mean value $\overline{f}$, and thus covariance with $(f-\overline{f})^2$ quantifies the type of selection operating in the system: A negative correlation is indicative of stabilizing selection, and a positive correlation is indicative of disruptive (\emph{i.e.} diversifying) selection. An extreme case of diversifying selection for fitness occurs if the mean fitness is at a local minimum for fitness but $f_i \not\equiv \overline{f}$ (\emph{i.e.} the population still exhibits some variation in $f$). In this case, if the variation in $f$ is associated with a variation in fitness, then $\textrm{Cov}(w,(f - \overline{f})^2)$ is strongly positive and the population experiences a sudden explosion in variance, causing the emergence of polymorphism in the population. If $\textrm{Cov}(w,(f - \overline{f})^2)$ is still positive after the initial emergence of multiple morphs, evolution eventually leads to the emergence of stable coexisting polymorphisms in the population - This phenomenon is a slight generalization of the idea of evolutionary branching that occurs in frameworks such as adaptive dynamics \citep{geritz_evolutionarily_1998}.

The $\textrm{Cov}\left(\partial f/\partial t,f\right)$ term once again represents the effect of eco-evolutionary feedback loops due to rapid change in $f$ that is not solely due to changes in $\mathbf{p}$. The $M_{\sigma^2_f}(\mathbf{p},N_K)$ term quantifies the effect of mutations on the variance of $f$. Note that each $Q_i(\mathbf{p}) \geq 0$ by its definition in \eqref{nD_functional_forms_for_replicator} and thus $\sum_i Q_i(\mathbf{p}) > 0$ if there are any mutational effects (and $=0$ otherwise). Furthermore, the total population size $KN_K > 2$ for most interesting evolutionary questions. Thus, from \eqref{variance_price_mutation_term}, it is clear that when $\mu > 0$ (\emph{i.e.} there is mutation in the system), we have $M_{\sigma^2_f}(\mathbf{p},N_K) > 0$, meaning that mutations always increase the variance of $f$ in the population.

The $\overline{\tau}\sigma^2_{f}$ term represents a loss of diversity due to stochastic extinctions (i.e. demographic stochasticity). To see this, it is instructive to consider the case in which this is the only force at play. Let us imagine a population of asexual organisms in which each $f_i$ is simply a label or mark arbitrarily assigned to individuals in the population at the start of an experiment/observational study and subsequently passed on to offspring - For example, a neutral genetic tag in a part of the genome that experiences a negligible mutation rate. Let us set $\mu = 0$ so that the labels cannot change between parents to offspring. This means that we have $M_{\sigma^2_f}(\mathbf{p},N_K) \equiv 0$. Further, since the labels are arbitrary and have no effect whatsoever on the biology of the organisms, we have $\textrm{Cov}\left(w,(f - \overline{f})^2\right) \equiv \textrm{Cov}\left(\tau,(f - \overline{f})^2\right) \equiv 0$. Since the labels do not change over time, we also have $\textrm{Cov}\left(\partial f/\partial t,f\right) = 0$. From \eqref{nD_stochastic_Price_variance}, we see that in this case, the variance changes as
\begin{equation}
d\sigma^2_f = - \frac{\overline{\tau}\sigma^2_{f}}{KN_K}dt + \frac{1}{\sqrt{K}N_{K}(t)}dB_{\sigma^2_{f}}
\end{equation}
On taking expectations, the second term on the RHS vanishes and we see that the expected variance in the population obeys
\begin{equation}
\label{neutral_example_for_variances}
\frac{d \mathbb{E}[\sigma^2_f]}{dt} = - \left(\mathbb{E}\left[\frac{\overline{\tau}}{KN_K}\right]\right)\mathbb{E}[\sigma^2_{f}]
\end{equation}
where we have decomposed the expectation of the product on the RHS into a product of expectations, which is admissible since the label $f$ is completely arbitrary and thus independent of both $\overline{\tau}$ and $N_K(t)$. Equation \eqref{neutral_example_for_variances} is a simple linear ODE and has the solution
\begin{equation}
\mathbb{E}[\sigma^2_f](t) = \sigma^2_f(0)e^{-\mathbb{E}\left[\frac{\overline{\tau}}{KN_K}\right]t}
\end{equation}
which tells us that the expected diversity (variance) of labels in the population decreases exponentially over time. The rate of loss is $\mathbb{E}\left[\overline{\tau}(KN_K)^{-1}\right]$, and thus, populations with higher mean turnover $\overline{\tau}$  and/or lower population size $KN_K$ lose diversity faster. This is because populations with higher $\overline{\tau}$ experience more stochastic events per unit time (a gambler's ruin type scenario), while extinction is `easier' in smaller populations because a smaller number of deaths is required to eliminate a label from the population completely. Note that \emph{which} labels/individuals are lost is entirely random (since all labels are arbitrary), but nevertheless, labels tend to be stochastically lost until only a small number of labels remain in the population.

\section{A stochastic field theory for quantitative traits}\label{sec_disc_field_eqns}

In chapter \ref{chap_infD_processes}, we formulated a `field equation' for the evolution of one-dimensional quantitative traits in populations. To recap, given a one-dimensional quantitative trait that takes values in a trait space $\mathcal{T} \subseteq \mathbb{R}$, we defined the set $\mathcal{M}(\mathcal{T}) =  \left\{\sum_{i=1}^{n}\delta_{x_i} \ | \ n \in \mathbb{N}, x_i \in \mathcal{T}\right\}$, where each $\delta_{x_i}$ is a Dirac mass centered at $x_i$. We then formulated a model in which the population at time $t$ is characterized as a whole by a randomly varying density distribution (`stochastic field') $\nu^{(t)}: \mathcal{M}(\mathcal{T}) \to [0,\infty)$ such that for any subset $A \subseteq \mathcal{T}$, the number of individuals that have trait value in $A$ is given by integrating $\nu^{(t)}$ over $A$. The change of this field is determined entirely by two functionals, $b(x|\nu)$ and $d(x|\nu)$ from $\mathcal{T}$ to $[0,\infty)$, that respectively describe the birth rate and death rate of individuals of type $x$ in a population $\nu$. Under the assumption that there exists a suitable system size parameter $K > 0$, we moved from the space of `number' distribution functions $\mathcal{M}(
\mathcal{T})$ to the space of `density' distribution functions $\mathcal{M}_{K}(\mathcal{T}) =  \left\{\sum_{i=1}^{n}\delta_{x_i}/K \ | \ n \in \mathbb{N}, x_i \in \mathcal{T}\right\}$ via a functional analog of the system size expansion. By appropriately rescaling the birth and death rate functionals, we then determined that $P(\phi,t)$, the probability that the population is described by the distribution $\phi \in \mathcal{M}_{K}(\mathcal{T})$ at time $t$, (approximately) satisfies the `stochastic field equation':
\begin{equation}
\label{disc_functional_field_eqns}
\resizebox{\textwidth}{!}{$\displaystyle
\frac{\partial P}{\partial t}(\phi,t) = \int\limits_{\mathcal{T}}\left[-
\frac{\delta}{\delta\phi(x)} \{ \left(\phi(x) w(x|\phi) + \mu Q(x|\phi)\right) P(\phi,t) \} + \frac{1}{2K}\frac{\delta^2}{\delta\phi(x)^2}\{\left(\phi(x) \tau(x|\phi) + \mu Q(x|\phi)\right)P(\phi,t)\}\right]dx$}
\end{equation}
where $w(x|\phi)$ and $\tau(x|\phi)$ are functionals that respectively describe the Malthusian fitness and per-capita turnover rate of type $x \in \mathcal{T}$ (now a continuous variable) in a population $\phi$. We then saw that this equation yields some well-known frameworks of quantitative genetics in the infinite population ($K \to \infty$) limit, thus illustrating consistency with known theories. If the `intuitive' version of It\^{o}'s formula for $\mathcal{M}_K(\mathcal{T})$ valued stochastic processes (obtained by `taking the limit' $m \to \infty$ in the It\^{o}'s formula for $m$-dimensional stochastic processes) holds, then the exact same steps carried out in Appendices \ref{App_density_to_freq} and \ref{App_stoch_var_eqns} will `go through' essentially unchanged for quantitative traits and yield the equations obtained by simply taking $m \to \infty$ in \eqref{nD_stochastic_RM}, \eqref{nD_stochastic_Price} and \eqref{nD_stochastic_Price_variance} (with sums replaced by integrals and derivatives replaced by functional derivatives as appropriate) as the `fundamental theorems' for the evolution of quantitative traits. Indeed,~\cite{week_white_2021} have recently proposed exactly the It\^o formula we would need via a heuristic It\^o multiplication table for $L^2(\mathbb{R}, m)$ valued processes with a.s. finite Hilbert-Schmidt norm and have shown that their heuristics are equivalent to the rigorous infinite-dimensional stochastic calculus proposed by~\cite{da_prato_stochastic_2014} for more general Hilbert space valued processes\footnote{If this sentence sounds like arcane gibberish to you, don't worry too much about the details - The essence is that~\cite{week_white_2021} have proven the `correct' formula we need holds whenever $b(x|\nu), d(x|\nu)$ and $\mathcal{M}_K(\mathcal{T})$ together fulfill certain technical requirements. My heuristics here are only a `first step' and have focused on accessibility over mathematical propriety, and thus make no attempt to verify whether and when these technical requirements are satisfied.}. Furthermore, using these `spacetime white noise heuristics' together with a particular functional form for (weak) solutions to certain SPDEs,~\cite{week_white_2021} have arrived at precisely the infinite-dimensional version of equations \eqref{nD_stochastic_Price} and \eqref{nD_stochastic_Price_variance} obtained by `taking the limit' $m \to \infty$ in \eqref{nD_stochastic_Price} and \eqref{nD_stochastic_Price_variance} from the `SPDE side'\footnote{If the well-known equivalence between SDEs and Fokker-Planck equations (see section \ref{sec_math_background}) carries through to infinite-dimensions, it becomes an equivalence between SPDEs and functional Fokker-Planck equations.} under certain additional assumptions required to ensure existence/uniqueness of solutions to the relevant SPDEs. Equation \eqref{disc_functional_field_eqns} can thus be argued to be the (functional) Fokker-Planck view of the stochastic processes for the evolutionary ecology of quantitative traits recently studied in~\cite{week_white_2021}. Like~\cite{week_white_2021}, I believe that rigorously establishing the precise criteria needed for general existence and uniqueness of solutions to equations of the form \eqref{disc_functional_field_eqns} (or the SPDE versions presented in~\cite{week_white_2021}, which resemble our equation \eqref{functional_langevin}) presents an important mathematical direction for future work that may yield new insights into both the biology of quantitative traits and the mathematics of measure-valued stochastic processes.

More generally, the intimate relation between It\^o SDEs and Fokker-Planck equations has been extensively exploited in the applied mathematics literature~\citep{van_kampen_stochastic_1981,oksendal_stochastic_1998, gardiner_stochastic_2009} since different problems are more appropriately attacked from one side than the other. In the infinite-dimensional case, the functional Fokker-Planck equation \eqref{disc_functional_field_eqns} (the `physics' side) allows us to attack questions of phenotypic clustering and evolutionary branching via powerful tools such as Fourier analysis (\cite{rogers_demographic_2012, rogers_modes_2015}; See also Appendix \ref{App_examples} for a general model-independent pedagogical derivation) that would have been significantly harder if one only used SPDEs. On the other hand, some questions are much more easily handled using tools from the `math' side of SPDEs such as infinitesimal generators and the martingale problem (see~\cite{champagnat_unifying_2006} or~\cite{week_white_2021}) rather than being forced to work with functional Fokker-Planck equations. Rigorously establishing a relation between functional Fokker-Planck equations of the form \eqref{disc_functional_field_eqns} and SPDEs of the form \eqref{functional_langevin} (or those studied in~\cite{week_white_2021}) for measure-valued branching processes could thus prove very fruitful for developing more integrative eco-evolutionary theory since it allows us to seamlessly transition between two alternative views of the same object. Importantly, the formalism I develop here in terms of functional Fokker-Planck equations likely does \emph{not} carry over to the study of higher dimensional quantitative traits (or populations which vary in two or more one-dimensional quantitative traits) because these processes are routinely badly behaved in higher dimensions: In particular, a (smooth) probability density function $P(\phi,t)$ frequently does not even exist in higher dimensions if one has any biologically non-trivial features such as interactions between types (\cite{fleming_measure-valued_1979, walsh_introduction_1986}; Also see for example~\cite{evans_measure-valued_1994}), making equation \eqref{disc_functional_field_eqns} entirely meaningless. My (admittedly limited) understanding is that such processes are also rather difficult to study in two or more dimensions from the SPDE side for similar reasons --- one is forced to work with distribution valued (rather than function valued) solutions and thus needs to use a considerable amount of advanced functional analysis to make progress~\citep{walsh_introduction_1986,carmona_stochastic_1999,balan_gentle_2018}.

To the best of my knowledge, a general formulation of stochastic field equations for the population dynamics of quantitative traits from the functional Fokker-Planck side in the manner I have carried out here has not been done before. Similar equations have been formulated for some specific stochastic models of quantitative trait evolution~\citep{rogers_demographic_2012,rogers_modes_2015} and population ecology of size-structured communities~\citep{odwyer_integrative_2009}. Stochastic field equations are also known in mathematical neurobiology~\citep{buice_field-theoretic_2007,bressloff_stochastic_2010,coombes_neural_2014}, and have recently been proposed in a model of collective motion~\citep{o_laighleis_minimal_2018}, suggesting that the study and formalization of such equations by pure mathematicians could have wide-spread applications. As is, such field equations are not on very solid mathematical footing, and are currently primarily used by physicists working in areas such as statistical field theory and are largely attacked using ingenious (and often heuristic) arguments and tools that are not necessarily mathematically well understood~\citep{carmona_stochastic_1999}. Equation \eqref{disc_functional_field_eqns} opens up the study of quantitative trait dynamics in finite fluctuating populations to analysis using various powerful tools of physics and stochastic processes such as duality~\citep{greenman_duality_2020}, the Fock space representation~\citep{dodd_many-body_2009, del_razo_probabilistic_2022}, and the path integral formalism~\citep{doi_second_1976, peliti_path_1985, dodd_many-body_2009, chow_path_2015, weber_master_2017}. Equation \eqref{disc_functional_field_eqns} also provides a more accessible alternative formulation that does not (explicitly) rely on tools from relatively `advanced' mathematical fields such as martingale theory and measure-valued processes that are typically used to study the evolution of quantitative traits in finite, fluctuating populations~\citep{dawson_stochastic_1975,fleming_measure-valued_1979,ethier_markov_1986,champagnat_unifying_2006,etheridge_mathematical_2011,week_white_2021}, thus hopefully making the field more accessible to theorists without formal training in mathematical areas such as measure theory and functional analysis.


\section{Discussion \& Outlook}\label{sec_disc}

Stochastic finite population models often exhibit behaviors that are markedly different from their deterministic limits. Since real-life populations are stochastic and finite, it is thus imperative that modellers work with stochastic first-principles models instead of their deterministic limits, lest they risk missing important phenomena that are unique to stochastic systems~\citep{black_stochastic_2012,schreiber_does_2022,hastings_transients_2004,shoemaker_integrating_2020}. In trying to work with stochastic models, several theorists have called for a reformulation of eco-evolutionary dynamics of finite populations starting from stochastic birth-death processes on the grounds that such a formulation is more fundamental and mechanistic~\citep{metcalf_why_2007,geritz_mathematical_2012,doebeli_towards_2017}. Part \ref{part_theory} of this thesis develops such a mechanistic theory for both populations of individuals that vary in some discrete character (chapter \ref{chap_BD}) as well as populations of individuals that vary in a single one-dimensional quantitative character (chapter \ref{chap_infD_processes}). The central result of this reformulation is an equation for change in type frequencies in the population (equation \eqref{nD_stochastic_RM}) that generalizes the replicator-mutator equation to finite, fluctuating, closed populations evolving in continuous time. From this, we can derive an equation for changes in the population mean of a type-level quantity (equation \eqref{nD_stochastic_Price}) that is a generalization of the Price equation to finite populations, as well as an equation for changes in population variance of a type-level quantity (equation \eqref{nD_stochastic_Price_variance}) in such populations. My work thus generalizes some fundamental formal structures of eco-evolutionary population dynamics to finite, fluctuating populations. In particular, my equations generalize the unifying formalism described in~\cite{lion_theoretical_2018}\footnote{which itself is a reformulation of the relatively well-known unification of eco-evolutionary dynamics via the Price equation~\citep{frank_natural_2012, queller_fundamental_2017, luque_mirror_2021} in a dynamically sufficient continuous time framework} to finite, fluctuating populations --- Taking $K \to \infty$ in equations \eqref{nD_stochastic_RM}, \eqref{nD_stochastic_Price}, and \eqref{nD_stochastic_Price_variance} recover equations (6), (11), and (14) in~\cite{lion_theoretical_2018} respectively as their infinite population limits. Several theorists have called for a reformulation of eco-evolutionary dynamics starting from stochastic birth-death processes on the grounds that such a formulation is more fundamental and mechanistic~\citep{metcalf_why_2007,geritz_mathematical_2012,doebeli_towards_2017}. My work provides some fundamental relations that any such reformulation must satisfy. These relations deal with biologically important quantities, lend themselves to simple biological interpretation, and are very general. They thus fulfill the criteria to be called `fundamental theorems' (in the sense of~\cite{queller_fundamental_2017}), or `unifying principles' (in the sense of~\cite{lion_theoretical_2018}) for the dynamics of finite populations. While my equations recover standard results such as the Price equation and the replicator-mutator equation in the infinite population limit, they also predict that these results do not completely capture the behavior of finite populations.

More precisely, the formalism studied in this thesis predicts a directional evolutionary force acting on variation in per-capita turnover rate $\tau$ that I call `noise-induced selection'. Noise-induced selection is only seen in finite populations and arises due to different types of individuals in the population experiencing a different number of stochastic events (birth and death) in a given time interval. Several specific finite population models illustrate that evolution in finite population can proceed in the direction opposite to that predicted by the infinite population limit, a phenomenon sometimes referred to as `reversing the direction of deterministic selection'~\citep{houchmandzadeh_selection_2012, houchmandzadeh_fluctuation_2015, behar_fluctuations-induced_2016, parsons_pathogen_2018,mcleod_social_2019}. This reversal has been thought to be a more general phenomenon, with many models of social evolution predicting noise-induced selection effects~\citep{parsons_consequences_2010,constable_demographic_2016,mcleod_social_2019}.  Most recently, a study showed that in a very broad class of competition models, noise-induced selection in finite populations can act in opposition to the direction of natural selection predicted in the infinite population limit~\citep{mazzolini_universality_2022}. My thesis generalizes this particular result of~\cite{mazzolini_universality_2022} to models with arbitrary interaction types and presents the relevant equations in a formalism that is more in accordance with standard biological models such as the Price equation. It is worth noting that the existence of noise-induced selection directly implies (from equation \eqref{nD_stochastic_RM} or \eqref{nD_stochastic_Price}) that evolution is not expected to maximize fitness in finite populations even if fitness is entirely frequency independent as long as there is some (heritable) variation in the turnover rates $\tau_i$, further underscoring the now well-appreciated point that the view of evolution as `climbing a hill' on a fitness landscape and thereby maximizing fitness is rather limited~\citep{grodwohl_theory_2017}.

The equations of section \ref{sec_fun_theorems} also imply that for the evolution of a trait to be \emph{truly} neutral in finite populations (in the sense of all $m$ types in a system having equal fixation/extinction probability if we start with an initial state in which every type has frequency $1/m$), it is not sufficient for the trait in question to be neutral with respect to fitness $w$. Instead, we also require the trait to be neutral with respect to turnover rate $\tau$.  Indeed, in ecological models, previous work has numerically shown that in finite, fluctuating populations, the equal growth rate of types is not sufficient to ensure equal fixation probabilities and that there is a slight biasing for types with lower turnover rates, sometimes interpreted as a selection `for longevity'~\citep{lin_features_2012, oliveira_advantage_2017,balasekaran_quasi-neutral_2022}. In models of evolutionary game theory in fluctuating finite populations, individuals with lower death rates have higher fixation probability even when growth rates are equalized~\citep{huang_stochastic_2015, czuppon_fixation_2018}. Similarly, models of cell cycle dynamics find that selection favors cell types that periodically arrest their cell cycle relative to non-arresting cells even when their growth rates are equal~\citep{wodarz_effect_2017}. In the language of our birth-death formalism, all of these studies equalize the growth rates $w$ of competing types but allow the turnover to vary (by reducing the birth rate through arresting the cell cycle, for example), thus allowing noise-induced selection for reduced turnover to operate in the system. My derivations show analytically that such deviations from neutrality in finite populations are a generic phenomenon explained by noise-induced selection and should be expected whenever there is variation in turnover rates. Selection on turnover rates also leads to insights on life-history evolution, and these insights have been extensively reported in a recent paper that uses certain discrete time stochastic processes and their approximation via techniques reminiscent of numerical stochastic integration to arrive at a formalism mathematically equivalent to our equations ~\citep{kuosmanen_turnover_2022}. The equivalent of my stochastic equations has also recently been derived for quantitative traits from a very different starting point\footnote{See Section \ref{sec_disc_field_eqns} above for a detailed discussion on how the field equation approach I used in \ref{chap_infD_processes} for quantitative traits is related to~\cite{week_white_2021}'s SPDE approach} using the theory of measure-valued branching processes~\citep{week_white_2021} --- Equations (21b) and (21c) in \cite{week_white_2021} are exactly the $m \to \infty$ version of our equations for changes in the mean value of a type-level quantity and changes in the variance of a type-level quantity respectively for the special case in which the type-level quantity is the value of the quantitative trait being studied.

On the practical side, the existence of noise-induced selection implies that simulation studies working with evolutionary individual-based or agent-based models should be careful about whether interaction effects are incorporated into birth rates or death rates since this seemingly arbitrary choice can have unintended consequences due to noise-induced selection, thus potentially biasing results~\citep{mcleod_social_2019,kuosmanen_turnover_2022}. Being mindful of noise-induced selection is also important for applied fields like conservation and population management which regularly deal with small populations. For example, when trying to increase the population of a hypothetical desired species in a multispecies community, increasing the birth rate is \emph{not} equivalent to reducing the death rate even though both result in an increase in the Malthusian fitness (growth rate) $w_i$. Decreasing the death rate leads to a decrease in $\tau_i$, which leads to positive noise-induced selection, whereas increasing the birth rate leads to an increase in $\tau_i$, which leads to noise-induced selection acting to reduce the abundance of the focal species from the community. If the total community size is small, increasing the birth rate of a species can thus lead to noise-induced selection completely eliminating the focal species from the community despite the fact that we \emph{increased} the growth rate of this species.~\cite{raatz_promoting_2023} have recently used a similar birth-death framework to study cancer treatment and thus provide a concrete example of the consequences of this asymmetry between changing birth rates and death rates: Due to the presence of noise-induced selection, the potential of a tumorous growth to adapt to treatments and experience evolutionary rescue depends inversely on the per-capita turnover $\tau_i$ of the constituent cancer cells in their model, with obvious implications for optimal treatment strategies. My results also indicate that measuring the growth rate of populations is not, in general, sufficient for accurate prediction/inference of future trajectories of the relative abundance of a species (or phenotype, allele, etc.) from empirical data even in completely controlled environments. The growth rate $w_i = b^{\textrm{(ind)}}_{i} - d^{\textrm{(ind)}}_{i}$ of a species $i$ only specifies the difference between its per-capita birth and death rates. In contrast, the complete stochastic dynamics also depend on the total turnover $\tau_i = b^{\textrm{(ind)}}_{i} + d^{\textrm{(ind)}}_{i}$ (\emph{i.e.} the sum of the per-capita birth and death rates).

At first glance, the idea of an evolutionary force that selects individuals with lower birth and death rates over individuals with higher birth and death rates may be reminiscent of notions in life-history evolution like $r$ vs $K$ selection or selection on the pace of life in stochastic models~\citep{stearns_evolution_1977}. However, it is unclear whether this similarity reflects some deep principle or whether it is just superficial. Models in life-history theory are often primarily concerned with spatio-temporally fluctuating (external) environments, and thus the stochasticity in those models is extrinsic to the population. Ecological frameworks such as modern coexistence theory deal with similar questions about population dynamics as our model and would benefit from a first principles stochastic birth-death formulation, also generally work with fluctuating external environments~\citep{chesson_multispecies_1994}. We have entirely neglected such extrinsic factors in our formalism. In principle, it is possible to make the birth and death rates \eqref{nD_functional_forms_for_replicator} in our model also depend on a temporally varying external environment $E(t)$ (whose variation may possibly depend on the population $\mathbf{v}(t)$). Incorporating such a term would ensure that the `ecological feedback' terms in equations \eqref{nD_stochastic_Price} and \eqref{nD_stochastic_Price_variance} are non-zero, but may also lead to much more complex dynamics. Furthermore, if the variation of the environment $E(t)$ has some associated stochasticity, the final dynamics would be the result of interactions between two qualitatively different forms of noise --- \emph{extrinsic} noise from the environment, and \emph{intrinsic} noise from the finiteness of the population --- and thus will be rather complicated and likely analytically intractable. Indeed, experimental studies indicate that (fluctuating) finite population size and fluctuating environments can interact in complex and sometimes counter-intuitive ways, with implications for evolvability and adaptation in spatio-temporally fluctuating environments~\citep{chavhan_larger_2020, chavhan_interplay_2021}. Thus, while integrating the birth-death framework I outline here with ecological ideas such as the pace-of-life syndrome~\citep{mathot_models_2018, wright_life-history_2019} or modern coexistence theory~\citep{chesson_multispecies_1994, johnson_resolving_2022} is biologically appealing, it is likely far from trivial and may present a promising avenue for future work.

~\cite{lion_theoretical_2018} has pointed out that in the dynamic setting (for infinite populations), the replicator-mutator equation \eqref{nD_replicator_mutator} is in some sense the `most fundamental' of the lot, and equations like the Price equation are best viewed as a hierarchy of moment equations for the population mean, population variance, etc. of a type-level quantity. This is also true in my framework - Equation \eqref{nD_stochastic_RM} is the most fundamental equation for population dynamics, and equations like \eqref{nD_stochastic_Price} and \eqref{nD_stochastic_Price_variance} can then be derived from \eqref{nD_stochastic_RM} through repeated application of It\^o's formula, in principle for any moment of the distribution of the quantity $f$ in the population (though this quickly becomes too tedious to actually carry out in practice). If we additionally assume that the quantity $f$ follows a Gaussian distribution in the population, then the mean and variance completely characterize the distribution of $f$, and equations \eqref{nD_stochastic_RM}, \eqref{nD_stochastic_Price}, and \eqref{nD_stochastic_Price_variance} together specify the complete stochastic dynamics of the system. Note that even in this case, actually solving these equations analytically for equilibrium/stationary state distributions of $\mathbf{p}$, $\overline{f}$, and $\sigma^2_f$ will quickly become impossibly difficult if the birth and death rate functions are complicated. Indeed, previous studies indicate that in high dimensions, evolutionary birth-death models can exhibit a dizzying array of complicated phenomena, including limit cycles and evolutionary chaos~\citep{doebeli_diversity_2017}. My framework also works with unstructured populations, neglecting any population structure due to age, class, developmental stage, or space, all of which can lead to complex dynamics and greatly reduce analytical tractability. Lastly, I have also neglected any potential complications introduced by genotype-phenotype maps and genetic processes such as dominance, epistasis, and pleiotropy since they can (in principle) be incorporated while defining the per-capita birth and death rates and our equations hold for arbitrarily complicated birth and death rates as long as they are of the functional form \eqref{nD_functional_forms_for_replicator}.

Despite these practical impediments to actually solving equations for particular models, my equations are very general, since part \ref{part_theory} makes essentially no assumptions other than density dependence, the impossibility of infinite growth starting from finite population size, and the ability to define per-capita birth and death rates. Further, the terms of the equations I derive always lend themselves to simple biological interpretation. Thus, like the Price equation, the utility of these equations lies not in their solutions for specific models, but instead in their generality and the fact that their terms help us clearly understand the various forces operating in biological populations~\citep{frank_natural_2012,luque_one_2017, luque_mirror_2021}. The general approach of working with `high-level' processes without specifying system-specific details is part of a broader pursuit of `model-independent' eco-evolutionary theory that has recently been gaining popularity in the literature ~\citep{grafen_formal_2014, queller_fundamental_2017, doebeli_towards_2017, lion_theoretical_2018, allen_mathematical_2019, rice_universal_2020, week_white_2021, wickman_theoretical_2022, kuosmanen_turnover_2022, mazzolini_universality_2022,lion_extending_2023}. Such model-independent descriptions that `abstract away' system-specific details almost inevitably come at the cost of precision~\citep{levins_strategy_1966, potochnik_idealization_2018}, and are thus intended to complement rather than supersede the study of specific models of specific systems.
