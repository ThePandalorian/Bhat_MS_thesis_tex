We first recall the version of the multi-dimensional It\^{o}'s formula that will be relevant to us. Consider an $m$-dimensional real It\^{o} process $\mathbf{X}_t$ given by the solution to
\begin{equation*}
d\mathbf{X}_t = \boldsymbol{\mu}(\mathbf{X}_t)dt + \boldsymbol{\sigma}(\mathbf{X}_t)d\mathbf{W}_t
\end{equation*}
where $\boldsymbol{\mu}: \mathbb{R}^m \to \mathbb{R}^m$ is the `drift vector' and $\boldsymbol{\sigma}: \mathbb{R}^{m} \to \mathbb{R}^{m \times m}$ is the `diffusion matrix'. Let $f: \mathbb{R}^m \to \mathbb{R}$ be an arbitrary $C^2(\mathbb{R}^m)$ function. Then, It\^{o}'s formula (\cite{oksendal_stochastic_1998}, Section 4.2) states that  the stochastic process $f(\mathbf{X}_t)$ must satisfy:
\begin{equation}
\label{nD_Ito_formula}
df(\mathbf{X}_t) = \left[\left(\nabla_{\mathbf{X}}f\right)^{\mathrm{T}}\boldsymbol{\mu} + \frac{1}{2}\mathrm{Tr}[\boldsymbol{\sigma}^{\mathrm{T}}(H_{\mathbf{X}}f)\boldsymbol{\sigma}]\right]dt + \left(\nabla_{\mathbf{X}}f\right)^{\mathrm{T}}\boldsymbol{\sigma}d\mathbf{W}_t
\end{equation}
where $\mathrm{Tr}[\cdot]$ denotes the trace of a matrix, $\left(\cdot\right)^{\mathrm{T}}$ denotes the transpose, and we have suppressed the $\mathbf{X}_t$ dependence of $\boldsymbol{\mu}$ and $\boldsymbol{\sigma}$ to reduce clutter. Here, $\nabla_{\mathbf{X}}f$ is the $m$-dimensional \emph{gradient vector} of $f$ and $H_{\mathbf{X}}f$ is the $m \times m$ \emph{Hessian matrix} of $f$, respectively defined for $f([x_1,\ldots,x_m]^{\mathrm{T}})$ as:
\begin{align*}
\left(\nabla_{\mathbf{x}} f\right)_j &= \frac{\partial f}{\partial x_j}\\
\left(H_{\mathbf{x}} f\right)_{jk} &= \frac{\partial^2 f}{\partial x_j \partial x_k } 
\end{align*}
In our case, we have the It\^{o} process given by \eqref{nD_Ito_SDE}, which defines how the density of each type of individual changes over time. We thus have $\boldsymbol{\mu}(\mathbf{X}_t) = \mathbf{A}^{-}(\mathbf{X}_t)$ and $\boldsymbol{\sigma}(\mathbf{X}_t) = \mathbf{D}(\mathbf{X}_t)/\sqrt{K}$.  For each fixed $i \in \{1,2,\ldots,m\}$, let us define a scalar function $f_i: \mathbb{R}^m \to \mathbb{R}$ as
\begin{equation*}
f_i(\mathbf{x}) = \frac{x_i}{\sum\limits_{j=1}^{m}x_j}
\end{equation*}
Thus, $f_i(\mathbf{X}_t)$ gives us the frequency of type $i$ individuals when the population is described by the vector $\mathbf{X}_t$. This function is obviously $C^2(\mathbb{R}^m)$, and we can thus use It\^{o}'s formula \eqref{nD_Ito_formula} to describe how it changes over time. The $j\textsuperscript{th}$ element of the gradient of $f_i$ is given by:
\begin{align}
\left(\nabla_{\mathbf{x}} f_i\right)_j &= \frac{\partial }{\partial x_j}\left(\frac{x_i}{\sum\limits_{k=1}^{m}x_k}\right)\nonumber\\
&= \left(\left(\frac{1}{\sum\limits_{r=1}^{m}x_r}\right)\frac{\partial x_i}{\partial x_j} 
- \left(\frac{x_i}{\left(\sum\limits_{r=1}^{m}x_r\right)^2}\right)\sum\limits_{k=1}^{m}\frac{\partial x_k}{\partial x_j}\right)\nonumber\\
&= \frac{1}{\sum\limits_{r=1}^{m}x_r}\left(\delta_{ij}-p_i\right)\label{nD_jacobian_for_ito}
\end{align}
where we have defined the frequency of the $i\textsuperscript{th}$ type $p_i = f_i(\mathbf{x})$ and used the fact that $\frac{\partial x_j}{\partial x_k} = \delta_{jk}$.
The $jk\textsuperscript{th}$ element of the Hessian is given by:
\begin{align}
\left(H_{\mathbf{x}} f_i\right)_{jk} &= \frac{\partial^2 }{\partial x_j \partial x_k}\left( \frac{x_i}{\sum\limits_{l=1}^{m}x_l}\right)\nonumber\\
&= \frac{\partial}{\partial x_j}\left(\frac{\delta_{ik}}{\sum\limits_{r=1}^{m}x_r}-\frac{x_i}{\left(\sum\limits_{r=1}^{m}x_r\right)^2}\right)\nonumber\\
&= \frac{1}{\left(\sum\limits_{r=1}^{m}x_r\right)^2}\left(2p_i - \delta_{ij}-\delta_{ik}\right)\label{nD_hessian_for_ito}
\end{align}
Thus, for the first term of \eqref{nD_Ito_formula}, we have:
\begin{align}
\left(\nabla_{\mathbf{X}}f_i\right)^{\mathrm{T}}\boldsymbol{\mathbf{A}^{-}} &= \sum\limits_{j=1}^{m}\left(\left(\nabla_{\mathbf{x}} f_i\right)_j\right)A^{-}_{j} \nonumber\\
&= \frac{1}{\sum\limits_{r=1}^{m}x_r}\sum\limits_{j=1}^{m}\left(\delta_{ij}-p_i\right)A^{-}_{j}\nonumber\\
&= \frac{1}{\sum\limits_{r=1}^{m}x_r}\left(A^{-}_{i} - p_i\sum\limits_{j=1}^{m}A^{-}_{j}\right)\label{nD_for_Ito_first_term}
\end{align}
This term describes the effects of selection and mutation at the infinite population limit. However, the finiteness of the population adds a second directional term to these dynamics, described by the second term that multiplies $dt$ in \eqref{nD_Ito_formula}. To calculate it, we first calculate:
\begin{align}
\frac{1}{\sqrt{K}}\left(H_{\mathbf{x}} f_i \mathbf{D}\right)_{jk} &= \frac{1}{\sqrt{K}}\sum\limits_{l=1}^{m} \left(H_{\mathbf{x}} f_i \right)_{jl}\left(\mathbf{D}\right)_{lk}\nonumber\\
&= \frac{1}{\sqrt{K}\left(\sum\limits_{r=1}^{m}x_r\right)^2}\sum\limits_{l=1}^{m}\left(2p_i - \delta_{ij} - \delta_{il}\right)\delta_{lk}\left(A^{+}_{l}A^{+}_{k}\right)^{\frac{1}{4}}\\
&=  \frac{1}{\sqrt{K}\left(\sum\limits_{r=1}^{m}x_r\right)^2}\left(\left(2p_i -\delta_{ij}\right)(A^{+}_{k})^{\frac{1}{2}} -\delta_{ik}\left(A^{+}_{i}A^{+}_{k}\right)^{\frac{1}{4}}\right)\\
&= \frac{1}{\sqrt{K}\left(\sum\limits_{r=1}^{m}x_r\right)^2}\left(2p_i -\delta_{ij} -\delta_{ik}\right)(A^{+}_{k})^{\frac{1}{2}}
\end{align}
and thus:
\begin{align}
\frac{1}{K}\left(\mathbf{D}^{\mathrm{T}} H_{\mathbf{x}} f_i \mathbf{D}\right)_{lk} &=\frac{1}{K}\sum\limits_{j=1}^{m}\left(\mathbf{D}^{\mathrm{T}}\right)_{lj}\left(H_{\mathbf{x}} f_i \mathbf{D}\right)_{jk}\nonumber\\
&=  \frac{1}{K\left(\sum\limits_{r=1}^{m}x_r\right)^2}\sum\limits_{j=1}^{m}\delta_{lj}\left(A^{+}_{l}A^{+}_{j}\right)^{\frac{1}{4}}(A^{+}_{k})^{\frac{1}{2}}\left(2p_i -\delta_{ij} -\delta_{ik}\right)\\
&=  \frac{1}{K\left(\sum\limits_{r=1}^{m}x_r\right)^2}(A^{+}_{k})^{\frac{1}{2}}\left(2p_i(A^{+}_{l})^{\frac{1}{2}} - (A^{+}_{i})^{\frac{1}{2}}\delta_{il} - (A^{+}_{l})^{\frac{1}{2}}\delta_{ik}\right)
\end{align}
Using this, we see that the trace of this matrix is given by:
\begin{align}
\frac{1}{K}\mathrm{Tr}[\mathbf{D}^{\mathrm{T}} H_{\mathbf{x}} f_i \mathbf{D}] &= \frac{1}{K}\sum\limits_{k=1}^{m}\left(\mathbf{D}^{\mathrm{T}} H_{\mathbf{x}} f_i \mathbf{D}\right)_{kk}\nonumber\\
&= \frac{1}{K\left(\sum\limits_{r=1}^{m}x_r\right)^2}\sum\limits_{k=1}^{m}\left(2p_i(A^{+}_{k}A^{+}_{k})^{\frac{1}{2}} - (A^{+}_{i}A^{+}_{k})^{\frac{1}{2}}\delta_{ik} - (A^{+}_{k}A^{+}_{k})^{\frac{1}{2}}\delta_{ik}\right)\\
&= \frac{1}{K\left(\sum\limits_{r=1}^{m}x_r\right)^2}\left(2p_i\left(\sum\limits_{k=1}^{m} A^{+}_k\right) - 2A^{+}_{i}\right)
\end{align}
and thus, the second term multiplying $dt$ in \eqref{nD_Ito_formula} is given by:
\begin{equation}
\frac{1}{2K}\mathrm{Tr}[\mathbf{D}^{\mathrm{T}} H_{\mathbf{x}} f_i \mathbf{D}] =  \frac{-1}{K\left(\sum\limits_{r=1}^{m}x_r\right)^2}\left(A^{+}_{i}-p_i\left(\sum\limits_{k=1}^{m} A^{+}_k\right)\right)\label{nD_for_Ito_second_term}
\end{equation}
Finally, denoting $d\mathbf{W}_t = [dW^{(1)}_t,dW^{(2)}_t, \ldots, dW^{(m)}_t]^{\mathrm{T}}$ where each $dW^{(j)}_t$ is an independent one dimensional Wiener process, we have:
\begin{align}
\left(\mathbf{D}d\mathbf{W}_t\right)_j &= \sum\limits_{k=1}^{m}\mathbf{D}_{jk}dW^{(k)}_t\nonumber\\
&= \sum\limits_{k=1}^{m}\delta_{jk}\left(A^{+}_{j}A^{+}_{k}\right)^{\frac{1}{4}}dW^{(k)}_t\\
&= \left(A^{+}_{j}\right)^{1/2}dW^{(j)}_t
\end{align}
Thus, using \eqref{nD_jacobian_for_ito}, we see that the last term on the RHS of \eqref{nD_Ito_formula} is given by:
\begin{align}
\frac{1}{\sqrt{K}}\left(\nabla_{\mathbf{X}}f\right)^{\mathrm{T}}\mathbf{D}d\mathbf{W}_t &= \frac{1}{\sqrt{K}}\sum\limits_{j=1}^{m}\left(\nabla_{\mathbf{x}} f_i\right)_j\left(\mathbf{D}d\mathbf{W}_t\right)_j\nonumber\\
&=  \frac{1}{\left(\sum\limits_{r=1}^{m}x_r\right)\sqrt{K}}\sum\limits_{j=1}^{m}\left(\delta_{ij}-p_i\right)\left(A^{+}_{j}\right)^{1/2}dW^{(j)}_t\\
&= \frac{1}{\left(\sum\limits_{r=1}^{m}x_r\right)\sqrt{K}}\left(A^{+}_{i}\right)^{1/2}dW^{(i)}_t - p_i\sum\limits_{j=1}^{m}\left(A^{+}_{j}\right)^{1/2}dW^{(j)}_t\label{nD_for_Ito_third_term}
\end{align}
Putting equations \eqref{nD_for_Ito_first_term}, \eqref{nD_for_Ito_second_term} and \eqref{nD_for_Ito_third_term} into \eqref{nD_Ito_formula} and letting $N_K(t) = \sum\limits_{r=1}^{m}x_r$ we see that $p_i = f_i(\mathbf{X})_t$, the frequency of the $i\textsuperscript{th}$ type in the population $\mathbf{X}_t$, changes according to the equation:
\begin{equation}
\label{nD_general_stoch_freq_change}
\begin{aligned}
dp_i &= \underbrace{\frac{1}{N_K(t)}\left(A^{-}_{i} - p_i\sum\limits_{j=1}^{m}A^{-}_{j}\right)dt}_{\text{$K \to \infty$ prediction}} - \frac{1}{K}\underbrace{\frac{1}{N_K^2(t)}\left(A^{+}_{i}-p_i\left(\sum\limits_{k=1}^{m} A^{+}_k\right)\right)dt}_{\substack{\text{Directional finite size effects}\\\text{due to differential turnover rates}}}\\
&+ \frac{1}{\sqrt{K} N_K(t)}\underbrace{\left[\left(A^{+}_{i}\right)^{1/2}dW^{(i)}_t - p_i\sum\limits_{j=1}^{m}\left(A^{+}_{j}\right)^{1/2}dW^{(j)}_t\right]}_{\substack{\text{Non-directional finite size effects}\\\text{due to stochastic fluctuations}}}
\end{aligned}
\end{equation}
Plugging the functional forms of \eqref{nD_functional_forms_for_replicator} and the definitions of $w_i$ and $\tau_i$ into the definitions of $A^{-}_i$ and $A^{+}_i$, we obtain the relations
\begin{equation}
\label{nD_det_limit_fitness_and_turnover}
\begin{aligned}
A^{-}_i &= x_iw_i(\mathbf{x}) + \mu Q_i(\mathbf{x})\\
A^{+}_i &= x_i\tau_i(\mathbf{x}) + \mu Q_i(\mathbf{x})
\end{aligned}
\end{equation}
Thus, for the first term of \eqref{nD_general_stoch_freq_change}, we have
\begin{align*}
\frac{1}{N_K(t)}\left(A^{-}_{i} - p_i\sum\limits_{j=1}^{m}A^{-}_{j}\right) &=  \frac{1}{N_K(t)}\left[w_i(\mathbf{x})x_i + \mu Q_i(\mathbf{x})\right] - \frac{p_i}{N_K(t)}\sum\limits_{j=1}^{m}\left[w_j(\mathbf{x})x_j + \mu Q_j(\mathbf{x})\right]\\
&= w_i(\mathbf{x})p_i + \frac{\mu}{N_K(t)}Q_i(\mathbf{x}) - p_i\sum\limits_{j=1}^{m}\left[w_j(\mathbf{x})p_j + \frac{\mu}{N_K(t)}Q_j(\mathbf{x})\right]
\end{align*}
Where we have used the definition of $p_i$ from \eqref{nD_tot_pop_and_prop_inds_defn}. Now using the definition of mean fitness from \eqref{nD_mean} and rearranging terms gives us
\begin{equation}
\frac{1}{N_K(t)}\left(A^{-}_{i} - p_i\sum\limits_{j=1}^{m}A^{-}_{j}\right) = (w_i(\mathbf{x}) - \overline{w})p_i + \mu\left[Q_i(\mathbf{p}) - p_i\left(\sum\limits_{j=1}^{m}Q_j(\mathbf{p})\right)\right]
\end{equation}
where we have defined $Q_j(\mathbf{p}) = Q_j(\mathbf{x})/N_K(t)$. Repeating the exact same calculations for the $A^{+}_i$ terms in the second term of \eqref{nD_general_stoch_freq_change} now yields equation \eqref{nD_eqn_for_frequencies} (which is also equation \eqref{nD_stochastic_RM}) in the main text.