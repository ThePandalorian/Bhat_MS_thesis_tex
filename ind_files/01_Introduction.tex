\epigraph{\justifying The epistemic aim of science is not truth, but understanding}{Angela Potochnik}

More than 150 years have passed since Ernst Haeckel first coined the term ‘ecology’ in 1866. Today, ecology is incredibly interdisciplinary, borrowing techniques and ideas from diverse fields such as computer science, statistics, economics, dynamical systems, physics, and information theory. With this development has come an array of tools intended to help the theoretician use abstract entities which only exist on paper to capture the essence of the real biological systems we see around us. Consequently, there has also been a boom in models that seek to explain natural phenomena using these tools. Despite the apparent overabundance of such seemingly disparate models in the literature, theoretical ecologists often think in terms of a few very general organizing principles which can be adapted in various ways to model specific biological systems \citep{vellend_theory_2016}. There is thus value to formulating general, idealized models in abstract terms to underscore these organizing principles and better illustrate the fundamental processes that are required to capture the `essence' of an ecological pattern. Such abstractions and idealizations are vital to model building and are useful in gaining insights into complex phenomena such as those that abound in ecology. However, it is important to remember that abstractions, are, ultimately, unrealistic by construction. As such, it is important to investigate whether predictions of general abstract theories are robust to failure of some of the more unrealistic assumptions that they make. Some of the central questions of ecology boil down to the question of investigating the phenomenon of multiple interacting ‘eco-variants’ (ecologically relevant phenotypic variants: Some examples in section \ref{synthesis}) coexisting in sympatry. In section \ref{idealization}, I reiterate some old arguments about the utility and pitfalls of general idealized models in ecology and evolution. In section \ref{synthesis}, I highlight diverse ecologically important contexts in which the coexistence of multiple discrete `eco-variants' is relevant. One general class of models, called `adaptive diversification', has gained immense popularity in trying to answer such questions of coexisting phenotypic variants. Chapter \ref{chap_AD} introduces this research program. I summarize three distinct modelling approaches to adaptive diversification in section \ref{AD} that differ in their simplifying assumptions and present some important general predictions of these models in sections \ref{high_dim} and \ref{dem_stoch}. I identify a problematic idealization within these existing models, and outline my goal of addressing this shortcoming for my thesis in section \ref{goals}. Chapter \ref{chap_BD} develops stochastic models of eco-evolutionary population dynamics from first principles, and I show that various known models from evolutionary game theory, population genetics, quantitative dynamics, and adaptive dynamics arise as special cases in the deterministic limit. This section also highlights the connections between the three modelling paradigms of adaptive diversification mentioned in section \ref{AD} by explicitly deriving them. This modelling exercise results in a useful organizational framework for general population models in ecology and evolution and shows how tools from statistical physics and mathematics can be useful in analyzing the models. Some approximations derived in this chapter are used in Chapter \ref{chap_examples} to analytically investigate adaptive diversification in finite populations.

\section{Idealization and generality in ecology}\label{idealization}
The push and pull between the search for general patterns and the specification of minutiae has a long and torturous history in ecology \citep{kingsland_modeling_1985}. One of the first mathematical idealizations in ecology came in the form of the logistic equation formulated by Pearl and Verhulst, and shortly thereafter, equations for the populations of interacting predator and prey species, put forth independently by Lotka and Volterra. These models were immediately controversial, and for good reason: Many ecologists felt that they were overly idealistic and neglected many important truths about real biological populations \citep{kingsland_modeling_1985}. These models were nevertheless quite good at predicting the patterns of such populations, and have proved valuable to the field of population ecology. Today, these models are viewed as `classical' and are regularly used even by hard-line empiricists, not because we believe them to be true in all their gory biological details, but because we recognize that they can be \emph{useful} despite being blatantly false generalizations. This is a single instance of a more general philosophical idea concerning the goals and ideals of science. As suggested by the epigraph at the beginning of the chapter, philosophers of science \citep{potochnik_idealization_2018} have recently argued that science does not seek truth, but instead seeks understanding. The fact that idealization is part and parcel of science is clear if one looks at the actual practice, be it theorists making unrealistic assumptions on paper to model specific phenomena or experimentalists creating artificially controlled conditions in the laboratory to test specific hypotheses \citep{zuk_models_2018}. Relatively simple models of complex eco-evolutionary processes are therefore desirable as a way to shine light on these phenomena. To reiterate once more, the goal of such simple models is not truth (whatever that is), predictive power (as with models in physics), or detailed description (as with detailed individual-based simulations, very flexible statistical models, or machine learning), but \emph{understanding}. Since the world is complicated and humans are limited, such understanding inevitably comes at the cost of other desirable qualities such as the ability to make precise quantitative predictions. It is important to remember at the outset that the models I speak about in this thesis may seem to be over-idealized and too general, and will make only qualitative predictions. This is \textit{by design}, in pursuit of general insight over precise quantitative prediction.\\
Vellend has recently argued that conceptual synthesis in community ecology requires ``shifting the emphasis away from an organizational structure based on the useful lines of inquiry carved out by researchers, to one based on the fundamental processes that underlie community dynamics and patterns" \citep{vellend_theory_2016}. Vellend's assertion is based on the fact that population genetics has managed to come up with reasonably comprehensive theory due to its focus on the abstract `high-level' processes of selection, mutation, drift, and gene flow instead of the myriad `low-level' processes that may be responsible for generating them. In contrast, he believes that practitioners of community ecology often focus on specific `low-level' processes such as predation rate, limiting resources ($R^*$), storage effects, priority effects, senescence, and niche partitioning, leading to a plethora of models (see Table 5.1 in \citep{vellend_theory_2016} for an in-exhaustive list of 24 such models) and the conclusion that community ecology `is a mess'. Vellend proposes organizing ecological models according to the `high-level' processes of selection, ecological drift (demographic stochasticity), dispersal, and speciation.\\ Of course, no such general organization will be perfect or all-encompassing. As Robert MacArthur once remarked, ``general events are only seen by ecologists with rather blurred vision. The very sharp-sighted always find discrepancies and are able to see that there is no generality, only a spectrum of special cases”\citep{kingsland_modeling_1985}. However, I believe the act of looking past the myriad low-level processes present in biological systems and categorizing theories, models, and concepts in terms of a small number of general fundamental high-level processes provides a powerful unifying tool to organize concepts in biology. This is perhaps best exemplified by Darwin’s theory of natural selection as first proposed in \emph{The Origin of Species}. Darwin famously painstakingly collected a series of ‘low-level’ facts and observations regarding breeds, wild populations, and the geographic record to support his hypothesis. However, ultimately, these observations culminated in a synthesis whereby they were all unified under a single, abstract, ‘high-level’ process, namely evolution by natural selection.
In the spirit of this approach, the questions I seek to answer in this thesis are in terms of rather abstract sounding `high-level' processes. In the next section, I survey some of the existing literature to provide concrete examples which motivate these general models.

\section{Coexistence of eco-variants in diverse contexts}\label{synthesis}
The phenomenon of coexistence of multiple phenotypically distinct, ecologically relevant variants of an organism in sympatry occurs in several different contexts in the natural world. This is best illustrated through examples, which I provide below.
\subsection{Alternative Reproductive Tactics}
Alternative reproductive tactics (ARTs) are discrete polymorphisms that occur in sympatry in one or both sexes of a species in response to intrasexual competition for mates. More precisely, two or more traits which are expressed in individuals of a given sex of a given species can be said to be ARTs if \citep{oliveira_alternative_2008}:
\begin{enumerate}[label=(\alph*)]
    \item They are \emph{alternative}, in the sense that a given individual can only express one of the behaviors at any given point in time.
    \item They are \emph{reproductive}, in the sense that the traits are directly relevant to the process of obtaining a mate and are intended to mitigate conspecific intrasexual competition.
    \item They are \emph{tactics}, in the sense of serving a well-defined adaptive function and thus having fitness consequences. Usually, different ARTs have different associated fitness effects \emph{ceteris paribus}.
\end{enumerate}
ARTs are widespread in the animal kingdom, and may be seen either in morphological traits, or as more complex behavioral polymorphisms (see \citep{oliveira_alternative_2008} for an overview). Most (if not all) species which exhibit ARTs only show a small number of polymorphisms (usually 2-3, 5 in a handful of cases, and very rarely 7-8) - \hl{I can maybe make a table for this: Somehow wasn't able to find one in the literature}. ARTs have obvious consequences for reproductive life history, and can strongly influence the ecology and evolutionary trajectory of organisms.

\subsection{Trophic resource polymorphisms}

Trophic resource polymorphism is defined as ``the occurrence of discrete intraspecific morphs showing differential niche use, usually through differences in feeding biology and habitat use" \citep{skulason_resource_1995}. Species which exhibit trophic resource polymorphisms often show marked, discontinuous morphological variations adapted for specific niche usage. For example, arctic charr and african cichlid fishes both often exhibit discrete trophic polymorphisms in which different morphs (of the same species) are specialized for feeding at different strata of the lakes they live in \citep{recknagel_ecosystem_2017}. Such polymorphisms also generally occur in relatively small numbers in sympatry (see Table 1 in \citep{smith_evolutionary_1996}). They are relatively widespread in vertebrates, and are thought to have important evolutionary consequences by acting as the `substrate' for adaptive radiations \citep{smith_evolutionary_1996}. 

\subsection{Mating types in isogamous species}
Isogamous organisms are sexual, but do not have gametes which can be classified into `male' and `female' based on size. Gametes of such species can, however, be divided into distinct `mating types' according to chemical self-incompatibility. There is substantial variation in the number of mating types present in a given species: While most organisms have only two mating types, this is by no means the rule \citep{phadke_rapid_2009,constable_rate_2018}, with a handful of organisms having 7-10 mating types, and one species of fungus (\textit{Schizophyllum commune}) exhibiting over 23,000 mating types. The study of mating types in isogamous species is ecologically and evolutionarily relevant, not only due to the intrinsic importance of mating and self-incompatibility, but due to the fact that isogamy is thought to be the evolutionary precursor to anisogamy: Thus, studying the factors governing the number of mating types in isogamous species may shed light on why anisogamous species have only two distinct types of gametes.

\subsection{Phenotypic clustering and sympatric speciation}
\hl{Write stuff about sympatric speciation, adaptive radiations, and phenotypic clustering.}
\\
\\
In all these cases, we are interested in monitoring the changes in the number of distinct phenotypic entities, in a scenario in which the phenotype is ecologically relevant for fitness. Existing literature has tended to use the word `morph' for this concept, and I will bow down to convention and do the same. However, it is important to remember that in this context, `morphs' are only defined in terms of traits that \textit{affect fitness}, in contrast to the more general use of the word in biology to mean any difference whatsoever (as in single nucleotide polymorphisms, for example). Note that in many of these cases, even though the emergent alternative morphs are discretely distributed, the underlying traits need not be, and indeed are often continuously varying, and only categorized into discrete bins by human experimentalists. Such discrete morphs are often persistent in phylogenetically similar lineages \citep{jamie_persistence_2020}, suggesting they are evolutionarily predictable to some degree, and are also thought to be important for very general evolutionary processes \citep{west-eberhard_alternative_1986}. Intraspecific variation also provides an important metric for the diversity harboured by populations and can have important ecological consequences. Two important questions immediately arise:

\begin{enumerate}
    \item How does a population that is initially monomorphic for a (selectively non-neutral) trait evolve to become polymorphic for this trait in a seemingly discrete manner?
    \item What governs the number of discrete morphs that co-occur in a population?
\end{enumerate}

This very general setting motivates a view in which an `individual' is simply characterized by its phenotype as a point in some abstract `trait space'. These questions then concern the distribution of points in this space (analogous to the frequency of each `type' of individual) changes over evolutionary time due to ecological fitness determinants (through the traits that the individuals possess). More formally, we can visualize an abstract trait space $\mathcal{T}$ which captures all possible phenotypes that an individual may exhibit. Any phenotype can then be viewed as some point $x \in \mathcal{T}$. We are interested in the distribution of trait values in the population subject to some ecological and evolutionary rules, and a `morph' or `eco-variant' is just a cluster within this trait space (a mode in the distribution).
% \subsection{Sympatric speciation}
% The origin of species is relatively well understood when populations diverge in allopatry (separated by geographic barriers, for example). In contrast, sympatric speciation, first formally proposed by Maynard Smith \hl{cite}, is much less well-understood, despite empirical evidence that it has occurred in nature \hl{cite}. Since speciation is one of the fundamental processes in 

% \section{Frequency-dependent selection}\label{FDS}
% One well-known mechanism of coexistence is `niche partitioning', whereby coexisting types occupy different areas in niche space and thereby reduce competition between types. This of course immediately immediately begs the question of how niche partitioning occurs in the first place. In general, coexistence is also possible without niche partitioning. Such coexistence in the presence of selective pressure (\textit{i.e} when eco-variants are not selectively neutral) occurs through frequency-dependent selection.



