\epigraph{\justifying The theory of evolution by natural selection is an ecological theory—founded on ecological observation by perhaps the greatest of all ecologists. It has been adopted by and brought up by the science of genetics, and ecologists, being modest people, are apt to forget their distinguished parenthood}{John Harper~\citep{harper_darwinian_1967}}

More than 150 years have passed since Charles Darwin first published \textit{Origin} (in 1859) and Ernst Haeckel first coined the term ‘ecology’ (in 1866). Today, both ecology and evolution are incredibly interdisciplinary fields, borrowing techniques and ideas from diverse areas such as computer science, statistics, economics, dynamical systems, physics, and information theory. With this development has come a cornucopia of models that try to understand biological phenomena in the language of these borrowed tools and techniques. Though many of these ideas are used to accurately describe and analyze specific systems, there is also value to formulating general models in abstract terms that only incorporate a small number of `fundamental' processes and try to capture the `essence' of a biological pattern~\citep{frank_natural_2012, vellend_theory_2016,luque_mirror_2021}. Such general organizing models are vital to theory-building~\citep{luque_mirror_2021} because they help clarify conceptual similarities and unifying factors between apparently disparate modelling frameworks, helping us seamlessly translate essential ideas from one theoretical language to another.

\section{Idealization and generality}\label{idealization}
The push and pull between the search for general patterns and the specification of minutiae has a long and torturous history in ecology~\citep{kingsland_modeling_1985} and evolution~\citep{provine_origins_2001}. The fact that idealization is part and parcel of science is clear if one looks at the actual practice, be it theorists making unrealistic assumptions on paper to model specific phenomena or experimentalists creating artificially controlled conditions in the laboratory to test specific hypotheses~\citep{zuk_models_2018}. Indeed, some philosophers of science recently argued that ``the epistemic goal of science is not truth, but understanding''~\citep{potochnik_idealization_2018}, an idea generally echoed by practicing scientists~\citep{levins_strategy_1966,zuk_models_2018,grainger_empiricists_2022}.  In other words, since the world is complicated and humans are limited, general understanding inevitably comes at the cost of other desirable qualities such as the ability to make precise quantitative predictions. This is especially true for complex phenomena such as those that are the domain of ecology and evolution, where we are often not even aware of all the factors that are at play or how they interact. It is important to remember at the outset that the models I speak about in this thesis may seem to be too abstract/general and may seem to ignore many relevant biological details. This is \textit{by design}, in pursuit of general insight over precise quantitative prediction.

An approach like this will often not be perfect or all-encompassing. As Robert MacArthur once remarked, ``general events are only seen by ecologists with rather blurred vision. The very sharp-sighted always find discrepancies and are able to see that there is no generality, only a spectrum of special cases”\citep{kingsland_modeling_1985}. However, formulating and studying such generalizations can be greatly beneficial as an aid to thinking, sometimes by the very virtue of the fact that it can be done through `blurry eyed' thinking that only looks for general broad-brush regularities\footnote{It bears noting that when a biologist says something is `general', they typically mean that it occurs in the majority of the relevant cases studied. In contrast, when a mathematician says something is `general', they mean that it is \emph{always} true. Thus, an event that occurs in 70\% of cases is `generally true' to biologists but `not true in general' to mathematicians. In this thesis I write as a biologist.}. This is perhaps best illustrated by the success of statistical mechanics in physics - Statistical mechanics was essentially born from the idea that statistical regularities of systems can emerge without the need for knowing the excruciating details of every single moving part, and indeed, starting from first principles, this sort of explicitly `blurry-eyed' thinking was shown to be able to recover the phenomenological laws of thermodynamics as \emph{statistical laws}. In biology, arguably the greatest such general organizing framework was the idea of evolution by natural selection as synthesized from myriad detailed observations by Charles Darwin. Theoretical population genetics has also had a long-standing tradition in building general organizing frameworks that `abstract away' some biological specificities in favor of a small number of `fundamental' forces, and here, the best analog of statistical mechanics is the Price equation, which was able to recover several known formalisms of evolution from first principles (see section \ref{sec_history} below). In a similar spirit, Vellend has recently argued that conceptual synthesis in community ecology requires
``shifting the emphasis away from an organizational structure based on the useful lines of
inquiry carved out by researchers, to one based on the fundamental processes that underlie community dynamics and patterns''~\citep{vellend_theory_2016}. Vellend's assertion is based on the
fact that population genetics has managed to come up  with reasonably comprehensive theory due to its focus on the abstract `high-level’ processes of selection, mutation, drift, and
gene flow (see section below) instead of the myriad `low-level' processes that may be responsible for generating them. In contrast, he believes that practitioners of community ecology often focus on
specific `low-level' processes such as predation rate, limiting resources (R\textsuperscript{*}), storage effects,
priority effects, senescence, and niche partitioning, leading to a plethora of models (see Table
5.1 in~\cite{vellend_theory_2016} for an in-exhaustive list of 24 such models) and the conclusion that
community ecology `is a mess'. Vellend instead proposes organizing ecological models according to
 the `high-level’ processes of selection, ecological drift (demographic stochasticity), dispersal,
and speciation.

\section{A (very, very) brief history of high-level modelling frameworks in population biology}\label{sec_history}

In population genetics, the relevant high-level forces are the standard evolutionary forces of natural selection, genetic drift, dispersal, and mutation. The description of evolution in terms of these forces was first laid out in formal mathematical terms during the Modern Synthesis by authors such as Wright, Fisher, and Haldane, an extremely successful venture that unified two major schools of thought --- Mendelian genetics and Darwinian evolution --- that were, at the time, considered to be incompatible~\citep{provine_origins_2001}. It is currently thought that this unification would have been unlikely or would have taken much longer if the architects of the Modern Synthesis had stuck to verbal arguments instead of working with formal models in explicitly mathematical terms~\citep{walsh_darwins_2014}. 

Formal models in classical population genetics often regarded forces like selection and mutation as fundamental, and the success of this approach during the Modern Synthesis illustrates the value of formulating high-level, abstract models that only provide a `high-level' description of the fundamental processes required to capture the essence of a biological pattern. However, this evolutionary play famously unfolds in the ecological theatre~\citep{hutchinson_ecological_1965}. Thus, quantities like fitness are not truly fundamental but instead emerge as the net result of various ecological interactions, tradeoffs, and constraints~\citep{metz_how_1992}, a fact that can have important consequences for evolution~\citep{coulson_putting_2006, kokko_can_2017}. Trying to understand such `eco-evolutionary dynamics' has sprouted a rich body of theoretical literature under the broad heading of `evolutionary ecology' and led to the development of theoretical frameworks like evolutionary game theory and adaptive dynamics, fields which have greatly enriched our understanding of biological populations~\citep{brown_why_2016}. Eco-evolutionary population dynamics can (very) broadly be organized under a single unifying framework, the Price equation, that yields all the relevant formal structures (such as evolutionary game theory and classic population genetics) as special cases~\citep{page_unifying_2002, queller_fundamental_2017, lion_theoretical_2018}. The Price equation partitions changes in population composition into multiple terms, each of which lends itself to a straightforward interpretation in terms of the high-level evolutionary forces of selection, mutation, and drift, thus providing a useful conceptual framework for thinking about how populations change over time~\citep{frank_natural_2012}. The Price equation also leads to a small number of simple yet insightful `fundamental theorems' of population biology~\citep{queller_fundamental_2017, lion_theoretical_2018, lehtonen_price_2018} and unifies several various seemingly disjoint formal structures under a single theoretical banner~\citep{ lehtonen_price_2020, luque_mirror_2021}

One of the general guiding principles of much of this mathematization has been the assumption that incorporating the reality of finite population sizes into models leads to no major qualitative differences in behavior, only `adding noise' or `blurring out' the predictions of simpler infinite population models~\citep{page_unifying_2002}. Consequently, several major theoretical frameworks in the field, such as adaptive dynamics, are explicitly formulated in deterministic terms at the infinite population size limit. However, this assumption is largely unjustified, and since populations in the real world are finite and stochastic, checking whether stochastic models differ from their deterministic analogs is vital to furthering our understanding of the fundamentals of population biology~\citep{hastings_transients_2004, coulson_skeletons_2004, shoemaker_integrating_2020}. Today, we increasingly recognize that incorporating the finite and stochastic nature of the real world routinely has much stronger consequences than simply `adding noise' to deterministic expectations~\citep{boettiger_noise_2018}, with important consequences for both ecological~\citep{schreiber_does_2022} and evolutionary~\citep{delong_stochasticity_2023} theory. In ecology and evolution, stochastic models often do not exhibit phenomena that occur in deterministic models~\citep{proulx_what_2005, johansson_will_2006, claessen_delayed_2007,  wakano_evolutionary_2013, debarre_evolutionary_2016, johnson_two-dimensional_2021}, exhibit phenomena that do not occur in deterministic models~\citep{rogers_demographic_2012, rogers_spontaneous_2012, rogers_modes_2015, veller_drift-induced_2017, delong_stochasticity_2023}, and can even completely reverse the predictions of deterministic models~\citep{houchmandzadeh_selection_2012,houchmandzadeh_fluctuation_2015,constable_demographic_2016,mcleod_social_2019}. Studies of neutral or near-neutral dynamics in population and quantitative genetics usually do take stochasticity seriously, explicitly modeling finite populations that follow stochastic dynamics. Unfortunately, the classic or standard stochastic models in both population genetics~\citep{fisher_genetical_1930,wright_evolution_1931, moran_random_1958, kimura_diffusion_1964} and quantitative genetics~\citep{crow_introduction_1970, lande_natural_1976} typically assume a fixed total population size and their validity is therefore rather restrictive since population sizes routinely fluctuate in the real world. Importantly, the Price equation itself is usually formulated in a deterministic, dynamically-insufficient manner (but see~\cite{rice_stochastic_2008} for a stochastic formulation of the Price equation). Since real-life populations are stochastic, finite, and of non-constant population size, it is thus imperative that we develop a theoretical framework that can handle such systems directly, instead of only working with deterministic, infinite-population approximations.

Incorporating stochasticity into deterministic systems is a tricky business, and, if done in a phenomenological manner by adding noise to a `deterministic skeleton'~\citep{coulson_skeletons_2004} in an ad-hoc fashion, can lead to nonsensical predictions and inconsistencies~\citep{strang_how_2019}. Stochastic individual-based models, in which (probabilistic) rules are specified at the level of the individual and population level dynamics are systematically derived from first principles, are self-consistent~\citep{strang_how_2019}, much more natural~\citep{black_stochastic_2012}, and can fundamentally differ from the predictions made by simply adding noise terms to a deterministic model~\citep{strang_how_2019}. Formulating the fundamental formal structures of evolutionary biology in terms of the mechanistic demographic processes of birth and death at the individual level is also greatly desirable for biological reasons~\citep{metcalf_why_2007,geritz_mathematical_2012} because `all paths to fitness lead through demography'~\citep{metcalf_all_2007}. In other words, since demographic processes such as birth and death rates explicitly account for the ecology of the system, they can more accurately reflect the complex interplay between ecological and evolutionary processes and provide a more fundamental mechanistic description of the relevant evolutionary forces and population dynamics~\citep{doebeli_towards_2017}. In this thesis, I present a formulation of population dynamics constructed from mechanistic first principles grounded in individual-level birth and death. The mathematical formalism itself is very general and applies equally to the `high level' forces of population genetics as postulated during the Modern Synthesis and the `high level' forces of community ecology as postulated by~\cite{vellend_theory_2016}, though I will mostly stick to the population genetics interpretation in my discussions.

\section{Outline of the outline of the rest of this thesis}
This section provides a bullet-point outline for those who are really short of time. A more detailed outline of the thesis is provided in the next section.

This thesis is structured as follows:

\begin{itemize}
	\item \textbf{Part \ref{part_theory}}
	develops mathematical formalism for describing finite fluctuating populations from first principles.
		\begin{itemize}
		\item Chapter \ref{chap_math_background} provides the necessary mathematical background and provides a toy example studying the size of a population of identical individuals that illustrates the major ideas used. If I am required to shoehorn the structure into the `intro-methods-results-discussion' format, then this chapter can be thought of as providing a mathematical `introduction' and illustration of the `methods' that will be used/generalized in the next chapters.
		\item Chapter \ref{chap_BD} develops a formalism for describing the evolution of finite fluctuating populations that vary in one or more \emph{discrete} characters. This yields equations that generalize the classic Price equation and replicator-mutator equation to finite fluctuating populations. This chapter is biologically original but uses well understood mathematics that has been used in biological contexts before.
		\item Chapter \ref{chap_infD_processes} extends the ideas developed in chapter \ref{chap_BD} to populations that vary in a single one-dimensional \emph{quantitative} character and derives a so-called `stochastic field theory' that describes evolution in such populations. This also results in some mathematical equations that may be of independent interest to physicists and applied mathematicians. This chapter is entirely original, both biologically, and (to the best of my knowledge) in its use of certain mathematical structures that extends and generalize previous statistical physics studies of certain ecological models.
		\end{itemize}
	\item \textbf{Part \ref{part_summary}} presents the major results of all the formalism developed up until this point.
		\begin{itemize}
			\item Chapter \ref{chap_unification} provides a technical summary of the major results by deriving three important stochastic differential equations. These are equations for type frequencies, the population mean value of an arbitrary type-level quantity, and the population variance of an arbitrary type-level quantity, and respectively generalize the replicator-mutator equation, the Price equation, and \cite{lion_theoretical_2018}'s variance equation to finite, stochastically fluctuating populations. These equations predict a directional evolutionary force called `noise-induced selection' that is only seen in finite, fluctuating populations. Some implications for social evolution and community ecology are discussed.
			\item Section \ref{sec_disc_field_eqns} of chapter \ref{chap_unification} discusses the use of the stochastic field equations formulated in chapter \ref{chap_infD_processes}, connects them to some previous studies in biology, and briefly discusses the use of and knowledge about such mathematical objects in applied mathematics fields such as statistical physics. This section is quite technical due to the nature of the subject.
			\item Chapter \ref{sec_disc} provides a non-technical discussion and summary of all the major results, and discusses biological implications, connections with previous studies, and opportunities for future work. This chapter has no equations (!). The mathematically averse can skip all other chapters and only read chapter \ref{sec_disc} if they are interested only in the final results and takeaways\footnote{Though I promise I have tried my best to make sure the journey to this destination is also pleasant :)} 
		\end{itemize}
\end{itemize}

\section{Outline of the rest of this thesis}

This section provides a more detailed outline of the thesis, for those who want more details than the bullet points above provide.

The rest of this thesis is divided into two parts. Part \ref{part_theory} provides all the gory mathematical details in a (hopefully) accessible pedagogical style. Part \ref{part_summary} then presents some major results and discusses their implications and connections with previous studies.

Chapter \ref{chap_math_background} provides the basic mathematical background required and illustrates the major ideas that we will use. To facilitate readership by a broad audience, I only assume passing familiarity with calculus (derivatives, integrals, Taylor expansions) and probability. Familiarity with stochastic calculus is helpful for some sections but is not required. I present a brief introduction to the relevant mathematics in section \ref{sec_math_background} and present a toy example of tracking population size of a population of identical individuals in section \ref{sec_1D_processes}. I introduce a description of the system via a `master equation', and then conduct a `system-size expansion' to obtain a Fokker-Planck equation for the system. Finally, I conduct a weak noise approximation to arrive at a linear Fokker-Planck equation which can be solved exactly using some stochastic calculus to arrive at a closed-form solution given by a time-dependent Ornstein-Uhlenbeck process, thus illustrating all the major tools required.

Chapter \ref{chap_BD} deals with the evolution of discrete traits. In this case, the system is finite-dimensional, since we can completely specify the state of the system by simply listing out the number of individuals of each type in a vector. I introduce a general multivariate process to describe the evolution of discretely varying traits, and use the system size expansion to arrive at a continuous description of change in trait frequencies as an SDE under mild assumptions on the functional forms of the birth and death rates. Unlike many classic stochastic formulations in evolutionary theory~\citep{fisher_genetical_1930,wright_evolution_1931,moran_random_1958,crow_introduction_1970, lande_natural_1976}, I do not assume a fixed (effective) population size and instead allow the total population size to fluctuate over time. I show that the deterministic limit of this process is the well-known replicator-mutator equation (or equivalently, the dynamic version of the Price equation), thus establishing the microscopic basis of well-known equations from stochastic first principles. I also illustrate some general predictions that can be made using the weak noise approximation for the sake of completeness. While the mathematics of this chapter is standard and well-understood, it has, to the best of my knowledge, not been used in this context in the generality we use here. Several specific models of specific systems do use these mathematical techniques, but these papers are often written assuming familiarity with notions in physics and/or mathematics and thus may not be very accessible to theoretical ecologists who do not have formal training in these subjects (but see~\cite{czuppon_understanding_2021} for a recent pedagogical review on the general approach as applied to Wright-Fisher and Moran processes, where total population size is constant). As such, chapters \ref{chap_math_background} and \ref{chap_BD} together also serve as a tutorial and technical introduction to some theoretical ideas: For ecologists, the chapter introduces `system size expansions' and illustrates their use in a general setting, and can be seen as a tutorial on modelling finite populations analytically with minimal assumptions; For population geneticists, the chapter illustrates how system-size approximations (`diffusion approximations' in the population genetics literature) can be carried out without assuming a constant (effective) population size and how this generalization has important consequences for the evolutionary forces at play; For physicists and applied mathematicians, the chapter presents a study of the consequences of applying the system-size expansion to the kind of density-dependent birth-death processes that are widely applicable in ecology and evolution - Unlike many physical systems, though processes occur in terms of numbers or densities, predictions in evolution are typically in terms of frequencies of types, and this fact has subtle consequences that can be overlooked if one only works with densities.

Chapter \ref{chap_infD_processes} introduces a function-valued process\footnote{This is really a measure-valued process but we will pretend we don't know this in the interest of accessibility} to model the evolution of quantitative traits such as body size, which can take on uncountably many values. This function-valued process can then also be analyzed via an analog of the system-size approximation to arrive at a `functional' Fokker-Planck equation in which derivatives are replaced by functional derivatives. I show that classic equations from quantitative genetics such as Kimura's infinite alleles model and Lande's gradient dynamics can be derived as the infinite population limit of this stochastic process. I also conduct a weak noise approximation to arrive at a linear functional Fokker-Planck equation that can be analyzed for specific systems as required. Unlike the systems studied in Chapter \ref{chap_BD}, formalizing the study of the kind of processes we study in Chapter \ref{chap_infD_processes} is an active area of mathematical research~\citep{carmona_stochastic_1999,da_prato_stochastic_2014,prevot_concise_2007,liu_stochastic_2015,bogachev_fokker-planck-kolmogorov_2015,balan_gentle_2018} and the mathematics itself is far from settled. Chapter \ref{chap_infD_processes} generalizes the work of Tim Rogers and colleagues~\citep{rogers_demographic_2012,rogers_spontaneous_2012,rogers_modes_2015}, and to the best of my knowledge, has never been presented in full generality before. Mathematically, chapter \ref{chap_infD_processes} presents heuristic, accessible alternatives to the rigorous tools of martingale theory and measure-valued branching processes that are usually employed to describe the evolution of quantitative traits~\citep{champagnat_unifying_2006,etheridge_mathematical_2011, week_white_2021} by generalizing the idea of a system size expansion of density-dependent (finite-dimensional) birth-death processes to the infinite-dimensional case using the notion of functional differentiation. Biologically, Chapter \ref{chap_infD_processes} provides `stochastic field equations' that describe the dynamics of one-dimensional quantitative traits in finite populations and illustrates that these equations are consistent with well-known formalisms in quantitative genetics at the infinite population limit. Some concrete models are presented as examples in Appendix \ref{App_examples} for clarity regarding the major ideas.

Part \ref{part_summary} summarizes the major results of our formalism and presents some simple equations that can be argued to be `fundamental theorems' of population biology in the sense of~\cite{queller_fundamental_2017}, which together form a `unifying perspective' in the sense of~\cite{lion_theoretical_2018}. These equations reduce to well-known results such as the Price equation, the replicator-mutator equation from evolutionary game theory, and Fisher's fundamental theorem from population genetics in the infinite population limit. For finite populations, these same equations predict a new evolutionary force, `noise-induced selection', that has still not found its way into the standard formal canon of evolutionary biology and whose significance is only recently being recognized~\citep{constable_demographic_2016,mcleod_social_2019,mazzolini_universality_2022, kuosmanen_turnover_2022}. Implications of noise-induced selection are also discussed in part \ref{part_summary}. Readers who are averse to or do not care for mathematical details can safely skip to Chapter \ref{sec_disc} for the major takeaways of this thesis, though I strongly recommend working through the formalism properly if possible.

%\section{Coexistence of eco-variants in diverse contexts}\label{synthesis}
%The phenomenon of coexistence of multiple phenotypically distinct, ecologically relevant variants of an organism in sympatry occurs in several different contexts in the natural world. This is best illustrated through examples, which I provide below.
%\subsection{Alternative Reproductive Tactics}
%Alternative reproductive tactics (ARTs) are discrete polymorphisms that occur in sympatry in one or both sexes of a species in response to intrasexual competition for mates. More precisely, two or more traits which are expressed in individuals of a given sex of a given species can be said to be ARTs if~\citep{oliveira_alternative_2008}:
%\begin{enumerate}[label=(\alph*)]
%    \item They are \emph{alternative}, in the sense that a given individual can only express one of the behaviors at any given point in time.
%    \item They are \emph{reproductive}, in the sense that the traits are directly relevant to the process of obtaining a mate and are intended to mitigate conspecific intrasexual competition.
%    \item They are \emph{tactics}, in the sense of serving a well-defined adaptive function and thus having fitness consequences. Usually, different ARTs have different associated fitness effects \emph{ceteris paribus}.
%\end{enumerate}
%ARTs are widespread in the animal kingdom, and may be seen either in morphological traits, or as more complex behavioral polymorphisms (see~\citep{oliveira_alternative_2008} for an overview). Most (if not all) species which exhibit ARTs only show a small number of polymorphisms (usually 2-3, 5 in a handful of cases, and very rarely 7-8) - \hl{I can maybe make a table for this: Somehow wasn't able to find one in the literature}. ARTs have obvious consequences for reproductive life history, and can strongly influence the ecology and evolutionary trajectory of organisms.
%
%\subsection{Trophic resource polymorphisms}
%
%Trophic resource polymorphism is defined as ``the occurrence of discrete intraspecific morphs showing differential niche use, usually through differences in feeding biology and habitat use"~\citep{skulason_resource_1995}. Species which exhibit trophic resource polymorphisms often show marked, discontinuous morphological variations adapted for specific niche usage. For example, arctic charr and african cichlid fishes both often exhibit discrete trophic polymorphisms in which different morphs (of the same species) are specialized for feeding at different strata of the lakes they live in~\citep{recknagel_ecosystem_2017}. Such polymorphisms also generally occur in relatively small numbers in sympatry (see Table 1 in~\citep{smith_evolutionary_1996}). They are relatively widespread in vertebrates, and are thought to have important evolutionary consequences by acting as the `substrate' for adaptive radiations~\citep{smith_evolutionary_1996}. 
%
%\subsection{Mating types in isogamous species}
%Isogamous organisms are sexual, but do not have gametes which can be classified into `male' and `female' based on size. Gametes of such species can, however, be divided into distinct `mating types' according to chemical self-incompatibility. There is substantial variation in the number of mating types present in a given species: While most organisms have only two mating types, this is by no means the rule~\citep{phadke_rapid_2009,constable_rate_2018}, with a handful of organisms having 7-10 mating types, and one species of fungus (\textit{Schizophyllum commune}) exhibiting over 23,000 mating types. The study of mating types in isogamous species is ecologically and evolutionarily relevant, not only due to the intrinsic importance of mating and self-incompatibility, but due to the fact that isogamy is thought to be the evolutionary precursor to anisogamy: Thus, studying the factors governing the number of mating types in isogamous species may shed light on why anisogamous species have only two distinct types of gametes.
%
%\subsection{Phenotypic clustering and sympatric speciation}
%\hl{Write stuff about sympatric speciation, adaptive radiations, and phenotypic clustering.}
%\\
%\\
%In all these cases, we are interested in monitoring the changes in the number of distinct phenotypic entities, in a scenario in which the phenotype is ecologically relevant for fitness. Existing literature has tended to use the word `morph' for this concept, and I will bow down to convention and do the same. However, it is important to remember that in this context, `morphs' are only defined in terms of traits that \textit{affect fitness}, in contrast to the more general use of the word in biology to mean any difference whatsoever (as in single nucleotide polymorphisms, for example). Note that in many of these cases, even though the emergent alternative morphs are discretely distributed, the underlying traits need not be, and indeed are often continuously varying, and only categorized into discrete bins by human experimentalists. Such discrete morphs are often persistent in phylogenetically similar lineages~\citep{jamie_persistence_2020}, suggesting they are evolutionarily predictable to some degree, and are also thought to be important for very general evolutionary processes~\citep{west-eberhard_alternative_1986}. Intraspecific variation also provides an important metric for the diversity harboured by populations and can have important ecological consequences. Two important questions immediately arise:
%
%\begin{enumerate}
%    \item How does a population that is initially monomorphic for a (selectively non-neutral) trait evolve to become polymorphic for this trait in a seemingly discrete manner?
%    \item What governs the number of discrete morphs that co-occur in a population?
%\end{enumerate}
%
%This very general setting motivates a view in which an `individual' is simply characterized by its phenotype as a point in some abstract `trait space'. These questions then concern the distribution of points in this space (analogous to the frequency of each `type' of individual) changes over evolutionary time due to ecological fitness determinants (through the traits that the individuals possess). More formally, we can visualize an abstract trait space $\mathcal{T}$ which captures all possible phenotypes that an individual may exhibit. Any phenotype can then be viewed as some point $x \in \mathcal{T}$. We are interested in the distribution of trait values in the population subject to some ecological and evolutionary rules, and a `morph' or `eco-variant' is just a cluster within this trait space (a mode in the distribution).
% \subsection{Sympatric speciation}
% The origin of species is relatively well understood when populations diverge in allopatry (separated by geographic barriers, for example). In contrast, sympatric speciation, first formally proposed by Maynard Smith \hl{cite}, is much less well-understood, despite empirical evidence that it has occurred in nature \hl{cite}. Since speciation is one of the fundamental processes in 

% \section{Frequency-dependent selection}\label{FDS}
% One well-known mechanism of coexistence is `niche partitioning', whereby coexisting types occupy different areas in niche space and thereby reduce competition between types. This of course immediately immediately begs the question of how niche partitioning occurs in the first place. In general, coexistence is also possible without niche partitioning. Such coexistence in the presence of selective pressure (\textit{i.e} when eco-variants are not selectively neutral) occurs through frequency-dependent selection.


%Raymond Pearl, one of the pioneers of mathematical ecology as a discipline, wrote in 1927 that ``What we want to know is how the biological forces of natality and mortality are so integrated and correlated in their action as to lead to a final result in size of population which follows this particular curve rather than some other one''~\citep{pearl_growth_1976}. Pearl realized 95 years ago that population dynamics must be ultimately explained by the mechanistic processes of birth and death. Today, there are mounting calls for more mechanistic models of evolution that are grounded in these fundamental processes of birth and death~\citep{geritz_mathematical_2012,doebeli_towards_2017}. Ecologists also increasingly recognize that incorporating stochasticity is vital to developing more realistic ecological models~\citep{hastings_transients_2004, coulson_skeletons_2004, boettiger_noise_2018, shoemaker_integrating_2020,schreiber_does_2022} and does more than `add noise' to deterministic expectations. Individual-based models, where ecological rules are specified at the level of the individual, are a powerful mathematical tool for mechanistic descriptions of stochastic population dynamics~\citep{black_stochastic_2012}. Birth-death processes are a very general class of stochastic processes that can be used to capture a wide range of eco-evolutionary processes. `System-size expansions' and their subsequent analysis using the `weak noise approximation' are common tools for analyzing stochastic birth-death processes that are well-known in the statistical physics and applied mathematics communities~\citep{gardiner_stochastic_2009}. However, these tools remain relatively underappreciated in ecology, despite being relatively easy to understand and extremely well-motivated in scenarios germane to ecology and evolution. Here, I present a formulation of population dynamics constructed from first principles grounded in birth-death processes. To facilitate readership by a broad audience, only passing familiarity with calculus (derivatives, integrals, Taylor expansions) and probability are assumed. Familiarity with stochastic calculus is helpful for some sections but is not required.  In section \ref{sec_1D_processes}, I show how fluctuating population size of populations of identical individuals can be tracked through a one-dimensional birth-death process. I introduce a description of the system via a `master equation', and then conduct a `system-size expansion' to obtain a Fokker-Planck equation for the system. Finally, I conduct a weak noise approximation to arrive at a linear Fokker-Planck equation which can be solved exactly using some stochastic calculus to arrive at a closed-form solution given by a time-dependent Ornstein-Uhlenbeck process. As an example, I illustrate the complete process for a stochastic version of the logistic equation. In section \ref{sec_nD_processes}, I present a multivariate process to describe the evolution of discretely varying traits, and, as before, illustrate the system size expansion and the weak noise approximation. Under mild assumptions, I show that the deterministic limit of this process is the well-known replicator-mutator equation (or equivalently, Eigen's quasispecies equation), thus establishing the microscopic basis of well-known equations in evolutionary game theory. I also show that the mean value of the trait in the population changes according to the Price equation in the deterministic limit. Chapter \ref{chap_infD_processes} introduces a function-valued process to model the evolution of quantitative traits such as body size, which can take on uncountably many values. This function-valued process can then also be analyzed via a system-size approximation to arrive at a `functional' Fokker-Planck equation, in terms of functional derivatives. Under mild assumptions, I show that classic equations such as Kimura's infinite alleles model and the canonical equation of adaptive dynamics can be derived as the deterministic limits of this stochastic process. I also conduct a weak noise approximation to arrive at a linear functional Fokker-Planck equation.  Sections \ref{sec_1D_processes} and \ref{sec_nD_processes} apply techniques that are well-known in physics, as applied to population biology. Chapter \ref{chap_infD_processes} generalizes the work of Tim Rogers and colleagues~\citep{rogers_demographic_2012,rogers_spontaneous_2012,rogers_modes_2015}, and to the best of my knowledge, is entirely original. Chapter \ref{chap_infD_processes} also presents a simple heuristic derivation of quantitative genetics models and adaptive dynamics from stochastic first principles that is much simpler than the rigorous mathematical derivations grounded in measure theory and martingale/markov theory that are currently standard reference in theoretical ecology~\citep{champagnat_individual_2008}. I illustrate the utility of all this abstract formalism through examples in Chapter \ref{chap_examples}, and discuss the major implications in \ref{chap_unification}.

