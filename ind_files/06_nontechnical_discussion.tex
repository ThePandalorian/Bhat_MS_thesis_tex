\epigraph{\justifying Not only is algebraic reasoning exact; it imposes an exactness on the verbal postulates made before algebra can start which is usually lacking in the first verbal formulations of scientific principles.}{J.B.S. Haldane~\citep{haldane_defense_1964}}

Stochastic finite population models often exhibit behaviors that are markedly different from their deterministic limits. Since real-life populations are stochastic and finite, it is thus imperative that modellers work with stochastic first-principles models instead of their deterministic limits, lest they risk missing important phenomena that are unique to stochastic systems~\citep{black_stochastic_2012,schreiber_does_2022,hastings_transients_2004,shoemaker_integrating_2020}. In trying to work with stochastic models, several theorists have called for a reformulation of eco-evolutionary dynamics of finite populations starting from stochastic birth-death processes on the grounds that such a formulation is more fundamental and mechanistic~\citep{metcalf_why_2007,geritz_mathematical_2012,doebeli_towards_2017}. Part \ref{part_theory} of this thesis develops such a mechanistic theory for both populations of individuals that vary in discrete characters (Chapter \ref{chap_BD}) as well as populations of individuals that vary in a single one-dimensional quantitative character (Chapter \ref{chap_infD_processes}). The central result of this reformulation is an equation for change in type frequencies in the population (equation \eqref{nD_stochastic_RM}) that generalizes the replicator-mutator equation to finite, fluctuating, closed populations evolving in continuous time. From this, we can derive an equation for changes in the population mean of a type-level quantity (equation \eqref{nD_stochastic_Price}) that is a generalization of the Price equation to finite populations, as well as an equation for changes in population variance of a type-level quantity (equation \eqref{nD_stochastic_Price_variance}) in such populations. My work thus generalizes some fundamental formal structures of eco-evolutionary population dynamics to finite, fluctuating populations.

In particular, my equations generalize the unifying formalism described in~\cite{lion_theoretical_2018}\footnote{which itself is a reformulation of the relatively well-known unification of eco-evolutionary dynamics via the Price equation~\citep{frank_natural_2012, queller_fundamental_2017, luque_mirror_2021} in a dynamically sufficient continuous time framework} to finite, fluctuating populations - Taking $K \to \infty$ in equations \eqref{nD_stochastic_RM}, \eqref{nD_stochastic_Price}, and \eqref{nD_stochastic_Price_variance} recover equations (6), (11), and (14) in~\cite{lion_theoretical_2018} respectively as their infinite population limits. Several theorists have called for a reformulation of eco-evolutionary dynamics starting from stochastic birth-death processes on the grounds that such a formulation is more fundamental and mechanistic~\citep{metcalf_why_2007,geritz_mathematical_2012,doebeli_towards_2017}. My work provides some fundamental relations that any such reformulation must satisfy. These relations deal with biologically important quantities, lend themselves to simple biological interpretation, and are very general. They thus fulfill the criteria to be called `fundamental theorems' (in the sense of~\cite{queller_fundamental_2017}), or `unifying principles' (in the sense of~\cite{lion_theoretical_2018}) for the dynamics of finite populations. While my equations recover standard results such as the Price equation and the replicator-mutator equation in the infinite population limit, they also predict that these results do not completely capture the behavior of finite populations.

More precisely, the formalism studied in this thesis predicts a directional evolutionary force acting on variation in per-capita turnover rate $\tau$ that I call `noise-induced selection'. Noise-induced selection is only seen in finite populations and arises due to different types of individuals in the population experiencing a different number of stochastic events (birth and death) in a given time interval. Several specific finite population models illustrate that evolution in finite population can proceed in a direction different from that predicted by the infinite population limit~\citep{parsons_consequences_2010, houchmandzadeh_selection_2012, houchmandzadeh_fluctuation_2015, behar_fluctuations-induced_2016, constable_demographic_2016,abu_awad_effects_2018,parsons_pathogen_2018,mcleod_social_2019}. Indeed, in some stochastic models, the outcome of finite population models can be exactly opposite to that of the infinite population limit, a phenomenon sometimes referred to as `reversing the direction of deterministic selection'~\citep{constable_demographic_2016,mcleod_social_2019}. Most recently, a study showed that in a very broad class of competition models, noise-induced selection in finite populations can act in opposition to the direction of natural selection predicted in the infinite population limit~\citep{mazzolini_universality_2022}. My thesis generalizes this particular result of~\cite{mazzolini_universality_2022} to models with arbitrary interaction types and presents the relevant equations in a formalism that is more in accordance with standard biological models such as the Price equation. It is worth noting that the existence of noise-induced selection directly implies (from equation \eqref{nD_stochastic_RM} or \eqref{nD_stochastic_Price}) that evolution is not expected to maximize fitness in finite populations even if fitness is entirely frequency independent as long as there is some (heritable) variation in the turnover rates $\tau_i$, further underscoring the now well-appreciated point that the view of evolution as `climbing a hill' on a fitness landscape and thereby maximizing fitness is rather limited~\citep{grodwohl_theory_2017}.

The equations of section \ref{sec_fun_theorems} also imply that for the evolution of a trait to be \emph{truly} neutral in finite populations (in the sense of all $m$ types in a system having equal fixation/extinction probability if we start with an initial state in which every type has frequency $1/m$), it is not sufficient for the trait in question to be neutral with respect to fitness $w$. Instead, we also require the trait to be neutral with respect to turnover rate $\tau$.  Indeed, in ecological models, previous work has numerically shown that in finite, fluctuating populations, the equal growth rate of types is not sufficient to ensure equal fixation probabilities and that there is a slight biasing for types with lower turnover rates, sometimes interpreted as a selection `for longevity'~\citep{lin_features_2012, oliveira_advantage_2017,balasekaran_quasi-neutral_2022}. In models of evolutionary game theory in fluctuating finite populations, individuals with lower death rates have higher fixation probability even when growth rates are equalized~\citep{huang_stochastic_2015, czuppon_fixation_2018}. Similarly, models of cell cycle dynamics find that selection favors cell types that periodically arrest their cell cycle relative to non-arresting cells even when their growth rates are equal~\citep{wodarz_effect_2017}. In the language of our birth-death formalism, all of these studies equalize the growth rates $w$ of competing types but allow the turnover to vary (by reducing the birth rate through arresting the cell cycle, for example), thus allowing noise-induced selection for reduced turnover to operate in the system. My derivations show analytically that such deviations from neutrality in finite populations are a generic phenomenon explained by noise-induced selection and should be expected whenever there is variation in turnover rates. Selection on turnover rates also leads to insights on life-history evolution, and these insights have been extensively reported in a recent paper that independently arrives at my equations for type frequencies (equation \eqref{nD_stochastic_RM}) and the change of mean fitness and turnover in the population (equations \eqref{nD_stochastic_Fisher} and \eqref{nD_stochastic_Fisher_turnover} respectively) using certain discrete time stochastic processes and their approximation via techniques reminiscent of numerical stochastic integration~\citep{kuosmanen_turnover_2022}. The equivalent of my stochastic equations has also recently been derived for quantitative traits from a very different starting point\footnote{See Section \ref{sec_disc_field_eqns} above for a detailed discussion on how the field equation approach I used in \ref{chap_infD_processes} for quantitative traits is related to~\cite{week_white_2021}'s SPDE approach} using the theory of measure-valued branching processes~\citep{week_white_2021} - Equations (21b) and (21c) in \cite{week_white_2021} are exactly the $m \to \infty$ version of our equations for changes in the mean value of a type-level quantity and changes in the variance of a type-level quantity respectively for the special case in which the type-level quantity is the value of the quantitative trait being studied.

On the practical side, the existence of noise-induced selection implies that simulation studies working with evolutionary individual-based or agent-based models should be careful about whether interaction effects are incorporated into birth rates or death rates since this seemingly arbitrary choice can have unintended consequences due to noise-induced selection, thus potentially biasing results~\citep{mcleod_social_2019,kuosmanen_turnover_2022}. Being mindful of noise-induced selection is also important for applied fields like conservation and population management which regularly deal with small populations. For example, when trying to increase the population of a hypothetical desired species in a multispecies community, increasing the birth rate is \emph{not} equivalent to reducing the death rate even though both result in an increase in the Malthusian fitness (growth rate) $w_i$. Decreasing the death rate leads to a decrease in $\tau_i$, which leads to positive noise-induced selection, whereas increasing the birth rate leads to an increase in $\tau_i$, which leads to noise-induced selection acting to reduce the abundance of the focal species from the community. If the total community size is small, increasing the birth rate of a species can thus lead to noise-induced selection completely eliminating the focal species from the community despite the fact that we \emph{increased} the growth rate of this species.~\cite{raatz_promoting_2023} have recently used a similar birth-death framework to study cancer treatment and thus provide a concrete example of the consequences of this asymmetry between changing birth rates and death rates: Due to the presence of noise-induced selection, the potential of a tumorous growth to adapt to treatments and experience evolutionary rescue depends inversely on the per-capita turnover $\tau_i$ of the constituent cancer cells in their model, with obvious implications for optimal treatment strategies. My results also indicate that measuring the growth rate of populations is not, in general, sufficient for accurate prediction/inference of future trajectories of the relative abundance of a species (or phenotype, allele, etc.) from empirical data even in completely controlled environments. The growth rate $w_i = b^{\textrm{(ind)}}_{i} - d^{\textrm{(ind)}}_{i}$ of a species $i$ only specifies the difference between its per-capita birth and death rates. In contrast, the complete stochastic dynamics also depend on the total turnover $\tau_i = b^{\textrm{(ind)}}_{i} + d^{\textrm{(ind)}}_{i}$ (\emph{i.e.} the sum of the per-capita birth and death rates).

Lastly, noise-induced selection is particular to fluctuating populations and does not occur in models with fixed population sizes\footnote{If $\sum_j x_j$ is a constant, the map $x_i \to x_i/\sum_j x_j$ becomes a linear map and we no longer need It\^o's formula to move from densities to frequencies in the derivation I conduct in Appendix \ref{App_density_to_freq}; Thus, simply dividing equation \eqref{nD_Ito_SDE} by the (now constant) total population size provides the complete dynamics of the system in frequency space: Note that the directional terms in equation \eqref{nD_Ito_SDE} depend only on $\mathbf{A}^-$, which in turn depends only on the fitness $w_i$ and the mutation terms, and this system thus has no noise-induced selection.} such as the neutral Wright-Fisher or Moran models, suggesting that working with such constant population frameworks is not sufficient to accurately capture the dynamics of real populations. Some previous theoretical studies have pointed out that approximating the population size of natural populations (which fluctuates) through a constant `effective population size' obtained as the harmonic mean of population size over is not always valid~\citep{sjodin_meaning_2005,abu_awad_effects_2018,kuosmanen_turnover_2022}. Recent experimental evolution studies have also directly shown that the harmonic mean of population size cannot always capture evolutionary population dynamics in fluctuating populations~\citep{chavhan_larger_2019}. The fact that noise-induced selection is only seen in fluctuating populations further underscores the general message from these studies - Approximating fluctuating populations via a constant (effective) population size may inadvertently remove important evolutionary properties of the systems under study.

At first glance, the idea of an evolutionary force that selects individuals with lower birth and death rates over individuals with higher birth and death rates may be reminiscent of notions in life-history evolution like $r$ vs $K$ selection or selection on the pace of life in stochastic models~\citep{stearns_evolution_1977}. However, it is unclear whether this similarity reflects some deep principle or whether it is just superficial. Models in life-history theory are often primarily concerned with spatio-temporally fluctuating (external) environments, and thus the stochasticity in those models is extrinsic to the population. Ecological frameworks such as modern coexistence theory, which deal with similar questions about population dynamics as our model and would benefit from a first principles stochastic birth-death formulation, also typically work with fluctuating external environments~\citep{chesson_stabilizing_1982,chesson_multispecies_1994}. We have entirely neglected such extrinsic factors in our formalism. In principle, it is possible to make the birth and death rates \eqref{nD_functional_forms_for_replicator} in our model also depend on a temporally varying external environment $E(t)$ (whose variation may possibly depend on the population $\mathbf{n}(t)$). Incorporating such a term would ensure that the `ecological feedback' terms in equations \eqref{nD_stochastic_Price} and \eqref{nD_stochastic_Price_variance} are non-zero, but may also lead to much more complex dynamics. If the variation of the environment $E(t)$ has some associated stochasticity, the complete dynamics of the system would be the result of interactions between two qualitatively different forms of noise --- \emph{extrinsic} noise from the environment, and \emph{intrinsic} noise from the finiteness of the population --- and thus will be rather complicated and likely difficult to handle analytically. Indeed, experimental studies indicate that (fluctuating) finite population size and fluctuating environments can interact in complex and sometimes counter-intuitive ways, with implications for evolvability and adaptation in spatio-temporally fluctuating environments~\citep{chavhan_larger_2020, chavhan_interplay_2021}. Thus, while integrating the birth-death framework I outline here with ecological ideas such as the pace-of-life syndrome~\citep{mathot_models_2018, wright_life-history_2019} or modern coexistence theory~\citep{chesson_multispecies_1994, johnson_resolving_2022} is biologically appealing, it is likely far from trivial and may present a promising avenue for future work.

~\cite{lion_theoretical_2018} has pointed out that in the dynamic setting (for infinite populations), the replicator-mutator equation \eqref{nD_replicator_mutator} is in some sense the `most fundamental' of the lot, and equations like the Price equation are best viewed as a hierarchy of moment equations for the population mean, population variance, etc. of a type-level quantity. This is also true in my framework - Equation \eqref{nD_stochastic_RM} is the most fundamental equation for population dynamics, and equations like \eqref{nD_stochastic_Price} and \eqref{nD_stochastic_Price_variance} can then be derived from \eqref{nD_stochastic_RM} through repeated application of It\^o's formula, in principle for any moment of the distribution of the quantity $f$ in the population (though this quickly becomes too tedious to actually carry out in practice). If we additionally assume that the quantity $f$ follows a Gaussian distribution in the population, then the mean and variance completely characterize the distribution of $f$, and equations \eqref{nD_stochastic_RM}, \eqref{nD_stochastic_Price}, and \eqref{nD_stochastic_Price_variance} together specify the complete stochastic dynamics of the system. Note that even in this case, actually solving these equations analytically for equilibrium/stationary state distributions of $\mathbf{p}$, $\overline{f}$, and $\sigma^2_f$ will quickly become impossibly difficult if the birth and death rate functions are complicated. Indeed, previous studies indicate that in high dimensions, evolutionary birth-death models can exhibit a dizzying array of complicated phenomena, including limit cycles and evolutionary chaos~\citep{doebeli_diversity_2017}. My framework also works with unstructured populations, neglecting any population structure due to age, class, developmental stage, or space, all of which can lead to complex dynamics and greatly reduce analytical tractability. Lastly, I have also neglected any potential complications introduced by genotype-phenotype maps and genetic processes such as dominance, epistasis, and pleiotropy since they can (in principle) be incorporated while defining the per-capita birth and death rates and our equations hold for arbitrarily complicated birth and death rates as long as they are of the functional form \eqref{nD_functional_forms_for_replicator}.

The points outlined above present practical impediments to actually solving the equations I present for particular models. However, I argue that trying to solve these equations `misses the point' of formulating our model. The equations of section \ref{sec_fun_theorems} are very general, since part \ref{part_theory} makes essentially no assumptions other than density dependence, the impossibility of infinite growth starting from finite population size, and the ability to define per-capita birth and death rates. Further, as we saw in section \ref{sec_fun_theorems}, the terms of the equations lend themselves to simple biological interpretation. Like the classical Price equation, the utility of the equations of section \ref{sec_fun_theorems} lies not (necessarily) in their solutions for specific models, but instead in their generality and the fact that their terms help us clearly understand the various forces operating in biological populations~\citep{frank_natural_2012,luque_one_2017, luque_mirror_2021}. The general spirit of this thesis is thus in line with the idea of building a `model-independent' eco-evolutionary theory that has recently been gaining popularity in the literature~\citep{grafen_formal_2014, queller_fundamental_2017, lion_theoretical_2018, allen_mathematical_2019, rice_universal_2020, week_white_2021, wickman_theoretical_2022, kuosmanen_turnover_2022, mazzolini_universality_2022,lion_extending_2023,allen_natural_2023}. Descriptions such as the classical Price equation and the equations I present in this thesis `abstract away' system-specific details and almost inevitably come at the cost of precision~\citep{levins_strategy_1966, potochnik_idealization_2018}. These approaches are thus intended to complement empirical studies and modelling approaches that carefully study specific systems. To quote Robert Millikan~\citep{millikan_electron_1924}, ``Science walks forward on two feet, namely theory and experiment [...] Sometimes it is one foot which is put forward first, sometimes the other, but continuous progress is only made by the use of both - by theorizing and then testing, or by finding new relations in the process of experimenting and then bringing the theoretical foot up and pushing it on beyond, and so on in unending alterations.''