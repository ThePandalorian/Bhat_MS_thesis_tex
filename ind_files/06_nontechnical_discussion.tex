\epigraph{\justifying Writing in the last months of this millenium, it is clear that the prime intellectual task of the future lies in constructing an appropriate theoretical framework for biology}{Sydney Brenner~\citep{brenner_theoretical_1999}}

\justifying
In this thesis, we have seen how stochastic birth-death processes can be used to construct and analyze mechanistic individual-based models for the dynamics of finite populations. In doing so, we have also seen that various well-known equations of evolutionary dynamics can be recovered in the infinite population size limit. In the finite-dimensional case, the infinite population limit corresponds to the equations of population genetics and evolutionary game theory. In the infinite-dimensional case, we instead obtain the equations of quantitative genetics. In both cases, the mean value of the trait in the population changes according to an equation resembling the Price equation. My derivation highlights the natural connections between various equations of population dynamics --- For example, the same procedures that lead to the replicator-mutator equation in the case of discretely varying traits yield Kimura's continuum-of-alleles model in the quantitative case --- and extends these similarities to finite, fluctuating populations (Table \ref{table_summary}).

{\centering\begin{sideways}
		\begin{minipage}{\textheight}
			\resizebox{\textheight}{!}{%
				\setstretch{1.5}
				\begin{tabular}{ %I manually specified the width of each column by trial and error
						|p{\dimexpr.25\linewidth-2\tabcolsep-1.3333\arrayrulewidth}% column 1
						|p{\dimexpr.27\linewidth-2\tabcolsep-1.3333\arrayrulewidth}% column 2
						|p{\dimexpr.33\linewidth-2\tabcolsep-1.3333\arrayrulewidth}% column 3
						|p{\dimexpr.25\linewidth-2\tabcolsep-1.3333\arrayrulewidth}% column 4
						|p{\dimexpr.4\linewidth-2\tabcolsep-1.3333\arrayrulewidth}|% column 5
					}
					\hline
					\centering \textbf{Number of possible distinct types/trait variants ($m$)} & \centering \textbf{State Space} &
					\centering \textbf{Model parameters} & \centering \textbf{Mesoscopic description} & \centering\arraybackslash \textbf{Infinite population limit} \\
					\hline
					$m = 1$ \newline (Identical individuals)  & $[0,1,2,3,\ldots]$ \newline (Population size) & Two non-negative functions, $b(n)$ and $d(n)$, describing the birth and death rate of individuals when the population size is $n$ & Univariate Fokker-Planck equation \newline (one-dimensional SDEs) & Dynamics of populations of identical individuals\\ 
					\hline
					$1 < m < \infty$ \newline (Discrete traits) & $[0,1,2,3,\ldots]^{m}$ \newline (Number of individuals of each trait variant) & $2m$ non-negative functions, $b_i(\mathbf{n})$ and $d_i(\mathbf{n})$ (for $1 \leq i \leq m$) describing the birth and death rate of trait variant $i$ when the population is $\mathbf{n}=[n_1,\ldots,n_m]$ & Multivariate Fokker-Planck equation \newline ($m$-dimensional SDEs) &  Evolutionary game theory \newline
					Lotka-Volterra competition \newline Quasispecies equation \newline Price equation (discrete traits) \\
					\hline
					$m = \infty$ \newline (Quantitative traits) & $ \left\{\sum\limits_{i=1}^{n}\delta_{x_i} \ \bigg{|} \ n \in \mathbb{N}, \ x_i \in \mathcal{T} \subseteq \mathbb{R}\right\}$ \newline \newline (Each Dirac mass $\delta_{x_i}$ is an individual with trait value $x_i$ in the trait space $\mathcal{T}$) & Two non-negative functionals $b(x|\nu)$ and $d(x|\nu)$ describing the birth and death rate of trait variant $x$ when the population is the (stochastic) field/function $\nu$ & Functional Fokker-Planck equation/Field theory \newline (SPDEs) & Kimura's continuum-of-alleles model \newline~\cite{sasaki_oligomorphic_2011}'s Oligomorphic Dynamics \newline~\cite{wickman_theoretical_2022}'s Trait Space Equations for intraspecific trait variation \newline Gradient Dynamics \newline Price equation (quantitative traits)\\
					\hline
				\end{tabular}
			}
			%\renewcommand\thetable{1}
			\captionof{table}{Summary of the various birth-death processes studied in this thesis}
			\label{table_summary}
		\end{minipage}
	\end{sideways}\par}
\clearpage

Stochastic models in biology often exhibit behaviors that are markedly different from their deterministic limits~\citep{jafarpour_noise-induced_2017,boettiger_noise_2018,jhawar_noise-induced_2020,coomer_noise_2022}. Since real-life populations are stochastic and finite, it is thus imperative that modellers work with stochastic models instead of their deterministic limits, lest they risk missing important phenomena that are unique to stochastic systems~\citep{black_stochastic_2012,hastings_transients_2004,shoemaker_integrating_2020,schreiber_does_2022}. In the context of population biology, the finiteness of populations is an important source of stochasticity that can lead to behavior not captured in corresponding infinite population limits~\citep{black_stochastic_2012,rogers_demographic_2012,debarre_evolutionary_2016,delong_stochasticity_2023}. When actually incorporating this stochasticity into eco-evolutionary theory, it is important to build up the stochastic theory from first principles instead of adding noise to existing deterministic models in an ad-hoc fashion, since the latter procedure can very easily lead to inconsistent models~\citep{strang_how_2019}. Indeed, several theorists have called for a reformulation of eco-evolutionary dynamics of finite populations starting from stochastic birth-death processes on the grounds that such a formulation is more fundamental and mechanistic~\citep{metcalf_why_2007,geritz_mathematical_2012,doebeli_towards_2017}.

In this thesis, I present a mathematical framework for such a reformulation. Part \ref{part_theory} of this thesis develops a formalism for both populations of individuals that vary in discrete characters (Chapter \ref{chap_BD}) as well as populations of individuals that vary in a single one-dimensional quantitative character (Chapter \ref{chap_infD_processes}). The central result of this reformulation is a series of stochastic differential equations derived in Chapter \ref{chap_unification}. To begin, I derive an equation for change in type frequencies in the population (equation \eqref{nD_stochastic_RM}) that generalizes the replicator-mutator equation to finite, fluctuating, closed populations evolving in continuous time. From this, I show how one can derive an equation for changes in the population mean of an arbitrary type-level quantity (equation \eqref{nD_stochastic_Price}) that generalizes the dynamic, continuous time version of the Price equation to finite, fluctuating populations. I also derive an equation for changes in the population variance of an arbitrary type-level quantity (equation \eqref{nD_stochastic_Price_variance}) in such populations that generalizes a recent infinite population formulation of the same~\citep{lion_theoretical_2018}. My work thus generalizes some fundamental formal structures of eco-evolutionary population dynamics to finite, fluctuating populations. In particular, my equations generalize the unifying formalism described in~\cite{lion_theoretical_2018}\footnote{which itself is a reformulation of the relatively well-known unification of eco-evolutionary dynamics via the Price equation~\citep{frank_natural_2012, queller_fundamental_2017, luque_mirror_2021} in a dynamically sufficient, continuous time framework} to finite, fluctuating populations - Taking $K \to \infty$ in equations \eqref{nD_stochastic_RM}, \eqref{nD_stochastic_Price}, and \eqref{nD_stochastic_Price_variance} recover equations (6), (11), and (14) in~\cite{lion_theoretical_2018} respectively as their infinite population limits. While my equations thus recover standard results such as the Price equation and the replicator-mutator equation in the infinite population limit, they also predict that these results do not completely capture the behavior of finite populations.

In Chapter \ref{chap_infD_processes}, I also postulate a `stochastic field theory' approach to modelling the evolution of quantitative traits in finite, fluctuating populations. I then show that this approach is consistent with known frameworks in theoretical population biology, reproducing results from quantitative genetics in the infinite population limit. This formulation also highlights a (somewhat heuristic) approach to the study of spacetime stochastic processes and related mathematical objects that avoids measure-theoretic tools and may be of independent interest to applied mathematicians. Broadly similar stochastic field theoretic approaches have also been proposed in mathematical neurobiology~\citep{buice_field-theoretic_2007,bressloff_stochastic_2010,coombes_neural_2014} and models of collective motion~\citep{o_laighleis_minimal_2018}. I  conjecture that the stochastic `fundamental equations' I derive in section \ref{sec_fun_theorems} for discrete traits should also hold for quantitative traits under mild assumptions on the trait space of the quantitative traits, though no proof is attempted in this thesis. We can, however, take encouragement from the observation that~\cite{week_white_2021} have already derived some special cases of these equations\footnote{In particular, equations (21b) and (21c) in \cite{week_white_2021} are precisely the $m \to \infty$ limit of my equations for changes in the mean value of a type-level quantity (equation \ref{nD_stochastic_Price}) and changes in the variance of a type-level quantity (equation \ref{nD_stochastic_Price_variance}) respectively for the special case in which the type-level quantity is the value of the quantitative trait being studied. See section \ref{sec_disc_field_eqns} for a more detailed technical discussion.} for quantitative traits using an approach grounded in the theory of measure-valued branching processes under certain technical assumptions on the stochastic process under study, the most biologically important assumption being that the traits are normally distributed in the population. My formalism and \cite{week_white_2021}'s formalism are complementary to each other, and reflect a deep duality between related mathematical structures that is well-appreciated in the applied mathematics literature (see section \ref{sec_disc_field_eqns}). These two complementary approaches to modeling stochastic processes have long been recognized in population genetics, where the (simpler analogs of the) general approach in models of fixed total population size go by the names `branching processes approach'~(\cite{week_white_2021}'s formalism; In the classic literature, seen in work like~\cite{haldane_mathematical_1927,fisher_distribution_1931}) and `diffusion theory approach'~(My formalism; In the classic literature, seen in work like~\cite{fisher_dominance_1923,wright_evolution_1931,kimura_problems_1957}).
  
The equations of section \ref{sec_fun_theorems} generically predict a directional evolutionary force acting on variation in per-capita turnover rate $\tau$ that I call `noise-induced selection' (following previous studies such as~\cite{constable_demographic_2016} and~\cite{week_white_2021}). Noise-induced selection is only seen in finite populations, is seen whenever there is differential turnover rate $\tau$ in the system, and arises due to different types of individuals in the population experiencing a different number of stochastic events (birth and death) in a given time interval. Several specific finite population models, especially in social evolution, illustrate that evolution in finite population can proceed in a direction different from that predicted by the infinite population limit~\citep{parsons_consequences_2010,melbinger_evolutionary_2010, houchmandzadeh_selection_2012, houchmandzadeh_fluctuation_2015,chotibut_evolutionary_2015,debarre_evolutionary_2016, behar_fluctuations-induced_2016, constable_demographic_2016,abu_awad_effects_2018,parsons_pathogen_2018,mcavoy_public_2018,mcleod_social_2019}. Indeed, in some stochastic models, the outcome of finite population models can be exactly opposite to that of the infinite population limit, a phenomenon sometimes referred to as `reversing the direction of deterministic selection'~\citep{constable_demographic_2016,mcleod_social_2019}. Chapter \ref{chap_unification} explains how the  results of all these studies can be explained using noise-induced selection. Most recently, a study showed that in a very broad class of competition models, finite populations can directionally evolve in a manner opposite to the direction of evolution predicted in the infinite population limit, essentially due to the effects of noise-induced selection~\citep{mazzolini_universality_2022}. My thesis generalizes this particular result of~\cite{mazzolini_universality_2022} to models with arbitrary interaction types and presents the relevant equations in a form that highlights connections with standard results in evolutionary theory such as the Price equation. It is worth noting that the existence of noise-induced selection directly implies (from equation \eqref{nD_stochastic_RM} or \eqref{nD_stochastic_Price}) that evolution is not expected to maximize fitness in finite populations even if fitness is entirely frequency independent as long as there is some (heritable) variation in the turnover rates $\tau_i$, further underscoring the now well-appreciated point that the view of evolution as `climbing a hill' on a fitness landscape and thereby maximizing fitness is rather limited~\citep{grodwohl_theory_2017}.

The equations of section \ref{sec_fun_theorems} also imply that for the evolution of a trait to be \emph{truly} neutral in finite populations (in the sense of all $m$ types in a system having equal fixation/extinction probability if we start with an initial state in which every type has frequency $1/m$), it is not sufficient for the trait in question to be neutral with respect to fitness $w$. Instead, we also require the trait to be neutral with respect to turnover rate $\tau$.  Indeed, in ecological models, previous work has numerically shown that in finite, fluctuating populations, the equal growth rate of types is not sufficient to ensure equal fixation probabilities and that there is a slight biasing for types with lower turnover rates, sometimes interpreted as a selection `for longevity'~\citep{lin_features_2012, oliveira_advantage_2017,balasekaran_quasi-neutral_2022}. In models of evolutionary game theory in fluctuating finite populations, individuals with lower death rates have higher fixation probability even when growth rates are equalized~\citep{huang_stochastic_2015, czuppon_fixation_2018}. Similarly, models of cell cycle dynamics find that selection favors cell types that periodically arrest their cell cycle relative to non-arresting cells even when their growth rates are equal~\citep{wodarz_effect_2017}. In the language of the birth-death formalism I develop in this thesis, all of these studies equalize the growth rates $w$ of competing types but allow the turnover to vary (by reducing the birth rate through arresting the cell cycle, for example), thus allowing noise-induced selection for reduced turnover to operate in the system. My derivations show analytically that such deviations from neutrality in finite populations are a generic phenomenon explained by noise-induced selection and should be expected whenever there is variation in turnover rates. Selection on turnover rates also leads to insights on life-history evolution, and these insights have been extensively reported in a recent pre-print that independently arrives at my equations for type frequencies (equation \eqref{nD_stochastic_RM}) and the change of mean fitness and turnover in the population (equations \eqref{nD_stochastic_Fisher} and \eqref{nD_stochastic_Fisher_turnover} respectively) using certain discrete time stochastic processes and their approximation via techniques reminiscent of numerical stochastic integration~\citep{kuosmanen_turnover_2022}. 

On the practical side, the existence of noise-induced selection implies that simulation studies working with evolutionary individual-based or agent-based models should be careful about whether interaction effects are incorporated into birth rates or death rates since this seemingly arbitrary choice can have unintended consequences due to noise-induced selection, thus potentially biasing results~\citep{mcleod_social_2019,kuosmanen_turnover_2022}. My results also indicate that measuring the growth rate of populations is not, in general, sufficient for accurate prediction/inference of future trajectories of the relative abundance of a species (or phenotype, allele, etc.) from empirical data even in completely controlled environments. The growth rate $w_i = b^{\textrm{(ind)}}_{i} - d^{\textrm{(ind)}}_{i}$ of a species $i$ only specifies the difference between its per-capita birth and death rates. In contrast, the complete stochastic dynamics also depend on the total turnover $\tau_i = b^{\textrm{(ind)}}_{i} + d^{\textrm{(ind)}}_{i}$ (\emph{i.e.} the sum of the per-capita birth and death rates).

Being mindful of noise-induced selection is also important for applied fields like conservation and population management which regularly deal with small populations. For example, when trying to increase the population of a hypothetical desired species in a multispecies community, increasing the birth rate is \emph{not} equivalent to reducing the death rate even though both result in an increase in the Malthusian fitness (growth rate) $w_i$. Decreasing the death rate leads to a decrease in $\tau_i$, which leads to positive noise-induced selection, whereas increasing the birth rate leads to an increase in $\tau_i$, which leads to noise-induced selection acting to reduce the abundance of the focal species from the community. If the total community size is small, increasing the birth rate of a species can thus lead to noise-induced selection completely eliminating the focal species from the community despite the fact that we \emph{increased} the growth rate of this species. Indeed, numerical investigations of `burst-death' dynamics in fluctuating viral populations show that increasing Malthusian fitness by boosting the survival rate (\emph{i.e.} reducing the death rate) leads to a greater increase in fixation probability than if an exactly equivalent increase in Malthusian fitness is achieved via increasing the burst rate (\emph{i.e.} increasing the birth rate) of viral particles~\citep{alexander_fixation_2008}.  A recent finite population birth-death model for cancer treatment provides another concrete example of the consequences of the asymmetry between changing birth rates and death rates: Due to the presence of noise-induced selection, the potential of a tumorous growth to adapt to treatments and experience evolutionary rescue depends inversely on the per-capita turnover $\tau_i$ of the constituent cancer cells, with obvious implications for optimal treatment strategies~\citep{raatz_promoting_2023}.

Lastly, noise-induced selection is particular to fluctuating populations and does not occur in models with fixed population sizes\footnote{If $\sum_j x_j$ is a constant, the map $x_i \to x_i/\sum_j x_j$ becomes a linear map and we no longer need It\^o's formula to move from densities to frequencies in the derivation I conduct in Appendix \ref{App_density_to_freq}; Thus, simply dividing equation \eqref{nD_Ito_SDE} by the (now constant) total population size provides the complete dynamics of the system in frequency space: Note that the directional terms in equation \eqref{nD_Ito_SDE} depend only on $\mathbf{A}^-$, which in turn depends only on the fitness $w_i$ and the mutation terms, and this system thus has no noise-induced selection.} such as the neutral Wright-Fisher or Moran models, suggesting that working with such constant population frameworks is not sufficient to accurately capture the dynamics of real populations. Some previous theoretical studies have pointed out that approximating the population size of fluctuating populations through a constant `effective population size' obtained as the harmonic mean of population size over time is not always valid~\citep{sjodin_meaning_2005,parsons_consequences_2010,iizuka_effective_2010,abu_awad_effects_2018,kuosmanen_turnover_2022}. Recent experimental evolution studies have also directly shown that the harmonic mean of population size need not be a good proxy to capture (even approximate) evolutionary dynamics in fluctuating populations~\citep{chavhan_larger_2019}. The fact that noise-induced selection is only seen in fluctuating populations further underscores the general message from these studies - Approximating fluctuating populations via a constant (effective) population size may inadvertently remove important evolutionary properties of the systems under study.

At first thought, the idea of an evolutionary force that selects individuals with lower birth and death rates over individuals with higher birth and death rates may be reminiscent of notions in life-history evolution like $r$ vs $K$ selection or selection on the pace of life~\citep{stearns_evolution_1977}. However, it is not possible to conclude whether this similarity reflects some deep principle or whether it is just superficial based solely on the work conducted in this thesis. Models in life-history theory are often primarily concerned with spatiotemporally fluctuating external environments, and thus the stochasticity at the core of life-history models is extrinsic to the population. Ecological frameworks such as modern coexistence theory, which also deal with questions about similar population dynamics and would benefit from a first principles stochastic birth-death formulation, also typically work with fluctuating external environments~\citep{chesson_stabilizing_1982,chesson_multispecies_1994}. I have entirely neglected such extrinsic factors in my formalism. In principle, it is possible to make the birth and death rates \eqref{nD_functional_forms_for_replicator} in my framework also depend on a temporally varying external environment $E(t)$ (whose variation may possibly depend on the population $\mathbf{n}(t)$). Incorporating such a term would ensure that the `ecological feedback' terms in equations \eqref{nD_stochastic_Price} and \eqref{nD_stochastic_Price_variance} are non-zero, but may also lead to much more complex dynamics. If the variation of the environment $E(t)$ has some associated stochasticity, the complete dynamics of the system would be the result of interactions between two qualitatively different forms of noise --- \emph{extrinsic} noise from the environment, and \emph{intrinsic} noise from the finiteness of the population --- and thus can be rather complicated~\citep{gokhale_eco-evolutionary_2016} and likely difficult to handle analytically. Indeed, experimental studies indicate that (fluctuating) finite population size and fluctuating environments can interact in complex and sometimes counter-intuitive ways, with implications for evolvability and adaptation in spatiotemporally fluctuating environments~\citep{chavhan_larger_2020, chavhan_interplay_2021}. Thus, while integrating the birth-death framework I outline here with ecological ideas such as the pace-of-life syndrome~\citep{mathot_models_2018, wright_life-history_2019} or modern coexistence theory~\citep{chesson_multispecies_1994, johnson_resolving_2022} is biologically appealing, it is likely far from trivial and may present a promising avenue for future work.

~\cite{lion_theoretical_2018} has pointed out that in the dynamic setting (for infinite populations), the replicator-mutator equation \eqref{nD_replicator_mutator} is in some sense the `most fundamental' of the lot, and equations like the Price equation are best viewed as a hierarchy of moment equations for the population mean, population variance, etc. of a type-level quantity. This is also true in my framework - Equation \eqref{nD_stochastic_RM} is the most fundamental equation for population dynamics, and equations like \eqref{nD_stochastic_Price} and \eqref{nD_stochastic_Price_variance} can then be derived from \eqref{nD_stochastic_RM} through repeated application of It\^o's formula, in principle for any moment of the distribution of the quantity $f$ in the population (though this quickly becomes too tedious to actually carry out in practice). If we additionally assume that the quantity $f$ follows a Gaussian distribution in the population, then the mean and variance completely characterize the distribution of $f$, and equations \eqref{nD_stochastic_RM}, \eqref{nD_stochastic_Price}, and \eqref{nD_stochastic_Price_variance} together specify the complete stochastic dynamics of the system. Note that even in this case, actually solving these equations analytically for equilibrium/stationary state distributions of $\mathbf{p}$, $\overline{f}$, and $\sigma^2_f$ will quickly become impossibly difficult if the birth and death rate functions are complicated. 

Previous studies indicate that in high dimensions, evolutionary birth-death models can exhibit a dizzying array of complicated phenomena, including limit cycles and evolutionary chaos~\citep{doebeli_diversity_2017}. I have also neglected any potential complications introduced by genotype-phenotype maps and genetic processes such as dominance, epistasis, and pleiotropy since they can (in principle) be incorporated while defining the per-capita birth and death rates and the equations I derive hold for arbitrarily complicated birth and death rates as long as they are of the functional form \eqref{nD_functional_forms_for_replicator}. However, while these facts may present impediments to formulating exact solutions to the equations I formulate, the point of these equations is not necessarily to solve them to begin with.

Indeed, the most important takeaway from this thesis is that the equations of section \ref{sec_fun_theorems} are very general, since part \ref{part_theory} makes essentially no assumptions other than density dependence, the impossibility of infinite growth starting from finite population size, and the ability to define per-capita birth and death rates, and thus, these birth and death rate function(al)s can in general be almost arbitrarily complicated. Further, as we saw in section \ref{sec_fun_theorems}, the terms of the equations lend themselves to simple biological interpretation. Like the classical Price equation, the utility of the equations of section \ref{sec_fun_theorems} lies not (necessarily) in their solutions for specific models, but instead in their generality and the fact that their terms help us clearly think about the various evolutionary phenomena operating in biological populations~\citep{frank_natural_2012,luque_one_2017, luque_mirror_2021}. The general spirit of this thesis is thus in line with the general idea of trying to formulate `model-independent' eco-evolutionary theory that has recently been (rapidly) gaining popularity in the literature~\citep{grafen_formal_2014, queller_fundamental_2017, lion_theoretical_2018, allen_mathematical_2019, rice_universal_2020, week_white_2021, wickman_theoretical_2022, kuosmanen_turnover_2022, mazzolini_universality_2022,lion_extending_2023,allen_natural_2023}. However, there are some biologically important factors that I have neglected in my formalism. My framework works with unstructured populations, neglecting any potential effects of groups, age, class, sex, developmental stage, or space\footnote{Of course,  since position in continuous space is just a special case of a quantitative trait, we have technically incorporated space in a very limited sense: The formalism of chapter \ref{chap_infD_processes} can equally well describe a finite population of exactly identical individuals moving through one-dimensional space, and so can technically describe phenomena like range expansion of clonal populations. However, I am only noting this as a technicality - Such populations are likely somewhat `evolutionarily boring', since all individuals in the model must always be exactly identical in all aspects other than spatial location.}, all of which can lead to very complex and often surprising dynamics. Explicitly incorporating features such as population structure, sex, and space from first principles in an analytically tractable framework is a formidable task that may need innovative new mathematical and biological arguments, and presents a fantastic opportunity for future studies.

Descriptions such as the classic Price equation and the equations I present in this thesis `abstract away' system-specific details and almost inevitably come at the cost of precision~\citep{levins_strategy_1966, potochnik_idealization_2018}. These approaches are thus intended to complement empirical studies and modelling approaches that carefully study specific systems and generate vital knowledge about how these systems behave. To quote Robert Millikan~\citep{millikan_electron_1924}, ``Science walks forward on two feet, namely theory and experiment [...] Sometimes it is one foot which is put forward first, sometimes the other, but continuous progress is only made by the use of both - by theorizing and then testing, or by finding new relations in the process of experimenting and then bringing the theoretical foot up and pushing it on beyond, and so on in unending alterations.''