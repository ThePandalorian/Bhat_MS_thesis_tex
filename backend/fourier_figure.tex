%from various answers on https://tex.stackexchange.com/questions/127375/replicate-the-fourier-transform-time-frequency-domains-correspondence-illustrati

\documentclass[tikz]{standalone}
\usepackage{pgfplots}
\pgfplotsset{compat=1.8}
\usepackage{xcolor}

\usetikzlibrary{3d}
%Packages

%references
\usepackage[utf8]{inputenc}
\usepackage[english]{babel}
\usepackage{csquotes}
\usepackage[
			sorting=nyc,
            backend=biber, %backend to use
            style=authoryear, %citation style
            uniquelist=false,
            natbib,
            block=ragged,
            maxnames=2,
            maxbibnames=99]{biblatex}

%Make the title of the bibliography say 'References'
\DefineBibliographyStrings{english}{%
	bibliography = {References},
}

%Sort by name-year-cite order (biblatex default is to sort by name-year-title).
\DeclareSortingTemplate{nyc}{
	\sort{
		\field{presort}
	}
	\sort[final]{
		\field{sortkey}
	}
	\sort{
		\field{sortname}
		\field{author}
		\field{editor}
		\field{translator}
		\field{sorttitle}
		\field{title}
	}
	\sort{
		\field{sortyear}
		\field{year}
	}
	\sort{\citeorder}
}

%%%%%%%%%%%%%%
%% Modify citations to follow the style of Cell
%% from https://tex.stackexchange.com/a/404787
\usepackage{xpatch}

% Some general changes
\DeclareNameAlias{sortname}{last-first}
\renewcommand*{\bibinitdelim}{}
\renewbibmacro*{in:}{%
    \iffieldequalstr{entrytype}{inproceedings}{%
        \printtext{\bibstring{in}\addspace}%
    }{}%
}

% Changes for Book
\csletcs{abx@macro@publisher+location+date@orig}{abx@macro@publisher+location+date}
\renewbibmacro*{publisher+location+date}{%
    \printtext[parens]{\usebibmacro{publisher+location+date@orig}}
}
\DeclareFieldFormat[book]{title}{#1\printunit{\addspace}}

% Changes for inproceedings
\DeclareFieldFormat[inproceedings]{title}{#1\isdot}
\DeclareFieldFormat{booktitle}{#1\addcomma}
\xpatchbibmacro{byeditor+others}{%
    \usebibmacro{byeditor+othersstrg}%
    \setunit{\addspace}%
    \printnames[byeditor]{editor}%
    \clearname{editor}%
}{%
    \printnames[byeditor]{editor}%
    \clearname{editor}
    \addcomma\addspace
    \bibstring{editor}
    \setunit{\addspace}%
}{}{}

% Changes in Article
\DeclareFieldFormat[article]{title}{#1}
\DeclareFieldFormat[article]{journaltitle}{#1\isdot}
\DeclareFieldFormat[article]{volume}{\textit{#1}}
\DeclareFieldFormat[article]{pages}{#1}


%%%%%%%%%%%%%

\usepackage{titlesec} % Format chapter headings
%from https://tex.stackexchange.com/a/52474

%Add a line b/w chapter number and heading
%for numbered chapters
\titleformat{\chapter}[display]
  {\normalfont\bfseries\Huge}
  {\chaptertitlename~\thechapter}{1pc}
  {{\color{gray}\titlerule[2pt]}\vspace{1pc}}

%Don't add a line for unnumbered chapters  
\titleformat{name=\chapter,numberless}[display]
  {\normalfont\bfseries\Huge}{}{1pc}
  {}

%reduce spacing before and after section titles
%this is to be read {left spacing}{before spacing}{after spacing}
%spacing: how to read {12pt plus 4pt minus 2pt}
%           12pt is what we would like the spacing to be
%           plus 4pt means that TeX can stretch it by at most 4pt
%           minus 2pt means that TeX can shrink it by at most 2pt
\titlespacing\section{0pt}{0pt plus 0pt minus 1pt}{0pt plus 0pt minus 1pt}
\titlespacing\subsection{0pt}{0pt plus 0pt minus 1pt}{0pt plus 0pt minus 1pt}
\titlespacing\subsubsection{0pt}{0pt plus 0pt minus 1pt}{0pt plus 0pt minus 1pt}

%Add a custom strut to increase vertical space given to equations when combining with underbrace
%from https://tex.stackexchange.com/a/13864
\newcommand*\mystrut[1]{\vrule width0pt height0pt depth#1\relax}

%%%%%%%%%%%%%
\usepackage{epigraph} %for quotes
\renewcommand{\textflush}{flushright} %quotes are right aligned

\usepackage{amsthm, amsmath, amssymb} % Mathematical typesetting
\usepackage{mathrsfs} %fancy fonts for sigma-algebras
\usepackage{float} % Improved interface for floating objects
\usepackage[final, colorlinks = true, 
            linkcolor = black, 
            citecolor = black,
            breaklinks=true]{hyperref} % For hyperlinks in the PDF
\usepackage{graphicx, multicol} % Enhanced support for graphics
\usepackage{xcolor} % Driver-independent color extensions
\usepackage{framed}
\usepackage[normalem]{ulem} %underlining
\usepackage{amsfonts}
\usepackage{enumitem}
\usepackage{mathtools}
\usepackage{multicol}
\usepackage{color,soul}

\usepackage[toc, title]{appendix} %Appendix

\usepackage[labelfont=bf]{caption} %bold captions on figures and tables

%for tables
\usepackage{makecell,tabularx}
%\setlength{\extrarowheight}{12pt} %additive padding
\renewcommand{\arraystretch}{2} %multiplicative padding 
\renewcommand\theadfont{\small\bfseries}
\usepackage{rotating}
\usepackage{setspace}

%For pseudocode
\usepackage[boxed]{algorithm2e}
\DontPrintSemicolon

%A bunch of definitions that make my life easier
\newtheorem{theorem}{Theorem}[section]
\newtheorem{corollary}{Corollary}[section]
\theoremstyle{definition}
\newtheorem*{definition}{Definition}
\newtheorem{example}{Example}
\newtheorem*{note}{Note}
\newtheorem*{claim}{Claim}
\newcommand{\bproof}{\bigskip {\bf Proof. }}
\newcommand{\eproof}{\hfill\qedsymbol}
\newcommand{\Disp}{\displaystyle}
\newcommand{\qe}{\hfill\(\bigtriangledown\)}
\setlength{\columnseprule}{1 pt}

%for characters inside circles
%syntax is \circled{character}
\usepackage{tikz}
\newcommand*\circled[1]{\tikz[baseline=(char.base)]{
            \node[shape=circle,draw,inner sep=1pt] (char) {#1};}}


\usepackage{pgfplots}
\pgfplotsset{compat=1.8}


\usepackage[mode=buildnew]{standalone} %for loading precompiled tikz figures

%%%%%%%%%%%%%%%%%%%%%%%%
% Defines the `mycase` environment for cases in proofs
\newcounter{cases}
\newcounter{subcases}[cases]
\newenvironment{mycase}
{
    \setcounter{cases}{0}
    \setcounter{subcases}{0}
    \newcommand{\case}
    {
        \stepcounter{cases}\textbf{Case \thecases.}
    }
    \newcommand{\subcase}
    {
        \par\indent\stepcounter{subcases}\textit{Subcase (\thesubcases):}
    }
}
{
    \par
}
\renewcommand*\thecases{\arabic{cases}}
\renewcommand*\thesubcases{\roman{subcases}}

%For easily making figures
%syntax is:
%\myfig{scaling_factor}{name_of_file}{caption}{label}
\newcommand{\myfig}[4]{\begin{figure}[h] \begin{center} \includegraphics[width=#1\textwidth]{#2} \caption{#3} \label{#4} \end{center} \end{figure}}

% horizontal line across the page
\newcommand{\horz}{
\vspace{-.4in}
\begin{center}
\begin{tabular}{p{\textwidth}}\\
\hline
\end{tabular}
\end{center}
}

%Resize the summation symbol
%syntax is \sum[size] 
\newlength{\depthofsumsign}
\setlength{\depthofsumsign}{\depthof{$\sum$}}
\newlength{\totalheightofsumsign}
\newlength{\heightanddepthofargument}
\newcommand{\bigsum}[1][1.4]{% only for \displaystyle
    \mathop{%
        \raisebox
            {-#1\depthofsumsign+1\depthofsumsign}
            {\scalebox
                {#1}
                {$\displaystyle\sum$}%
            }
    }
}


\begin{document}


\definecolor{lightgray}{rgb}{0.8, 0.8, 0.8} %grid of coordinate system, axes
\definecolor{midgray}{rgb}{0.6, 0.6, 0.6} %layer of border
\definecolor{darkgray}{rgb}{0.4, 0.4, 0.4} %curves, fill under curve


%x=(-18:0.9), y=(90:0.9), z=(42:1.1) <-- full 3D view
    \begin{tikzpicture}[scale=2.4] %create tikz picture
    
        \begin{axis}[ %create 3d plot within tikz
         set layers=standard, %use predefined layers
         view={30}{35}, %perspective adjustment %25 30
        domain=0:10, %plot limit in trait space direction
        samples y=1, %samples for dual space direction
        unit vector ratio*=1 2 1, %rescale unit vectors
        hide axis, %do not plot axes
        xtick=\empty, ytick=\empty, ztick=\empty, %no ticks on coordinate axes
        clip=false %let me plot outside the coordinate system
    ]
        %limit variables
        \def\xmax{100} %limits for curves and layers
        \def\xmin{0}
        \def\ymax{55}
        \def\ymin{5}
        \def\zmax{25}
        \def\zmin{-5}
        \def\xlayer{110} %dual space layer
        \def\sumcurve{0} %sum curve of fourier modes
        \def\amp{0.3}

        %fourier mode curves
        \pgfplotsinvokeforeach{1,2,3,4,5}{ %for each fourier mode
            
            \xdef\amp{0.08*(6-#1)}        
            \draw [<->,on layer=background, lightgray] (axis cs:0,#1,0) -- (axis cs:9.5,#1,0); %axes
            \addplot3 [on layer=main, blue, smooth, samples=200,domain=-0.5:9.5] (x-1,#1,{\amp*cos(100*#1/2*(x) - 50*#1)}); %plot curves 

            \xdef\sumcurve{\sumcurve + \amp*cos(100*#1/2*(x) - 50*#1)} %add current curve to sumcurve
        }

        %transparent layers
        \fill[white,opacity=0.7] (\xmin,0,\zmin) -- (\xmin,0,\zmax) -- (\xmax,0,\zmax) -- (\xmax,0,\zmin) -- cycle; %transparent layer in trait space
        \fill[white,opacity=0.7] (\xlayer,\ymin,\zmin) -- (\xlayer,\ymin,\zmax) -- (\xlayer,\ymax,\zmax) -- (\xlayer,\ymax,\zmin) -- cycle; % transparent layer for dual space

        %grid lines
        \pgfplotsinvokeforeach{\xmin,\xmin+5,...,\xmax}{ %create horizontal grid lines (trait space layer)
            \draw[lightgray,opacity=0.6] (#1,0,\zmin) -- (#1,0,\zmax);
        }
        \pgfplotsinvokeforeach{\ymin,\ymin+2.5,...,\ymax}{ %create horizontal grid lines (dual space layer)
            \draw[lightgray,opacity=0.6] (\xlayer,#1,\zmin) -- (\xlayer,#1,\zmax);
        }
        \pgfplotsinvokeforeach{\zmin,\zmin+5,...,\zmax}{ %create vertical grid lines (both layers)
            \draw[lightgray,opacity=0.6] (\xmin,0,#1) -- (\xmax,0,#1);
            \draw[lightgray,opacity=0.6] (\xlayer,\ymin,#1) -- (\xlayer,\ymax,#1);
        }

        %borders
        \draw[midgray] (\xmin,0,\zmin) -- (\xmin,0,\zmax) -- (\xmax,0,\zmax) -- (\xmax,0,\zmin) -- cycle; %trait space layer borders
        \draw[midgray] (\xlayer,\ymin,\zmin) -- (\xlayer,\ymin,\zmax) -- (\xlayer,\ymax,\zmax) -- (\xlayer,\ymax,\zmin) -- cycle; %dual space layer borders

        %dots on dual space pane
        % \pgfplotsinvokeforeach{1,2,3,4,5}{ %for each fourier mode
        %         \node[circle,scale=1,fill=blue,on layer=main] at (axis cs:9.5,#1,0.4*5-0.4*#1) {}; %add dots on dual space pane
        % }
        
        %trait space curve
        \addplot3 [samples=1000,red,thick] (x-1,0,{\sumcurve});

    % dual space curve (for some reason removing this screws up the diagram, so I've just made it transparent)
    \addplot3 [samples=200,domain=0:5,green,opacity=0] (11.5,x,{2*sin((x-0.7)*180)^200*e^(-x/2)-0.3});

    %trait space domain axes and labels
    \begin{scope}
    \draw[<->,thick] (\xmin-5,0,\zmin) -- (\xmax+5,0,\zmin);
    \draw[->,thick] (0,0,\zmin) -- (0,0,\zmax+3);
    \node[scale=37, canvas is xz plane at y=0] at (0.45*\xmax,0,\zmin-8) {Trait Space $\mathcal{T}$};
    \node[scale=27, canvas is xz plane at y=0, rotate=90] at (\xmin-5,0,0.45*\zmax) {Density $\phi(x)$};
    \end{scope}

    %dual domain axes and labels
    \begin{scope}
    \draw[<->,thick] (\xlayer,\ymin,\zmin) -- (\xlayer,\ymax+4,\zmin);
    \draw[->,thick] (\xlayer,\ymax,\zmin) -- (\xlayer,\ymax,\zmax+3);
    \node[scale=20,canvas is yz plane at x=11.5] at (axis cs:0,3,-1.75) {Fourier dual space $\Tilde{\mathcal{T}}$};
    \node[scale=20,canvas is yz plane at x=11.5, rotate=90] at (axis cs:0,5.8,0.6) {Magnitude of $\phi_k$};
    \pgfplotsinvokeforeach{1,2,3,4,5}{
        \node[scale=20,blue,canvas is yz plane at x=11.5] at (axis cs:0,#1,-1.2) {$\phi_{#1}$}; %add labels

        \node[circle,scale=20,fill=blue,canvas is yz plane at x=11.5] at (axis cs:0,#1,0.4*5-0.4*#1) {};
    }
    \end{scope}
    \end{axis}
    \end{tikzpicture}


\end{document}