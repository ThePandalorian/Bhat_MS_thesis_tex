\documentclass[twoside, 12pt]{iiser-thesis}

%Bio dept wants all font in Arial (ew)
%Will keep it commented until the v end for my own sanity. I refuse to actively change a tex file to Arial permanently. Change compiler to XeLaTeX or LuaLaTeX for this to work
%\usepackage{fontspec}
%\setmainfont{Arial}

%my own list of packages
%Packages

%references
\usepackage[utf8]{inputenc}
\usepackage[english]{babel}
\usepackage{csquotes}
\usepackage[
			sorting=nyc,
            backend=biber, %backend to use
            style=authoryear, %citation style
            uniquelist=false,
            natbib,
            block=ragged,
            maxnames=2,
            maxbibnames=99]{biblatex}

%Make the title of the bibliography say 'References'
\DefineBibliographyStrings{english}{%
	bibliography = {References},
}

%Sort by name-year-cite order (biblatex default is to sort by name-year-title).
\DeclareSortingTemplate{nyc}{
	\sort{
		\field{presort}
	}
	\sort[final]{
		\field{sortkey}
	}
	\sort{
		\field{sortname}
		\field{author}
		\field{editor}
		\field{translator}
		\field{sorttitle}
		\field{title}
	}
	\sort{
		\field{sortyear}
		\field{year}
	}
	\sort{\citeorder}
}

%%%%%%%%%%%%%%
%% Modify citations to follow the style of Cell
%% from https://tex.stackexchange.com/a/404787
\usepackage{xpatch}

% Some general changes
\DeclareNameAlias{sortname}{last-first}
\renewcommand*{\bibinitdelim}{}
\renewbibmacro*{in:}{%
    \iffieldequalstr{entrytype}{inproceedings}{%
        \printtext{\bibstring{in}\addspace}%
    }{}%
}

% Changes for Book
\csletcs{abx@macro@publisher+location+date@orig}{abx@macro@publisher+location+date}
\renewbibmacro*{publisher+location+date}{%
    \printtext[parens]{\usebibmacro{publisher+location+date@orig}}
}
\DeclareFieldFormat[book]{title}{#1\printunit{\addspace}}

% Changes for inproceedings
\DeclareFieldFormat[inproceedings]{title}{#1\isdot}
\DeclareFieldFormat{booktitle}{#1\addcomma}
\xpatchbibmacro{byeditor+others}{%
    \usebibmacro{byeditor+othersstrg}%
    \setunit{\addspace}%
    \printnames[byeditor]{editor}%
    \clearname{editor}%
}{%
    \printnames[byeditor]{editor}%
    \clearname{editor}
    \addcomma\addspace
    \bibstring{editor}
    \setunit{\addspace}%
}{}{}

% Changes in Article
\DeclareFieldFormat[article]{title}{#1}
\DeclareFieldFormat[article]{journaltitle}{#1\isdot}
\DeclareFieldFormat[article]{volume}{\textit{#1}}
\DeclareFieldFormat[article]{pages}{#1}


%%%%%%%%%%%%%

\usepackage{titlesec} % Format chapter headings
%from https://tex.stackexchange.com/a/52474

%Add a line b/w chapter number and heading
%for numbered chapters
\titleformat{\chapter}[display]
  {\normalfont\bfseries\Huge}
  {\chaptertitlename~\thechapter}{1pc}
  {{\color{gray}\titlerule[2pt]}\vspace{1pc}}

%Don't add a line for unnumbered chapters  
\titleformat{name=\chapter,numberless}[display]
  {\normalfont\bfseries\Huge}{}{1pc}
  {}

%reduce spacing before and after section titles
%this is to be read {left spacing}{before spacing}{after spacing}
%spacing: how to read {12pt plus 4pt minus 2pt}
%           12pt is what we would like the spacing to be
%           plus 4pt means that TeX can stretch it by at most 4pt
%           minus 2pt means that TeX can shrink it by at most 2pt
\titlespacing\section{0pt}{0pt plus 0pt minus 1pt}{0pt plus 0pt minus 1pt}
\titlespacing\subsection{0pt}{0pt plus 0pt minus 1pt}{0pt plus 0pt minus 1pt}
\titlespacing\subsubsection{0pt}{0pt plus 0pt minus 1pt}{0pt plus 0pt minus 1pt}

%Add a custom strut to increase vertical space given to equations when combining with underbrace
%from https://tex.stackexchange.com/a/13864
\newcommand*\mystrut[1]{\vrule width0pt height0pt depth#1\relax}

%%%%%%%%%%%%%
\usepackage{epigraph} %for quotes
\renewcommand{\textflush}{flushright} %quotes are right aligned

\usepackage{amsthm, amsmath, amssymb} % Mathematical typesetting
\usepackage{mathrsfs} %fancy fonts for sigma-algebras
\usepackage{float} % Improved interface for floating objects
\usepackage[final, colorlinks = true, 
            linkcolor = black, 
            citecolor = black,
            breaklinks=true]{hyperref} % For hyperlinks in the PDF
\usepackage{graphicx, multicol} % Enhanced support for graphics
\usepackage{xcolor} % Driver-independent color extensions
\usepackage{framed}
\usepackage[normalem]{ulem} %underlining
\usepackage{amsfonts}
\usepackage{enumitem}
\usepackage{mathtools}
\usepackage{multicol}
\usepackage{color,soul}

\usepackage[toc, title]{appendix} %Appendix

\usepackage[labelfont=bf]{caption} %bold captions on figures and tables

%for tables
\usepackage{makecell,tabularx}
%\setlength{\extrarowheight}{12pt} %additive padding
\renewcommand{\arraystretch}{2} %multiplicative padding 
\renewcommand\theadfont{\small\bfseries}
\usepackage{rotating}
\usepackage{setspace}

%For pseudocode
\usepackage[boxed]{algorithm2e}
\DontPrintSemicolon

%A bunch of definitions that make my life easier
\newtheorem{theorem}{Theorem}[section]
\newtheorem{corollary}{Corollary}[section]
\theoremstyle{definition}
\newtheorem*{definition}{Definition}
\newtheorem{example}{Example}
\newtheorem*{note}{Note}
\newtheorem*{claim}{Claim}
\newcommand{\bproof}{\bigskip {\bf Proof. }}
\newcommand{\eproof}{\hfill\qedsymbol}
\newcommand{\Disp}{\displaystyle}
\newcommand{\qe}{\hfill\(\bigtriangledown\)}
\setlength{\columnseprule}{1 pt}

%for characters inside circles
%syntax is \circled{character}
\usepackage{tikz}
\newcommand*\circled[1]{\tikz[baseline=(char.base)]{
            \node[shape=circle,draw,inner sep=1pt] (char) {#1};}}


\usepackage{pgfplots}
\pgfplotsset{compat=1.8}


\usepackage[mode=buildnew]{standalone} %for loading precompiled tikz figures

%%%%%%%%%%%%%%%%%%%%%%%%
% Defines the `mycase` environment for cases in proofs
\newcounter{cases}
\newcounter{subcases}[cases]
\newenvironment{mycase}
{
    \setcounter{cases}{0}
    \setcounter{subcases}{0}
    \newcommand{\case}
    {
        \stepcounter{cases}\textbf{Case \thecases.}
    }
    \newcommand{\subcase}
    {
        \par\indent\stepcounter{subcases}\textit{Subcase (\thesubcases):}
    }
}
{
    \par
}
\renewcommand*\thecases{\arabic{cases}}
\renewcommand*\thesubcases{\roman{subcases}}

%For easily making figures
%syntax is:
%\myfig{scaling_factor}{name_of_file}{caption}{label}
\newcommand{\myfig}[4]{\begin{figure}[h] \begin{center} \includegraphics[width=#1\textwidth]{#2} \caption{#3} \label{#4} \end{center} \end{figure}}

% horizontal line across the page
\newcommand{\horz}{
\vspace{-.4in}
\begin{center}
\begin{tabular}{p{\textwidth}}\\
\hline
\end{tabular}
\end{center}
}

%Resize the summation symbol
%syntax is \sum[size] 
\newlength{\depthofsumsign}
\setlength{\depthofsumsign}{\depthof{$\sum$}}
\newlength{\totalheightofsumsign}
\newlength{\heightanddepthofargument}
\newcommand{\bigsum}[1][1.4]{% only for \displaystyle
    \mathop{%
        \raisebox
            {-#1\depthofsumsign+1\depthofsumsign}
            {\scalebox
                {#1}
                {$\displaystyle\sum$}%
            }
    }
}

%this file contains the bibliography
\addbibresource{refs.bib}

%%%%%%%%%%%%%%%%%%%
% Packages/Macros %
%%%%%%%%%%%%%%%%%%%
\usepackage{fullpage}
\setlength{\parskip}{1em}

\usepackage{setspace}
\onehalfspacing

\setcounter{tocdepth}{3}

\catcode`\@=11
\catcode`\@=12

%%%%%%%%%%%%
% Document %
%%%%%%%%%%%%
%\setcounter{chapter}{-1}

\title{Eco-evolutionary dynamics of finite populations from first principles}


\author{Ananda Shikhara Bhat}
\coordinator{Coordinator} \supervisor{Vishwesha Guttal}\cosupervisor{Rohini Balakrishnan} \sdesignation{}
\department{Centre for Ecological Sciences, Indian Institute of Science} \reader{Sutirth Dey}
%\reader{Reader 2}
\dedication{This thesis is dedicated to ?}
\graduationyear{2023}
\academicyear{2022-2023}
\graduationmonth{April}

\thesisabstract{
Some central questions of ecology and evolution ask how many distinct variants of an entity can emerge and coexist in sympatry. Examples include the study of standing genetic variation (alleles), polymorphisms (genotypes/phenotypes), and sympatric speciation (species). Historically, these questions have often been studied through phenomenological models formulated for infinitely large populations. Stochastic individual-based models (IbMs), where mechanistic rules are specified for each individual, are more realistic but are rarely studied analytically.\\
\\
In this thesis, I show how ideas from statistical physics can be used to analytically formulate and study IbMs from biological first principles. Starting from a stochastic ‘birth-death process’, I show how density dependence of the population dynamics can be used to obtain a continuous approximation of the system as a stochastic differential equation (SDE). I show that well-known ‘fundamental theorems’ of evolution, such as the Price equation, are recovered in the infinite population limit. My stochastic formulation reveals a new directional evolutionary force in finite populations that lends itself to a simple biological interpretation and has no analogue in infinitely large populations. As an application, I show how this formalism can be used to study cooperation in finite populations. Finally, I extend this theory to study quantitative traits through a stochastic field equation, which can be thought of as an infinite-dimensional SDE. In the infinite population limit, I recover well-known frameworks such as Kimura's continuum-of-alleles and adaptive dynamics. My work thus highlights the connections between various equations in ecology and evolution by revealing their common stochastic underpinnings.
}

\acknowledgments{Not more than 250 words}
\allowdisplaybreaks %allow align envts to go across pages   

\begin{document}
\thesisfront
\listoftables
\listoffigures	
\chapter{Introduction}
\epigraph{\justifying The theory of evolution by natural selection is an ecological theory—founded on ecological observation by perhaps the greatest of all ecologists. It has been adopted by and brought up by the science of genetics, and ecologists, being modest people, are apt to forget their distinguished parenthood}{John Harper~\citep{harper_darwinian_1967}}

%More than 150 years have passed since Charles Darwin first published \textit{Origin} (in 1859) and Ernst Haeckel first coined the term ‘ecology’ (in 1866). Today, both ecology and evolution are incredibly interdisciplinary fields, borrowing techniques and ideas from diverse areas such as computer science, statistics, economics, dynamical systems, physics, and information theory. With this development has come a cornucopia of models that try to understand biological phenomena in the language of these borrowed tools and techniques. Though many of these ideas are used to accurately describe and analyze specific systems, there is also value to formulating general models in abstract terms that only incorporate a small number of `fundamental' processes and try to capture the `essence' of a biological pattern~\citep{frank_natural_2012, vellend_theory_2016,luque_mirror_2021}. Such general organizing models are vital to theory-building~\citep{luque_mirror_2021} because they help clarify conceptual similarities and unifying factors between apparently disparate modelling frameworks, helping us seamlessly translate essential ideas from one theoretical language to another.
%
%\section{Idealization and generality}\label{idealization}
Idealization and generalization are part and parcel of science, and this is clear if one looks at the actual practice, be it theorists making unrealistic assumptions on paper to model specific phenomena or experimentalists creating artificially controlled conditions in the laboratory to test specific hypotheses~\citep{zuk_models_2018}. Indeed, some philosophers of science argued that ``the epistemic goal of science is not truth, but understanding''~\citep{potochnik_idealization_2018}, an idea generally echoed by practicing scientists~\citep{levins_strategy_1966,servedio_not_2014,zuk_models_2018,grainger_empiricists_2022}. In other words, since the world is complicated and humans are limited, general understanding inevitably comes at the cost of other desirable qualities such as the ability to make precise quantitative predictions. This is especially true for complex phenomena such as those that are the domain of ecology and evolution, where we are often not even aware of all the factors that are at play or how they interact. We thus benefit from formulating simple `general' models that provide simple qualitative predictions and help us think about the phenomena we study in a cohesive, unified framework that we can understand well~\citep{potochnik_idealization_2018,luque_mirror_2021}.   
 
A general approach need not (and often will not) be perfect or all-encompassing. As Robert MacArthur once remarked, ``general events are only seen by ecologists with rather blurred vision. The very sharp-sighted always find discrepancies and are able to see that there is no generality, only a spectrum of special cases”~\citep{kingsland_modeling_1985}. MacArthur was speaking primarily about biological generalities and special cases, but in a related vein, if our language of choice for expressing our general events is mathematics, making any non-trivial observations in complex fields such as ecology and evolution often requires \emph{mathematical} approximations and idealizations that may exclude or underplay several `low-level' model-specific details in favor of a more general description at a `higher' level achieved in some limit that only contains a small number of `model-independent' quantities, often derived from first principles. Formulating and studying such general frameworks can be greatly beneficial as an aid to thinking, sometimes precisely \emph{because} `blurry eyed' thinking that begins from a small set of fundamental first principles and only looks for general broad-brush regularities can be much more insightful than accounting for every little detail or special case. The success of such an approach is perhaps best illustrated by the success of statistical mechanics in physics - Statistical mechanics was essentially born from the idea that various useful statements about systems with many moving parts can be made without the need for knowing the excruciating details of every single moving part, and indeed, starting from first principles, this sort of explicitly `blurry-eyed' thinking that only looked at approximate properties was shown to be able to recover the phenomenological laws of thermodynamics as \emph{statistical} laws.

%In population ecology, Vellend has argued that conceptual synthesis requires
%``shifting the emphasis away from an organizational structure based on the useful lines of
%inquiry carved out by researchers, to one based on the fundamental processes that underlie community dynamics and patterns''~\citep{vellend_theory_2016}. Vellend's assertion is based on the
%fact that population genetics has managed to come up  with reasonably comprehensive theory due to its focus on the abstract `high-level’ processes of selection, mutation, drift, and
%gene flow (see section \ref{sec_history} below) instead of the myriad `low-level' processes that may be responsible for generating them. In contrast, he believes that practitioners of community ecology often focus on specific `low-level' processes such as predation rate, limiting resources (R\textsuperscript{*}), storage effects,
%priority effects, senescence, and niche partitioning, leading to a plethora of models (see Table
%5.1 in~\cite{vellend_theory_2016} for an in-exhaustive list of 24 such models) and the conclusion that
%community ecology `is a mess'. Vellend instead proposes speaking about ecological theory in terms of the `high-level’ processes of selection, ecological drift (demographic stochasticity), speciation,
%and dispersal, in direct analogy with the `high-level' processes of selection, genetic drift, mutation, and gene flow respectively in population genetics. Without dwelling on the practicality of such an organization, which has been written about in great detail in~\cite{vellend_theory_2016}, I will simply note that such an organization and broad analogy at the very least seems somewhat plausible from a theoretical viewpoint, and is desirable for providing a cohesive theoretical underpinning for speaking about various hypotheses and ideas in community ecology and how they relate to each other. This is especially relevant given the increasing recognition that population ecology and evolutionary dynamics (Here I really mean the theoretical frameworks of population genetics + quantitative genetics\footnote{Though one could perhaps argue that all evolutionary phenomena should be captured in this sort of framework for some sufficiently general definition of the words `population', `quantitative', and `genetics'. In Michael Lynch's words, ``Nothing in evolution makes sense except in the light of population genetics''~\citep{lynch_origins_2007}. I don't know if this is an extreme view, though, which is why these sentences are in a footnote :)}) are perhaps not as neatly separated as many early architects of the fields had thought~\citep{coulson_putting_2006,metcalf_why_2007,schoener_newest_2011,kokko_can_2017,lion_theoretical_2018,govaert_eco-evolutionary_2019,svensson_eco-evolutionary_2019,hendry_critique_2019}.

\section{A (very, very) brief summary of high-level modelling frameworks in population biology}\label{sec_history}

In biology, arguably the greatest such general organizing framework was the idea of evolution by natural selection as synthesized from myriad detailed observations by Charles Darwin. Theoretical population genetics has also had a long-standing tradition in building general organizing frameworks that `abstract away' some biological specificities in favor of a small number of `fundamental' notions like selection and mutation which act on a small number of `fundamental' quantities like fitness. The description of evolution in these general terms was first laid out in formal mathematical terms during the Modern Synthesis by authors such as Wright, Fisher, and Haldane, an extremely successful venture that unified two major schools of thought --- Mendelian genetics and Darwinian evolution --- that were, at the time, considered to be incompatible~\citep{provine_origins_2001}. It is currently thought that this unification would have been unlikely or would have taken much longer if the architects of the Modern Synthesis had stuck to verbal arguments instead of working with formal models in explicitly mathematical terms~\citep{walsh_darwins_2014}. 

The most general mathematical framework we have for evolutionary population biology is the Price equation. Indeed, much like statistical mechanics in physics, the Price equation is derived from a small number of very general first principles and is able to recover several standard equations as special cases. The Price equation partitions changes in population composition into multiple terms, each of which lends itself to a straightforward interpretation in terms of `high-level' evolutionary forces such as selection and mutation, thus providing a useful conceptual framework for thinking about how populations change over time~\citep{frank_natural_2012}. However, the greater complexity of ecology and evolution relative to physics has meant that the generality of the Price equation comes with a much bigger cost in predictive power - the Price equation in its most general formulation is dynamically insufficient\footnote{Since this is not very standard nomenclature outside theoretical biology and related fields: `Dynamically insufficient' means that the equation cannot be iterated, or presented in the form of a difference equation $x_{t+1} = F(x_t)$ or a differential equation $\dot{x} = F(x)$. Thus, we cannot predict a dynamic `trajectory' given an initial condition $x_0$. The term `insufficient' is to indicate that the equation requires `complete' information to be true and only relates two quantities `in retrospect'. It is thus `insufficient' for prediction. In our case, in its most general setting, the Price equation is usually formulated in a way that partitions a \emph{given} amount of phenotypic change between two populations (usually but not necessarily the same population at two different times) into change due to selection, transmission bias, etc., rather than \emph{predicting} a trajectory for how much phenotypic change will occur at various future times based on the \emph{current} phenotypic distribution.}~\citep{van_veelen_use_2005,frank_natural_2012,luque_one_2017}. However, this need not be the case, and indeed, several authors have put forth more predictive versions of the Price equation by moving to a continuous time differential equation framework in which the Price equation is dynamically sufficient but manifests in a slightly less general form~\citep{page_unifying_2002,lion_theoretical_2018,day_price_2020}.

These general formulations are still often very difficult to coax concrete quantitative predictions from, but they do lend themselves to simple biological interpretation, and in the dynamically sufficient formulations, often provide \emph{qualitative} predictions. These qualitative predictions, as well as the decomposition of terms in the original Price equation, are useful primarily for their generality --- The Price equation gives us a clear idea of which evolutionary forces operate in which systems and when in an almost entirely `model-independent' language~\citep{okasha_evolution_2006,frank_natural_2012,queller_fundamental_2017,luque_one_2017}. It also leads to a small number of simple yet insightful `fundamental theorems' of population biology~\citep{queller_fundamental_2017, lion_theoretical_2018, lehtonen_price_2018} that serve a similar function, and unifies several various seemingly disjoint formal structures of evolution under a single theoretical banner~\citep{ lehtonen_price_2020, luque_mirror_2021}.

The success of the Modern Synthesis illustrates the value of formulating abstract mathematical models that only provide a `high-level' description of the fundamental processes required to capture the essence of biological evolution~\citep{provine_origins_2001,walsh_darwins_2014}. However, the evolutionary play that architects of the Modern Synthesis studied famously unfolds in the ecological theatre~\citep{hutchinson_ecological_1965}. Thus, quantities like fitness are not truly fundamental but instead emerge as the net result of various ecological interactions, tradeoffs, and constraints~\citep{metz_how_1992}, a fact that can have important consequences for evolution~\citep{coulson_putting_2006,schoener_newest_2011, kokko_can_2017}. Trying to understand such `eco-evolutionary feedbacks' or `eco-evolutionary dynamics' has sprouted a rich body of literature under the broad heading of `evolutionary ecology' that has greatly enriched our understanding of biological populations~\citep{coulson_putting_2006,metcalf_why_2007,schoener_newest_2011,brown_why_2016,kokko_can_2017,lion_theoretical_2018,govaert_eco-evolutionary_2019,svensson_eco-evolutionary_2019,hendry_critique_2019}. Several major theoretical frameworks in the slightly more general setting of eco-evolutionary dynamics --- such as evolutionary game theory and adaptive dynamics --- as well as the standard equations of population genetics and quantitative genetics, can still be recovered (in a very general sense) as special cases of a slightly reformulated version of the Price equation~\citep{page_unifying_2002,lion_theoretical_2018}.

One of the general guiding principles of much of this mathematization has been the assumption that incorporating the reality of finite population sizes into models leads to no major qualitative differences in behavior, only `adding noise' or `blurring out' the predictions of simpler infinite population models~\citep{page_unifying_2002}. Consequently, several major theoretical frameworks in the field, such as adaptive dynamics, are explicitly formulated in deterministic terms at the infinite population size limit. However, this assumption is largely unjustified, and since populations in the real world are finite and stochastic, checking whether stochastic models differ from their deterministic analogs is vital to furthering our understanding of the fundamentals of population biology~\citep{hastings_transients_2004, coulson_skeletons_2004, shoemaker_integrating_2020}. Today, we increasingly recognize that incorporating the finite and stochastic nature of the real world routinely has much stronger consequences than simply `adding noise' to deterministic expectations~\citep{boettiger_noise_2018}, with important consequences for both ecological~\citep{schreiber_does_2022} and evolutionary~\citep{delong_stochasticity_2023} theory. In ecology and evolution, stochastic models need not exhibit phenomena predicted by their deterministic analogues~\citep{proulx_what_2005, johansson_will_2006, claessen_delayed_2007,  wakano_evolutionary_2013, debarre_evolutionary_2016, johnson_two-dimensional_2021}. In addition, they exhibit novel phenomena not predicted by the deterministic approximations~\citep{rogers_demographic_2012, rogers_spontaneous_2012, rogers_modes_2015, veller_drift-induced_2017, delong_stochasticity_2023}, for example, even completely `reversing' the predictions of deterministic models~\citep{houchmandzadeh_selection_2012,houchmandzadeh_fluctuation_2015,constable_demographic_2016,mcleod_social_2019}.

Studies of neutral or near-neutral dynamics in population and quantitative genetics usually do take stochasticity seriously, explicitly modeling finite populations that follow stochastic dynamics. Unfortunately, the classic or standard stochastic models in both population genetics~\citep{fisher_genetical_1930,wright_evolution_1931, moran_random_1958, kimura_diffusion_1964} and quantitative genetics~\citep{crow_introduction_1970, lande_natural_1976} typically assume a fixed total population size, thus restricting their validity in a world where population sizes routinely fluctuate. Those models which study evolution in populations with non-constant size usually impose deterministic and typically phenomenological rules for how the total population size must vary~\citep{kimura_probability_1974,ewens_probability_1967,otto_probability_1997, engen_fixation_2009, waxman_unified_2011}. These rules further usually do not depend on population composition. Such models are thus somewhat artificial since demography and population size are forced to be independent quantities even though this is obviously not the case in natural populations, where population size is a `bulk' property whose value emerges from an intricate interplay of the individual-level demographic processes of birth and death~\citep{metcalf_why_2007,doebeli_towards_2017}. Notably, the most general framework we have, the Price equation, is typically formulated in a deterministic setting ~\citep{page_unifying_2002,frank_natural_2012,queller_fundamental_2017,lion_theoretical_2018,day_price_2020} that ignores stochasticity (but see~\cite{rice_stochastic_2008,rice_universal_2020} for a discrete time formulation that is stochastic but, as far as I can tell, is dynamically insufficient, just like the original formulation of the Price equation). Since real-life populations are stochastic, finite, and of non-constant population size, this is somewhat of a problem, since we know that such deterministic approximations that may not capture important dynamics of the real systems of interest.

Incorporating stochasticity into deterministic systems is a tricky business, and, if done in a phenomenological manner by adding noise to a `deterministic skeleton'~\citep{coulson_skeletons_2004} in an ad-hoc fashion, can lead to models that are not self-consistent~\citep{strang_how_2019}. Further, this procedure of adding noise in an ad-hoc manner provides no insight into the mechanistic factors actually responsible for the stochasticity in the first place. Stochastic individual-based models, in which (probabilistic) rules are specified at the level of the individual and population level dynamics are systematically derived from first principles, are self-consistent, much more natural, and can fundamentally differ from the predictions made by simply adding noise terms to a deterministic model~\citep{black_stochastic_2012,strang_how_2019}. Formulating the fundamental formal structures of evolutionary biology in terms of the mechanistic demographic processes of birth and death at the individual level is also greatly desirable for biological reasons~\citep{metcalf_why_2007,geritz_mathematical_2012} because `all paths to fitness lead through demography'~\citep{metcalf_all_2007}. In other words, since demographic processes such as birth and death rates explicitly account for the ecology of the system, they can more accurately reflect the complex interplay between ecological and evolutionary processes and provide a more fundamental mechanistic description of the relevant evolutionary forces and population dynamics~\citep{doebeli_towards_2017}. In this thesis, I present a formulation of population dynamics constructed from mechanistic first principles grounded in individual-level birth and death. The mathematical formalism itself is very general and applies equally to the `high level' forces of population genetics and the `high level' forces of community ecology as postulated by~\cite{vellend_theory_2016}, though I will mostly stick to the population genetics interpretation in my discussions.

\section{A very brief outline of the rest of this thesis}
This section provides a bullet-point chapter-wise outline for convenience. A more detailed outline of the thesis, with explanations of the technical content covered, originality, etc., is provided in the next section.

This thesis develops a mathematical formalism for describing finite fluctuating populations from first principles and is structured as follows. The rest of this thesis is divided into two parts. Part \ref{part_theory} provides the complete formalism in all its gory mathematical detail, but in a (hopefully) accessible pedagogical style. Part \ref{part_summary} then (re)presents some major results from part \ref{part_theory} and discusses their implications and connections with previous studies. Appendix \ref{App_examples} presents some concrete examples of models for clarity regarding the major ideas.

\begin{itemize}
		\item Chapter \ref{chap_math_background} provides the necessary mathematical background and provides a toy example studying the size of a population of identical individuals that illustrates the major ideas used. If I am required to shoehorn this thesis into a `intro-methods-results-discussion' format, then chapter \ref{chap_math_background} can be thought of as providing a mathematical `introduction' and illustration of the `methods' that will be used (in chapter \ref{chap_BD}) and generalized (in chapter \ref{chap_infD_processes}) in a biological context to get `results'.
		\item Chapter \ref{chap_BD} develops a formalism for describing the evolution of finite fluctuating populations of individuals that come in arbitrarily many `types' which vary in a \emph{discrete} character. This yields equations that generalize the classic Price equation and replicator-mutator equation to finite fluctuating populations.
		\item Chapter \ref{chap_infD_processes} extends the ideas developed in chapter \ref{chap_BD} to populations that vary in a single one-dimensional \emph{quantitative} character and derives a so-called `stochastic field theory' that describes evolution in such populations. This also results in some mathematical equations that may be of independent interest to physicists and applied mathematicians.
		\item Chapter \ref{chap_unification} provides a technical summary of the major results by deriving three important stochastic differential equations. These are equations for type frequencies, the population mean value of an arbitrary type-level quantity, and the population variance of an arbitrary type-level quantity, and respectively generalize the replicator-mutator equation, the Price equation, and \cite{lion_theoretical_2018}'s variance equation to finite, stochastically fluctuating populations. These equations predict a directional evolutionary force called `noise-induced selection' that is only seen in finite, fluctuating populations. Some implications for social evolution and community ecology are discussed. I also briefly discuss the field equation formalism I develop in chapter \ref{chap_infD_processes}.
		\item Chapter \ref{sec_disc} provides a quick summary of the major results and discusses biological implications, connections with previous studies, and opportunities for future work. This chapter has no equations (!). Thus, readers who do not want any mathematical equations can skip all other chapters and directly read chapter \ref{sec_disc} if they are interested only in the final results and takeaways. 
\end{itemize}
\section{A more expanded outline of the rest of this thesis}

Chapter \ref{chap_math_background} provides the basic mathematical background required and illustrates the major ideas that we will use. To facilitate readership by a broad audience, I only assume passing familiarity with calculus (derivatives, integrals, Taylor expansions) and probability. Familiarity with stochastic calculus is helpful for some sections but is not required, and I present a brief introduction to the relevant notions from both Markov theory and stochastic calculus in section \ref{sec_math_background}. In section \ref{sec_1D_processes}, I present a toy example of tracking population size of a population of identical individuals in section. I introduce a description of the system via a `master equation', and then conduct a `system-size expansion' to obtain a Fokker-Planck equation for the system, thus illustrating all the major tools required. For completeness, I also conduct a weak noise approximation to arrive at a so-called `linear' Fokker-Planck equation that can be solved exactly to arrive at a closed-form solution.

Chapter \ref{chap_BD} deals with the evolution of discrete traits. In this case, the system is finite-dimensional, since we can completely specify the state of the system by simply listing out the number of individuals of each type in a vector. I introduce a general multivariate process to describe the evolution of discretely varying traits, and use the system size expansion to arrive at a continuous description of change in trait frequencies as an SDE under mild assumptions on the functional forms of the birth and death rates. Unlike many classic stochastic formulations in evolutionary theory~\citep{fisher_genetical_1930,wright_evolution_1931,moran_random_1958,crow_introduction_1970, lande_natural_1976,kimura_probability_1974}, I do not assume a fixed (effective) population size and instead allow the total population size to be a natural emergent property from the demographic processes of birth and death. I show that the deterministic limit of this process is the well-known replicator-mutator equation (or equivalently, the dynamic version of the Price equation), thus establishing the microscopic basis of well-known equations from stochastic first principles. I also illustrate some general predictions that can be made using the weak noise approximation for the sake of completeness.

While the mathematics of chapter \ref{chap_BD} is standard and well-understood, it has, to the best of my knowledge, not been applied before in the generality and context we use here. Several specific models of specific systems do use these mathematical techniques, but these papers are often written assuming familiarity with notions in physics and/or mathematics and thus may not be very accessible to theoretical ecologists who do not have formal training in these subjects (but see~\cite{czuppon_understanding_2021} for a recent pedagogical review on the general approach applied to Wright-Fisher and Moran processes, where total population size is constant). As such, chapters \ref{chap_math_background} and \ref{chap_BD} together also serve as a tutorial and technical introduction to some theoretical ideas: For ecologists, the chapter introduces `system size expansions' and illustrates their use in a general setting, and can be seen as a tutorial on modelling finite populations analytically with minimal assumptions; For population geneticists, the chapter illustrates how system-size approximations (`diffusion approximations' in the population genetics literature) can be carried out without assuming a constant (effective) population size and how this generalization has important consequences for the evolutionary forces at play; For physicists and applied mathematicians, the chapter presents a study of the consequences of applying the system-size expansion to the kind of density-dependent birth-death processes that are widely applicable in ecology and evolution - Unlike many physical systems, though demographic processes like birth and death over ecological timescales are usually formulated in terms of population numbers or densities, predictions in evolution are typically in terms of frequencies of types, and this fact has subtle consequences that can be overlooked if one only works with densities, as is often done in physics studies of eco-evolutionary systems.

Chapter \ref{chap_infD_processes} introduces a function-valued process to model the evolution of quantitative traits such as body size, which can take on uncountably many values. This function-valued process can then also be analyzed via an analog of the system-size approximation to arrive at a `functional' Fokker-Planck equation in which derivatives are replaced by functional derivatives. I show that classic equations from quantitative genetics such as Kimura's cotinuum-of-alleles model~\citep{kimura_stochastic_1965} and Lande's gradient dynamics~\citep{lande_quantitative_1982} can be derived as the infinite population limit of this stochastic process. I also conduct a weak noise approximation to arrive at a linear functional Fokker-Planck equation that can be analyzed for specific systems as required. Unlike the systems studied in Chapter \ref{chap_BD}, formalizing the study of the kind of processes we study in Chapter \ref{chap_infD_processes} is an active area of mathematical research~\citep{carmona_stochastic_1999,da_prato_stochastic_2014,prevot_concise_2007,liu_stochastic_2015,bogachev_fokker-planck-kolmogorov_2015,balan_gentle_2018} and the mathematics itself is far from settled.

Chapter \ref{chap_infD_processes} generalizes the work of Tim Rogers and colleagues~\citep{rogers_demographic_2012,rogers_spontaneous_2012,rogers_modes_2015} to a wide class of eco-evolutionary systems, and to the best of my knowledge, has never been presented in full generality before. Mathematically, chapter \ref{chap_infD_processes} presents heuristic, accessible alternatives to the rigorous tools of martingale theory and measure-valued branching processes that are usually employed to describe the evolution of quantitative traits~\citep{champagnat_unifying_2006,etheridge_mathematical_2011, week_white_2021} by generalizing the idea of a system size expansion of density-dependent (finite-dimensional) birth-death processes to the infinite-dimensional case using the notion of functional differentiation. Biologically, Chapter \ref{chap_infD_processes} provides `stochastic field equations' that describe the dynamics of one-dimensional quantitative traits in finite populations and illustrates that these equations are consistent with well-known formalisms in quantitative genetics at the infinite population limit.

Part \ref{part_summary} summarizes the major results of the formalism developed in Part \ref{part_theory} and presents some simple equations that can be argued to be `fundamental equations' of population biology in the sense of~\cite{queller_fundamental_2017}, and together form a `unifying perspective' in the sense of~\cite{lion_theoretical_2018}. These equations reduce to well-known results such as the Price equation, the replicator-mutator equation from evolutionary game theory, and Fisher's fundamental theorem from population genetics in the infinite population limit. For finite populations, these same equations predict a new evolutionary force, `noise-induced selection', that has still not found its way into the standard formal canon of evolutionary biology and whose significance is only recently being recognized~\citep{constable_demographic_2016,mcleod_social_2019,mazzolini_universality_2022, kuosmanen_turnover_2022}. Implications of noise-induced selection are also discussed in part \ref{part_summary}. Readers who are okay with mathematical equations but do not want any intermediate derivations (or just trust my math) can skip to Chapter \ref{chap_unification} for the major equations that emerge as being important, though I strongly encourage working through the entire formalism properly if possible. Readers who are averse to or do not care for equations can safely skip to Chapter \ref{sec_disc} directly for the major takeaways of this thesis.


\chapter{Adaptive diversification: a crash course}\label{chap_AD}
\epigraph{\justifying From so simple a beginning, endless forms most beautiful and wonderful have been, and are being, evolved}{Charles Darwin}

\section{Adaptive diversification}\label{AD}
The questions proposed at the end of the previous chapter are best addressed by a modeling framework called `adaptive diversification', where an initially monomorphic population evolves to become polymorphic due to (disruptive) frequency-dependent selection. The framework has been widely used in evolutionary ecology and evolutionary game theory, especially in the context of sympatric speciation \citep{dieckmann_origin_1999}. Models of adaptive diversification can be broadly classified into three classes of models which differ starkly in their approach and assumptions \citep{doebeli_adaptive_2011}. I introduce all three approaches below.

\subsection{The `classic' adaptive dynamics approach}
The classic approach to modelling adaptive diversification was first articulated in the late 90s by J.J. Metz's group \citep{geritz_evolutionarily_1998}. The crux of the approach relies on the observation that evolutionary stability (in the ESS sense of evolutionary game theory) and asymptotic convergence stability (in the sense of dynamical systems) need not always coincide.\\
More concretely, consider an infinitely large asexual population that is monomorphic for some quantitative trait such that every individual of the population has the trait value $x \in \mathcal{T} \subseteq \mathbb{R}^n$. We are interested in the dynamics of $x$ over evolutionary time. There is assumed to be a separation of timescales between ecology and evolution such that whenever we observe the population, it is at ecological equilibrium (This is equivalent to a strong-selection + weak-mutation setting). The evolutionary dynamics of the trait value in the population are modelled as following the equation:
\begin{equation}
\label{AD_canonical_eqn}
\frac{dx}{dt} = g(x) = B(x)\left(\nabla_y f(y;x) \bigg{|}_{y=x}\right)
\end{equation}
Here, $B(x)$ describes the mutational process of the trait and is intended to model mutational biases. For the sake of simplicity, I will assume that $B(x) = 1$ below, but the essential results are not greatly changed by more complicated forms. $f(y;x)$ is the \emph{invasion fitness} function, and describes the expected fitness of a mutant type $y$ appearing in a population that is monomorphic $x$-valued.%\footnote{In evolutionary game theory notation, this is exactly the incentive function $h_{x,y,0}$ that characterizes whether a strategy is an ESS. Recall that $h_{p,q,\epsilon} = \mathbb{E}[p;(1-\epsilon)\delta_p+\epsilon\delta_q] - \mathbb{E}[q;(1-\epsilon)\delta_p+\epsilon\delta_q]$ is the `incentive' of switching from strategy $q$ to strategy $p$ in a population where a fraction $\epsilon$ of the individuals are $q$-players and the rest are $p$-players. Here, the strategies are continuous and the incentive is in terms of fitness.}
$x$ is sometimes called the `resident' trait value. Thus, evolution is modelled as following the gradient $\nabla_y f(y;x)$ of the invasion fitness, a quantity sometimes called the selection gradient. \eqref{AD_canonical_eqn} can be interpreted to mean that at each time step, the population `samples' all local mutations, and if a mutant can invade this resident population, then the trait rapidly spreads in the population, and by the time we next `observe' it, the entire population has adopted this mutant trait. Equation \eqref{AD_canonical_eqn} is called the \emph{canonical equation of adaptive dynamics}. Fixed points for this equation, \textit{i.e} points $x^*$ for which $g(x^*)=0$, are called \emph{evolutionary singularities}. These singularities can be characterized by two different stability notions, as we will see below, and the difference between the two notions captures the important phenomenon of \textit{evolutionary branching}, which we will encounter shortly.

\definition{\emph{(Convergence stability)}}{ A singularity $x^*$ is said to be convergent stable (CS) if it is an asymptotically stable fixed point for equation \eqref{AD_canonical_eqn}.}

Thus, singularities which are convergence stable are local attractors, in the sense that nearby populations will evolve towards this state.

\definition{\emph{(Evolutionary stability)}}{ A singularity $x^*$ is said to be evolutionarily stable (ES) if it cannot be invaded by any nearby mutants.}
\\
\\
In the one dimensional case, evolutionary stability requires that the invasion fitness be such that no mutant can invade the population. Since $g(x^*) = \frac{\partial}{\partial y}f(y;x^*) = 0$ by definition at a singularity, the condition for evolutionary stability is controlled by the second derivative, and is given by:
\begin{equation}
\label{ESS_condition}
\frac{\partial^2}{\partial y^2} f(y;x) \bigg{|}_{y=x=x^*} < 0
\end{equation}
On the other hand, from elementary non-linear dynamics, we know that convergence stability requires $\frac{dg}{dx} < 0$, \textit{i.e}:
\begin{equation}
\label{CS_condition}
\frac{\partial^2}{\partial x \partial y} f(y;x) \bigg{|}_{y=x=x^*} + \frac{\partial^2}{\partial y^2} f(y;x) \bigg{|}_{y=x=x^*} < 0
\end{equation}
Note that neither \eqref{ESS_condition} nor \eqref{CS_condition} imply the other, and thus, we arrive at the following classification of evolutionary singularities:
\begin{center}
    \begin{tabularx}{0.4\textwidth}{ 
  | >{\centering\arraybackslash}X 
  | >{\centering\arraybackslash}X 
  | >{\centering\arraybackslash}X | }
        \hline
           & \textbf{ES} & \textbf{not ES} \\
        \hline
        \textbf{CS} &  \circled{A}  &  \circled{B} \\ 
        \hline
        \textbf{not CS} & \circled{C}  & \circled{D} \\
        \hline
    \end{tabularx}
\end{center}

Points which are not convergent stable are not of interest to us because populations can only attain such a state if they begin there. Thus, \circled{C} and \circled{D} can be ignored for our purposes.\footnote{Singularities of type \circled{C} are sometimes called `garden of Eden' points, since a population that begins at such a point will remain there and cannot be invaded by nearby mutants, but if a population does not begin there (or is somehow driven out by external forces), it can never find its way back to the singularity.}\\
Points of type \circled{A} are both evolutionarily stable and convergent stable. Such points represent `endpoints' for evolution, since populations are attracted to such points and also cannot evolve away from them since they cannot be invaded by any nearby mutants. Points of type \circled{B} are CS but not ES. Populations are attracted to such points, but once they have arrived, they are susceptible to invasion by nearby mutants. Let $x^*$ be such a point, and let $x_1 < x^* < x_2$ be two points in the immediate vicinity of $x^*$.
\begin{claim}{(Protected Polymorphism)}
Each of $x_1$ and $x_2$ can invade the other
\end{claim}
\begin{proof}
We can Taylor expand the invasion fitness function as
\begin{align*}
    f(x_1 ; x_2) &= \underbrace{f(x_2 ; x_2)}_{=0} + \underbrace{(x_1 - x_2)}_{< 0 }\underbrace{\frac{\partial f}{\partial x_1}(x_1 ; x_2)\bigg{|}_{x_1 = x_2}}_{=g(x_2)} + \frac{(x_1-x_2)^2}{2} \frac{\partial^2 f}{\partial x_1^2}(x_1 ; x_2)\bigg{|}_{x_1 = x_2}
\end{align*}
where we have neglected terms that are higher than second order. Since $x^*$ is convergent stable, $\frac{d g}{dx}\bigg{|}_{x=x^*} < 0$ and we therefore see that $g$ is a decreasing function of $x$ in the immediate vicinity of $x^*$. Thus, since $x_2 > x^*$, we must have $g(x_2) < g(x^*) = 0$, and we can conclude that the second term in the RHS must be positive. Lastly, since $x^*$ is evolutionarily unstable, we must have $\frac{\partial^2 f}{\partial x_1^2}(x_1 ; x^*)\bigg{|}_{x_1 = x^*} > 0$ by the stability criterion. Thus, assuming $f$ is smooth, for $x_2$ sufficiently close to $x^*$, it must be the case that $\frac{\partial^2 f}{\partial x_1^2}(x_1 ; x_2)\bigg{|}_{x_1 = x_2} > 0$. Thus, the third term of the RHS is also positive, meaning that $f(x_1 ; x_2) > 0$, and that $x_1$ can thus invade $x_2$. An exactly analogous argument shows that $f(x_2 ; x_1) > 0$, thus completing the proof.
\end{proof}
Mutual invasibility of $x_1$ and $x_2$ results in a so-called `protected polymorphism' in which the population harbours some members with trait value $x_1$ and some members with trait value $x_2$. This can be shown in some cases to lead to further divergence where the polymorphisms grow further apart in trait space while both being maintained in the population. Thus, the population appears to have `branched' from a monomorphic state to a dimorphic state in trait space. Due to this reason, points of type \circled{B} are called \emph{branching points}, and the population is said to have undergone \emph{evolutionary branching} once it has gone from a monomorphism to a dimorphism. Note that once a population has branched, equation \eqref{AD_canonical_eqn} no longer describes the population, since it is no longer monomorphic - We instead need a system of \emph{two} equations, one for each morph.\\
In adaptive dynamics, the name of the game is thus formulating reasonable guesses for $B(x)$ and $f(y;x)$. For example, one common choice for modelling asexual resource competition is $B(x) \equiv 1$ and the invasion fitness function:
\begin{equation}
\label{AD_cts_logistic_invasion_fitness}
f(y:x) = 1 - \frac{\alpha(x,y)K(x)}{K(y)}
\end{equation}
where $\alpha(x,y)$ is called the \emph{competition kernel} and $K(x)$ is called the \emph{carrying capacity function} (formulated in analogy with Lotka-Volterra competition or the logistic equation). Invasion fitnesses are sometimes derived from more mechanistic processes (such as explicitly formulating birth/death functions), but the conceptual idea is the same regardless of how complicated these functions may be, and evolutionary branching easily arises in a very wide range of ecological scenarios as long as the frequency-dependence of the selective force is strong enough \citep{doebeli_evolutionary_2000,doebeli_adaptive_2011}.

\subsection{The PDE approach}

A slightly more general approach to adaptive diversification is to relax the assumption of separation of timescales while keeping the assumption of infinite population size. In this case, we instead wish to formulate an equation for the distribution $\phi(u)$ of trait values in $\mathcal{T}$. We thus seek PDEs of the form:
\begin{equation*}
    \frac{\partial \phi(u)}{\partial t} = F(u,\phi(u),t)
\end{equation*}
In PDE models, diversification shows up as patterns (in the Turing sense), and the existence of a polymorphism corresponds to a multimodal solution for $\phi(x)$. The functional form of $F$ is often formulated through biological principles. For example, in analogy to the logistic equation, one could postulate the continuous version:
\begin{equation}
\label{cts_logistic}
\frac{\partial \phi(u)}{\partial t} = r\phi(u)\left(1 - \frac{\phi(u)}{K(u)}\int\limits_{\mathcal{T}}\alpha(u,v)\phi(v)dv\right)
\end{equation}
where $r$ is a growth rate, $K(u)$ gives the carrying capacity of the environment for individuals with phenotype $u$, and $\alpha(u,v)$ is a competition kernel which determines the effect of an individual with phenotype $v$ on the growth of an individual with phenotype $u$, and the integral thus gives a measure of the effective density experienced by individuals with phenotype $u$. $\alpha(u,v)$ is generally chosen such that strength of competition decreases with phenotypic distance (for example, $\alpha(u,v) = \exp(|u-v|)$), in analogy with niche partitioning. Note that this is an integrodifferential equation, and as such, is usually not easy to solve for most functional forms of $\alpha(u,v)$. We thus need to resort to numerical methods to solve such equations. \\
PDE models can also incorporate space more easily than the classic adaptive dynamics approach. For example, Doebeli \textit{et al.} propose a spatial model of resource competition in which we track the density $\phi(x,u)$ of $u$-phenotype individuals at the location $x$ given by the PDE:
\begin{equation}
\begin{split}
\label{spatial_PDE}
\frac{\partial \phi(x,u)}{\partial t} = r\left(\int\limits_{\mathcal{T}}B(v)\phi(x,v)dv - \frac{\phi(x,u)}{K(x,u)}\int\limits_{\mathcal{S}}\int\limits_{\mathcal{T}}\alpha_p(u,v)\alpha_s(x,y)\phi(y,v)dvdy\right) \\ +m\left(\int\limits_{\mathcal{S}}D(x,y)\phi(y,u)dy - \phi(x,u)\right)
\end{split}
\end{equation}
where $B(v)$ is a birth kernel that specifies births and mutations, $K(x,u)$ is a carrying capacity function that varies with both phenotype and geographic location, $\alpha_p(u,v)$ describes how strength of competition varies with phenotypic distance, $\alpha_s(x,y)$ describes how strength of competition varies with spatial (geographic) distance, $m$ is a migration rate, and $D(x,y)$ is a dispersion kernel that describes the probability that an individual at location $y$ will migrate into location $x$.\\
PDE models sacrifice interpretability for increased generality. While these models can include space as well as incorporate polymorphic populations more easily, the equations themselves generally have to be solved numerically, and it is more difficult to gain understanding as to why a certain behavior is observed. Nevertheless, mathematically sophisticated techniques have shown that multimodality remains a robust phenomenon for a very wide class of PDE models \citep{elmhirst_pod_2008,doebeli_adaptive_2011}.

\subsection{Stochastic individual-based models}

The most general approach to modelling adaptive diversification is through individual-based models. In such models, birth rates, death rates, and interaction rules are specified for each individual, and system-level properties are observed by letting the system change according to these rules. These models can be shown to be more general, biologically explicit versions of both PDE models and adaptive dynamics models, as will be discussed in section \ref{dem_stoch}.

\subsection{An example of a prediction of adaptive diversification}

Consider the continuous version of the logistic equation given by the invasion fitness \eqref{AD_cts_logistic_invasion_fitness} in adaptive dynamics and the PDE \eqref{cts_logistic} for PDE models. Assume that we have $\alpha(u,v) = \exp{-(u-v)^2/(2\sigma^{2}_{\alpha})}$ and $K(u) = K_{0}\exp{-(u)^2/(2\sigma^{2}_{K})}$, \emph{i.e.} Gaussian carrying capacity and Gaussian competition kernel. Then, it is known that diversification fails (\emph{i.e.} the population remains monomorphic) if and only if:
\begin{equation}\label{AD_monomorphic_condn}
\sigma_K<\sigma_{\alpha}
\end{equation}.
One way that \ref{AD_monomorphic_condn} can be satisfied is if $\sigma_K$ is very small. Biologically, this means that the environment is such that only some very particular morphs are viable, and the limit where $\sigma_K = 0$ corresponds to a case where only a single morph is environmentally viable. Ecologists are well-acquainted with this notion, and refer to it as `habitat filtering', the phenomenon in which the habitat itself `selects' for certain phenotypes due to particular limiting abiotic factors.\\
\\
A second way to satisfy \ref{AD_monomorphic_condn} is if $\sigma_{\alpha}$ is very large. Biologically, this means that competition occurs across a wide range of phenotypes, and the limit where $\sigma_{\alpha} = \infty$ corresponds to frequency-independent competition. In other words, diversification can fail if competition cannot be alleviated through character displacement, \emph{i.e} selection is not disruptive (or has a very weak disruptive component). This could happen if, for example, the morphs are competing for a resource that has no alternatives and can only be acquired in a single way (Ex: Competition for sunlight in plants).

\section{Demographic stochasticity and stochastic models}\label{dem_stoch}

The probabilistic origins of adaptive diversification were first noted by Dieckmann and Law, who used a stochastic individual-based model in which ecological rules at the level of the individual led to the population evolving as a Markov chain in trait space in a `quasi-monomorphic' manner \citep{dieckmann_dynamical_1996}. This stochastic process was shown to yield the canonical equation of adaptive dynamics \eqref{AD_canonical_eqn} as a first order deterministic approximation, thus establishing a connection between stochastic individual-based rules (microscopic dynamics) and deterministic evolution via adaptive dynamics (macroscopic approximations). Champagnat \textit{et al.} have since used more sophisticated mathematical tools (measure-valued processes and their infinitesimal generators) to extend such models to include polymorphic populations, and have arrived at similar results \citep{champagnat_unifying_2006,champagnat_individual_2008}. These latter studies were able to show convergence of the expected paths of the stochastic process to both PDE models such as \eqref{cts_logistic} and the canonical equation \eqref{AD_canonical_eqn} based on the approximations used and the limits taken (large population size, weak mutation, separation of ecological and evolutionary timescales, etc), thus uniting the three modelling approaches introduced in section \ref{AD}. However, despite this explicit recognition of the probabilistic roots that unify adaptive diversification, the effects of demographic stochasticity on the process of diversification are relatively poorly studied.\\
\\
The effects of demographic stochasticity on evolutionary branching were first studied by Claessen \textit{et al.}, first for asexual populations \citep{claessen_delayed_2007} and later for sexual populations \citep{claessen_effect_2008}. Both studies used a two-patch resource competition models, and broadly first formulated deterministic adaptive dynamics for the system, and then used computational individual-based simulations in which the total population size is capped in order to study the effects of demographic stochasticity. Both studies found that smaller populations have longer waiting times to reach evolutionary branching, and found that below a critical threshold population size, populations did not branch and remained monomorphic. The sexual model \citep{claessen_effect_2008} was also empirically tested using data on resource polymorphism in populations of arctic charr fishes, where lake size acts as a proxy for the maximum allowed population size, and the authors indeed found that populations living in larger lakes tended to exhibit more polymorphism.\\

\myfig{0.8}{Media/2.1_dem_stoch_effects.png}{\textbf{Effect of population size on evolutionary branching}. Two different realizations of a stochastic individual-based model of resource competition in one dimension where individuals compete according to a continuous version of the logistic equation, written in Python. The infinite population limit of this equation is given by \eqref{cts_logistic} with $\sigma_{K} = 1.9, \sigma_{\alpha} = 0.7$. The PDE model (infinite population limit) predicts evolutionary branching for these parameter values. Each point represents an individual. The model on top has a population of around 1000 individuals and remains monomorphic. The model on the bottom has a population of around 10000 individuals and exhibits evolutionary branching, where an initially monomorphic population evolves to become dimorphic.}{fig_dem_stoch}

On the mathematical front, Wakano and Iwasa first provided analytical support for the phenomenon of delayed evolutionary branching in small monomorphic populations using a moment-based approach in a Wright-Fisher model where evolutionary branching was identified by an `explosion' in the variance of the trait distribution in the population \citep{wakano_evolutionary_2013}. These authors identified two different regimes of branching, which they call `deterministic branching' (meaning branching is guaranteed) and `stochastic branching' (meaning branches have a propensity to recombine soon after branching) based on population size and selection strength. They too found that polymorphism is more likely in larger populations, and that monomorphism is virtually guaranteed if the population size is small enough. Importantly, since their model was in a Wright-Fisher framework, the total population size was assumed to be fixed. Debarre and Otto have recently extended these results to cases in which population sizes are allowed to vary using a similar moment-based approach  \citep{debarre_evolutionary_2016}. These latter authors were able to further decompose the regimes of selection strength (and population size) identified by Wakano and Iwasa. Within Wakano and Iwasa's `stochastic branching' regime, Debarre and Otto identify a `no branching' regime, an `intermittent branching' regime, and a `branched' regime. The first and last of these are characterized by branching events that are either extremely infrequent or extremely frequent, leading to predictable dynamics. Only in the smaller intermediate range was branching `truly' unpredictable over biologically relevant timescales. In this intermediate regime, populations undergo repeated cycles of branching and collapse, and hence the number of polymorphisms in the population varies with time.\\

\section{Diversity in multi-dimensional trait space}\label{high_dim}
Organisms rarely (if ever) vary along a single independent phenotypic axis. However, adaptive dynamics in multi-dimensional trait space is not very easy to deal with due to mathematical technicalities that arise with identifying neccessary and sufficient criteria for evolutionary singularities to be branching points \citep{doebeli_adaptive_2011,leimar_multidimensional_2009}. As such, adaptive diversification in multi-dimensional trait spaces is usually studied through PDE models or individual-based models \citep{ispolatov_individual-based_2016}. As usual, all models discussed in this section operate in the (infinitely) large population limit. Surprisingly, while the question of saturating levels of diversity in adaptive diversification has not been extensively studied in either one dimensional models or explicit spatial models, it \textit{has} been examined in the context of the dependence on the dimensionality of the trait space. The motivation for these studies arises from the classic notions of competitive exclusion and limiting similarity, with position in trait space being thought of as analogous to the (Hutchinsonian) ecological niche.\\
Broadly, the results of these studies align with intuitive expectations: As the dimensionality of the trait space increases, expected diversity increases. For one-dimensional adaptive diversification, strong frequency-dependence is needed to ensure coexistence of multiple morphs/branches at equilibrium. Using the continuous logistic equation \eqref{cts_logistic} with Gaussian forms of $K(x)$ and $\alpha(x,y)$, Doebeli and Ispolatov have shown that as the dimensionality of the trait space increases, progressively weaker frequency dependence is sufficient to maintain multiple morphs at equilibrium \citep{doebeli_complexity_2010}. Furthermore, they have also shown that for a fixed level of frequency-dependence, the probability of having multiple morphs at equilibrium increases as the dimensionality of the trait space increases. Other authors have since arrived at broadly similar conclusions in more general settings \citep{debarre_multidimensional_2014,svardal_organismal_2014}. While these studies say that coexistence is \textit{easier} in higher dimensions, they do not comment on how many morphs are expected to coexist at equilibrium. This has only been addressed by a recent study \citep{doebeli_diversity_2017} which used adaptive dynamics for a continuous analogue of the Lotka-Volterra competition equations to attack the question. They find that all else being equal, the number of coexisting morphs should increase exponentially with the dimensionality of the trait space that the organisms compete in. These authors also repeat this analysis using PDE models (Equation \eqref{cts_logistic} in particular) and computational individual-based simulations in place of adaptive dynamics and claim to find the same broad results.

\section{The importance of being finite (and stochastic)}\label{goals}
The previous sections paint an intriguing picture of the state of adaptive diversification as an explanatory paradigm. Distinct modeling approaches (adaptive dynamics, PDE models, and stochastic IbMs) have been proposed to model adaptive diversification, and all predict that diversification is a robust phenomenon \citep{doebeli_adaptive_2011}. However, theory predicts an exponential increase in the number of coexisting eco-variants as the dimensionality of the trait space increases. Since organisms generally live in rather high-dimensional trait spaces, if we were to to by the theory, we would expect extremely high diversity of eco-variants to be the norm in nature. However, empirical literature (see section \ref{synthesis}) suggests that this is not the case. What, then, may explain this apparent disagreement of theory with the real world, and how may we remedy it?\\
I posit here that the culprit is the assumption of infinite population size. Indeed, we have seen that theoretical studies predict that when demographic stochasticity is taken into account, waiting time to branching increases, and below a threshold population size, the system fails to diversify. Though this is only an indirect measure of the expected number of coexisting eco-variants, it is nevertheless rather suggestive. Furthermore, empirical literature also suggests that population size can have important effects on the diversity of eco-variants that a population can harbour: In several species of lake fish known to harbour adaptive phenotypic polymorphisms, studies \citep{claessen_effect_2008,recknagel_crater_2014,recknagel_ecosystem_2017} show that population size is positively correlated lake size, and that larger lakes harbour more intraspecific polymorphism. However, since larger lakes also present more ecological opportunity in the form of resource niches, it is important to remember that population size is not the only factor at play here. Studies have also shown that higher population density is positively correlated with color polymorphism in African land snails \citep{owen_polymorphism_1963} and the occurrence of a discrete dimorphism in forcep size in European earwigs \citep{tomkins_population_2004}.\\
Intuitively, larger demographic stochasticity can lead to lower diversity due to the stochastic elimination of rare mutants regardless of their fitness advantage, in stark contrast to the `invasion implies fixation' paradigm of deterministic models such as adaptive dynamics. Larger demographic stochasticity also leaves populations with less `access' to mutations per unit time, which may explain why studies have consistently observed a larger wait time to diversification. Lastly, theories such as adaptive dynamics require the population to be exactly located at evolutionary singularities, whereas with finite, stochastic populations, we observe a spread around these singularities, further hindering diversification.\\
\\
In this thesis, I will directly use stochastic birth-death derived from first principles to investigate the expected number of distinct eco-variants that can be harboured in a system. In the process, we will develop a general organizational paradigm for developing stochastic individual-level population models from first principles, and show that many well-known models from population genetics arise as special cases in the infinite population size limit. Using certain approximation schemes, we will arrive at a tractable approximate process that describes how weak 
stochastic fluctuations drive populations away from deterministic expectations.


\chapter{Population dynamics from stochastic first principles}\label{chap_BD}
\epigraph{\justifying Somewhere [...] between the specific that has no meaning and the general that has no content there must be, for each purpose and at each level of abstraction, an optimum degree of generality}{Kenneth~\citet{boulding_general_1956}}

Let us now consider the situation we are actually interested in. Assume that our population is \emph{not} composed of identical organisms, but instead can contain up to $m$ different kinds of organisms - For example, individuals may come in one of $m$ colors, or a gene may have $m$ different alleles. The specific interpretation of the different variants is irrelevant to our formalism, and I, therefore, refer to each distinct variant of an organism simply as a `type'. Unlike many classic stochastic formulations in evolutionary theory~\citep{fisher_genetical_1930,wright_evolution_1931,moran_random_1958, kimura_problems_1957, kimura_diffusion_1964, kimura_number_1964, crow_introduction_1970, lande_natural_1976}, I do not assume a fixed (effective) population size and instead allow the total population size to fluctuate naturally over time. 

\section{Description of the process and the Master Equation}
Given a population that can contain up to $m$ different (fixed) kinds of organisms, it can be entirely characterized by specifying the number of organisms of each type (Figure \ref{fig_nD_pop_description}A). Thus, the state of the population at a given time $t$ is an $m$-dimensional \emph{vector} of the form $\mathbf{n}(t) = [n_1(t),n_2(t),\ldots,n_m(t)]^{\mathrm{T}}$, where $n_i(t)$ is the number of individuals of type $i$ in the population at time $t$.

Given a state $\mathbf{n}(t)$,  we also need to describe how this vector can change over time due to births and deaths (ecology). In this case, a birth or death could result in an individual belonging to one of $m$ different types. Thus, whereas before we had two functions $b(n)$ and $d(n)$ which take in a number as an input, we now require $2m$ functions that take in a vector as an input (Figure \ref{fig_nD_pop_description}B). In other words, for each type $i \in \{1,2,\ldots,m\}$, we must specify a birth rate $b_i(\mathbf{n})$ and a death rate $d_i(\mathbf{n})$. By `rates', I mean that if we know that \emph{either a birth or a death} occurs, then the probability that this event is the birth of an individual of type $i$ is given by
\begin{equation*}
\mathbb{P}\big[\textrm{ Birth of a type $i$ individual } \big{|} \textrm{ something happened }\big] = \frac{b_i(\mathbf{n})}{\sum\limits_{j=1}^{m}(b_j(\mathbf{n})+d_j(\mathbf{n}))}
\end{equation*}
and the probability that the event is the death of an individual of type $i$ is
\begin{equation*}
\mathbb{P}\big[\textrm{ Death of a type $i$ individual } \big{|} \textrm{ something happened }\big] = \frac{d_i(\mathbf{n})}{\sum\limits_{j=1}^{m}(b_j(\mathbf{n})+d_j(\mathbf{n}))}
\end{equation*}

\myfig{0.9}{figures/3.1_BD_process_2D.png}{\textbf{Schematic description of a multi-dimensional birth-death process}. \textbf{(A)} Consider a population of birds in which individuals are either {\color{red}red} or {\color{blue}blue}. In this case, we have $m=2$, since there are two types of individuals in the population. \textbf{(B)} The state of the system can be described by a vector containing the number of individuals of each discrete type (in this case, the number of red and blue birds in the population). Births and deaths result in changes in the elements of the state vector.}{fig_nD_pop_description}

As before, we can describe the rate of change of $P(\mathbf{n},t)$, the probability of finding the population in a state $\mathbf{n}$ at time $t$, by measuring the inflow and outflow rates. Since the population changes in units of exactly one individual (by definition of a birth-death process; see section \ref{sec_math_background}), we know that these inflow and outflow rates must only involve populations that are a single individual away from our focal population. In other words, for a population $\mathbf{n} = [n_1,\ldots,n_{m}]^{\mathrm{T}}$, the `inflow' is from all populations of the form $[n_1,\ldots,n_{i}-1,\dots,n_{m}]^{\mathrm{T}}$ through a birth of a type $i$ individual, and from all populations of the form $[n_1,\ldots,n_{i}+1,\dots,n_{m}]^{\mathrm{T}}$ through the death of a type $i$ individual. Thus, we have the inflow rate
\begin{equation}
\label{nD_rate_in}
\begin{split}
R_{\textrm{in}}(\mathbf{n},t) &= \sum\limits_{j=1}^{m}b_{j}([n_1,\ldots,n_{j}-1,\ldots,n_m]^{\mathrm{T}})P([n_1,\ldots,n_{j}-1,\ldots,n_m]^{\mathrm{T}},t) \\
& +\sum\limits_{j=1}^{m}d_{j}([n_1,\ldots,n_{j}+1,\ldots,n_m]^{\mathrm{T}})P([n_1,\ldots,n_{j}+1,\ldots,n_m]^{\mathrm{T}},t)
\end{split}
\end{equation}
Outflow is through births and deaths of individuals in the population $\mathbf{n}$ itself, and thus we have:
\begin{equation}
\label{nD_rate_out}
R_{\textrm{out}}(\mathbf{n},t) = \sum\limits_{j=1}^{m}b_{j}(\mathbf{n})P(\mathbf{n},t) + \sum\limits_{j=1}^{m}d_{j}(\mathbf{n})P(\mathbf{n},t)
\end{equation}
We will now define step operators, both for notational ease and in anticipation of the system size expansion. For each $i \in \{1,\ldots,m\}$, let us define two step operators $\mathcal{E}_{i}^{\pm}$ by their action on any function $f([n_1,\ldots,n_m],t)$ as:
\begin{equation}
\label{nD_step_operators}
\mathcal{E}_{i}^{\pm}f([n_1,\ldots,n_i,\ldots,n_m]^{\mathrm{T}},t) = f([n_1,\ldots,n_i \pm 1, \ldots n_m]^{\mathrm{T}},t)
\end{equation}
In other words, $\mathcal{E}_{i}^{\pm}$ just changes the population through the addition or removal of one type $i$ individual. We can now the rate of change of $P(\mathbf{n},t)$ as
\begin{equation}
\label{nD_intermediate_for_M_eqn}
\frac{\partial P}{\partial t}(\mathbf{n},t) = R_{\textrm{in}}(\mathbf{n},t) - R_{\textrm{out}}(\mathbf{n},t)
\end{equation}
Substituting \eqref{nD_rate_in}, \eqref{nD_rate_out}, and \eqref{nD_step_operators} into equation \eqref{nD_intermediate_for_M_eqn}, we obtain:
\begin{equation}
\label{nD_M_eqn}
\frac{\partial P}{\partial t}(\mathbf{n},t) = \sum\limits_{j=1}^{m}\left[(\mathcal{E}_j^{-}-1)b_j(\mathbf{n})P(\mathbf{n},t) + (\mathcal{E}_j^{+}-1)d_j(\mathbf{n})P(\mathbf{n},t)\right]
\end{equation}
This is the master equation of our $m$-dimensional process.

\section{The system-size expansion}
As explained in Section \ref{sec_intuition_sys_size}, we will now assume (on ecological grounds) that there exists a system-size parameter $K > 0$ such that the discrete jumps between states happen in units of $1/K$ and the total population size is controlled by $K$, with $K = \infty$ corresponding to an infinitely large population. In particular, we assume that the birth and death rates scale such that we can make the substitutions
\begin{align*}
\mathbf{x} &\coloneqq \frac{\mathbf{n}}{K}\\
b^{(K)}_i(\mathbf{x}) &\coloneqq \frac{1}{K}b_i(\mathbf{n})\\
d^{(K)}_i(\mathbf{x}) &\coloneqq \frac{1}{K}d_i(\mathbf{n})
\end{align*}
where $\mathbf{x} = \mathbf{n}/K$ measures population \emph{density} instead of population numbers. We now define new step operators $\Delta_{i}^{\pm}$ by their action on any real-valued function $f(\mathbf{x},t)$ as
\begin{equation}
\label{nD_step_operators_rescaled}
\Delta_{i}^{\pm}f([x_1,\ldots,x_m]^{\mathrm{T}},t) = f([x_1,\ldots,x_i \pm \frac{1}{K}, \ldots x_m]^{\mathrm{T}},t)
\end{equation}
In terms of these new variables, \eqref{nD_M_eqn} becomes
\begin{equation}
\label{nd_M_eqn_rescaled}
\frac{\partial P}{\partial t}(\mathbf{x},t) = K\sum\limits_{j=1}^{m}\left[(\Delta_j^{-}-1)b^{(K)}_j(\mathbf{x})P(\mathbf{x},t) + (\Delta_j^{+}-1)d^{(K)}_j(\mathbf{x})P(\mathbf{x},t)\right]
\end{equation}
If $K$ is large, we can once again Taylor expand the action of the step operators as
\begin{equation*}
f([x_1,\ldots,x_i \pm \frac{1}{K}, \ldots x_m]^{\mathrm{T}},t) = f(\mathbf{x},t) \pm \frac{1}{K}\frac{\partial f}{\partial x_i}(\mathbf{x},t) + \frac{1}{2K^2}\frac{\partial^2f}{\partial x_i^2}(\mathbf{x},t) + \mathcal{O}(K^{-3})
\end{equation*}
which, after substituting into \eqref{nd_M_eqn_rescaled} and neglecting $ \mathcal{O}(K^{-3})$ terms, yields the equation
\begin{equation}
\label{nD_FPE}
\setlength{\fboxsep}{2\fboxsep}\boxed{\frac{\partial P}{\partial t}(\mathbf{x},t) = \sum\limits_{j=1}^{m}\left[-\frac{\partial}{\partial x_j}\{A_j^{-}(\mathbf{x})P(\mathbf{x},t)\} + \frac{1}{2K}\frac{\partial^2}{\partial x_j^2}\{A_j^{+}(\mathbf{x})P(\mathbf{x},t)\}\right]}
\end{equation}
where
\begin{equation*}
A_{i}^{\pm}(\mathbf{x}) = b^{(K)}_i(\mathbf{x})\pm d^{(K)}_i(\mathbf{x})
\end{equation*}
Equation \eqref{nD_FPE} is an $m$-dimensional Fokker-Planck equation, and corresponds to the $m$-dimensional It\^o process
\begin{equation}
\label{nD_Ito_SDE}
d\mathbf{X}_{t} = \mathbf{A^-}(\mathbf{X}_t)dt + \frac{1}{\sqrt{K}}\mathbf{D}(\mathbf{X}_t)d\mathbf{W}_t
\end{equation}
where $\mathbf{A^-}(\mathbf{X}_t)$ is an $m$-dimensional vector with $i$\textsuperscript{th} element $ = A^{-}_{i}(\mathbf{X}_t)$. $\mathbf{D}(\mathbf{X}_t)$ is an $m \times m$ matrix with $ij$th element $\left(\mathbf{D}(\mathbf{X}_t)\right)_{ij} = \delta_{ij}\left(A^{+}_{i}A^{+}_{j}\right)^{\frac{1}{4}}$, where $\delta_{ij}$ is the Kronecker delta symbol, defined by
\begin{equation*}
\delta_{ij} = 
\begin{cases}
1 & i=j\\
0 & i\neq j
\end{cases}
\end{equation*}
Finally, $\mathbf{W}_t$ is the $m$-dimensional Wiener process and can be thought of as a vector of independent one-dimensional Wiener processes (which have been defined in \ref{intro_SDE}). This is the `mesoscopic' description of our process.

\section{Functional forms of the birth and death rates}

I assume that the birth and death rate functions have the functional form
\begin{equation}
\label{nD_functional_forms_for_replicator}
\begin{aligned}
b^{(K)}_i(\mathbf{x}) &= x_ib^{\textrm{(ind)}}_{i}(\mathbf{x}) + \mu Q_i(\mathbf{x})\\
d^{(K)}_i(\mathbf{x}) &= x_id^{\textrm{(ind)}}_i(\mathbf{x})
\end{aligned}
\end{equation}
where $b^{\textrm{(ind)}}_{i}(\mathbf{x})$ and $d^{\textrm{(ind)}}_{i}(\mathbf{x})$ are non-negative functions that respectively describe the per-capita birth and death rate of type $i$ individuals, $\mu \geq 0$ is a constant describing the mutation rate in the population, and $Q_i(\mathbf{x})$ is a non-negative function that describes the additional birth rate of type $i$ individuals due to mutations in the population $\mathbf{x}$ that cannot be captured in the per-capita birth rate\footnote{When $x_i = 0$, \emph{i.e.} there are no type $i$ individuals in the population, individuals of type $i$ may still be born through mutations during births of the other types. This cannot be captured in $b^{\textrm{(ind)}}_{i}(\mathbf{x})$ because the term $x_ib^{\textrm{(ind)}}_{i}(\mathbf{x})$ vanishes when $x_i = 0$. Note that no analogous problem exists for the death rate, since the death rate of type $i$ individuals must be 0 when $x_i$ is 0 to ensure that we never have negative population densities.}. My assumptions on the functional forms \eqref{nD_functional_forms_for_replicator} thus amount to saying that birth and death rates can be separated into mutational and non-mutational components, and furthermore that the density dependence of the birth and death rates due to non-mutational effects is in a form that allows us to write down per-capita birth and death rates for each type. Let us define the \emph{Malthusian fitness} of the $i\textsuperscript{th}$ type as $w_i(\mathbf{x}) \coloneqq b^{\textrm{(ind)}}_{i}(\mathbf{x}) - d^{\textrm{(ind)}}_{i}(\mathbf{x})$, and the \emph{per-capita turnover rate} of the $i\textsuperscript{th}$ type as $\tau_i(\mathbf{x}) = b^{\textrm{(ind)}}_{i}(\mathbf{x}) + d^{\textrm{(ind)}}_{i}(\mathbf{x})$.  The quantity $w_i(\mathbf{x})$ describes the per-capita growth rate of type $i$ individuals in a population $\mathbf{x}$ discounting mutation. Ecologists often denote this quantity by the symbol $r_i$ and simply call it the (exponential) growth rate of type $i$, but I will stick to $w_i$ and `fitness' here. $\tau_i$ is a measure of the number of events (birth events + death events) that a type $i$ individual experiences in a given time interval --- Individuals with higher turnover rates experience more events (on average) than those with lower turnover rates. This can be thought of as a measure of the `pace of life'. I briefly note that the quantity $\tau_i$ has also been called the `variability in the reproductive output' in the literature~\citep{gillespie_natural_1974}. It is notable that both $w_i$ and $\tau_i$ depend on the state of the population as a whole (\textit{i.e.} $\mathbf{x}$) and not just on the density of the focal type. Thus, in general, the fitness and the turnover rate in our model are both frequency and density-dependent.

\section{Statistical measures for population-level quantities}\label{sec_stat_measures}

Though the causes of evolution are generally through ecological phenomena that occur at the \emph{individual} level like we have been working with, we are very often interested in tracking the effects of evolution on quantities described at a \emph{population} level, such as the mean fitness or mean phenotype in the population~\citep{bourrat_evolution_2019}. Furthermore, the relevant quantities at the individual level, such as individual fitness or phenotype, are typically equal for all individuals of the same type (in some sense this is our basis for defining different types in the first place). I use the term `type-level quantities' henceforth to refer to such quantities that are equal for all individuals that are of the same type. To facilitate the description of such quantities, given any state $\mathbf{x}(t)$ that describes our population at time $t$, let us first define the total (scaled) population size and the frequency of each type in the population as:
\begin{equation}
	\label{nD_tot_pop_and_prop_inds_defn}
	\begin{aligned}
		N_{K}(t) &\coloneqq \sum\limits_{i=1}^{m}x_i(t) = \frac{1}{K}\sum\limits_{i=1}^{m}n_i(t)\\
		p_i(t) &\coloneqq \frac{x_i(t)}{N_{K}(t)} = \frac{n_i(t)}{\sum\limits_{j=1}^{m}n_i(t)}
	\end{aligned}
\end{equation}
Now, let $f$ be any type-level quantity with (possibly time-dependent) value $f_i$ for the $i\textsuperscript{th}$ type. For example, if each type is a phenotype for a trait such as height, which can be assigned a numerical value, then setting $f_i = \textit{value of $i\textsuperscript{th}$ phenotype}$ gives us the mean trait value in the population. We can compute the statistical mean value of any such quantity in the population as
\begin{equation}
\label{nD_mean}
\overline{f}(t) \coloneqq \sum\limits_{i=1}^{m}f_ip_{i}
\end{equation}
the statistical covariance of two such quantities $f$ and $g$ as
\begin{equation}
\label{nD_cov}
\textrm{Cov}(f,g) \coloneqq \overline{fg} - \overline{f}\overline{g}
\end{equation}
and the statistical variance of a quantity $f$ as $\sigma^2_f \coloneqq \textrm{Cov}(f,f)$.

It is important to recognize that these quantities are distinct from and independent of the \emph{probabilistic} expectation, variance, and covariance obtained by integrating over realizations in the underlying probability space. Indeed, for finite populations, the statistical mean, statistical variance, and statistical covariance are all themselves stochastic processes: For each instant of time, these population-level quantities are a random variable (\emph{i.e.} a \emph{function} and not just a number) depending on $\mathbf{p}$, the (random) vector of type frequencies in the population. For infinite populations, the statistical mean, variance, and covariance are entirely deterministic time-dependent quantities that simply describe how $f$ is distributed across the population. Failure to clearly make this distinction between statistical operations and probabilistic operations has led to much confusion with regard to the infinite population Price equation, which is entirely deterministic and does not incorporate features of finite populations such as drift in its original formulation~\citep{frank_price_1997,van_veelen_use_2005,frank_natural_2012}.

\section{Stochastic Trait Frequency Dynamics}\label{sec_nD_freq_eqns}

In appendix \ref{App_density_to_freq}, I use a multivariate version of It\^{o}'s formula to derive a general stochastic equation for the frequencies of each type in the population. Letting $\overline{w} = \sum w_ip_i$ and $\overline{\tau} = \sum \tau_i p_i$ be the average population fitness and turnover respectively, I show in appendix \ref{App_density_to_freq} that the frequency of the $i\textsuperscript{th}$ type in the population $\mathbf{x}(t)$ changes according to the equation:
\begin{equation}
\label{nD_eqn_for_frequencies}
\begin{aligned}
dp_i(t) &= \left[(w_i(\mathbf{x}) - \overline{w})p_i + \mu\left\{Q_i(\mathbf{p}) - p_i\left(\sum\limits_{j=1}^{m}Q_j(\mathbf{p})\right)\right\}\right]dt\\
&- \frac{1}{K}\frac{1}{N_{K}(t)}\left[(\tau_i(\mathbf{x}) - \overline{\tau})p_i + \mu\left\{Q_i(\mathbf{p}) - p_i\left(\sum\limits_{j=1}^{m}Q_j(\mathbf{p})\right)\right\}\right]dt\\
&+ \frac{1}{\sqrt{K}}\frac{1}{N_{K}(t)}\left[\left(A^{+}_{i}\right)^{1/2}dW^{(i)}_t - p_i\sum\limits_{j=1}^{m}\left(A^{+}_{j}\right)^{1/2}dW^{(j)}_t\right]
\end{aligned}
\end{equation}
where $W^{(1)}_t,W^{(2)}_t, \ldots, W^{(m)}_t$ are $m$ independent one-dimensional standard Brownian motion processes and I have used the notation $Q_i(\mathbf{p}) = Q_i(\mathbf{x})/N_K(t)$ for notational clarity. I will show below that the first term in this expression describes directional changes in the population composition due to `classical' evolutionary forces such as selection and mutation that occur in deterministic infinite population models. The second term is an additional directional force on population composition that is only seen in finite populations and can be thought of as a biasing `selection' for reduced turnover rate due to an effect similar to gambler's ruin in probability theory. The consequences of this term, as well as connections with previous studies, are discussed in detail in part \ref{part_summary}. Finally, the last term of equation \eqref{nD_eqn_for_frequencies} describes non-directional stochastic effects due to fluctuations and has a `spreading effect'.

\section{The infinite population limit}\label{sec_nD_det_limit}
Like in \ref{sec_1D_processes}, we can once again take $K \to \infty$ in \eqref{nD_Ito_SDE} to obtain a deterministic expression. Here, the expression reads
\begin{equation}
\label{nD_det_limit}
\frac{d\mathbf{x}}{dt} = \mathbf{A^-}(\mathbf{x}) = \mathbf{b}^{(K)}(\mathbf{x}) - \mathbf{d}^{(K)}(\mathbf{x})
\end{equation}
where the $m$-dimensional vector-valued functions $\mathbf{b}^{(K)}(\mathbf{x})$ and $\mathbf{d}^{(K)}(\mathbf{x})$ on the RHS are defined as having $i$\textsuperscript{th} element $b^{(K)}_i(\mathbf{x})$ and $d^{(K)}_i(\mathbf{x})$ respectively. For the trait frequencies, by taking $K \to \infty$ in \eqref{nD_eqn_for_frequencies}, we obtain a deterministic equation that reads:
\begin{equation}
\label{nD_replicator_mutator}
\setlength{\fboxsep}{2\fboxsep}\boxed{\frac{dp_i}{dt} = (w_i(\mathbf{x}) - \overline{w})p_i + \mu\left[Q_i(\mathbf{p}) - p_i\left(\sum\limits_{j=1}^{m}Q_j(\mathbf{p})\right)\right]}
\end{equation}
The first term of \eqref{nD_replicator_mutator} describes changes due to faithful (non-mutational) replication, and the second describes changes due to mutation. For this reason, equation \eqref{nD_replicator_mutator} is called the \emph{replicator-mutator equation} in the evolutionary game theory literature, where the individual `types' are interpreted to be pure strategies~\citep{hofbauer_evolutionary_1998,page_unifying_2002,cressman_replicator_2014}. If in addition, each $w_i(\mathbf{x})$ is linear in $\mathbf{x}$, meaning we can write $w_i(\mathbf{x}) = \sum_{j}a_{ij}x_j$ for some set of constants $a_{ij}$, then we recover the more well-known replicator-mutator equation for matrix games in which the constants $a_{ij}$ form the `payoff matrix' (See the example presented in Appendix~\ref{App_examples}). As is well-known, the replicator equation (without mutation) for matrix games with $m$ pure strategies is equivalent to the generalized Lotka-Volterra equations for a community with $m-1$ species~\citep{hofbauer_evolutionary_1998}, providing the connection to community ecology.  Equation \eqref{nD_replicator_mutator} is also equivalent to Eigen's \emph{quasispecies equation} from molecular evolution if each `type' is interpreted as a genetic sequence and each $w_i(\mathbf{x})$ is a constant function\footnote{Mutational effects are often additionally assumed to act through direct `transmission probabilities' of mutating from one type to another. This means that we can write $Q_i(\mathbf{p}) = \sum_j Q_{ij}p_j$, where $Q_{ii} = 0$, and for each $j \neq i$, $Q_{ij} \geq0$ is a constant describing the probability of a $j \to i$ mutation (conditioned on the occurrence of a mutation). Substituting this into \eqref{nD_replicator_mutator} yields an equation in terms of `$Q$-matrices' or `mutation matrices' that may be more familiar to some.}~\citep{page_unifying_2002}. We can now calculate how the mean of any `type level' quantity $f$, defined as $f_i$ for the $i$\textsuperscript{th} type, changes in the population (For example, if each type is a phenotype for a trait such as height, which can be assigned a numerical value, then setting $f_i = \textit{value of $i\textsuperscript{th}$ phenotype}$ gives us the mean trait value in the population). The product rule of calculus tells us that we have the relation
\begin{equation}
\label{product_rule_for_nD_price}
\frac{d}{dt}\left(\sum\limits_{i=1}^{m}f_ip_i\right) = \sum\limits_{i=1}^{m}\left(f_i\frac{\partial p_i}{\partial t} + p_i\frac{\partial f_i}{\partial t}\right) = \sum\limits_{i=1}^{m}f_i\frac{\partial p_i}{\partial t} + \overline{\left(\frac{\partial f}{\partial t}\right)}
\end{equation}
Multiplying both sides of equation \eqref{nD_replicator_mutator} by $f_i$ and summing over all $i$, we obtain
\begin{align*}
\sum\limits_{i=1}^{m}f_i\frac{\partial p_i}{\partial t} &= \sum\limits_{i=1}^{m}f_iw_i(\mathbf{x})p_i - \overline{w}\sum\limits_{i=1}^{m}f_ip_i + \mu\left[\sum\limits_{i=1}^{m}Q_i(\mathbf{p})f_i - \left(\sum\limits_{j=1}^{m}Q_j(\mathbf{p})\sum\limits_{i=1}^{m}p_if_i\right)\right]\\
\Rightarrow \frac{d\overline{f}}{dt} &= \overline{wf}-(\overline{w})(\overline{f}) + \mu\left[\sum\limits_{i=1}^{m}Q_i(\mathbf{p})f_i - \left(\sum\limits_{j=1}^{m}Q_j(\mathbf{p})\right)\overline{f}\right]
\end{align*}
Using the definition of statistical covariance from \eqref{nD_cov}, we obtain
\begin{equation}
\sum\limits_{i=1}^{m}f_i\frac{\partial p_i}{\partial t} = \mathrm{Cov}(w,f) + \mu\left[\sum\limits_{i=1}^{m}Q_i(\mathbf{p})f_i - \left(\sum\limits_{j=1}^{m}Q_j(\mathbf{p})\right)\overline{f}\right] 
\end{equation}
Thus, substituting this into \eqref{product_rule_for_nD_price}, we get
\begin{equation}
\label{nD_Price_time_dependent}
\setlength{\fboxsep}{2\fboxsep}\boxed{\frac{d\overline{f}}{dt} = \mathrm{Cov}(w,f) + \mu\left[\sum\limits_{i=1}^{m}Q_i(\mathbf{p})f_i - \left(\sum\limits_{j=1}^{m}Q_j(\mathbf{p})\right)\overline{f}\right] + \overline{\left(\frac{\partial f}{\partial t}\right)}}
\end{equation}
This is a Price equation for quantities $f_i$ which can vary over time~\citep{lion_theoretical_2018,day_price_2020}. To obtain the more familiar Price equation seen in textbooks, we can consider time-independent $f_i$, \emph{i.e.} situations in which each $f_i$ is constant over time, and thus changes in $\overline{f}$ are purely due to changes in the composition of the population. For such quantities, we have $\frac{\partial f_i}{\partial t} = 0 \ \forall \ i$ and thus obtain the dynamic version of the famous Price equation~\citep{page_unifying_2002,lion_theoretical_2018}:
\begin{equation}
\label{nD_Price}
\setlength{\fboxsep}{2\fboxsep}\boxed{
	\frac{d\overline{f}}{dt} = \mathrm{Cov}(w,f) + \mu\left[\sum\limits_{i=1}^{m}Q_i(\mathbf{p})f_i - \left(\sum\limits_{j=1}^{m}Q_j(\mathbf{p})\right)\overline{f}\right]}
\end{equation}
The first term of the RHS describes the statistical covariance between the quantity $f$ and the fitness $w$. The second term describes `transmission bias' due to mutational effects - The first summation is the `inflow' of $f$ due to mutations, and the second is the `outflow'.

\section{Stochastic fluctuations and the weak noise approximation}

As in the one-dimensional case, we can go a little further if the noise is sufficiently weak. Let the deterministic trajectory obtained by solving \eqref{nD_det_limit} be given by $\boldsymbol{\alpha}(t) = [\alpha_1(t), \alpha_2(t), \ldots, \alpha_m(t)]^{\mathrm{T}}$.  We can once again track 
stochastic fluctuations from the deterministic trajectory by introducing the new variables
\begin{equation}
\begin{aligned}
\mathbf{y} &= \sqrt{K}(\mathbf{x} - \boldsymbol{\alpha}(t))\\
s&=t\\
\tilde{P}(\mathbf{y},s) &= \frac{1}{\sqrt{K}}P(\mathbf{x},t)
\end{aligned}
\end{equation}
Then, after some algebra that follows the exact same steps as in section \ref{sec_1D_WNA} and retaining only the highest order terms in $\sqrt{K}$, we obtain the equation:
\begin{equation}
\label{nD_WNA_intermediate}
\frac{\partial \Tilde{P}_{0}}{\partial s}(\mathbf{y},s) = \sum\limits_{j=1}^{m}\left(-\frac{\partial}{\partial y_j}\left\{(A^{-}_{j})_{1}(s)\Tilde{P}_{0}(\mathbf{y},s)\right\}+\frac{1}{2}{A_j}^{+}(\boldsymbol{\alpha}(s))\frac{\partial^2}{\partial{y_j}^2}\{\Tilde{P}_{0}(\mathbf{y},s)\}\right)
\end{equation}
where $(A^{-}_{j})_{1}(s)$ is the $\mathcal{O}(1/\sqrt{K})$ term of the power series expansion
\begin{equation*}
A^-_{j}(\boldsymbol{\alpha} + \frac{\mathbf{y}}{\sqrt{K}}) = \sum\limits_{n=0}^{\infty}(A^{-}_{j})_{n}(s)\left(\frac{\mathbf{y}}{\sqrt{K}}\right)^n
\end{equation*}
In the case where the series expansion is a Taylor expansion, then the first-order term of this expansion is given by
\begin{equation}
\label{nD_WNA_taylor_term}
(A^{-}_{j})_{1}(s) = \sum\limits_{i=1}^{m} y_i\left(\frac{\partial A^{-}_j(\mathbf{x})}{\partial x_i}\bigg{|}_{\mathbf{x}=\boldsymbol{\alpha}(s)}\right)
\end{equation}
In multi-variable calculus, the directional derivative\footnote{Some authors use the notation $\partial_{\mathbf{v}}f(\mathbf{x})$ or $\mathbf{v}\cdot\nabla f(\mathbf{x})$ for this object.} $D_{\mathbf{v}}(f(\mathbf{x}))$ of a multidimensional function $f: \mathbb{R}^m \to \mathbb{R}$ along a vector $\mathbf{v}$ is the function defined by:
\begin{equation}
\label{directional_derivative_defn}
D_{\mathbf{v}}(f(\mathbf{x})) \coloneqq \sum\limits_{i=1}^{m}\left(\frac{\partial f(\mathbf{x})}{\partial x_i}\right)v_i = \lim_{h \to 0}\frac{f(\mathbf{x}+h\mathbf{v})-f(\mathbf{x})}{h}
\end{equation}
Comparing with \eqref{nD_WNA_taylor_term}, we see that the weak-noise approximation of our process is:
\begin{equation}
\label{nD_WNA}
\frac{\partial P}{\partial t}(\mathbf{y},t) = \sum\limits_{j=1}^{m}\left(-\frac{\partial}{\partial y_j}\left\{D_{\mathbf{y}}(A_j^-(\boldsymbol{\alpha}))(t)P(\mathbf{y},t)\right\}+\frac{1}{2}{A_j}^{+}(\boldsymbol{\alpha}(t))\frac{\partial^2}{\partial{y_j}^2}\{P(\mathbf{y},t)\}\right)
\end{equation}
where we have dropped the tildes and gone back from $s$ to $t$ for notational clarity. The directional derivative of the population turnover rate $A_j^-$ `in the direction' of the stochastic fluctuation $\mathbf{y}$ at the deterministic point $\boldsymbol{\alpha}(s)$ is precisely the multidimensional analogue of the derivative we had in \eqref{1D_WNA}. The meaning of equation \eqref{nD_WNA} is clearer if we compute how the moments of the fluctuation $y_i$ in the density of type $i$ individuals (for some $i$) change over time. Let $n > 0$. We have:
\begin{align}
\frac{d}{dt}\mathbb{E}[y_i^n] &= \frac{d}{dt}\int\limits_{\mathbb{R}^m}y_i^nP(\mathbf{y},t)d\mathbf{y}\\
&= \int\limits_{\mathbb{R}^m}y_i^n\frac{\partial P}{\partial t}(\mathbf{y},t)d\mathbf{y}\label{nD_change_of_moments_defn}
\end{align}
where I have assumed that $y_i^n$ and $P(\mathbf{y},t)$ vary sufficiently smoothly to allow us to interchange the order of derivatives and integrals. By the Leibniz integral rule, this only requires the map $(\mathbf{y},t) \to y_i^nP(\mathbf{y},t)$ to be bounded and $C^1$ in an open subset of $\mathbb{R}^m \times [0,\infty)$. We have also used the notation $\displaystyle \int\limits_{\mathbb{R}^m} \ f(\mathbf{y}) \ d\mathbf{y} \coloneqq \int\limits_{\mathbb{R}}\int\limits_{\mathbb{R}}\cdots\int\limits_{\mathbb{R}} \ f(\mathbf{y}) \ dy_1 dy_2 \ldots dy_m$. The one-dimensional integrals are over the entire real line and not just over $[0,\infty)$ because fluctuations can be either positive (greater than $\boldsymbol{\alpha}(t)$) or negative (lesser than $\boldsymbol{\alpha}(t)$). For notational brevity, let us use the shorthand $D_j \coloneqq D_{\mathbf{y}}(A_j^-(\boldsymbol{\alpha}))(t)$. We can now substitute \eqref{nD_WNA} into \eqref{nD_change_of_moments_defn} to obtain
\begin{align}
\frac{d}{dt}\mathbb{E}[y_i^n] &= \int\limits_{\mathbb{R}^m} y_i^n \left(\sum\limits_{j=1}^{m}\left(-\frac{\partial}{\partial y_j}\left\{D_{j}P(\mathbf{y},t)\right\}+\frac{1}{2}{A_j}^{+}(\boldsymbol{\alpha}(t))\frac{\partial^2}{\partial{y_j}^2}\{P(\mathbf{y},t)\}\right)\right)d\mathbf{y}\\
&= \sum\limits_{j=1}^{m}\left[-\int\limits_{\mathbb{R}^m} y_i^n\frac{\partial}{\partial y_j}\left\{D_{j}P(\mathbf{y},t)\right\}d\mathbf{y} + \frac{{A_j}^{+}(\boldsymbol{\alpha}(t))}{2}\int\limits_{\mathbb{R}^m} y_i^n\frac{\partial^2}{\partial{y_j}^2}\{P(\mathbf{y},t)\}d\mathbf{y}\right]\label{nD_intermediate_for_moments}
\end{align}
We will evaluate the integrals on the RHS of \eqref{nD_intermediate_for_moments} using integration by parts. Recall that for any two functions $u$ and $v$ defined on a domain $\Omega$, the general formula for integration by parts is given by:
\begin{equation}
\label{int_by_parts_general_formula}
\int\limits_{\Omega}\frac{\partial u}{\partial x_i}vd\mathbf{x} = -\int\limits_{\Omega}u\frac{\partial v}{\partial x_i}d\mathbf{x} + \int\limits_{\partial\Omega}uv\gamma_{i}dS(\mathbf{x})
\end{equation}
where $\partial \Omega$ is the boundary of $\Omega$, $dS$ is the surface element of this boundary, and $\gamma_i$ is the $i\textsuperscript{th}$ component of the unit outward normal to the boundary. In our case, we have $\Omega = \mathbb{R}^m$, and thus the boundary conditions are evaluated as $\|y\| \to \infty$. I assume that the magnitude of stochastic fluctuations is bounded, and therefore impose the condition $\displaystyle \lim_{\|y\| \to \infty}  P(\mathbf{y},t) = 0$. Further, I assume that this decay is fast enough that $\displaystyle \lim_{\|y\| \to \infty}D_jP(\mathbf{y},t) = 0\ \forall \ j$. Under these conditions, we can evaluate the two integrals in the RHS of \eqref{nD_intermediate_for_moments} by using integration by parts and discarding the boundary term (The second term on the RHS of \eqref{int_by_parts_general_formula}). Note that since the $y_i$s are orthogonal to each other, we have the relation:
\begin{equation*}
\frac{\partial y_i ^{n}}{\partial y_j} = \delta_{ij}ny_i^{n-1}
\end{equation*}
Using this relation and then using integration by parts on the RHS of \eqref{nD_intermediate_for_moments} (once for the first term and twice for the second term), we obtain the considerably simpler expression
\begin{align}
\frac{d}{dt}\mathbb{E}[y_i^n] &= n\int\limits_{\mathbb{R}^m} y_i^{n-1}D_{i}P(\mathbf{y},t)d\mathbf{y} + \frac{n(n-1)}{2}A_i^+(\boldsymbol{\alpha}(t))\int\limits_{\mathbb{R}^m} y_i^{n-2}P(\mathbf{y},t)d\mathbf{y}\\
\Rightarrow \frac{d}{dt}\mathbb{E}[y_i^n] &= n\mathbb{E}[y_i^{n-1}D_{i}]+\frac{n(n-1)}{2}A_i^+(\boldsymbol{\alpha}(t))\mathbb{E}[y_i^{n-2}]\label{nD_general_moment_eqns}
\end{align}
Of particular interest are the cases $n=1$ (corresponding to the expected value of $y_i$) and $n=2$ (which can be used along with the expected value to compute the variance of $y_i$). We have:
\begin{align}
\frac{d}{dt}\mathbb{E}[y_i] &= \mathbb{E}[D_{i}]\label{nD_moment_eqn_mean}\\
\frac{d}{dt}\mathbb{E}[y_i^2] &= 2\mathbb{E}[y_iD_{i}] +  A_i^+(\boldsymbol{\alpha}(t)) = 2\langle y_i,D_i \rangle + 2\mathbb{E}[y_i]\mathbb{E}[D_i]+A_i^+(\boldsymbol{\alpha}(t))\label{nD_moment_eqn_2nd_mom}
% \Rightarrow \frac{d}{dt}\mathrm{Var}(y_i) &= - \mathbb{E}[D_{i}]^2 + 2\mathbb{E}[y_iD_{i}] +  A_i^+(\boldsymbol{\alpha}(t))\label{nD_moment_eqn_var}
\end{align}
Where $\langle X, Y \rangle$ is the \emph{probability} covariance between two random variables $X$ and $Y$, defined as $\langle X, Y \rangle \coloneqq \mathbb{E}[XY] -  \mathbb{E}[X]\mathbb{E}[Y]$. This is not to be confused with the \emph{statistical} covariance defined by \eqref{nD_cov} that appears in the deterministic Price equation \eqref{nD_Price}. Thus, whether stochastic fluctuations are expected to grow or decay is controlled by $D_i$, a measure of how the growth rate ($b_i - d_i$) changes along the direction of the fluctuation, whereas the spread of the fluctuations (the variance) has contributions from the net turnover rate ($A_i^+ = b_i + d_i$) and the (probability) covariance between the fluctuation and $D_i$. In the case of the functional forms given by \eqref{nD_functional_forms_for_replicator}, we have:
\begin{equation}
A_i^-(\mathbf{x}) = w_i(\mathbf{x})x_i + \mu Q_i(\mathbf{x})
\end{equation}
and thus, from \eqref{nD_WNA_taylor_term}, we can calculate the directional derivative $D_i$ as
\begin{align}
D_i &= \sum\limits_{k=1}^{m} y_k\left(\frac{\partial A^{-}_i(\mathbf{x})}{\partial x_k}\bigg{|}_{\mathbf{x}=\boldsymbol{\alpha}(t)}\right)\\
&= \sum\limits_{k=1}^{m} y_k\left(\frac{\partial}{\partial x_k}\left( w_i(\mathbf{x})x_i + \mu Q_i(\mathbf{x})\right)\bigg{|}_{\mathbf{x}=\boldsymbol{\alpha}(t)}\right)\\
&= \sum\limits_{k=1}^{m} y_k\left(\alpha_i\frac{\partial w_i}{\partial x_k}\bigg{|}_{\mathbf{x}=\boldsymbol{\alpha}(t)}\right) + y_iw_i(\boldsymbol{\alpha}) + \mu\sum\limits_{k=1}^{m} y_k\left(\frac{\partial Q_i}{\partial x_k}(\mathbf{x})\bigg{|}_{\mathbf{x}=\boldsymbol{\alpha}(t)}\right)\\
&= y_iw_i(\boldsymbol{\alpha}) + \alpha_iD_{\mathbf{y}}(w_i(\boldsymbol{\alpha})) + \mu D_{\mathbf{y}}(Q_i(\boldsymbol{\alpha}))\label{nD_WNA_directional_derivative_for_replicator_eqns}
\end{align}
Using this in \eqref{nD_moment_eqn_mean}, we see that the expected change of a fluctuation in the density of type $i$ individuals evolves as:
\begin{equation}
\label{nD_moment_eqn_mean_replicator}
\frac{d}{dt}\mathbb{E}[y_i] = \underbrace{w_i(\boldsymbol{\alpha})\mathbb{E}[y_i]}_{\substack{\text{Current fitness of type $i$} \\ \text{at deterministic trajectory $\boldsymbol{\alpha}$} \\ \text{(scaled by expected fluctuation $\mathbb{E}[y_i]$)}}} + \underbrace{\alpha_i\mathbb{E}[D_{\mathbf{y}}(w_i(\boldsymbol{\alpha}))]}_{\substack{\text{Expected change in fitness} \\ \text{ of type $i$ in going from $\boldsymbol{\alpha}$ to $\mathbf{y}$} \\ \text{(scaled by deterministic density $\alpha_i$)}}} + \underbrace{\mu\mathbb{E}[D_{\mathbf{y}}(Q_i(\boldsymbol{\alpha}))]}_{\substack{\text{Expected effect of} \\ \text{mutations}}}
\end{equation}
Thus, the expected behavior of fluctuations in the weak noise limit is controlled purely by fitness differences and mutational effects. If $\mathbb{E}[y_i] \equiv 0 \ \forall \ i$ is a stable fixed point for the system of equations defined by \eqref{nD_moment_eqn_mean_replicator}, then stochastic fluctuations are expected to decay, meaning that the deterministic point $\boldsymbol{\alpha}$ is a stable configuration for the complete dynamics when fluctuations are weak. In the case of 2-strategy matrix games (\emph{i.e.} when $m=2$ and the fitness functions $w_i(p)$ are of the form $w_i = \sum_j a_{ij}p_j$ for some constants $a_{ij}$), it has been shown that $\mathbb{E}[y_i] \equiv 0 \ \forall \ i$ is a stable fixed point for \eqref{nD_moment_eqn_mean_replicator} if and only if the population state $\boldsymbol{\alpha}$ is an evolutionarily stable strategy (ESS) for the determinstic game defined by \eqref{nD_det_limit} (which of course reduces to the replicator dynamics defined by \eqref{nD_replicator_mutator}), thus recovering a stochastic version of a classic result in evolutionary game theory~\citep{tao_stochastic_2007}.

\chapter{Examples and Applications}\label{chap_examples}
\epigraph{\justifying Our future advances will not be concerned with universal laws, but instead with universal approaches to tackling particular problems}{Peter Kareiva}

In the previous chapter, we used various tools and arguments to illustrate how to obtain approximate equations for very general stochastic birth-death processes. In this chapter, we put this theory into use through examples to study extinction and diversity in finite populations.

\section{Example in one dimension: The stochastic logistic equation}
Consider the functional forms of example \ref{ex_1D_stoch_logistic}, given by
\begin{equation}
\label{ex_1D_stoch_logistic_BD_eqns}
\begin{aligned}
    b(n) &= \lambda n\\
    d(n) &= \left(\mu + (\lambda-\mu)\frac{n}{K}\right)n
\end{aligned}
\end{equation}
Here, $K$ is the system-size parameter. Introducing the new variable $x=n/K$, we obtain
\begin{align*}
    b_K(x) &= \frac{1}{K}b(n) = \frac{1}{K}\lambda Kx\\
    d_K(x) &= \frac{1}{K}d(n) = \frac{1}{K}\left(\mu + (\lambda-\mu)\frac{Kx}{K}\right)Kx
\end{align*}
Thus, we have
\begin{equation*}
    A^{\pm}(x) = b_K(x)\pm d_K(x) = x\left(\lambda \pm \left(\left(\mu + (\lambda-\mu)x\right)\right) \right)
\end{equation*}
Defining $r=\lambda-\mu$ and $v=\lambda+\mu$ and using equation \eqref{1D_SDE}, we 
see that the `mesoscopic view' of the system is given by the solution of the SDE
\begin{equation}\label{ex_1D_stoch_logistic_full_SDE}
dX_t =  rX_t(1-X_t)dt + \sqrt{\frac{X_t(v+rX_t)}{K}}dB_t
\end{equation}
From equation \eqref{1D_det_limit}, we see that the deterministic dynamics are
\begin{equation}\label{ex_1D_stoch_logistic_det_limit}
\frac{dx}{dt} = A^-(x) = rx(1-x)
\end{equation}
showing that in the infinite population limit, we
obtain the logistic equation. Letting $\alpha(t)$ be the solution of the logistic equation \eqref{ex_1D_stoch_logistic_det_limit}, We can Taylor expand $A^{\pm}(x)$ for the weak noise approximation, and we find:
\begin{align*}
A^-_1(x) &= \frac{d}{dx}(rx(1-x))\biggl{|}_{x=\alpha} = r(1 - 2\alpha(t))\\
A^+_0(x) &= \alpha(t)(v+r\alpha(t))
\end{align*}
Thus, the weak noise approximation of \ref{ex_1D_stoch_logistic_BD_eqns} is given by
\begin{equation}
    X_t = \alpha(t) + \frac{1}{\sqrt{K}}Y_t
\end{equation}
where the stochastic process $Y_t$ is an Ornstein-Uhlenbeck process given by the solution to the linear SDE
\begin{align}
    dY_t &= A^-_1(t)Y_tdt + \sqrt{A^+_0(t)}dB_t\nonumber\\
    \Rightarrow dY_t &= r(1 - 2\alpha(t))Y_tdt + \sqrt{\alpha(t)(v+r\alpha(t))}dB_t\label{ex_1D_stoch_logistic_WNA}
\end{align}

The time series predicted by these three processes look qualitatively similar and all seem to fluctuate about the deterministic steady state (Figure \ref{fig_1D_stoch_logistic_timeseries}).

\myfig{1}{Media/4.1_stoch_logistic_timeseries.png}{Comparison of a single realization of the exact birth-death process \eqref{ex_1D_stoch_logistic_BD_eqns}, the deterministic trajectory \eqref{ex_1D_stoch_logistic_det_limit}, the non-linear Fokker-Planck equation \eqref{ex_1D_stoch_logistic_full_SDE}, and the weak noise approximation \eqref{ex_1D_stoch_logistic_WNA} for \textbf{(A)} $K = 500$, \textbf{(B)} $K = 1000$, and \textbf{(C)} $K = 10000$. $\lambda = 2, \mu = 1$ for all thee cases.}{fig_1D_stoch_logistic_timeseries}

The deterministic trajectory \eqref{ex_1D_stoch_logistic_det_limit} has two fixed points, one at $x=0$ (extinction) and one at $x=1$ (corresponding to a population size of $n=K$). For $r > 0$, $x=0$ is unstable and $x=1$ is a global attractor, meaning in the deterministic limit, when $r > 0$, all populations end up at $x=1$ given enough time. The stochastic dynamics \eqref{ex_1D_stoch_logistic_full_SDE} and \eqref{ex_1D_stoch_logistic_WNA} depend not only on $r$, but also on $v$, the sum of the birth and death rates. It has been proven that $X_t = 0$ is the only recurrent state for the full stochastic dynamics \eqref{ex_1D_stoch_logistic_full_SDE}, meaning that every population is guaranteed to go extinct\footnote{This can be proven using tools from Markov chain theory. For those interested, the proof uses ergodicity to arrive at a contradiction if any state other than $0$ exhibits a non-zero density at steady state.} given enough time \citep{nasell_extinction_2001}, thus illustrating an important difference between finite and infinite populations. $X_t = 0$ is also an `absorbing' state since once a population goes extinct, it has no way of being revived in this model. However, if $K$ is large enough, the eventual extinction of the population may take a very long time. In fact, we can make the expected time to extinction arbitrarily long by making $K$ sufficiently large. Thus, for moderately large values of $K$, it is biologically meaningful only to look at a weaker version of the steady state distribution by imposing the condition that the population does not go extinct and looking at the `transient' dynamics \citep{hastings_transients_2004}. Conditioned on non-extinction, the solution to \eqref{ex_1D_stoch_logistic_full_SDE} has a `quasistationary' distribution about the deterministic attractor $X_t = 1$, with some variance reflecting the effect of noise-induced fluctuations in population size \citep{nasell_extinction_2001} due to the finite size of the population. The weak-noise approximation \eqref{ex_1D_stoch_logistic_WNA} implicitly assumes non-extinction by only measuring small fluctuations from the deterministic solution to \eqref{ex_1D_stoch_logistic_det_limit} and thus, at steady state, naturally describes a quasistationary distribution centered about $X_t = 1$. The steady-state density (probability density function as $t \to \infty$) of the exact birth-death process \eqref{ex_1D_stoch_logistic_BD_eqns} is compared with that predicted by \eqref{ex_1D_stoch_logistic_full_SDE} and \eqref{ex_1D_stoch_logistic_WNA} for various values of $K$ in figure \ref{fig_1D_stoch_logistic_densities}.

\myfig{0.85}{Media/4.2_stoch_logistic_distributions.png}{Comparison of the steady-state densities given by \eqref{ex_1D_stoch_logistic_BD_eqns}, \eqref{ex_1D_stoch_logistic_full_SDE}, and \eqref{ex_1D_stoch_logistic_WNA} for \textbf{(A)} $K = 500$, \textbf{(B)} $K = 1000$, and \textbf{(C)} $K = 10000$. $\lambda = 2, \mu = 1$ for all thee cases. Each curve was obtained using $1000$ independent realizations.}{fig_1D_stoch_logistic_densities}

\section{Example for discrete traits: Lotka-Volterra and matrix games in finite populations}

The methods outlined in the above section have recently been used to study the population dynamics of a finite population playing a so-called `matrix game' (An evolutionary game for which you can write down a payoff matrix) with 2 pure strategies \citep{tao_stochastic_2007}. Based on the interpretation of what each type represents, this is mathematically equivalent to studying frequency-dependent selection on a one-locus two-allele gene (with a bijective genotype-phenotype map and no mutations) or studying two-species competitive Lotka-Volterra dynamics, as we will show below. The stochastic Lotka-Volterra competition model shown below has also been proved to be equivalent to an $m$-allele Moran model under certain limits \citep{constable_mapping_2017}.\\
\\
Let us imagine a population with $m$ types of individuals that are competing for resources. Let the state of the population be characterized by the vector $\mathbf{v}(t) = [v_1(t),v_2(t),\ldots,v_m(t)]$, where $v_i(t)$ is the number of type $i$ individuals at time $t$. Let the birth and death rates of the $i$th type be given by:
\begin{equation}
\label{nD_example_numbers_b_d_rates}
\begin{aligned}
b_i(\mathbf{v}) &= \lambda v_i\\
d_i(\mathbf{v}) &= \left(\mu + \frac{1}{K}\left(\sum\limits_{j=1}^{m}M_{ij}v_j\right)\right)v_i
\end{aligned}
\end{equation}
where $K > 0$ is our system size parameter (and represents a global carrying capacity across all types), $\lambda > 0$ and $\mu > 0$ are suitable positive constants representing the baseline natality and mortality common to all types, and $M_{ij} \geq 0$ is a non-negative constant describing the effect of type $j$ individuals on the death rate of type $i$ individuals due to competition. We assume that $M_{ij} \ll K$. The values $M_{ij}$ are often collected in an $m \times m$ matrix $\mathbf{M}$. The negative of this matrix, $-\mathbf{M}$, is called the `payoff matrix' (in evolutionary game theory) or `interaction matrix' (in Lotka-Volterra models), which is why the infinite population analogues of such models are called `matrix games' in the evolutionary game theory literature. Lotka-Volterra models also frequently assume that the diagonal elements $M_{ii}$ are all equal, though we will not make that assumption here.\\
\\
Going from population numbers $\mathbf{v}$ to densities $\mathbf{x} = \mathbf{v}/K$, we obtain the birth and death rates:
\begin{equation}
\label{nD_example_density_b_d_rates}
\begin{aligned}
b^{(K)}_i(\mathbf{x}) &= \lambda x_i\\
d^{(K)}_i(\mathbf{x}) &= \left(\mu + \sum\limits_{j=1}^{m}M_{ij}x_j\right)x_i
\end{aligned}
\end{equation}
Thus, we have
\begin{equation*}
A^{\pm}_{i} = x_i\left(\lambda \pm \left(\mu + \sum\limits_{j=1}^{m}M_{ij}x_j\right)\right)
\end{equation*}
Defining $r = \lambda - \mu$ and $v = \lambda + \mu$, we see from equation \eqref{nD_Ito_SDE} that the mesoscopic view is the $m$ dimensional SDE given by
\begin{equation}
\label{nD_example_SDE}
d\mathbf{X}_{t} = \mathbf{A^-}(\mathbf{X}_t)dt + \frac{1}{\sqrt{K}}\mathbf{D}(\mathbf{X}_t)d\mathbf{B}_t
\end{equation}
where 
\begin{equation*}
\mathbf{A^-}_i = {(\mathbf{X}_{t})}_i(r - \sum\limits_{j=1}^{m}M_{ij}{(\mathbf{X}_{t})}_j) 
\end{equation*}
and
\begin{equation*}
(\mathbf{D}\mathbf{D}^{\mathrm{T}})_i = {(\mathbf{X}_{t})}_i(v + \sum\limits_{j=1}^{m}M_{ij}{(\mathbf{X}_{t})}_j) 
\end{equation*}
From \eqref{nD_det_limit}, we see that the deterministic limit is a set of $m$ coupled ODEs given by
\begin{equation}
\label{nD_example_det_limit}
\frac{d x_i}{dt} = x_i\left(r - \sum\limits_{j=1}^{m}M_{ij}x_j\right)
\end{equation}
These are precisely the competitive Lotka-Volterra equations for a system of $m$ species. By matching the terms of \eqref{nD_example_density_b_d_rates} with those of \eqref{nD_functional_forms_for_replicator}, we can identify that we have $\mu = 0$ and
\begin{equation}
\label{nD_example_fitness}
\begin{aligned}
b^{(\textrm{int})}_{i}(\mathbf{x}) &= \lambda\\
d^{(\textrm{int})}_i(\mathbf{x}) &= \mu + \sum\limits_{j=1}^{m}M_{ij}x_j\\
w_i(\mathbf{x}) &= r - \sum\limits_{j=1}^{m}M_{ij}x_j
\end{aligned}
\end{equation}
If $p_i(t)$ is the frequency of type $i$ individuals in the population at time $t$ and $N_K(t) = \sum_i x_i(t)$, then the mean fitness is given by
\begin{align}
\overline{w}(t) &= \sum\limits_{i=1}^{m}w_ip_i\\
&= \sum\limits_{i=1}^{m}\left(r - \sum\limits_{j=1}^{m}M_{ij}x_j\right)p_i\\
&= r - \sum\limits_{i=1}^{m}p_i\left(\sum\limits_{j=1}^{m}M_{ij}x_j\right)
\end{align}
where we have used the fact that $\sum_i p_i = 1$ in the last line. Using \eqref{nD_replicator_mutator} to write down the equations for the frequencies $p_i$, we obtain
\begin{equation}
\frac{dp_i}{dt} = \left[\sum\limits_{i=1}^{m}p_i\left(\sum\limits_{j=1}^{m}M_{ij}x_j\right) - \sum\limits_{j=1}^{m}M_{ij}x_j\right]p_i
\end{equation}
Let us define the payoff matrix $\mathbf{B} = - \mathbf{M}$. Then, the dynamics are
\begin{equation}
\frac{1}{N_K(t)}\frac{dp_i}{dt} = \left[(\mathbf{Bp})_i - \mathbf{p}\cdot\mathbf{Bp}\right]p_i   
\end{equation}
which is the familiar version of the replicator equation seen in most textbooks, with an extra $N_K(t)$ factor to account for the fact that $\sum_i x_i$ is allowed to fluctuate in our model. If instead $N_K$ was a constant for all time, it could simply be absorbed into the definition of the payoff matrix $B$ to obtain exactly the replicator equation as presented in most ecology/evolution textbooks. We briefly note here that if $m$ is large, then both the stochastic dynamics \eqref{nD_example_SDE} and the deterministic limit \eqref{nD_example_det_limit} can be simplified from an $m$ dimensional system to an $m-1$ dimensional system by a coordinate transformation. If we go from the variables $x_1,\ldots,x_m$ to the variables $p_1,\ldots,p_{m-1},N_K$, then, we can exploit the fact that $N_K$ varies much less than the $p_i$ terms to project the system onto a `slow manifold' in which $N_K$ is approximately constant, thus obtaining an $m-1$ dimensional system of equations and recovering the relation between the Lotka-Volterra equations for $m$ species and the replicator equation for $m-1$ tactics \citep{constable_mapping_2017,parsons_dimension_2017}. However, we will not explore this further in this manuscript, and refer the reader to \citep{constable_stochastic_2013} and \citep{parsons_dimension_2017} for a review of the ideas of (stochastic) dynamics on slow manifolds.\\
Let the solution to the equations \eqref{nD_example_det_limit} be given by $\mathbf{a}(t) = [a_1(t),\ldots,a_m(t)]$. For the weak noise approximation, we can Taylor expand $A^{\pm}_i$ and use \eqref{nD_WNA_directional_derivative_for_replicator_eqns} to compute the directional derivative as:
\begin{align}
D_i &= y_iw_i(\mathbf{a}) + a_i\sum\limits_{k=1}^{m}y_k\left(\frac{\partial w_i}{\partial x_k}\bigg{|}_{\mathbf{x}=\mathbf{a}(t)}\right)\\
&= y_iw_i(\mathbf{a}) + a_i\sum\limits_{k=1}^{m}y_k\left(\frac{\partial}{\partial x_k}(r+\sum\limits_{j=1}^{m}B_{ij}x_j)\bigg{|}_{\mathbf{x}=\mathbf{a}(t)}\right)\\
&= y_iw_i(\mathbf{a}) + a_i\sum\limits_{k=1}^{m}y_kB_{ik}\\
\Rightarrow D_i &= y_iw_i(\mathbf{a}) + a_iw_i(\mathbf{y}) - ra_i\label{nD_example_directional_derivative}
\end{align}
where we have used the fact that $w_i(\mathbf{y}) = r + \sum\limits_{k=1}^{m}y_kB_{ik}$ (from \eqref{nD_example_fitness}) in the last step. Thus, in the weak noise approximation of our process, the dynamics are given by
\begin{equation}
\mathbf{x}(t) = \mathbf{a}(t) + \frac{1}{\sqrt{K}}\mathbf{y}(t)
\end{equation}
where the stochastic fluctuations $\mathbf{y}(t)$ satisfy the linear Fokker-Planck equation
\begin{equation}
\resizebox{1.1\textwidth}{!}{$\displaystyle\frac{\partial P}{\partial t}(\mathbf{y},t) = \sum\limits_{i=1}^{m}\left(-\frac{\partial}{\partial y_i}\left\{\left(y_iw_i(\mathbf{a}) + a_iw_i(\mathbf{y}) - ra_i\right)P(\mathbf{y},t)\right\}+\frac{1}{2}\left(a_i\left(v - \sum\limits_{j=1}^{m}B_{ij}a_j\right)\right)\frac{\partial^2}{\partial{y_i}^2}P(\mathbf{y},t)\right)$}
\end{equation}
Using \eqref{nD_example_directional_derivative} in \eqref{nD_moment_eqn_mean}, we see that the fluctuations are expected to evolve as:
\begin{equation}
\label{nD_example_moment_eqn_mean}
\frac{d}{dt}\mathbb{E}[y_i] = w_i(\mathbf{a})\mathbb{E}[y_i] + a_i\sum\limits_{k=1}^{m}B_{ik}\mathbb{E}[y_k]
\end{equation}
or, in matrix form:
\begin{equation}
\resizebox{0.93\textwidth}{!}{$\displaystyle
	\frac{d}{dt}\begin{bmatrix}
	\mathbb{E}[y_1]\\
	\mathbb{E}[y_2]\\
	\vdots\\
	\mathbb{E}[y_i]\\
	\vdots\\
	\mathbb{E}[y_m]
	\end{bmatrix}
	=
	\begin{bmatrix}
	(r + \sum\limits_{j=1}^{m}B_{1j}a_j + a_1B_{11}) & a_1B_{12} & a_1B_{13} & \dots & \dots & \dots & a_1B_{1m}\\
	a_2B_{21} & (r + \sum\limits_{j=1}^{m}B_{2j}a_j + a_2B_{22}) & a_2B_{23} & \dots & \dots & \dots & a_2B_{2m}\\
	\vdots &  & \ddots & &  & & \vdots\\
	a_{i}B_{i1} & a_iB_{i2} & a_iB_{i3} & \dots & (r + \sum\limits_{j=1}^{m}B_{ij}a_j + a_iB_{ii}) & \dots & a_iB_{im}\\
	\vdots &  &  & & & \ddots & \vdots\\
	a_mB_{m1} & a_mB_{m2} & a_mB_{m3} & \dots & \dots & \dots & (r + \sum\limits_{j=1}^{m}B_{mj}a_j + a_mB_{mm})
	\end{bmatrix}
	\begin{bmatrix}
	\mathbb{E}[y_1]\\
	\mathbb{E}[y_2]\\
	\vdots\\
	\mathbb{E}[y_i]\\
	\vdots\\
	\mathbb{E}[y_m]
	\end{bmatrix}
	$}
\end{equation}
The eigenvalues of the first matrix on the RHS will tell us whether the fixed point $\mathbb{E}[y_i] = 0 \ \forall \ i$ (the only fixed point of this system) is stable, or whether fluctuations are expected to grow (up to the point where the fluctuations are so large that the WNA is no longer valid).
\hl{can compare time series or something here if we think it is helpful. Visualization may be little hard though, because we always have $\geq 3$ dimensions ($m$, + 1 additional for time)}

\section{Example for quantitative traits}

\subsection{Interlude: Detecting clusters through Fourier analysis}

In section \ref{sec_infD_processes}, we used various approximations to arrive at the linear functional Fokker-Planck equation
\begin{equation}
\label{functional_WNE_chap_4}
    \frac{\partial P}{\partial t}(\zeta,t) = \int\limits_{\mathcal{T}}\left(-\frac{\delta}{\delta \zeta(x)}\left\{\mathcal{D}_{\zeta}[\mathcal{A}^{-}](x)P(\zeta,t)\right\}+\frac{1}{2}\mathcal{A}^{+}(x|\psi)\frac{\delta^2}{\delta\zeta(x)^2}\{P(\zeta,t)\}\right)dx
\end{equation}
for describing stochastic fluctuations $\zeta$ from the deterministic solution obtained by solving \eqref{deterministic_traj}. Our goal is now to find a method to effectively detect and describe evolutionary branches (modes in trait space, corresponding to individual morphs) for this process. We will do this by measuring the autocorrelation of the distribution of the population over trait space. A convenient theorem due to Weiner and Khinchin relates the autocorrelation of a probability distribution to its power spectral density via Fourier transformation. This has been extensively used in spatial ecology, and we too will make use of it here. Specifically, we will carry out a basis expansion of our functions in the Fourier basis $\{e^{ikx}\}_{k\in\mathbb{Z}}$. If $\mathcal{D}_{\zeta}[\mathcal{A}^{-}]$ is a translation-invariant\footnote{This is horrible nomenclature by the mathematicians. Though `invariant' is the conventional name for this concept, the intended meaning is not really invariant but `equivariant'. Formally, let $\mathcal{F}$ be a suitable function space of real valued functions. For any $c \in \mathbb{R}$, let $T_c: \mathcal{F} \to \mathcal{F}$ be the translation operator on this space, defined by $T_c[f(x)] = f(x+c)$. An operator $L: \mathcal{F} \to \mathcal{F}$ is said to be translation-invariant if it commutes with $T_c$ for every $c \in \mathbb{R}$, \emph{i.e.} $T_c[L[f]] = L[T_c[f]] \ \forall \ f \in \mathcal{F} \ \forall \ c \in \mathbb{R}$} linear operator, then $\exp(ikx)$ acts as an eigenfunction, significantly simplifying the calculations. We therefore assume this below.



%\includestandalone[width=.8\textwidth]{backend/fourier_figure}
\myfig{0.8}{Media/4.3_fourier_fig.png}{\textbf{Schematic description of Fourier analysis}. A function $\phi(x)$ (shown in red) over the trait space can be decomposed as the sum of infinitely many Fourier modes (shown in blue) $\phi_k$. In the Fourier dual space, we can look at the peaks of each of these Fourier modes: The magnitude of $\phi_k$ tells us how much it contributes to the actual function of interest $\phi$.}{fig_Fourier}

Assume that $\mathcal{D}_{\zeta}[\mathcal{A}^{-}]$ takes the form:
\begin{equation*}
 \mathcal{D}_{\zeta}[\mathcal{A}^-](x,t) = L[\zeta(x,t)]   
\end{equation*}
for a translation-invariant linear operator $L$ that only depends on $x$ and $t$. The presence of phenotypic clustering and polymorphisms can be analyzed by examining the power spectrum of $\Tilde{P}_{0}(\zeta,s)$ over the trait space.\\
We assume that $\zeta$, and $\mathcal{A}^{+}(x|\psi)$ admit the Fourier basis representations:
\begin{equation}
\label{fourier_representations_functions}
\begin{aligned}
\zeta(x,t) &= \sum\limits_{k=-\infty}^{\infty}e^{ikx}\zeta_k(t) \ \ ; \ \ \zeta_k(t) = \int\limits_{\mathcal{T}}\zeta(x,t)e^{-ikx}dx\\
\mathcal{A}^{+}(x|\psi) &= \sum\limits_{k=-\infty}^{\infty}e^{ikx}A_k(t) \ \ ; \ \ A_k(t) = \int\limits_{\mathcal{T}}\mathcal{A}^{+}(x|\psi)e^{-ikx}dx
\end{aligned}
\end{equation}
In this case, the functional derivative operator obeys:
\begin{equation}
\label{fourier_representations_derivative}
    \frac{\delta}{\delta \zeta(x)} = \sum\limits_{k=-\infty}^{\infty}e^{-ikx}\frac{\partial}{\partial \zeta_k}
\end{equation}
and since $L$ is linear and translation-invariant, we also have the relation\footnote{This is because $\exp(ikx)$ acts as an eigenfunction for translation invariant linear operators, and therefore, for any function $\varphi = \sum\varphi_k\exp(ikx)$, we have the relation $L[\varphi] = L[\sum\varphi_k\exp(ikx)]=\sum\varphi_kL[\exp(ikx)]=\sum\varphi_kL_k\exp(ikx)$, where $L_k$ is the eigenvalue of $L$ associated with the eigenfunction $\exp(ikx)$. It is helpful to draw the analogy with eigenvectors of matrices and view $L_k\varphi_k$ as the projection of $L[\varphi]$ along the $k$th eigenvector $e_k = \exp(ikx)$.}:
\begin{equation}
\label{fourier_representation_linear_operator}
    L[\zeta] = \sum\limits_{k=-\infty}^{\infty}L_{k}\zeta_ke^{ikx}
\end{equation}
where 
\begin{equation*}
    L_k = e^{-ikx}L[e^{ikx}]
\end{equation*}
Lastly, by definition of Fourier modes, we have, for any differentiable real function $F$ and any fixed time $t > 0$:
\begin{equation}
\label{fourier_mode_relation}
\frac{\partial}{\partial \zeta_j(t)}F(\zeta_i(t)) = \delta_{ij}F'(\zeta_j(t))
\end{equation}
where $\delta_{ij}$ is the Kronecker delta symbol.
Using \eqref{fourier_representations_functions}, \eqref{fourier_representations_derivative}, and \eqref{fourier_representation_linear_operator} in \eqref{functional_WNE_zeroth_order}, we get, for the first term of the RHS:
\begin{gather}
-\int\limits_{\mathcal{T}}\frac{\delta}{\delta \zeta(x)}\left\{L[\zeta(x,t)]P(\zeta,t)\right\}dx\nonumber\\
= -\int\limits_{\mathcal{T}}\sum\limits_{k}e^{-ikx}\frac{\partial}{\partial \zeta_k}\{\sum\limits_{n}e^{inx}L_n\zeta_nP\}dx\nonumber\\
= -\int\limits_{\mathcal{T}}\sum\limits_{k}\sum\limits_{n}e^{-i(k-n)x}\frac{\partial}{\partial \zeta_k}\{L_n\zeta_nP\}dx\nonumber\\
= -2\pi\sum\limits_{k}L_{k}\frac{\partial}{\partial \zeta_k}\{\zeta_kP\}\label{fourier_FPE_first_term}
\end{gather}
and for the second:
\begin{gather}
\int\limits_{\mathcal{T}}\sum\limits_{k}e^{ikx}A_k\left(\sum\limits_{m}\sum\limits_{n}e^{-i(m+n)x}\frac{\partial}{\partial \zeta_m}\frac{\partial}{\partial \zeta_n}P\right)dx\nonumber\\
= \int\limits_{\mathcal{T}}\sum\limits_{k}\sum\limits_{m}\sum\limits_{n}e^{i(k-m-n)x}A_k\frac{\partial}{\partial \zeta_m}\frac{\partial}{\partial \zeta_n}\{P\}dx\nonumber\\
= 2\pi\sum\limits_{m}\sum\limits_{n}A_{m+n}\frac{\partial}{\partial \zeta_m}\frac{\partial}{\partial \zeta_{n}}\{P\}\label{fourier_FPE_second_term}
\end{gather}
Substituting \eqref{fourier_FPE_first_term} and \eqref{fourier_FPE_second_term} into \eqref{functional_FPE}, we see that the Fokker-Planck equation in Fourier space reads:
\begin{equation}
\label{fourier_FPE}
\frac{\partial P}{\partial t} = -2\pi\sum\limits_{k}L_{k}\frac{\partial}{\partial \zeta_k}\{\zeta_kP\} + \pi\sum\limits_{m}\sum\limits_{n}A_{m+n}\frac{\partial}{\partial \zeta_m}\frac{\partial}{\partial \zeta_{n}}\{P\}
\end{equation}
It is important to remember that since $\zeta(x,t)$ is a stochastic process, $\zeta_i$ is really a stochastic process and thus $\zeta_i(t)$ is actually shorthand for the random variable $(\zeta_i)_{t}(\omega)$, where $\omega$ is a sample path in the Fourier dual of our original probability space. Multiplying both sides of \eqref{fourier_FPE} by $\zeta_r$ and integrating over the probability space to obtain expectation values, we see that
\begin{align}
\frac{d}{dt}\mathbb{E}[\zeta_r] &= - 2\pi \sum\limits_{k}\int\zeta_rL_k\frac{\partial}{\partial \zeta_k}\{\zeta_k P\}d\omega + \pi\sum\limits_{m}\sum\limits_{n}A_{m+n}\int\zeta_r\frac{\partial}{\partial \zeta_m}\frac{\partial}{\partial \zeta_{n}}(P)d\omega\nonumber\\
&=  2\pi \sum\limits_{k}L_k\int\zeta_k\frac{\partial \zeta_r}{\partial \zeta_k}Pd\omega + \pi\sum\limits_{m}\sum\limits_{n}A_{m+n}\int\frac{\partial^2 \zeta_r}{\partial \zeta_m\partial \zeta_{n}}Pd\omega\nonumber\\
&=  2\pi L_{r}\mathbb{E}[\zeta_r]\label{fourier_mode_mean}
\end{align}
where we have used integration by parts and neglected the boundary term in the second step (assuming once again that $P$ decays rapidly enough near the boundaries that this is doable), and then used \eqref{fourier_mode_relation} to arrive at the final expression. Similarly, multiplying \eqref{fourier_FPE} by $\zeta_r\zeta_s$, integrating over the probability space and using integration by parts, we get:
\begin{align}
\frac{d}{dt}\mathbb{E}[\zeta_r\zeta_s] &= 2\pi \sum\limits_{k}L_{k}\int\zeta_kP\frac{\partial}{\partial \zeta_k}\{\zeta_r\zeta_s\}d\omega + \pi\sum\limits_{m}\sum\limits_{n}A_{m+n}\int\limits_{-\infty}^{\infty}P\frac{\partial}{\partial \zeta_m}\frac{\partial}{\partial \zeta_{n}}\{\zeta_r\zeta_s\}d\omega\nonumber\\
&= 2\pi (L_{r} + L_{s})\mathbb{E}[\zeta_r\zeta_s] + \pi (A_{2r}+A_{2s})\label{fourier_mode_covariance}
\end{align}
At the stationary state, the LHS must be zero by definition, and we must therefore have, for every $r,s \in \mathbb{Z}$,:
\begin{equation}
\label{fourier_mode_covariance_stationary}
\mathbb{E}[\zeta_r\zeta_s] = -   \frac{A_{2r}+A_{2s}}{2(L_{r}+L_{s})}
\end{equation}
Recall that the Fourier modes of any real function $\varphi$ must satisfy $\varphi_{-r} = \overline{\varphi}_r$. Since $\zeta$, $A$ and $L$ are all real, we can substitute $s=-r$ in equation \eqref{fourier_mode_covariance_stationary} to obtain the autocovariance relation:
\begin{equation}
\label{fourier_mode_autocovariance}
\mathbb{E}[|\zeta_r|^2] =- \frac{\mathrm{Re}(A_{2r})}{2\mathrm{Re}(L_{r})}
\end{equation}

The presence of phenotypic clustering can be detected using the `spatial covariance' of our original process $\phi$, defined as \citep{rogers_demographic_2012}:
\begin{equation}
\label{spatial_covariance_defn}
\Xi[x] = m(\mathcal{T})\int\limits_{\mathcal{T}}\mathbb{E}[\phi_{\infty}(x)\phi_{\infty}(y-x)]dy
\end{equation}
where $\phi_{\infty}$ is the stationary state distribution of $\{\phi_t\}_{t}$ and $m$ is the Lebesgue measure. We can use a spatial analogue of the Wiener-Khinchin theorem to calculate:
\begin{equation}
\label{spatial_covariance_zeta}
\Xi[x] = m(\mathcal{T})\left[\int\limits_{\mathcal{T}}\psi_{\infty}(x)\psi_{\infty}(y-x)dy + \frac{1}{K}\sum\limits_{r=-\infty}^{\infty}\mathbb{E}[|\zeta_r|^2]e^{irx}\right]
\end{equation}
where the expectations in the second term are for the stationary state. A flat $\Xi[x]$ indicates that there are no clusters, and peaks indicate the presence of clusters.

\subsection{An example: The quantitative logistic equation}

As an example, let us carry out the functional Kramers-Moyal expansion for the birth and death functionals given by \eqref{Rogers_logistic_BD}. In terms of the scaled variable $\phi = K\nu$, these functions read:
\begin{equation}
\label{Rogers_logistic_BD_scaled}
\begin{aligned}
b_K(x|\phi) &= \frac{1}{K}b(x|\nu) = \frac{1}{K}\left( r\int\limits_{\mathcal{T}}m(x,y)K\phi(y)dy\right)\\
    d_K(x|\phi) &= \frac{1}{K}d(x|\nu) =  \frac{1}{K}\left(\frac{K\phi(x)}{Kn(x)}\int\limits_{\mathcal{T}}\alpha(x,y)K\phi(y)dy\right)
\end{aligned}
\end{equation}
Thus, using equation \eqref{deterministic_traj}, the deterministic trajectory becomes:
\begin{equation}
\label{Rogers_logistic_BD_deterministic}
\frac{\partial \psi}{\partial t}(x,t) = r\int\limits_{\mathcal{T}}m(x,y)\psi(y,t)dy-\frac{1}{n(x)}\psi(x,t)\int\limits_{\mathcal{T}}\alpha(x,y)\psi(y,t)dy
\end{equation}
Note that if we employ the change of variables $\Psi = K\psi$ to go back from $\mathcal{M}_{K}$ (\textit{i.e} $\phi^{(t)}$) to $\mathcal{M}$ (\textit{i.e} $\nu^{(t)}$), we recover the familiar continuous logistic equation (Equation \eqref{cts_logistic}) as the deterministic limit:
\begin{align*}
\frac{\partial \Psi}{\partial t}(x,t) &= r\int\limits_{\mathcal{T}}m(x,y)\Psi(y,t)dy-\frac{\Psi(x,t)}{Kn(x)}\int\limits_{\mathcal{T}}\alpha(x,y)\Psi(y,t)dy \\
&\approx r\Psi(x,t) -\frac{\Psi(x,t)}{K(x)}\int\limits_{\mathcal{T}}\alpha(x,y)\Psi(y,t)dy + D_m\nabla^2_{x}\Psi(x,t)
\end{align*}
where $K(x) = Kn(x)$ is the carrying capacity experienced by an individual of phenotype $x$, and $D_m = r \sigma_m^2/2$ measures the `diffusion rate' of the population in trait space. It is left as an exercise for the reader to verify by the same steps that if we instead have the birth rate functional $b(x|\phi) = \lambda\int m(x,y)\phi(y)dy$ (with $m(x,y)$ as defined in \eqref{Rogers_logistic_BD}) and the death rate functional $d(x|\phi) = \phi(x)\left(\mu+(\lambda-\mu)\phi(x)/K\right)$, the macroscopic limit yields the famous Fisher-KPP equation with growth rate $r=\lambda-\mu$ and diffusion constant $D = \lambda \sigma_m^2/2$.\\
\\
In any case, for the system defined by \eqref{Rogers_logistic_BD_scaled}, we can also calculate $\mathcal{D}_{\zeta}[\mathcal{A}^-]$ as
\begin{align*}
\mathcal{D}_{\zeta}[\mathcal{A}^-] &= \frac{d}{d\epsilon}\left( r\int\limits_{\mathcal{T}} m(x,y)(\psi(y)+\epsilon\zeta(y))dy - \frac{\psi(x)+\epsilon\zeta(x)}{n(x)}\int\limits_{\mathcal{T}}\alpha(x,y)(\psi(y)+\epsilon\zeta(y))dy\right) \biggl{|}_{\epsilon = 0}\\
&= r\int\limits_{\mathcal{T}}m(x,y)\zeta(y)dy - \frac{1}{n(x)}\left(\psi(x)\int\limits_{\mathcal{T}}\alpha(x,y)\zeta(y)dy + \zeta(x)\int\limits_{\mathcal{T}}\alpha(x,y)\psi(y)dy\right)
\end{align*}
% \chapter{}
% \input{ind_files/chapter_05}
% \chapter{} <-- uncomment and add as required
% \input{ind_files/chapter_06}

\begin{appendices}
\chapter{Deriving the Fokker-Planck equations for It\^{o} SDEs}\label{App_SDE_FPE}
Here, I present a simple (informal) derivation of the Fokker-Planck equation (FPE) for a one-dimensional It\^{o} process. The result for the multi-dimensional case follows from the same logic but is more notationally cumbersome.
\\
Consider a one-dimensional real It\^{o} process given by $dX_t = \mu(X_t,t)dt + \sigma(X_t,t)dB_t$ on a filtered probability space $\Omega \subseteq \mathbb{R}$ that admits a probability density function $P(x,t)$ on $\Omega \times [0,\infty)$. Let $f:\mathbb{R}\to\mathbb{R}$ be an arbitrary $C^2(\mathbb{R})$ function. By It\^{o}'s formula, we have:
\begin{align*}
    df(X_t) = \left(\mu f' + \frac{\sigma^2}{2}f''\right)dt + \sigma f' dB_t
\end{align*}
Writing this in integral form and taking expectations on both sides yields:
\begin{align}
\label{eq_expectation_Ito}
    \mathbb{E}[f(X_t)] = \mathbb{E}\left[\int\limits_{0}^{t}\left(\mu f' + \frac{\sigma^2}{2}f''\right)ds\right] + \mathbb{E}\left[\int\limits_{0}^{t}\sigma f' dB_s\right]
\end{align}
Since the Brownian motion is a martingale, as long as $X_t$ and $\sigma(X_t,t)$ are reasonably `nice', the stochastic integral in the second term of the RHS of \eqref{eq_expectation_Ito} will be a continuous $L^2(\mathbb{P})$ martingale starting at the origin, and its expectation will therefore be 0. Using the definition of the expectation value, we are thus left with:
\begin{equation*}
    \int\limits_{\Omega}f(X_t)P(x,t)dx = \int\limits_{\Omega}\left(\int\limits_{0}^{t}\mu f' + \frac{\sigma^2}{2}f''ds\right)P(x,t)dx
\end{equation*}
Assuming derivatives and expectations commute, we can now differentiate with respect to time on both sides and use the fundamental theorem of calculus to write
\begin{equation}
\label{eq_Ito_to_FPE_for_parts}
\int\limits_{\Omega}f(X_t)\frac{\partial P}{\partial t}(x,t)dx = \underbrace{\int\limits_{\Omega}\mu f'P(x,t)dx}_{I(x,t)} + \underbrace{\int\limits_{\Omega}\frac{\sigma^2}{2}f''P(x,t)dx}_{J(x,t)}
\end{equation}
We will now use integration by parts to further evaluate $I(x,t)$ and $J(x,t)$. Recall that the general formula for integration by parts is given by:
\begin{equation*}
    \int\limits_{\Omega}u_{x_i}vdx = -\int\limits_{\Omega}uv_{x_i}dx + \int\limits_{\partial\Omega}uv\gamma_{i}dS(x)
\end{equation*}
where subscript indicates differentiation and $\gamma$ is the unit outward normal. In our case, assuming that $P(x,t) \equiv 0$ on $\partial \Omega$, the boundary term (second term of the RHS) vanishes and we can use integration by parts once on $I(x,t)$ to obtain
\begin{equation}
\label{eq_Ito_to_FPE_I_term}
    I(x,t) = - \int\limits_{\Omega}f(X_t)\left(\frac{\partial}{\partial x}\mu P(x,t)\right)dx
\end{equation}
and twice on $J(x,t)$ to obtain
\begin{align}
    J(x,t) &= - \frac{1}{2}\int\limits_{\Omega}f'(X_t)\left(\frac{\partial}{\partial x}\sigma^2 P(x,t)\right)dx\nonumber\\
    &= \frac{1}{2}\int\limits_{\Omega}f(X_t)\left(\frac{\partial^2}{\partial x^2}\sigma^2 P(x,t)\right)dx\label{eq_Ito_to_FPE_J_term}
\end{align}
Substituting \eqref{eq_Ito_to_FPE_I_term} and \eqref{eq_Ito_to_FPE_J_term} into \eqref{eq_Ito_to_FPE_for_parts} and collecting terms yields
\begin{equation*}
    \int\limits_{\Omega}f(X_t)\frac{\partial P}{\partial t}(x,t)dx = \int\limits_{\Omega}f(X_t)\left[-\frac{\partial}{\partial x}(\mu P(x,t)) + \frac{1}{2}\frac{\partial^2}{\partial x^2}(\sigma^2P(x,t))\right]dx
\end{equation*}
Since this is true for an arbitrary choice of $f(x)$ (as long as $f$ is $C^2$), we are thus led to conclude that the density function $P(x,t)$ must satisfy:
\begin{equation}
\label{FPE}
\frac{\partial P}{\partial t}(x,t) =-\frac{\partial}{\partial x}\left(\mu(x,t) P(x,t)\right) + \frac{1}{2}\frac{\partial^2}{\partial x^2}\left((\sigma(x,t))^2P(x,t)\right)
\end{equation}
Equation \eqref{FPE} is the Fokker-Planck equation in one dimension. Using the exact same strategy, the multidimensional Fokker-Planck equation for the $n$ dimensional It\^{o} Process $d\mathbf{X}_t = \mu(\mathbf{X}_t,t)dt + \sigma(\mathbf{X}_t,t)dB_t$ is found to be:
\begin{equation}
\label{FPE_ndim}
\frac{\partial P}{\partial t}(\mathbf{x},t) =-\sum\limits_{i=1}^{n}\frac{\partial}{\partial x_i}\left(\mu_i(\mathbf{x},t) P(\mathbf{x},t)\right) + \frac{1}{2}\sum\limits_{i=1}^{n}\sum\limits_{j=1}^{n}\frac{\partial^2}{\partial x_ix_j}\left(D_{ij}P(\mathbf{x},t)\right)
\end{equation}
where $\mathbf{D} = \mathbf{\sigma}\mathbf{\sigma}^T$.
\chapter{Deriving stochastic trait frequency dynamics using It\^{o}'s formula}\label{App_density_to_freq}
We first recall the version of the multi-dimensional It\^{o}'s formula that will be relevant to us. Consider an $m$-dimensional real It\^{o} process $\mathbf{X}_t$ given by the solution to
\begin{equation*}
d\mathbf{X}_t = \boldsymbol{\mu}(\mathbf{X}_t)dt + \boldsymbol{\sigma}(\mathbf{X}_t)d\mathbf{B}_t
\end{equation*}
where $\boldsymbol{\mu}: \mathbb{R}^m \to \mathbb{R}^m$ is the `drift vector' and $\boldsymbol{\sigma}: \mathbb{R}^{m} \to \mathbb{R}^{m \times m}$ is the `diffusion matrix'. Let $f: \mathbb{R}^m \to \mathbb{R}$ be an arbitrary $C^2(\mathbb{R}^m)$ function. Then, It\^{o}'s formula (\cite{oksendal_stochastic_1998}, Section 4.2) states that  the stochastic process $f(\mathbf{X}_t)$ must satisfy:
\begin{equation}
\label{nD_Ito_formula}
df(\mathbf{X}_t) = \left[\left(\nabla_{\mathbf{X}}f\right)^{\mathrm{T}}\boldsymbol{\mu} + \frac{1}{2}\mathrm{Tr}[\boldsymbol{\sigma}^{\mathrm{T}}(H_{\mathbf{X}}f)\boldsymbol{\sigma}]\right]dt + \left(\nabla_{\mathbf{X}}f\right)^{\mathrm{T}}\boldsymbol{\sigma}d\mathbf{B}_t
\end{equation}
where $\mathrm{Tr}[\cdot]$ denotes the trace of a matrix, $\left(\cdot\right)^{\mathrm{T}}$ denotes the transpose, and we have suppressed the $\mathbf{X}_t$ dependence of $\boldsymbol{\mu}$ and $\boldsymbol{\sigma}$ to reduce clutter. Here, $\nabla_{\mathbf{X}}f$ is the $m$ dimensional \emph{gradient vector} of $f$ and $H_{\mathbf{X}}f$ is the $m \times m$ \emph{Hessian matrix} of $f$, respectively defined for $f(x_1,\ldots,x_m)$ as:
\begin{align*}
\left(\nabla_{\mathbf{x}} f\right)_j &= \frac{\partial f}{\partial x_j}\\
\left(H_{\mathbf{x}} f\right)_{jk} &= \frac{\partial^2 f}{\partial x_j \partial x_k } 
\end{align*}
In our case, we have the It\^{o} process given by \eqref{nD_Ito_SDE}, which defines how the density of each type of individual changes over time. We thus have $\boldsymbol{\mu}(\mathbf{X}_t) = \mathbf{A}^{-}(\mathbf{X}_t)$ and $\boldsymbol{\sigma}(\mathbf{X}_t) = \mathbf{D}(\mathbf{X}_t)/\sqrt{K}$.  For each fixed $i \in \{1,2,\ldots,m\}$, let us define a scalar function $f_i: \mathbb{R}^m \to \mathbb{R}$ as
\begin{equation*}
f_i(\mathbf{x}) = \frac{x_i}{\sum\limits_{j=1}^{m}x_j}
\end{equation*}
Thus, $f_i(\mathbf{X}_t)$ gives us the frequency of type $i$ individuals when the population is described by the vector $\mathbf{X}_t$. This function is obviously $C^2(\mathbb{R}^m)$, and we can thus use It\^{o}'s formula \eqref{nD_Ito_formula} to describe how it changes over time. The $j\textsuperscript{th}$ element of the gradient of $f_i$ is given by:
\begin{align}
\left(\nabla_{\mathbf{x}} f_i\right)_j &= \frac{\partial }{\partial x_j}\left(\frac{x_i}{\sum\limits_{k=1}^{m}x_k}\right)\nonumber\\
&= \left(\frac{1}{N}\frac{\partial x_i}{\partial x_j} 
- \frac{x_i}{N^2}\sum\limits_{k=1}^{m}\frac{\partial x_k}{\partial x_j}\right)\nonumber\\
&= \frac{1}{N}\left(\delta_{ij}-p_i\right)\label{nD_jacobian_for_ito}
\end{align}
where we have defined the total (scaled) population size\footnote{This is $N_{K}(t)$ in the main text, but we omit the subscript here to reduce notational clutter}  $N = \sum_i x_i$ and the frequency of the $i\textsuperscript{th}$ type $p_i = f_i(x)$ and used the fact that $\frac{\partial x_j}{\partial x_k} = \delta_{jk}$. The $jk\textsuperscript{th}$ element of the Hessian is given by:
\begin{align}
\left(H_{\mathbf{x}} f_i\right)_{jk} &= \frac{\partial^2 }{\partial x_j \partial x_k}\left( \frac{x_i}{\sum\limits_{l=1}^{m}x_l}\right)\nonumber\\
&= \frac{\partial}{\partial x_j}\left(\frac{\delta_{ik}}{N}-\frac{x_i}{N^2}\right)\nonumber\\
&= \frac{1}{N^2}\left(2p_i - \delta_{ij}-\delta_{ik}\right)\label{nD_hessian_for_ito}
\end{align}
Thus, for the first term of \eqref{nD_Ito_formula}, we have:
\begin{align}
\left(\nabla_{\mathbf{X}}f_i\right)^{\mathrm{T}}\boldsymbol{\mathbf{A}^{-}} &= \sum\limits_{j=1}^{m}\left(\left(\nabla_{\mathbf{x}} f_i\right)_j\right)A^{-}_{j} \nonumber\\
&= \frac{1}{N}\sum\limits_{j=1}^{m}\left(\delta_{ij}-p_i\right)A^{-}_{j}\nonumber\\
&= \frac{1}{N}\left(A^{-}_{i} - p_i\sum\limits_{j=1}^{m}A^{-}_{j}\right)\label{nD_for_Ito_first_term}
\end{align}
If the birth and death rates take the form of \eqref{nD_functional_forms_for_replicator}, then it is easy to see using \eqref{nD_det_limit_fitess_defn} that equation \eqref{nD_for_Ito_first_term} is exactly the RHS of \eqref{nD_replicator_intermediate_1}. Thus, when the birth and death rates take the form of \eqref{nD_functional_forms_for_replicator}, then \eqref{nD_for_Ito_first_term} describes the deterministic component of the dynamics as described by the replicator-mutator equation, Price equation, etc. in the infinite population limit. These are the effects of selection and mutation at the infinite population limit. However, the finiteness of the population adds a second term to these dynamics, described by the second term that multiplies $dt$ in \eqref{nD_Ito_formula}. To calculate it, we first calculate:
\begin{align}
\frac{1}{\sqrt{K}}\left(H_{\mathbf{x}} f_i \mathbf{D}\right)_{jk} &= \frac{1}{\sqrt{K}}\sum\limits_{l=1}^{m} \left(H_{\mathbf{x}} f_i \right)_{jl}\left(\mathbf{D}\right)_{lk}\nonumber\\
&= \frac{1}{\sqrt{K}N^2}\sum\limits_{l=1}^{m}\left(2p_i - \delta_{ij} - \delta_{il}\right)\delta_{lk}\left(A^{+}_{l}A^{+}_{k}\right)^{\frac{1}{4}}\\
&=  \frac{1}{\sqrt{K}N^2}\left(\left(2p_i -\delta_{ij}\right)(A^{+}_{k})^{\frac{1}{2}} -\delta_{ik}\left(A^{+}_{i}A^{+}_{k}\right)^{\frac{1}{4}}\right)\\
&= \frac{1}{\sqrt{K}N^2}\left(2p_i -\delta_{ij} -\delta_{ik}\right)(A^{+}_{k})^{\frac{1}{2}}
\end{align}
and thus:
\begin{align}
\frac{1}{K}\left(\mathbf{D}^{\mathrm{T}} H_{\mathbf{x}} f_i \mathbf{D}\right)_{lk} &=\frac{1}{K}\sum\limits_{j=1}^{m}\left(\mathbf{D}^{\mathrm{T}}\right)_{lj}\left(H_{\mathbf{x}} f_i \mathbf{D}\right)_{jk}\nonumber\\
&=  \frac{1}{KN^2}\sum\limits_{j=1}^{m}\delta_{lj}\left(A^{+}_{l}A^{+}_{j}\right)^{\frac{1}{4}}(A^{+}_{k})^{\frac{1}{2}}\left(2p_i -\delta_{ij} -\delta_{ik}\right)\\
&=  \frac{1}{KN^2}(A^{+}_{k})^{\frac{1}{2}}\left(2p_i(A^{+}_{l})^{\frac{1}{2}} - (A^{+}_{i})^{\frac{1}{2}}\delta_{il} - (A^{+}_{l})^{\frac{1}{2}}\delta_{ik}\right)
\end{align}
Using this, we see that the trace of this matrix is given by:
\begin{align}
\frac{1}{K}\mathrm{Tr}[\mathbf{D}^{\mathrm{T}} H_{\mathbf{x}} f_i \mathbf{D}] &= \frac{1}{K}\sum\limits_{k=1}^{m}\left(\mathbf{D}^{\mathrm{T}} H_{\mathbf{x}} f_i \mathbf{D}\right)_{kk}\nonumber\\
&= \frac{1}{KN^2}\sum\limits_{k=1}^{m}\left(2p_i(A^{+}_{k}A^{+}_{k})^{\frac{1}{2}} - (A^{+}_{i}A^{+}_{k})^{\frac{1}{2}}\delta_{ik} - (A^{+}_{k}A^{+}_{k})^{\frac{1}{2}}\delta_{ik}\right)\\
&= \frac{1}{KN^2}\left(2p_i\left(\sum\limits_{k=1}^{m} A^{+}_k\right) - 2A^{+}_{i}\right)
\end{align}
and thus, the second term multiplying $dt$ in \eqref{nD_Ito_formula} is given by:
\begin{equation}
\frac{1}{2K}\mathrm{Tr}[\mathbf{D}^{\mathrm{T}} H_{\mathbf{x}} f_i \mathbf{D}] =  \frac{-1}{KN^2}\left(A^{+}_{i}-p_i\left(\sum\limits_{k=1}^{m} A^{+}_k\right)\right)\label{nD_for_Ito_second_term}
\end{equation}
Finally, denoting $d\mathbf{B}_t = [dB^{(1)}_t,dB^{(2)}_t, \ldots, dB^{(m)}_t]^{\mathrm{T}}$ where each $dB^{(j)}_t$ is an independent one dimensional Wiener process, we have:
\begin{align}
\left(\mathbf{D}d\mathbf{B}_t\right)_j &= \sum\limits_{k=1}^{m}\mathbf{D}_{jk}dB^{(k)}_t\nonumber\\
&= \sum\limits_{k=1}^{m}\delta_{jk}\left(A^{+}_{j}A^{+}_{k}\right)^{\frac{1}{4}}dB^{(k)}_t\\
&= \left(A^{+}_{j}\right)^{1/2}dB^{(j)}_t
\end{align}
Thus, using \eqref{nD_jacobian_for_ito}, we see that the last term on the RHS of \eqref{nD_Ito_formula} is given by
\begin{align}
\frac{1}{\sqrt{K}}\left(\nabla_{\mathbf{X}}f\right)^{\mathrm{T}}\mathbf{D}d\mathbf{B}_t &= \frac{1}{\sqrt{K}}\sum\limits_{j=1}^{m}\left(\nabla_{\mathbf{x}} f_i\right)_j\left(\mathbf{D}d\mathbf{B}_t\right)_j\nonumber\\
&=  \frac{1}{N\sqrt{K}}\sum\limits_{j=1}^{m}\left(\delta_{ij}-p_i\right)\left(A^{+}_{j}\right)^{1/2}dB^{(j)}_t\\
&= \frac{1}{N\sqrt{K}}\left(A^{+}_{i}\right)^{1/2}dB^{(i)}_t - p_i\sum\limits_{j=1}^{m}\left(A^{+}_{j}\right)^{1/2}dB^{(j)}_t\label{nD_for_Ito_third_term}
\end{align}
Putting equations \eqref{nD_for_Ito_first_term}, \eqref{nD_for_Ito_second_term} and \eqref{nD_for_Ito_third_term} into \eqref{nD_Ito_formula}, we see that $p_i = f_i(\mathbf{X})_t$, the frequency of the $i\textsuperscript{th}$ type in the population $\mathbf{X}_t$, changes according to the equation:
\begin{equation}
\begin{aligned}
N(t) dp_i &= \left(A^{-}_{i} - p_i\sum\limits_{j=1}^{m}A^{-}_{j}\right)dt - \frac{1}{K}\frac{1}{N(t)}\left(A^{+}_{i}-p_i\left(\sum\limits_{k=1}^{m} A^{+}_k\right)\right)dt\\
&+ \frac{1}{\sqrt{K}}\left[\left(A^{+}_{i}\right)^{1/2}dB^{(i)}_t - p_i\sum\limits_{j=1}^{m}\left(A^{+}_{j}\right)^{1/2}dB^{(j)}_t\right]
\end{aligned}
\end{equation}
Substituting the functional forms given by \eqref{nD_functional_forms_for_replicator} and repeating calculations for the $A^{+}_i$ terms exactly analogous to those done in going from \eqref{nD_replicator_intermediate_1} to \eqref{nD_replicator_mutator} now yields equation \eqref{nD_eqn_for_frequencies} in the main text.
\end{appendices}

\printbibliography[title=References]
\end{document}
